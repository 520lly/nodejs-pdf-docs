\section{Stream}

\begin{Shaded}
\begin{Highlighting}[]
\DataTypeTok{Stability}\NormalTok{: }\DecValTok{2} \NormalTok{- Unstable}
\end{Highlighting}
\end{Shaded}

A stream is an abstract interface implemented by various objects in
Node. For example a
\href{http.html\#http\_http\_incomingmessage}{request to an HTTP server}
is a stream, as is
\href{process.html\#process\_process\_stdout}{stdout}. Streams are
readable, writable, or both. All streams are instances of
\href{events.html\#events\_class\_events\_eventemitter}{EventEmitter}

You can load the Stream base classes by doing
\texttt{require('stream')}. There are base classes provided for
\hyperref[stream_class_stream_readable]{Readable} streams,
\hyperref[stream_class_stream_writable]{Writable} streams,
\hyperref[stream_class_stream_duplex]{Duplex} streams, and
\hyperref[stream_class_stream_transform]{Transform} streams.

This document is split up into 3 sections. The first explains the parts
of the API that you need to be aware of to use streams in your programs.
If you never implement a streaming API yourself, you can stop there.

The second section explains the parts of the API that you need to use if
you implement your own custom streams yourself. The API is designed to
make this easy for you to do.

The third section goes into more depth about how streams work, including
some of the internal mechanisms and functions that you should probably
not modify unless you definitely know what you are doing.

\subsection{API for Stream Consumers}

Streams can be either
\hyperref[stream_class_stream_readable]{Readable},
\hyperref[stream_class_stream_writable]{Writable}, or both
(\hyperref[stream_class_stream_duplex]{Duplex}).

All streams are EventEmitters, but they also have other custom methods
and properties depending on whether they are Readable, Writable, or
Duplex.

If a stream is both Readable and Writable, then it implements all of the
methods and events below. So, a
\hyperref[stream_class_stream_duplex]{Duplex} or
\hyperref[stream_class_stream_transform]{Transform} stream is fully
described by this API, though their implementation may be somewhat
different.

It is not necessary to implement Stream interfaces in order to consume
streams in your programs. If you \textbf{are} implementing streaming
interfaces in your own program, please also refer to
\hyperref[stream_api_for_stream_implementors]{API for Stream
Implementors} below.

Almost all Node programs, no matter how simple, use Streams in some way.
Here is an example of using Streams in a Node program:

\begin{Shaded}
\begin{Highlighting}[]
\KeywordTok{var} \NormalTok{http = require(}\CharTok{'http'}\NormalTok{);}

\KeywordTok{var} \NormalTok{server = }\KeywordTok{http}\NormalTok{.}\FunctionTok{createServer}\NormalTok{(}\KeywordTok{function} \NormalTok{(req, res) \{}
  \CommentTok{// req is an http.IncomingMessage, which is a Readable Stream}
  \CommentTok{// res is an http.ServerResponse, which is a Writable Stream}

  \KeywordTok{var} \NormalTok{body = }\CharTok{''}\NormalTok{;}
  \CommentTok{// we want to get the data as utf8 strings}
  \CommentTok{// If you don't set an encoding, then you'll get Buffer objects}
  \KeywordTok{req}\NormalTok{.}\FunctionTok{setEncoding}\NormalTok{(}\CharTok{'utf8'}\NormalTok{);}

  \CommentTok{// Readable streams emit 'data' events once a listener is added}
  \KeywordTok{req}\NormalTok{.}\FunctionTok{on}\NormalTok{(}\CharTok{'data'}\NormalTok{, }\KeywordTok{function} \NormalTok{(chunk) \{}
    \NormalTok{body += chunk;}
  \NormalTok{\})}

  \CommentTok{// the end event tells you that you have entire body}
  \KeywordTok{req}\NormalTok{.}\FunctionTok{on}\NormalTok{(}\CharTok{'end'}\NormalTok{, }\KeywordTok{function} \NormalTok{() \{}
    \KeywordTok{try} \NormalTok{\{}
      \KeywordTok{var} \NormalTok{data = }\KeywordTok{JSON}\NormalTok{.}\FunctionTok{parse}\NormalTok{(body);}
    \NormalTok{\} }\KeywordTok{catch} \NormalTok{(er) \{}
      \CommentTok{// uh oh!  bad json!}
      \KeywordTok{res}\NormalTok{.}\FunctionTok{statusCode} \NormalTok{= }\DecValTok{400}\NormalTok{;}
      \KeywordTok{return} \KeywordTok{res}\NormalTok{.}\FunctionTok{end}\NormalTok{(}\CharTok{'error: '} \NormalTok{+ }\KeywordTok{er}\NormalTok{.}\FunctionTok{message}\NormalTok{);}
    \NormalTok{\}}

    \CommentTok{// write back something interesting to the user:}
    \KeywordTok{res}\NormalTok{.}\FunctionTok{write}\NormalTok{(}\KeywordTok{typeof} \NormalTok{data);}
    \KeywordTok{res}\NormalTok{.}\FunctionTok{end}\NormalTok{();}
  \NormalTok{\})}
\NormalTok{\})}

\KeywordTok{server}\NormalTok{.}\FunctionTok{listen}\NormalTok{(}\DecValTok{1337}\NormalTok{);}

\CommentTok{// $ curl localhost:1337 -d '\{\}'}
\CommentTok{// object}
\CommentTok{// $ curl localhost:1337 -d '"foo"'}
\CommentTok{// string}
\CommentTok{// $ curl localhost:1337 -d 'not json'}
\CommentTok{// error: Unexpected token o}
\end{Highlighting}
\end{Shaded}

\subsubsection{Class: stream.Readable}

The Readable stream interface is the abstraction for a \emph{source} of
data that you are reading from. In other words, data comes \emph{out} of
a Readable stream.

A Readable stream will not start emitting data until you indicate that
you are ready to receive it.

Readable streams have two ``modes'': a \textbf{flowing mode} and a
\textbf{paused mode}. When in flowing mode, data is read from the
underlying system and provided to your program as fast as possible. In
paused mode, you must explicitly call \texttt{stream.read()} to get
chunks of data out. Streams start out in paused mode.

\textbf{Note}: If no data event handlers are attached, and there are no
\hyperref[stream_readable_pipe_destination_options]{\texttt{pipe()}}
destinations, and the stream is switched into flowing mode, then data
will be lost.

You can switch to flowing mode by doing any of the following:

\begin{itemize}
\item
  Adding a \hyperref[stream_event_data]{\texttt{'data'} event} handler
  to listen for data.
\item
  Calling the \hyperref[stream_readable_resume]{\texttt{resume()}}
  method to explicitly open the flow.
\item
  Calling the
  \hyperref[stream_readable_pipe_destination_options]{\texttt{pipe()}}
  method to send the data to a
  \hyperref[stream_class_stream_writable]{Writable}.
\end{itemize}

You can switch back to paused mode by doing either of the following:

\begin{itemize}
\item
  If there are no pipe destinations, by calling the
  \hyperref[stream_readable_pause]{\texttt{pause()}} method.
\item
  If there are pipe destinations, by removing any
  \hyperref[stream_event_data]{\texttt{'data'} event} handlers, and
  removing all pipe destinations by calling the
  \hyperref[stream_readable_unpipe_destination]{\texttt{unpipe()}}
  method.
\end{itemize}

Note that, for backwards compatibility reasons, removing \texttt{'data'}
event handlers will \textbf{not} automatically pause the stream. Also,
if there are piped destinations, then calling \texttt{pause()} will not
guarantee that the stream will \emph{remain} paused once those
destinations drain and ask for more data.

Examples of readable streams include:

\begin{itemize}
\item
  \href{http.html\#http\_http\_incomingmessage}{http responses, on the
  client}
\item
  \href{http.html\#http\_http\_incomingmessage}{http requests, on the
  server}
\item
  \href{fs.html\#fs\_class\_fs\_readstream}{fs read streams}
\item
  \href{zlib.html}{zlib streams}
\item
  \href{crypto.html}{crypto streams}
\item
  \href{net.html\#net\_class\_net\_socket}{tcp sockets}
\item
  \href{child\_process.html\#child\_process\_child\_stdout}{child
  process stdout and stderr}
\item
  \href{process.html\#process\_process\_stdin}{process.stdin}
\end{itemize}

\paragraph{Event: `readable'}

When a chunk of data can be read from the stream, it will emit a
\texttt{'readable'} event.

In some cases, listening for a \texttt{'readable'} event will cause some
data to be read into the internal buffer from the underlying system, if
it hadn't already.

\begin{Shaded}
\begin{Highlighting}[]
\KeywordTok{var} \NormalTok{readable = getReadableStreamSomehow();}
\KeywordTok{readable}\NormalTok{.}\FunctionTok{on}\NormalTok{(}\CharTok{'readable'}\NormalTok{, }\KeywordTok{function}\NormalTok{() \{}
  \CommentTok{// there is some data to read now}
\NormalTok{\})}
\end{Highlighting}
\end{Shaded}

Once the internal buffer is drained, a \texttt{readable} event will fire
again when more data is available.

\paragraph{Event: `data'}

\begin{itemize}
\item
  \texttt{chunk} \{Buffer \textbar{} String\} The chunk of data.
\end{itemize}

Attaching a \texttt{data} event listener to a stream that has not been
explicitly paused will switch the stream into flowing mode. Data will
then be passed as soon as it is available.

If you just want to get all the data out of the stream as fast as
possible, this is the best way to do so.

\begin{Shaded}
\begin{Highlighting}[]
\KeywordTok{var} \NormalTok{readable = getReadableStreamSomehow();}
\KeywordTok{readable}\NormalTok{.}\FunctionTok{on}\NormalTok{(}\CharTok{'data'}\NormalTok{, }\KeywordTok{function}\NormalTok{(chunk) \{}
  \KeywordTok{console}\NormalTok{.}\FunctionTok{log}\NormalTok{(}\CharTok{'got %d bytes of data'}\NormalTok{, }\KeywordTok{chunk}\NormalTok{.}\FunctionTok{length}\NormalTok{);}
\NormalTok{\})}
\end{Highlighting}
\end{Shaded}

\paragraph{Event: `end'}

This event fires when no more data will be provided.

Note that the \texttt{end} event \textbf{will not fire} unless the data
is completely consumed. This can be done by switching into flowing mode,
or by calling \texttt{read()} repeatedly until you get to the end.

\begin{Shaded}
\begin{Highlighting}[]
\KeywordTok{var} \NormalTok{readable = getReadableStreamSomehow();}
\KeywordTok{readable}\NormalTok{.}\FunctionTok{on}\NormalTok{(}\CharTok{'data'}\NormalTok{, }\KeywordTok{function}\NormalTok{(chunk) \{}
  \KeywordTok{console}\NormalTok{.}\FunctionTok{log}\NormalTok{(}\CharTok{'got %d bytes of data'}\NormalTok{, }\KeywordTok{chunk}\NormalTok{.}\FunctionTok{length}\NormalTok{);}
\NormalTok{\})}
\KeywordTok{readable}\NormalTok{.}\FunctionTok{on}\NormalTok{(}\CharTok{'end'}\NormalTok{, }\KeywordTok{function}\NormalTok{() \{}
  \KeywordTok{console}\NormalTok{.}\FunctionTok{log}\NormalTok{(}\CharTok{'there will be no more data.'}\NormalTok{);}
\NormalTok{\});}
\end{Highlighting}
\end{Shaded}

\paragraph{Event: `close'}

Emitted when the underlying resource (for example, the backing file
descriptor) has been closed. Not all streams will emit this.

\paragraph{Event: `error'}

Emitted if there was an error receiving data.

\paragraph{readable.read({[}size{]})}

\begin{itemize}
\item
  \texttt{size} \{Number\} Optional argument to specify how much data to
  read.
\item
  Return \{String \textbar{} Buffer \textbar{} null\}
\end{itemize}

The \texttt{read()} method pulls some data out of the internal buffer
and returns it. If there is no data available, then it will return
\texttt{null}.

If you pass in a \texttt{size} argument, then it will return that many
bytes. If \texttt{size} bytes are not available, then it will return
\texttt{null}.

If you do not specify a \texttt{size} argument, then it will return all
the data in the internal buffer.

This method should only be called in paused mode. In flowing mode, this
method is called automatically until the internal buffer is drained.

\begin{Shaded}
\begin{Highlighting}[]
\KeywordTok{var} \NormalTok{readable = getReadableStreamSomehow();}
\KeywordTok{readable}\NormalTok{.}\FunctionTok{on}\NormalTok{(}\CharTok{'readable'}\NormalTok{, }\KeywordTok{function}\NormalTok{() \{}
  \KeywordTok{var} \NormalTok{chunk;}
  \KeywordTok{while} \NormalTok{(null !== (chunk = }\KeywordTok{readable}\NormalTok{.}\FunctionTok{read}\NormalTok{())) \{}
    \KeywordTok{console}\NormalTok{.}\FunctionTok{log}\NormalTok{(}\CharTok{'got %d bytes of data'}\NormalTok{, }\KeywordTok{chunk}\NormalTok{.}\FunctionTok{length}\NormalTok{);}
  \NormalTok{\}}
\NormalTok{\});}
\end{Highlighting}
\end{Shaded}

If this method returns a data chunk, then it will also trigger the
emission of a \hyperref[stream_event_data]{\texttt{'data'} event}.

\paragraph{readable.setEncoding(encoding)}

\begin{itemize}
\item
  \texttt{encoding} \{String\} The encoding to use.
\item
  Return: \texttt{this}
\end{itemize}

Call this function to cause the stream to return strings of the
specified encoding instead of Buffer objects. For example, if you do
\texttt{readable.setEncoding('utf8')}, then the output data will be
interpreted as UTF-8 data, and returned as strings. If you do
\texttt{readable.setEncoding('hex')}, then the data will be encoded in
hexadecimal string format.

This properly handles multi-byte characters that would otherwise be
potentially mangled if you simply pulled the Buffers directly and called
\texttt{buf.toString(encoding)} on them. If you want to read the data as
strings, always use this method.

\begin{Shaded}
\begin{Highlighting}[]
\KeywordTok{var} \NormalTok{readable = getReadableStreamSomehow();}
\KeywordTok{readable}\NormalTok{.}\FunctionTok{setEncoding}\NormalTok{(}\CharTok{'utf8'}\NormalTok{);}
\KeywordTok{readable}\NormalTok{.}\FunctionTok{on}\NormalTok{(}\CharTok{'data'}\NormalTok{, }\KeywordTok{function}\NormalTok{(chunk) \{}
  \KeywordTok{assert}\NormalTok{.}\FunctionTok{equal}\NormalTok{(}\KeywordTok{typeof} \NormalTok{chunk, }\CharTok{'string'}\NormalTok{);}
  \KeywordTok{console}\NormalTok{.}\FunctionTok{log}\NormalTok{(}\CharTok{'got %d characters of string data'}\NormalTok{, }\KeywordTok{chunk}\NormalTok{.}\FunctionTok{length}\NormalTok{);}
\NormalTok{\})}
\end{Highlighting}
\end{Shaded}

\paragraph{readable.resume()}

\begin{itemize}
\item
  Return: \texttt{this}
\end{itemize}

This method will cause the readable stream to resume emitting
\texttt{data} events.

This method will switch the stream into flowing mode. If you do
\emph{not} want to consume the data from a stream, but you \emph{do}
want to get to its \texttt{end} event, you can call
\hyperref[stream_readable_resume]{\texttt{readable.resume()}} to open
the flow of data.

\begin{Shaded}
\begin{Highlighting}[]
\KeywordTok{var} \NormalTok{readable = getReadableStreamSomehow();}
\KeywordTok{readable}\NormalTok{.}\FunctionTok{resume}\NormalTok{();}
\KeywordTok{readable}\NormalTok{.}\FunctionTok{on}\NormalTok{(}\CharTok{'end'}\NormalTok{, }\KeywordTok{function}\NormalTok{(chunk) \{}
  \KeywordTok{console}\NormalTok{.}\FunctionTok{log}\NormalTok{(}\CharTok{'got to the end, but did not read anything'}\NormalTok{);}
\NormalTok{\})}
\end{Highlighting}
\end{Shaded}

\paragraph{readable.pause()}

\begin{itemize}
\item
  Return: \texttt{this}
\end{itemize}

This method will cause a stream in flowing mode to stop emitting
\texttt{data} events, switching out of flowing mode. Any data that
becomes available will remain in the internal buffer.

\begin{Shaded}
\begin{Highlighting}[]
\KeywordTok{var} \NormalTok{readable = getReadableStreamSomehow();}
\KeywordTok{readable}\NormalTok{.}\FunctionTok{on}\NormalTok{(}\CharTok{'data'}\NormalTok{, }\KeywordTok{function}\NormalTok{(chunk) \{}
  \KeywordTok{console}\NormalTok{.}\FunctionTok{log}\NormalTok{(}\CharTok{'got %d bytes of data'}\NormalTok{, }\KeywordTok{chunk}\NormalTok{.}\FunctionTok{length}\NormalTok{);}
  \KeywordTok{readable}\NormalTok{.}\FunctionTok{pause}\NormalTok{();}
  \KeywordTok{console}\NormalTok{.}\FunctionTok{log}\NormalTok{(}\CharTok{'there will be no more data for 1 second'}\NormalTok{);}
  \NormalTok{setTimeout(}\KeywordTok{function}\NormalTok{() \{}
    \KeywordTok{console}\NormalTok{.}\FunctionTok{log}\NormalTok{(}\CharTok{'now data will start flowing again'}\NormalTok{);}
    \KeywordTok{readable}\NormalTok{.}\FunctionTok{resume}\NormalTok{();}
  \NormalTok{\}, }\DecValTok{1000}\NormalTok{);}
\NormalTok{\})}
\end{Highlighting}
\end{Shaded}

\paragraph{readable.pipe(destination, {[}options{]})}

\begin{itemize}
\item
  \texttt{destination}
  \{\hyperref[stream_class_stream_writable]{Writable} Stream\} The
  destination for writing data
\item
  \texttt{options} \{Object\} Pipe options
\item
  \texttt{end} \{Boolean\} End the writer when the reader ends. Default
  = \texttt{true}
\end{itemize}

This method pulls all the data out of a readable stream, and writes it
to the supplied destination, automatically managing the flow so that the
destination is not overwhelmed by a fast readable stream.

Multiple destinations can be piped to safely.

\begin{Shaded}
\begin{Highlighting}[]
\KeywordTok{var} \NormalTok{readable = getReadableStreamSomehow();}
\KeywordTok{var} \NormalTok{writable = }\KeywordTok{fs}\NormalTok{.}\FunctionTok{createWriteStream}\NormalTok{(}\CharTok{'file.txt'}\NormalTok{);}
\CommentTok{// All the data from readable goes into 'file.txt'}
\KeywordTok{readable}\NormalTok{.}\FunctionTok{pipe}\NormalTok{(writable);}
\end{Highlighting}
\end{Shaded}

This function returns the destination stream, so you can set up pipe
chains like so:

\begin{Shaded}
\begin{Highlighting}[]
\KeywordTok{var} \NormalTok{r = }\KeywordTok{fs}\NormalTok{.}\FunctionTok{createReadStream}\NormalTok{(}\CharTok{'file.txt'}\NormalTok{);}
\KeywordTok{var} \NormalTok{z = }\KeywordTok{zlib}\NormalTok{.}\FunctionTok{createGzip}\NormalTok{();}
\KeywordTok{var} \NormalTok{w = }\KeywordTok{fs}\NormalTok{.}\FunctionTok{createWriteStream}\NormalTok{(}\CharTok{'file.txt.gz'}\NormalTok{);}
\KeywordTok{r}\NormalTok{.}\FunctionTok{pipe}\NormalTok{(z).}\FunctionTok{pipe}\NormalTok{(w);}
\end{Highlighting}
\end{Shaded}

For example, emulating the Unix \texttt{cat} command:

\begin{Shaded}
\begin{Highlighting}[]
\KeywordTok{process.stdin}\NormalTok{.}\FunctionTok{pipe}\NormalTok{(}\KeywordTok{process}\NormalTok{.}\FunctionTok{stdout}\NormalTok{);}
\end{Highlighting}
\end{Shaded}

By default
\hyperref[stream_writable_end_chunk_encoding_callback]{\texttt{end()}}
is called on the destination when the source stream emits \texttt{end},
so that \texttt{destination} is no longer writable. Pass
\texttt{\{ end: false \}} as \texttt{options} to keep the destination
stream open.

This keeps \texttt{writer} open so that ``Goodbye'' can be written at
the end.

\begin{Shaded}
\begin{Highlighting}[]
\KeywordTok{reader}\NormalTok{.}\FunctionTok{pipe}\NormalTok{(writer, \{ }\DataTypeTok{end}\NormalTok{: }\KeywordTok{false} \NormalTok{\});}
\KeywordTok{reader}\NormalTok{.}\FunctionTok{on}\NormalTok{(}\CharTok{'end'}\NormalTok{, }\KeywordTok{function}\NormalTok{() \{}
  \KeywordTok{writer}\NormalTok{.}\FunctionTok{end}\NormalTok{(}\CharTok{'Goodbye\textbackslash{}n'}\NormalTok{);}
\NormalTok{\});}
\end{Highlighting}
\end{Shaded}

Note that \texttt{process.stderr} and \texttt{process.stdout} are never
closed until the process exits, regardless of the specified options.

\paragraph{readable.unpipe({[}destination{]})}

\begin{itemize}
\item
  \texttt{destination}
  \{\hyperref[stream_class_stream_writable]{Writable} Stream\}
  Optional specific stream to unpipe
\end{itemize}

This method will remove the hooks set up for a previous \texttt{pipe()}
call.

If the destination is not specified, then all pipes are removed.

If the destination is specified, but no pipe is set up for it, then this
is a no-op.

\begin{Shaded}
\begin{Highlighting}[]
\KeywordTok{var} \NormalTok{readable = getReadableStreamSomehow();}
\KeywordTok{var} \NormalTok{writable = }\KeywordTok{fs}\NormalTok{.}\FunctionTok{createWriteStream}\NormalTok{(}\CharTok{'file.txt'}\NormalTok{);}
\CommentTok{// All the data from readable goes into 'file.txt',}
\CommentTok{// but only for the first second}
\KeywordTok{readable}\NormalTok{.}\FunctionTok{pipe}\NormalTok{(writable);}
\NormalTok{setTimeout(}\KeywordTok{function}\NormalTok{() \{}
  \KeywordTok{console}\NormalTok{.}\FunctionTok{log}\NormalTok{(}\CharTok{'stop writing to file.txt'}\NormalTok{);}
  \KeywordTok{readable}\NormalTok{.}\FunctionTok{unpipe}\NormalTok{(writable);}
  \KeywordTok{console}\NormalTok{.}\FunctionTok{log}\NormalTok{(}\CharTok{'manually close the file stream'}\NormalTok{);}
  \KeywordTok{writable}\NormalTok{.}\FunctionTok{end}\NormalTok{();}
\NormalTok{\}, }\DecValTok{1000}\NormalTok{);}
\end{Highlighting}
\end{Shaded}

\paragraph{readable.unshift(chunk)}

\begin{itemize}
\item
  \texttt{chunk} \{Buffer \textbar{} String\} Chunk of data to unshift
  onto the read queue
\end{itemize}

This is useful in certain cases where a stream is being consumed by a
parser, which needs to ``un-consume'' some data that it has
optimistically pulled out of the source, so that the stream can be
passed on to some other party.

If you find that you must often call \texttt{stream.unshift(chunk)} in
your programs, consider implementing a
\hyperref[stream_class_stream_transform]{Transform} stream instead.
(See API for Stream Implementors, below.)

\begin{Shaded}
\begin{Highlighting}[]
\CommentTok{// Pull off a header delimited by \textbackslash{}n\textbackslash{}n}
\CommentTok{// use unshift() if we get too much}
\CommentTok{// Call the callback with (error, header, stream)}
\KeywordTok{var} \NormalTok{StringDecoder = require(}\CharTok{'string_decoder'}\NormalTok{).}\FunctionTok{StringDecoder}\NormalTok{;}
\KeywordTok{function} \NormalTok{parseHeader(stream, callback) \{}
  \KeywordTok{stream}\NormalTok{.}\FunctionTok{on}\NormalTok{(}\CharTok{'error'}\NormalTok{, callback);}
  \KeywordTok{stream}\NormalTok{.}\FunctionTok{on}\NormalTok{(}\CharTok{'readable'}\NormalTok{, onReadable);}
  \KeywordTok{var} \NormalTok{decoder = }\KeywordTok{new} \NormalTok{StringDecoder(}\CharTok{'utf8'}\NormalTok{);}
  \KeywordTok{var} \NormalTok{header = }\CharTok{''}\NormalTok{;}
  \KeywordTok{function} \NormalTok{onReadable() \{}
    \KeywordTok{var} \NormalTok{chunk;}
    \KeywordTok{while} \NormalTok{(null !== (chunk = }\KeywordTok{stream}\NormalTok{.}\FunctionTok{read}\NormalTok{())) \{}
      \KeywordTok{var} \NormalTok{str = }\KeywordTok{decoder}\NormalTok{.}\FunctionTok{write}\NormalTok{(chunk);}
      \KeywordTok{if} \NormalTok{(}\KeywordTok{str}\NormalTok{.}\FunctionTok{match}\NormalTok{(}\OtherTok{/}\BaseNTok{\textbackslash{}n\textbackslash{}n}\OtherTok{/}\NormalTok{)) \{}
        \CommentTok{// found the header boundary}
        \KeywordTok{var} \NormalTok{split = }\KeywordTok{str}\NormalTok{.}\FunctionTok{split}\NormalTok{(}\OtherTok{/}\BaseNTok{\textbackslash{}n\textbackslash{}n}\OtherTok{/}\NormalTok{);}
        \NormalTok{header += }\KeywordTok{split}\NormalTok{.}\FunctionTok{shift}\NormalTok{();}
        \KeywordTok{var} \NormalTok{remaining = }\KeywordTok{split}\NormalTok{.}\FunctionTok{join}\NormalTok{(}\CharTok{'\textbackslash{}n\textbackslash{}n'}\NormalTok{);}
        \KeywordTok{var} \NormalTok{buf = }\KeywordTok{new} \NormalTok{Buffer(remaining, }\CharTok{'utf8'}\NormalTok{);}
        \KeywordTok{if} \NormalTok{(}\KeywordTok{buf}\NormalTok{.}\FunctionTok{length}\NormalTok{)}
          \KeywordTok{stream}\NormalTok{.}\FunctionTok{unshift}\NormalTok{(buf);}
        \KeywordTok{stream}\NormalTok{.}\FunctionTok{removeListener}\NormalTok{(}\CharTok{'error'}\NormalTok{, callback);}
        \KeywordTok{stream}\NormalTok{.}\FunctionTok{removeListener}\NormalTok{(}\CharTok{'readable'}\NormalTok{, onReadable);}
        \CommentTok{// now the body of the message can be read from the stream.}
        \NormalTok{callback(null, header, stream);}
      \NormalTok{\} }\KeywordTok{else} \NormalTok{\{}
        \CommentTok{// still reading the header.}
        \NormalTok{header += str;}
      \NormalTok{\}}
    \NormalTok{\}}
  \NormalTok{\}}
\NormalTok{\}}
\end{Highlighting}
\end{Shaded}

\paragraph{readable.wrap(stream)}

\begin{itemize}
\item
  \texttt{stream} \{Stream\} An ``old style'' readable stream
\end{itemize}

Versions of Node prior to v0.10 had streams that did not implement the
entire Streams API as it is today. (See ``Compatibility'' below for more
information.)

If you are using an older Node library that emits \texttt{'data'} events
and has a \hyperref[stream_readable_pause]{\texttt{pause()}} method
that is advisory only, then you can use the \texttt{wrap()} method to
create a \hyperref[stream_class_stream_readable]{Readable} stream
that uses the old stream as its data source.

You will very rarely ever need to call this function, but it exists as a
convenience for interacting with old Node programs and libraries.

For example:

\begin{Shaded}
\begin{Highlighting}[]
\KeywordTok{var} \NormalTok{OldReader = require(}\CharTok{'./old-api-module.js'}\NormalTok{).}\FunctionTok{OldReader}\NormalTok{;}
\KeywordTok{var} \NormalTok{oreader = }\KeywordTok{new} \NormalTok{OldReader;}
\KeywordTok{var} \NormalTok{Readable = require(}\CharTok{'stream'}\NormalTok{).}\FunctionTok{Readable}\NormalTok{;}
\KeywordTok{var} \NormalTok{myReader = }\KeywordTok{new} \NormalTok{Readable().}\FunctionTok{wrap}\NormalTok{(oreader);}

\KeywordTok{myReader}\NormalTok{.}\FunctionTok{on}\NormalTok{(}\CharTok{'readable'}\NormalTok{, }\KeywordTok{function}\NormalTok{() \{}
  \KeywordTok{myReader}\NormalTok{.}\FunctionTok{read}\NormalTok{(); }\CommentTok{// etc.}
\NormalTok{\});}
\end{Highlighting}
\end{Shaded}

\subsubsection{Class: stream.Writable}

The Writable stream interface is an abstraction for a \emph{destination}
that you are writing data \emph{to}.

Examples of writable streams include:

\begin{itemize}
\item
  \href{http.html\#http\_class\_http\_clientrequest}{http requests, on
  the client}
\item
  \href{http.html\#http\_class\_http\_serverresponse}{http responses, on
  the server}
\item
  \href{fs.html\#fs\_class\_fs\_writestream}{fs write streams}
\item
  \href{zlib.html}{zlib streams}
\item
  \href{crypto.html}{crypto streams}
\item
  \href{net.html\#net\_class\_net\_socket}{tcp sockets}
\item
  \href{child\_process.html\#child\_process\_child\_stdin}{child process
  stdin}
\item
  \href{process.html\#process\_process\_stdout}{process.stdout},
  \href{process.html\#process\_process\_stderr}{process.stderr}
\end{itemize}

\paragraph{writable.write(chunk, {[}encoding{]}, {[}callback{]})}

\begin{itemize}
\item
  \texttt{chunk} \{String \textbar{} Buffer\} The data to write
\item
  \texttt{encoding} \{String\} The encoding, if \texttt{chunk} is a
  String
\item
  \texttt{callback} \{Function\} Callback for when this chunk of data is
  flushed
\item
  Returns: \{Boolean\} True if the data was handled completely.
\end{itemize}

This method writes some data to the underlying system, and calls the
supplied callback once the data has been fully handled.

The return value indicates if you should continue writing right now. If
the data had to be buffered internally, then it will return
\texttt{false}. Otherwise, it will return \texttt{true}.

This return value is strictly advisory. You MAY continue to write, even
if it returns \texttt{false}. However, writes will be buffered in
memory, so it is best not to do this excessively. Instead, wait for the
\texttt{drain} event before writing more data.

\paragraph{Event: `drain'}

If a {[}\texttt{writable.write(chunk)}{]}{[}{]} call returns false, then
the \texttt{drain} event will indicate when it is appropriate to begin
writing more data to the stream.

\begin{Shaded}
\begin{Highlighting}[]
\CommentTok{// Write the data to the supplied writable stream 1MM times.}
\CommentTok{// Be attentive to back-pressure.}
\KeywordTok{function} \NormalTok{writeOneMillionTimes(writer, data, encoding, callback) \{}
  \KeywordTok{var} \NormalTok{i = }\DecValTok{1000000}\NormalTok{;}
  \FunctionTok{write}\NormalTok{();}
  \KeywordTok{function} \FunctionTok{write}\NormalTok{() \{}
    \KeywordTok{var} \NormalTok{ok = }\KeywordTok{true}\NormalTok{;}
    \KeywordTok{do} \NormalTok{\{}
      \NormalTok{i -= }\DecValTok{1}\NormalTok{;}
      \KeywordTok{if} \NormalTok{(i === }\DecValTok{0}\NormalTok{) \{}
        \CommentTok{// last time!}
        \KeywordTok{writer}\NormalTok{.}\FunctionTok{write}\NormalTok{(data, encoding, callback);}
      \NormalTok{\} }\KeywordTok{else} \NormalTok{\{}
        \CommentTok{// see if we should continue, or wait}
        \CommentTok{// don't pass the callback, because we're not done yet.}
        \NormalTok{ok = }\KeywordTok{writer}\NormalTok{.}\FunctionTok{write}\NormalTok{(data, encoding);}
      \NormalTok{\}}
    \NormalTok{\} }\KeywordTok{while} \NormalTok{(i > }\DecValTok{0} \NormalTok{&& ok);}
    \KeywordTok{if} \NormalTok{(i > }\DecValTok{0}\NormalTok{) \{}
      \CommentTok{// had to stop early!}
      \CommentTok{// write some more once it drains}
      \KeywordTok{writer}\NormalTok{.}\FunctionTok{once}\NormalTok{(}\CharTok{'drain'}\NormalTok{, }\FunctionTok{write}\NormalTok{);}
    \NormalTok{\}}
  \NormalTok{\}}
\NormalTok{\}}
\end{Highlighting}
\end{Shaded}

\paragraph{writable.cork()}

Forces buffering of all writes.

Buffered data will be flushed either at \texttt{.uncork()} or at
\texttt{.end()} call.

\paragraph{writable.uncork()}

Flush all data, buffered since \texttt{.cork()} call.

\paragraph{writable.end({[}chunk{]}, {[}encoding{]}, {[}callback{]})}

\begin{itemize}
\item
  \texttt{chunk} \{String \textbar{} Buffer\} Optional data to write
\item
  \texttt{encoding} \{String\} The encoding, if \texttt{chunk} is a
  String
\item
  \texttt{callback} \{Function\} Optional callback for when the stream
  is finished
\end{itemize}

Call this method when no more data will be written to the stream. If
supplied, the callback is attached as a listener on the \texttt{finish}
event.

Calling {[}\texttt{write()}{]}{[}{]} after calling
\hyperref[stream_writable_end_chunk_encoding_callback]{\texttt{end()}}
will raise an error.

\begin{Shaded}
\begin{Highlighting}[]
\CommentTok{// write 'hello, ' and then end with 'world!'}
\KeywordTok{http}\NormalTok{.}\FunctionTok{createServer}\NormalTok{(}\KeywordTok{function} \NormalTok{(req, res) \{}
  \KeywordTok{res}\NormalTok{.}\FunctionTok{write}\NormalTok{(}\CharTok{'hello, '}\NormalTok{);}
  \KeywordTok{res}\NormalTok{.}\FunctionTok{end}\NormalTok{(}\CharTok{'world!'}\NormalTok{);}
  \CommentTok{// writing more now is not allowed!}
\NormalTok{\});}
\end{Highlighting}
\end{Shaded}

\paragraph{Event: `finish'}

When the
\hyperref[stream_writable_end_chunk_encoding_callback]{\texttt{end()}}
method has been called, and all data has been flushed to the underlying
system, this event is emitted.

\begin{Shaded}
\begin{Highlighting}[]
\KeywordTok{var} \NormalTok{writer = getWritableStreamSomehow();}
\KeywordTok{for} \NormalTok{(}\KeywordTok{var} \NormalTok{i = }\DecValTok{0}\NormalTok{; i < }\DecValTok{100}\NormalTok{; i ++) \{}
  \KeywordTok{writer}\NormalTok{.}\FunctionTok{write}\NormalTok{(}\CharTok{'hello, #'} \NormalTok{+ i + }\CharTok{'!\textbackslash{}n'}\NormalTok{);}
\NormalTok{\}}
\KeywordTok{writer}\NormalTok{.}\FunctionTok{end}\NormalTok{(}\CharTok{'this is the end\textbackslash{}n'}\NormalTok{);}
\FunctionTok{write}\NormalTok{.}\FunctionTok{on}\NormalTok{(}\CharTok{'finish'}\NormalTok{, }\KeywordTok{function}\NormalTok{() \{}
  \KeywordTok{console}\NormalTok{.}\FunctionTok{error}\NormalTok{(}\CharTok{'all writes are now complete.'}\NormalTok{);}
\NormalTok{\});}
\end{Highlighting}
\end{Shaded}

\paragraph{Event: `pipe'}

\begin{itemize}
\item
  \texttt{src} \{\hyperref[stream_class_stream_readable]{Readable}
  Stream\} source stream that is piping to this writable
\end{itemize}

This is emitted whenever the \texttt{pipe()} method is called on a
readable stream, adding this writable to its set of destinations.

\begin{Shaded}
\begin{Highlighting}[]
\KeywordTok{var} \NormalTok{writer = getWritableStreamSomehow();}
\KeywordTok{var} \NormalTok{reader = getReadableStreamSomehow();}
\KeywordTok{writer}\NormalTok{.}\FunctionTok{on}\NormalTok{(}\CharTok{'pipe'}\NormalTok{, }\KeywordTok{function}\NormalTok{(src) \{}
  \KeywordTok{console}\NormalTok{.}\FunctionTok{error}\NormalTok{(}\CharTok{'something is piping into the writer'}\NormalTok{);}
  \KeywordTok{assert}\NormalTok{.}\FunctionTok{equal}\NormalTok{(src, reader);}
\NormalTok{\});}
\KeywordTok{reader}\NormalTok{.}\FunctionTok{pipe}\NormalTok{(writer);}
\end{Highlighting}
\end{Shaded}

\paragraph{Event: `unpipe'}

\begin{itemize}
\item
  \texttt{src} \{\hyperref[stream_class_stream_readable]{Readable}
  Stream\} The source stream that
  \hyperref[stream_readable_unpipe_destination]{unpiped} this
  writable
\end{itemize}

This is emitted whenever the
\hyperref[stream_readable_unpipe_destination]{\texttt{unpipe()}}
method is called on a readable stream, removing this writable from its
set of destinations.

\begin{Shaded}
\begin{Highlighting}[]
\KeywordTok{var} \NormalTok{writer = getWritableStreamSomehow();}
\KeywordTok{var} \NormalTok{reader = getReadableStreamSomehow();}
\KeywordTok{writer}\NormalTok{.}\FunctionTok{on}\NormalTok{(}\CharTok{'unpipe'}\NormalTok{, }\KeywordTok{function}\NormalTok{(src) \{}
  \KeywordTok{console}\NormalTok{.}\FunctionTok{error}\NormalTok{(}\CharTok{'something has stopped piping into the writer'}\NormalTok{);}
  \KeywordTok{assert}\NormalTok{.}\FunctionTok{equal}\NormalTok{(src, reader);}
\NormalTok{\});}
\KeywordTok{reader}\NormalTok{.}\FunctionTok{pipe}\NormalTok{(writer);}
\KeywordTok{reader}\NormalTok{.}\FunctionTok{unpipe}\NormalTok{(writer);}
\end{Highlighting}
\end{Shaded}

\subsubsection{Class: stream.Duplex}

Duplex streams are streams that implement both the
\hyperref[stream_class_stream_readable]{Readable} and
\hyperref[stream_class_stream_writable]{Writable} interfaces. See
above for usage.

Examples of Duplex streams include:

\begin{itemize}
\item
  \href{net.html\#net\_class\_net\_socket}{tcp sockets}
\item
  \href{zlib.html}{zlib streams}
\item
  \href{crypto.html}{crypto streams}
\end{itemize}

\subsubsection{Class: stream.Transform}

Transform streams are \hyperref[stream_class_stream_duplex]{Duplex}
streams where the output is in some way computed from the input. They
implement both the \hyperref[stream_class_stream_readable]{Readable}
and \hyperref[stream_class_stream_writable]{Writable} interfaces. See
above for usage.

Examples of Transform streams include:

\begin{itemize}
\item
  \href{zlib.html}{zlib streams}
\item
  \href{crypto.html}{crypto streams}
\end{itemize}

\subsection{API for Stream Implementors}

To implement any sort of stream, the pattern is the same:

\begin{enumerate}[1.]
\item
  Extend the appropriate parent class in your own subclass. (The
  \href{util.html\#util\_util\_inherits\_constructor\_superconstructor}{\texttt{util.inherits}}
  method is particularly helpful for this.)
\item
  Call the appropriate parent class constructor in your constructor, to
  be sure that the internal mechanisms are set up properly.
\item
  Implement one or more specific methods, as detailed below.
\end{enumerate}

The class to extend and the method(s) to implement depend on the sort of
stream class you are writing:

Use-case

Class

Method(s) to implement

Reading only

\hyperref[stream_class_stream_readable_1]{Readable}

{[}\_read{]}{[}{]}

Writing only

\hyperref[stream_class_stream_writable_1]{Writable}

{[}\_write{]}{[}{]}

Reading and writing

\hyperref[stream_class_stream_duplex_1]{Duplex}

{[}\emph{read{]}{[}{]}, {[}}write{]}{[}{]}

Operate on written data, then read the result

\hyperref[stream_class_stream_transform_1]{Transform}

\emph{transform, }flush

In your implementation code, it is very important to never call the
methods described in \hyperref[stream_api_for_stream_consumers]{API
for Stream Consumers} above. Otherwise, you can potentially cause
adverse side effects in programs that consume your streaming interfaces.

\subsubsection{Class: stream.Readable}

\texttt{stream.Readable} is an abstract class designed to be extended
with an underlying implementation of the
\hyperref[stream_readable_read_size_1]{\texttt{\_read(size)}}
method.

Please see above under
\hyperref[stream_api_for_stream_consumers]{API for Stream Consumers}
for how to consume streams in your programs. What follows is an
explanation of how to implement Readable streams in your programs.

\paragraph{Example: A Counting Stream}

This is a basic example of a Readable stream. It emits the numerals from
1 to 1,000,000 in ascending order, and then ends.

\begin{Shaded}
\begin{Highlighting}[]
\KeywordTok{var} \NormalTok{Readable = require(}\CharTok{'stream'}\NormalTok{).}\FunctionTok{Readable}\NormalTok{;}
\KeywordTok{var} \NormalTok{util = require(}\CharTok{'util'}\NormalTok{);}
\KeywordTok{util}\NormalTok{.}\FunctionTok{inherits}\NormalTok{(Counter, Readable);}

\KeywordTok{function} \NormalTok{Counter(opt) \{}
  \KeywordTok{Readable}\NormalTok{.}\FunctionTok{call}\NormalTok{(}\KeywordTok{this}\NormalTok{, opt);}
  \KeywordTok{this}\NormalTok{._}\FunctionTok{max} \NormalTok{= }\DecValTok{1000000}\NormalTok{;}
  \KeywordTok{this}\NormalTok{._}\FunctionTok{index} \NormalTok{= }\DecValTok{1}\NormalTok{;}
\NormalTok{\}}

\KeywordTok{Counter.prototype}\NormalTok{._}\FunctionTok{read} \NormalTok{= }\KeywordTok{function}\NormalTok{() \{}
  \KeywordTok{var} \NormalTok{i = }\KeywordTok{this}\NormalTok{._}\FunctionTok{index}\NormalTok{++;}
  \KeywordTok{if} \NormalTok{(i > }\KeywordTok{this}\NormalTok{._}\FunctionTok{max}\NormalTok{)}
    \KeywordTok{this}\NormalTok{.}\FunctionTok{push}\NormalTok{(null);}
  \KeywordTok{else} \NormalTok{\{}
    \KeywordTok{var} \NormalTok{str = }\CharTok{''} \NormalTok{+ i;}
    \KeywordTok{var} \NormalTok{buf = }\KeywordTok{new} \NormalTok{Buffer(str, }\CharTok{'ascii'}\NormalTok{);}
    \KeywordTok{this}\NormalTok{.}\FunctionTok{push}\NormalTok{(buf);}
  \NormalTok{\}}
\NormalTok{\};}
\end{Highlighting}
\end{Shaded}

\paragraph{Example: SimpleProtocol v1 (Sub-optimal)}

This is similar to the \texttt{parseHeader} function described above,
but implemented as a custom stream. Also, note that this implementation
does not convert the incoming data to a string.

However, this would be better implemented as a
\hyperref[stream_class_stream_transform]{Transform} stream. See below
for a better implementation.

\begin{Shaded}
\begin{Highlighting}[]
\CommentTok{// A parser for a simple data protocol.}
\CommentTok{// The "header" is a JSON object, followed by 2 \textbackslash{}n characters, and}
\CommentTok{// then a message body.}
\CommentTok{//}
\CommentTok{// NOTE: This can be done more simply as a Transform stream!}
\CommentTok{// Using Readable directly for this is sub-optimal.  See the}
\CommentTok{// alternative example below under the Transform section.}

\KeywordTok{var} \NormalTok{Readable = require(}\CharTok{'stream'}\NormalTok{).}\FunctionTok{Readable}\NormalTok{;}
\KeywordTok{var} \NormalTok{util = require(}\CharTok{'util'}\NormalTok{);}

\KeywordTok{util}\NormalTok{.}\FunctionTok{inherits}\NormalTok{(SimpleProtocol, Readable);}

\KeywordTok{function} \NormalTok{SimpleProtocol(source, options) \{}
  \KeywordTok{if} \NormalTok{(!(}\KeywordTok{this} \NormalTok{instanceof SimpleProtocol))}
    \KeywordTok{return} \KeywordTok{new} \NormalTok{SimpleProtocol(options);}

  \KeywordTok{Readable}\NormalTok{.}\FunctionTok{call}\NormalTok{(}\KeywordTok{this}\NormalTok{, options);}
  \KeywordTok{this}\NormalTok{._}\FunctionTok{inBody} \NormalTok{= }\KeywordTok{false}\NormalTok{;}
  \KeywordTok{this}\NormalTok{._}\FunctionTok{sawFirstCr} \NormalTok{= }\KeywordTok{false}\NormalTok{;}

  \CommentTok{// source is a readable stream, such as a socket or file}
  \KeywordTok{this}\NormalTok{._}\FunctionTok{source} \NormalTok{= source;}

  \KeywordTok{var} \NormalTok{self = }\KeywordTok{this}\NormalTok{;}
  \KeywordTok{source}\NormalTok{.}\FunctionTok{on}\NormalTok{(}\CharTok{'end'}\NormalTok{, }\KeywordTok{function}\NormalTok{() \{}
    \KeywordTok{self}\NormalTok{.}\FunctionTok{push}\NormalTok{(null);}
  \NormalTok{\});}

  \CommentTok{// give it a kick whenever the source is readable}
  \CommentTok{// read(0) will not consume any bytes}
  \KeywordTok{source}\NormalTok{.}\FunctionTok{on}\NormalTok{(}\CharTok{'readable'}\NormalTok{, }\KeywordTok{function}\NormalTok{() \{}
    \KeywordTok{self}\NormalTok{.}\FunctionTok{read}\NormalTok{(}\DecValTok{0}\NormalTok{);}
  \NormalTok{\});}

  \KeywordTok{this}\NormalTok{._}\FunctionTok{rawHeader} \NormalTok{= [];}
  \KeywordTok{this}\NormalTok{.}\FunctionTok{header} \NormalTok{= null;}
\NormalTok{\}}

\KeywordTok{SimpleProtocol.prototype}\NormalTok{._}\FunctionTok{read} \NormalTok{= }\KeywordTok{function}\NormalTok{(n) \{}
  \KeywordTok{if} \NormalTok{(!}\KeywordTok{this}\NormalTok{._}\FunctionTok{inBody}\NormalTok{) \{}
    \KeywordTok{var} \NormalTok{chunk = }\KeywordTok{this}\NormalTok{._}\FunctionTok{source}\NormalTok{.}\FunctionTok{read}\NormalTok{();}

    \CommentTok{// if the source doesn't have data, we don't have data yet.}
    \KeywordTok{if} \NormalTok{(chunk === null)}
      \KeywordTok{return} \KeywordTok{this}\NormalTok{.}\FunctionTok{push}\NormalTok{(}\CharTok{''}\NormalTok{);}

    \CommentTok{// check if the chunk has a \textbackslash{}n\textbackslash{}n}
    \KeywordTok{var} \NormalTok{split = -}\DecValTok{1}\NormalTok{;}
    \KeywordTok{for} \NormalTok{(}\KeywordTok{var} \NormalTok{i = }\DecValTok{0}\NormalTok{; i < }\KeywordTok{chunk}\NormalTok{.}\FunctionTok{length}\NormalTok{; i++) \{}
      \KeywordTok{if} \NormalTok{(chunk[i] === }\DecValTok{10}\NormalTok{) \{ }\CommentTok{// '\textbackslash{}n'}
        \KeywordTok{if} \NormalTok{(}\KeywordTok{this}\NormalTok{._}\FunctionTok{sawFirstCr}\NormalTok{) \{}
          \NormalTok{split = i;}
          \KeywordTok{break}\NormalTok{;}
        \NormalTok{\} }\KeywordTok{else} \NormalTok{\{}
          \KeywordTok{this}\NormalTok{._}\FunctionTok{sawFirstCr} \NormalTok{= }\KeywordTok{true}\NormalTok{;}
        \NormalTok{\}}
      \NormalTok{\} }\KeywordTok{else} \NormalTok{\{}
        \KeywordTok{this}\NormalTok{._}\FunctionTok{sawFirstCr} \NormalTok{= }\KeywordTok{false}\NormalTok{;}
      \NormalTok{\}}
    \NormalTok{\}}

    \KeywordTok{if} \NormalTok{(split === -}\DecValTok{1}\NormalTok{) \{}
      \CommentTok{// still waiting for the \textbackslash{}n\textbackslash{}n}
      \CommentTok{// stash the chunk, and try again.}
      \KeywordTok{this}\NormalTok{._}\FunctionTok{rawHeader}\NormalTok{.}\FunctionTok{push}\NormalTok{(chunk);}
      \KeywordTok{this}\NormalTok{.}\FunctionTok{push}\NormalTok{(}\CharTok{''}\NormalTok{);}
    \NormalTok{\} }\KeywordTok{else} \NormalTok{\{}
      \KeywordTok{this}\NormalTok{._}\FunctionTok{inBody} \NormalTok{= }\KeywordTok{true}\NormalTok{;}
      \KeywordTok{var} \NormalTok{h = }\KeywordTok{chunk}\NormalTok{.}\FunctionTok{slice}\NormalTok{(}\DecValTok{0}\NormalTok{, split);}
      \KeywordTok{this}\NormalTok{._}\FunctionTok{rawHeader}\NormalTok{.}\FunctionTok{push}\NormalTok{(h);}
      \KeywordTok{var} \NormalTok{header = }\KeywordTok{Buffer}\NormalTok{.}\FunctionTok{concat}\NormalTok{(}\KeywordTok{this}\NormalTok{._}\FunctionTok{rawHeader}\NormalTok{).}\FunctionTok{toString}\NormalTok{();}
      \KeywordTok{try} \NormalTok{\{}
        \KeywordTok{this}\NormalTok{.}\FunctionTok{header} \NormalTok{= }\KeywordTok{JSON}\NormalTok{.}\FunctionTok{parse}\NormalTok{(header);}
      \NormalTok{\} }\KeywordTok{catch} \NormalTok{(er) \{}
        \KeywordTok{this}\NormalTok{.}\FunctionTok{emit}\NormalTok{(}\CharTok{'error'}\NormalTok{, }\KeywordTok{new} \NormalTok{Error(}\CharTok{'invalid simple protocol data'}\NormalTok{));}
        \KeywordTok{return}\NormalTok{;}
      \NormalTok{\}}
      \CommentTok{// now, because we got some extra data, unshift the rest}
      \CommentTok{// back into the read queue so that our consumer will see it.}
      \KeywordTok{var} \NormalTok{b = }\KeywordTok{chunk}\NormalTok{.}\FunctionTok{slice}\NormalTok{(split);}
      \KeywordTok{this}\NormalTok{.}\FunctionTok{unshift}\NormalTok{(b);}

      \CommentTok{// and let them know that we are done parsing the header.}
      \KeywordTok{this}\NormalTok{.}\FunctionTok{emit}\NormalTok{(}\CharTok{'header'}\NormalTok{, }\KeywordTok{this}\NormalTok{.}\FunctionTok{header}\NormalTok{);}
    \NormalTok{\}}
  \NormalTok{\} }\KeywordTok{else} \NormalTok{\{}
    \CommentTok{// from there on, just provide the data to our consumer.}
    \CommentTok{// careful not to push(null), since that would indicate EOF.}
    \KeywordTok{var} \NormalTok{chunk = }\KeywordTok{this}\NormalTok{._}\FunctionTok{source}\NormalTok{.}\FunctionTok{read}\NormalTok{();}
    \KeywordTok{if} \NormalTok{(chunk) }\KeywordTok{this}\NormalTok{.}\FunctionTok{push}\NormalTok{(chunk);}
  \NormalTok{\}}
\NormalTok{\};}

\CommentTok{// Usage:}
\CommentTok{// var parser = new SimpleProtocol(source);}
\CommentTok{// Now parser is a readable stream that will emit 'header'}
\CommentTok{// with the parsed header data.}
\end{Highlighting}
\end{Shaded}

\paragraph{new stream.Readable({[}options{]})}

\begin{itemize}
\item
  \texttt{options} \{Object\}
\item
  \texttt{highWaterMark} \{Number\} The maximum number of bytes to store
  in the internal buffer before ceasing to read from the underlying
  resource. Default=16kb, or 16 for \texttt{objectMode} streams
\item
  \texttt{encoding} \{String\} If specified, then buffers will be
  decoded to strings using the specified encoding. Default=null
\item
  \texttt{objectMode} \{Boolean\} Whether this stream should behave as a
  stream of objects. Meaning that stream.read(n) returns a single value
  instead of a Buffer of size n
\end{itemize}

In classes that extend the Readable class, make sure to call the
Readable constructor so that the buffering settings can be properly
initialized.

\paragraph{readable.\_read(size)}

\begin{itemize}
\item
  \texttt{size} \{Number\} Number of bytes to read asynchronously
\end{itemize}

Note: \textbf{Implement this function, but do NOT call it directly.}

This function should NOT be called directly. It should be implemented by
child classes, and only called by the internal Readable class methods.

All Readable stream implementations must provide a \texttt{\_read}
method to fetch data from the underlying resource.

This method is prefixed with an underscore because it is internal to the
class that defines it, and should not be called directly by user
programs. However, you \textbf{are} expected to override this method in
your own extension classes.

When data is available, put it into the read queue by calling
\texttt{readable.push(chunk)}. If \texttt{push} returns false, then you
should stop reading. When \texttt{\_read} is called again, you should
start pushing more data.

The \texttt{size} argument is advisory. Implementations where a ``read''
is a single call that returns data can use this to know how much data to
fetch. Implementations where that is not relevant, such as TCP or TLS,
may ignore this argument, and simply provide data whenever it becomes
available. There is no need, for example to ``wait'' until \texttt{size}
bytes are available before calling
\hyperref[stream_readable_push_chunk_encoding]{\texttt{stream.push(chunk)}}.

\paragraph{readable.push(chunk, {[}encoding{]})}

\begin{itemize}
\item
  \texttt{chunk} \{Buffer \textbar{} null \textbar{} String\} Chunk of
  data to push into the read queue
\item
  \texttt{encoding} \{String\} Encoding of String chunks. Must be a
  valid Buffer encoding, such as \texttt{'utf8'} or \texttt{'ascii'}
\item
  return \{Boolean\} Whether or not more pushes should be performed
\end{itemize}

Note: \textbf{This function should be called by Readable implementors,
NOT by consumers of Readable streams.}

The \texttt{\_read()} function will not be called again until at least
one \texttt{push(chunk)} call is made.

The \texttt{Readable} class works by putting data into a read queue to
be pulled out later by calling the \texttt{read()} method when the
\texttt{'readable'} event fires.

The \texttt{push()} method will explicitly insert some data into the
read queue. If it is called with \texttt{null} then it will signal the
end of the data (EOF).

This API is designed to be as flexible as possible. For example, you may
be wrapping a lower-level source which has some sort of pause/resume
mechanism, and a data callback. In those cases, you could wrap the
low-level source object by doing something like this:

\begin{Shaded}
\begin{Highlighting}[]
\CommentTok{// source is an object with readStop() and readStart() methods,}
\CommentTok{// and an `ondata` member that gets called when it has data, and}
\CommentTok{// an `onend` member that gets called when the data is over.}

\KeywordTok{util}\NormalTok{.}\FunctionTok{inherits}\NormalTok{(SourceWrapper, Readable);}

\KeywordTok{function} \NormalTok{SourceWrapper(options) \{}
  \KeywordTok{Readable}\NormalTok{.}\FunctionTok{call}\NormalTok{(}\KeywordTok{this}\NormalTok{, options);}

  \KeywordTok{this}\NormalTok{._}\FunctionTok{source} \NormalTok{= getLowlevelSourceObject();}
  \KeywordTok{var} \NormalTok{self = }\KeywordTok{this}\NormalTok{;}

  \CommentTok{// Every time there's data, we push it into the internal buffer.}
  \KeywordTok{this}\NormalTok{._}\FunctionTok{source}\NormalTok{.}\FunctionTok{ondata} \NormalTok{= }\KeywordTok{function}\NormalTok{(chunk) \{}
    \CommentTok{// if push() returns false, then we need to stop reading from source}
    \KeywordTok{if} \NormalTok{(!}\KeywordTok{self}\NormalTok{.}\FunctionTok{push}\NormalTok{(chunk))}
      \KeywordTok{self._source}\NormalTok{.}\FunctionTok{readStop}\NormalTok{();}
  \NormalTok{\};}

  \CommentTok{// When the source ends, we push the EOF-signalling `null` chunk}
  \KeywordTok{this}\NormalTok{._}\FunctionTok{source}\NormalTok{.}\FunctionTok{onend} \NormalTok{= }\KeywordTok{function}\NormalTok{() \{}
    \KeywordTok{self}\NormalTok{.}\FunctionTok{push}\NormalTok{(null);}
  \NormalTok{\};}
\NormalTok{\}}

\CommentTok{// _read will be called when the stream wants to pull more data in}
\CommentTok{// the advisory size argument is ignored in this case.}
\KeywordTok{SourceWrapper.prototype}\NormalTok{._}\FunctionTok{read} \NormalTok{= }\KeywordTok{function}\NormalTok{(size) \{}
  \KeywordTok{this}\NormalTok{._}\FunctionTok{source}\NormalTok{.}\FunctionTok{readStart}\NormalTok{();}
\NormalTok{\};}
\end{Highlighting}
\end{Shaded}

\subsubsection{Class: stream.Writable}

\texttt{stream.Writable} is an abstract class designed to be extended
with an underlying implementation of the
{[}\texttt{\_write(chunk, encoding, callback)}{]}{[}{]} method.

Please see above under
\hyperref[stream_api_for_stream_consumers]{API for Stream Consumers}
for how to consume writable streams in your programs. What follows is an
explanation of how to implement Writable streams in your programs.

\paragraph{new stream.Writable({[}options{]})}

\begin{itemize}
\item
  \texttt{options} \{Object\}
\item
  \texttt{highWaterMark} \{Number\} Buffer level when
  {[}\texttt{write()}{]}{[}{]} starts returning false. Default=16kb, or
  16 for \texttt{objectMode} streams
\item
  \texttt{decodeStrings} \{Boolean\} Whether or not to decode strings
  into Buffers before passing them to {[}\texttt{\_write()}{]}{[}{]}.
  Default=true
\end{itemize}

In classes that extend the Writable class, make sure to call the
constructor so that the buffering settings can be properly initialized.

\paragraph{writable.\_write(chunk, encoding, callback)}

\begin{itemize}
\item
  \texttt{chunk} \{Buffer \textbar{} String\} The chunk to be written.
  Will always be a buffer unless the \texttt{decodeStrings} option was
  set to \texttt{false}.
\item
  \texttt{encoding} \{String\} If the chunk is a string, then this is
  the encoding type. Ignore if chunk is a buffer. Note that chunk will
  \textbf{always} be a buffer unless the \texttt{decodeStrings} option
  is explicitly set to \texttt{false}.
\item
  \texttt{callback} \{Function\} Call this function (optionally with an
  error argument) when you are done processing the supplied chunk.
\end{itemize}

All Writable stream implementations must provide a
{[}\texttt{\_write()}{]}{[}{]} method to send data to the underlying
resource.

Note: \textbf{This function MUST NOT be called directly.} It should be
implemented by child classes, and called by the internal Writable class
methods only.

Call the callback using the standard \texttt{callback(error)} pattern to
signal that the write completed successfully or with an error.

If the \texttt{decodeStrings} flag is set in the constructor options,
then \texttt{chunk} may be a string rather than a Buffer, and
\texttt{encoding} will indicate the sort of string that it is. This is
to support implementations that have an optimized handling for certain
string data encodings. If you do not explicitly set the
\texttt{decodeStrings} option to \texttt{false}, then you can safely
ignore the \texttt{encoding} argument, and assume that \texttt{chunk}
will always be a Buffer.

This method is prefixed with an underscore because it is internal to the
class that defines it, and should not be called directly by user
programs. However, you \textbf{are} expected to override this method in
your own extension classes.

\subsubsection{writable.\_writev(chunks, callback)}

\begin{itemize}
\item
  \texttt{chunks} \{Array\} The chunks to be written. Each chunk has
  following format: \texttt{\{ chunk: ..., encoding: ... \}}.
\item
  \texttt{callback} \{Function\} Call this function (optionally with an
  error argument) when you are done processing the supplied chunks.
\end{itemize}

Note: \textbf{This function MUST NOT be called directly.} It may be
implemented by child classes, and called by the internal Writable class
methods only.

This function is completely optional to implement. In most cases it is
unnecessary. If implemented, it will be called with all the chunks that
are buffered in the write queue.

\subsubsection{Class: stream.Duplex}

A ``duplex'' stream is one that is both Readable and Writable, such as a
TCP socket connection.

Note that \texttt{stream.Duplex} is an abstract class designed to be
extended with an underlying implementation of the \texttt{\_read(size)}
and {[}\texttt{\_write(chunk, encoding, callback)}{]}{[}{]} methods as
you would with a Readable or Writable stream class.

Since JavaScript doesn't have multiple prototypal inheritance, this
class prototypally inherits from Readable, and then parasitically from
Writable. It is thus up to the user to implement both the lowlevel
\texttt{\_read(n)} method as well as the lowlevel
{[}\texttt{\_write(chunk, encoding, callback)}{]}{[}{]} method on
extension duplex classes.

\paragraph{new stream.Duplex(options)}

\begin{itemize}
\item
  \texttt{options} \{Object\} Passed to both Writable and Readable
  constructors. Also has the following fields:
\item
  \texttt{allowHalfOpen} \{Boolean\} Default=true. If set to
  \texttt{false}, then the stream will automatically end the readable
  side when the writable side ends and vice versa.
\end{itemize}

In classes that extend the Duplex class, make sure to call the
constructor so that the buffering settings can be properly initialized.

\subsubsection{Class: stream.Transform}

A ``transform'' stream is a duplex stream where the output is causally
connected in some way to the input, such as a \href{zlib.html}{zlib}
stream or a \href{crypto.html}{crypto} stream.

There is no requirement that the output be the same size as the input,
the same number of chunks, or arrive at the same time. For example, a
Hash stream will only ever have a single chunk of output which is
provided when the input is ended. A zlib stream will produce output that
is either much smaller or much larger than its input.

Rather than implement the
\hyperref[stream_readable_read_size_1]{\texttt{\_read()}} and
{[}\texttt{\_write()}{]}{[}{]} methods, Transform classes must implement
the \texttt{\_transform()} method, and may optionally also implement the
\texttt{\_flush()} method. (See below.)

\paragraph{new stream.Transform({[}options{]})}

\begin{itemize}
\item
  \texttt{options} \{Object\} Passed to both Writable and Readable
  constructors.
\end{itemize}

In classes that extend the Transform class, make sure to call the
constructor so that the buffering settings can be properly initialized.

\paragraph{transform.\_transform(chunk, encoding, callback)}

\begin{itemize}
\item
  \texttt{chunk} \{Buffer \textbar{} String\} The chunk to be
  transformed. Will always be a buffer unless the \texttt{decodeStrings}
  option was set to \texttt{false}.
\item
  \texttt{encoding} \{String\} If the chunk is a string, then this is
  the encoding type. (Ignore if \texttt{decodeStrings} chunk is a
  buffer.)
\item
  \texttt{callback} \{Function\} Call this function (optionally with an
  error argument) when you are done processing the supplied chunk.
\end{itemize}

Note: \textbf{This function MUST NOT be called directly.} It should be
implemented by child classes, and called by the internal Transform class
methods only.

All Transform stream implementations must provide a \texttt{\_transform}
method to accept input and produce output.

\texttt{\_transform} should do whatever has to be done in this specific
Transform class, to handle the bytes being written, and pass them off to
the readable portion of the interface. Do asynchronous I/O, process
things, and so on.

Call \texttt{transform.push(outputChunk)} 0 or more times to generate
output from this input chunk, depending on how much data you want to
output as a result of this chunk.

Call the callback function only when the current chunk is completely
consumed. Note that there may or may not be output as a result of any
particular input chunk.

This method is prefixed with an underscore because it is internal to the
class that defines it, and should not be called directly by user
programs. However, you \textbf{are} expected to override this method in
your own extension classes.

\paragraph{transform.\_flush(callback)}

\begin{itemize}
\item
  \texttt{callback} \{Function\} Call this function (optionally with an
  error argument) when you are done flushing any remaining data.
\end{itemize}

Note: \textbf{This function MUST NOT be called directly.} It MAY be
implemented by child classes, and if so, will be called by the internal
Transform class methods only.

In some cases, your transform operation may need to emit a bit more data
at the end of the stream. For example, a \texttt{Zlib} compression
stream will store up some internal state so that it can optimally
compress the output. At the end, however, it needs to do the best it can
with what is left, so that the data will be complete.

In those cases, you can implement a \texttt{\_flush} method, which will
be called at the very end, after all the written data is consumed, but
before emitting \texttt{end} to signal the end of the readable side.
Just like with \texttt{\_transform}, call \texttt{transform.push(chunk)}
zero or more times, as appropriate, and call \texttt{callback} when the
flush operation is complete.

This method is prefixed with an underscore because it is internal to the
class that defines it, and should not be called directly by user
programs. However, you \textbf{are} expected to override this method in
your own extension classes.

\paragraph{Example: \texttt{SimpleProtocol} parser v2}

The example above of a simple protocol parser can be implemented simply
by using the higher level
\hyperref[stream_class_stream_transform]{Transform} stream class,
similar to the \texttt{parseHeader} and \texttt{SimpleProtocol v1}
examples above.

In this example, rather than providing the input as an argument, it
would be piped into the parser, which is a more idiomatic Node stream
approach.

\begin{Shaded}
\begin{Highlighting}[]
\KeywordTok{var} \NormalTok{util = require(}\CharTok{'util'}\NormalTok{);}
\KeywordTok{var} \NormalTok{Transform = require(}\CharTok{'stream'}\NormalTok{).}\FunctionTok{Transform}\NormalTok{;}
\KeywordTok{util}\NormalTok{.}\FunctionTok{inherits}\NormalTok{(SimpleProtocol, Transform);}

\KeywordTok{function} \NormalTok{SimpleProtocol(options) \{}
  \KeywordTok{if} \NormalTok{(!(}\KeywordTok{this} \NormalTok{instanceof SimpleProtocol))}
    \KeywordTok{return} \KeywordTok{new} \NormalTok{SimpleProtocol(options);}

  \KeywordTok{Transform}\NormalTok{.}\FunctionTok{call}\NormalTok{(}\KeywordTok{this}\NormalTok{, options);}
  \KeywordTok{this}\NormalTok{._}\FunctionTok{inBody} \NormalTok{= }\KeywordTok{false}\NormalTok{;}
  \KeywordTok{this}\NormalTok{._}\FunctionTok{sawFirstCr} \NormalTok{= }\KeywordTok{false}\NormalTok{;}
  \KeywordTok{this}\NormalTok{._}\FunctionTok{rawHeader} \NormalTok{= [];}
  \KeywordTok{this}\NormalTok{.}\FunctionTok{header} \NormalTok{= null;}
\NormalTok{\}}

\KeywordTok{SimpleProtocol.prototype}\NormalTok{._}\FunctionTok{transform} \NormalTok{= }\KeywordTok{function}\NormalTok{(chunk, encoding, done) \{}
  \KeywordTok{if} \NormalTok{(!}\KeywordTok{this}\NormalTok{._}\FunctionTok{inBody}\NormalTok{) \{}
    \CommentTok{// check if the chunk has a \textbackslash{}n\textbackslash{}n}
    \KeywordTok{var} \NormalTok{split = -}\DecValTok{1}\NormalTok{;}
    \KeywordTok{for} \NormalTok{(}\KeywordTok{var} \NormalTok{i = }\DecValTok{0}\NormalTok{; i < }\KeywordTok{chunk}\NormalTok{.}\FunctionTok{length}\NormalTok{; i++) \{}
      \KeywordTok{if} \NormalTok{(chunk[i] === }\DecValTok{10}\NormalTok{) \{ }\CommentTok{// '\textbackslash{}n'}
        \KeywordTok{if} \NormalTok{(}\KeywordTok{this}\NormalTok{._}\FunctionTok{sawFirstCr}\NormalTok{) \{}
          \NormalTok{split = i;}
          \KeywordTok{break}\NormalTok{;}
        \NormalTok{\} }\KeywordTok{else} \NormalTok{\{}
          \KeywordTok{this}\NormalTok{._}\FunctionTok{sawFirstCr} \NormalTok{= }\KeywordTok{true}\NormalTok{;}
        \NormalTok{\}}
      \NormalTok{\} }\KeywordTok{else} \NormalTok{\{}
        \KeywordTok{this}\NormalTok{._}\FunctionTok{sawFirstCr} \NormalTok{= }\KeywordTok{false}\NormalTok{;}
      \NormalTok{\}}
    \NormalTok{\}}

    \KeywordTok{if} \NormalTok{(split === -}\DecValTok{1}\NormalTok{) \{}
      \CommentTok{// still waiting for the \textbackslash{}n\textbackslash{}n}
      \CommentTok{// stash the chunk, and try again.}
      \KeywordTok{this}\NormalTok{._}\FunctionTok{rawHeader}\NormalTok{.}\FunctionTok{push}\NormalTok{(chunk);}
    \NormalTok{\} }\KeywordTok{else} \NormalTok{\{}
      \KeywordTok{this}\NormalTok{._}\FunctionTok{inBody} \NormalTok{= }\KeywordTok{true}\NormalTok{;}
      \KeywordTok{var} \NormalTok{h = }\KeywordTok{chunk}\NormalTok{.}\FunctionTok{slice}\NormalTok{(}\DecValTok{0}\NormalTok{, split);}
      \KeywordTok{this}\NormalTok{._}\FunctionTok{rawHeader}\NormalTok{.}\FunctionTok{push}\NormalTok{(h);}
      \KeywordTok{var} \NormalTok{header = }\KeywordTok{Buffer}\NormalTok{.}\FunctionTok{concat}\NormalTok{(}\KeywordTok{this}\NormalTok{._}\FunctionTok{rawHeader}\NormalTok{).}\FunctionTok{toString}\NormalTok{();}
      \KeywordTok{try} \NormalTok{\{}
        \KeywordTok{this}\NormalTok{.}\FunctionTok{header} \NormalTok{= }\KeywordTok{JSON}\NormalTok{.}\FunctionTok{parse}\NormalTok{(header);}
      \NormalTok{\} }\KeywordTok{catch} \NormalTok{(er) \{}
        \KeywordTok{this}\NormalTok{.}\FunctionTok{emit}\NormalTok{(}\CharTok{'error'}\NormalTok{, }\KeywordTok{new} \NormalTok{Error(}\CharTok{'invalid simple protocol data'}\NormalTok{));}
        \KeywordTok{return}\NormalTok{;}
      \NormalTok{\}}
      \CommentTok{// and let them know that we are done parsing the header.}
      \KeywordTok{this}\NormalTok{.}\FunctionTok{emit}\NormalTok{(}\CharTok{'header'}\NormalTok{, }\KeywordTok{this}\NormalTok{.}\FunctionTok{header}\NormalTok{);}

      \CommentTok{// now, because we got some extra data, emit this first.}
      \KeywordTok{this}\NormalTok{.}\FunctionTok{push}\NormalTok{(}\KeywordTok{chunk}\NormalTok{.}\FunctionTok{slice}\NormalTok{(split));}
    \NormalTok{\}}
  \NormalTok{\} }\KeywordTok{else} \NormalTok{\{}
    \CommentTok{// from there on, just provide the data to our consumer as-is.}
    \KeywordTok{this}\NormalTok{.}\FunctionTok{push}\NormalTok{(chunk);}
  \NormalTok{\}}
  \NormalTok{done();}
\NormalTok{\};}

\CommentTok{// Usage:}
\CommentTok{// var parser = new SimpleProtocol();}
\CommentTok{// source.pipe(parser)}
\CommentTok{// Now parser is a readable stream that will emit 'header'}
\CommentTok{// with the parsed header data.}
\end{Highlighting}
\end{Shaded}

\subsubsection{Class: stream.PassThrough}

This is a trivial implementation of a
\hyperref[stream_class_stream_transform]{Transform} stream that
simply passes the input bytes across to the output. Its purpose is
mainly for examples and testing, but there are occasionally use cases
where it can come in handy as a building block for novel sorts of
streams.

\subsection{Streams: Under the Hood}

\subsubsection{Buffering}

Both Writable and Readable streams will buffer data on an internal
object called \texttt{\_writableState.buffer} or
\texttt{\_readableState.buffer}, respectively.

The amount of data that will potentially be buffered depends on the
\texttt{highWaterMark} option which is passed into the constructor.

Buffering in Readable streams happens when the implementation calls
\hyperref[stream_readable_push_chunk_encoding]{\texttt{stream.push(chunk)}}.
If the consumer of the Stream does not call \texttt{stream.read()}, then
the data will sit in the internal queue until it is consumed.

Buffering in Writable streams happens when the user calls
{[}\texttt{stream.write(chunk)}{]}{[}{]} repeatedly, even when
\texttt{write()} returns \texttt{false}.

The purpose of streams, especially with the \texttt{pipe()} method, is
to limit the buffering of data to acceptable levels, so that sources and
destinations of varying speed will not overwhelm the available memory.

\subsubsection{\texttt{stream.read(0)}}

There are some cases where you want to trigger a refresh of the
underlying readable stream mechanisms, without actually consuming any
data. In that case, you can call \texttt{stream.read(0)}, which will
always return null.

If the internal read buffer is below the \texttt{highWaterMark}, and the
stream is not currently reading, then calling \texttt{read(0)} will
trigger a low-level \texttt{\_read} call.

There is almost never a need to do this. However, you will see some
cases in Node's internals where this is done, particularly in the
Readable stream class internals.

\subsubsection{\texttt{stream.push('')}}

Pushing a zero-byte string or Buffer (when not in
\hyperref[stream_object_mode]{Object mode}) has an interesting side
effect. Because it \emph{is} a call to
\hyperref[stream_readable_push_chunk_encoding]{\texttt{stream.push()}},
it will end the \texttt{reading} process. However, it does \emph{not}
add any data to the readable buffer, so there's nothing for a user to
consume.

Very rarely, there are cases where you have no data to provide now, but
the consumer of your stream (or, perhaps, another bit of your own code)
will know when to check again, by calling \texttt{stream.read(0)}. In
those cases, you \emph{may} call \texttt{stream.push('')}.

So far, the only use case for this functionality is in the
\href{tls.html\#tls\_class\_cryptostream}{tls.CryptoStream} class, which
is deprecated in Node v0.12. If you find that you have to use
\texttt{stream.push('')}, please consider another approach, because it
almost certainly indicates that something is horribly wrong.

\subsubsection{Compatibility with Older Node Versions}

In versions of Node prior to v0.10, the Readable stream interface was
simpler, but also less powerful and less useful.

\begin{itemize}
\item
  Rather than waiting for you to call the \texttt{read()} method,
  \texttt{'data'} events would start emitting immediately. If you needed
  to do some I/O to decide how to handle data, then you had to store the
  chunks in some kind of buffer so that they would not be lost.
\item
  The \hyperref[stream_readable_pause]{\texttt{pause()}} method was
  advisory, rather than guaranteed. This meant that you still had to be
  prepared to receive \texttt{'data'} events even when the stream was in
  a paused state.
\end{itemize}

In Node v0.10, the Readable class described below was added. For
backwards compatibility with older Node programs, Readable streams
switch into ``flowing mode'' when a \texttt{'data'} event handler is
added, or when the
\hyperref[stream_readable_resume]{\texttt{resume()}} method is called.
The effect is that, even if you are not using the new \texttt{read()}
method and \texttt{'readable'} event, you no longer have to worry about
losing \texttt{'data'} chunks.

Most programs will continue to function normally. However, this
introduces an edge case in the following conditions:

\begin{itemize}
\item
  No \hyperref[stream_event_data]{\texttt{'data'} event} handler is
  added.
\item
  The \hyperref[stream_readable_resume]{\texttt{resume()}} method is
  never called.
\item
  The stream is not piped to any writable destination.
\end{itemize}

For example, consider the following code:

\begin{Shaded}
\begin{Highlighting}[]
\CommentTok{// WARNING!  BROKEN!}
\KeywordTok{net}\NormalTok{.}\FunctionTok{createServer}\NormalTok{(}\KeywordTok{function}\NormalTok{(socket) \{}

  \CommentTok{// we add an 'end' method, but never consume the data}
  \KeywordTok{socket}\NormalTok{.}\FunctionTok{on}\NormalTok{(}\CharTok{'end'}\NormalTok{, }\KeywordTok{function}\NormalTok{() \{}
    \CommentTok{// It will never get here.}
    \KeywordTok{socket}\NormalTok{.}\FunctionTok{end}\NormalTok{(}\CharTok{'I got your message (but didnt read it)\textbackslash{}n'}\NormalTok{);}
  \NormalTok{\});}

\NormalTok{\}).}\FunctionTok{listen}\NormalTok{(}\DecValTok{1337}\NormalTok{);}
\end{Highlighting}
\end{Shaded}

In versions of node prior to v0.10, the incoming message data would be
simply discarded. However, in Node v0.10 and beyond, the socket will
remain paused forever.

The workaround in this situation is to call the \texttt{resume()} method
to start the flow of data:

\begin{Shaded}
\begin{Highlighting}[]
\CommentTok{// Workaround}
\KeywordTok{net}\NormalTok{.}\FunctionTok{createServer}\NormalTok{(}\KeywordTok{function}\NormalTok{(socket) \{}

  \KeywordTok{socket}\NormalTok{.}\FunctionTok{on}\NormalTok{(}\CharTok{'end'}\NormalTok{, }\KeywordTok{function}\NormalTok{() \{}
    \KeywordTok{socket}\NormalTok{.}\FunctionTok{end}\NormalTok{(}\CharTok{'I got your message (but didnt read it)\textbackslash{}n'}\NormalTok{);}
  \NormalTok{\});}

  \CommentTok{// start the flow of data, discarding it.}
  \KeywordTok{socket}\NormalTok{.}\FunctionTok{resume}\NormalTok{();}

\NormalTok{\}).}\FunctionTok{listen}\NormalTok{(}\DecValTok{1337}\NormalTok{);}
\end{Highlighting}
\end{Shaded}

In addition to new Readable streams switching into flowing mode,
pre-v0.10 style streams can be wrapped in a Readable class using the
\texttt{wrap()} method.

\subsubsection{Object Mode}

Normally, Streams operate on Strings and Buffers exclusively.

Streams that are in \textbf{object mode} can emit generic JavaScript
values other than Buffers and Strings.

A Readable stream in object mode will always return a single item from a
call to \texttt{stream.read(size)}, regardless of what the size argument
is.

A Writable stream in object mode will always ignore the
\texttt{encoding} argument to \texttt{stream.write(data, encoding)}.

The special value \texttt{null} still retains its special value for
object mode streams. That is, for object mode readable streams,
\texttt{null} as a return value from \texttt{stream.read()} indicates
that there is no more data, and
\hyperref[stream_readable_push_chunk_encoding]{\texttt{stream.push(null)}}
will signal the end of stream data (\texttt{EOF}).

No streams in Node core are object mode streams. This pattern is only
used by userland streaming libraries.

You should set \texttt{objectMode} in your stream child class
constructor on the options object. Setting \texttt{objectMode}
mid-stream is not safe.

\subsubsection{State Objects}

\hyperref[stream_class_stream_readable]{Readable} streams have a
member object called \texttt{\_readableState}.
\hyperref[stream_class_stream_writable]{Writable} streams have a
member object called \texttt{\_writableState}.
\hyperref[stream_class_stream_duplex]{Duplex} streams have both.

\textbf{These objects should generally not be modified in child
classes.} However, if you have a Duplex or Transform stream that should
be in \texttt{objectMode} on the readable side, and not in
\texttt{objectMode} on the writable side, then you may do this in the
constructor by setting the flag explicitly on the appropriate state
object.

\begin{Shaded}
\begin{Highlighting}[]
\KeywordTok{var} \NormalTok{util = require(}\CharTok{'util'}\NormalTok{);}
\KeywordTok{var} \NormalTok{StringDecoder = require(}\CharTok{'string_decoder'}\NormalTok{).}\FunctionTok{StringDecoder}\NormalTok{;}
\KeywordTok{var} \NormalTok{Transform = require(}\CharTok{'stream'}\NormalTok{).}\FunctionTok{Transform}\NormalTok{;}
\KeywordTok{util}\NormalTok{.}\FunctionTok{inherits}\NormalTok{(JSONParseStream, Transform);}

\CommentTok{// Gets \textbackslash{}n-delimited JSON string data, and emits the parsed objects}
\KeywordTok{function} \NormalTok{JSONParseStream(options) \{}
  \KeywordTok{if} \NormalTok{(!(}\KeywordTok{this} \NormalTok{instanceof JSONParseStream))}
    \KeywordTok{return} \KeywordTok{new} \NormalTok{JSONParseStream(options);}

  \KeywordTok{Transform}\NormalTok{.}\FunctionTok{call}\NormalTok{(}\KeywordTok{this}\NormalTok{, options);}
  \KeywordTok{this}\NormalTok{._}\FunctionTok{writableState}\NormalTok{.}\FunctionTok{objectMode} \NormalTok{= }\KeywordTok{false}\NormalTok{;}
  \KeywordTok{this}\NormalTok{._}\FunctionTok{readableState}\NormalTok{.}\FunctionTok{objectMode} \NormalTok{= }\KeywordTok{true}\NormalTok{;}
  \KeywordTok{this}\NormalTok{._}\FunctionTok{buffer} \NormalTok{= }\CharTok{''}\NormalTok{;}
  \KeywordTok{this}\NormalTok{._}\FunctionTok{decoder} \NormalTok{= }\KeywordTok{new} \NormalTok{StringDecoder(}\CharTok{'utf8'}\NormalTok{);}
\NormalTok{\}}

\KeywordTok{JSONParseStream.prototype}\NormalTok{._}\FunctionTok{transform} \NormalTok{= }\KeywordTok{function}\NormalTok{(chunk, encoding, cb) \{}
  \KeywordTok{this}\NormalTok{._}\FunctionTok{buffer} \NormalTok{+= }\KeywordTok{this}\NormalTok{._}\FunctionTok{decoder}\NormalTok{.}\FunctionTok{write}\NormalTok{(chunk);}
  \CommentTok{// split on newlines}
  \KeywordTok{var} \NormalTok{lines = }\KeywordTok{this}\NormalTok{._}\FunctionTok{buffer}\NormalTok{.}\FunctionTok{split}\NormalTok{(}\OtherTok{/}\BaseNTok{\textbackslash{}r}\FloatTok{?}\BaseNTok{\textbackslash{}n}\OtherTok{/}\NormalTok{);}
  \CommentTok{// keep the last partial line buffered}
  \KeywordTok{this}\NormalTok{._}\FunctionTok{buffer} \NormalTok{= }\KeywordTok{lines}\NormalTok{.}\FunctionTok{pop}\NormalTok{();}
  \KeywordTok{for} \NormalTok{(}\KeywordTok{var} \NormalTok{l = }\DecValTok{0}\NormalTok{; l < }\KeywordTok{lines}\NormalTok{.}\FunctionTok{length}\NormalTok{; l++) \{}
    \KeywordTok{var} \NormalTok{line = lines[l];}
    \KeywordTok{try} \NormalTok{\{}
      \KeywordTok{var} \NormalTok{obj = }\KeywordTok{JSON}\NormalTok{.}\FunctionTok{parse}\NormalTok{(line);}
    \NormalTok{\} }\KeywordTok{catch} \NormalTok{(er) \{}
      \KeywordTok{this}\NormalTok{.}\FunctionTok{emit}\NormalTok{(}\CharTok{'error'}\NormalTok{, er);}
      \KeywordTok{return}\NormalTok{;}
    \NormalTok{\}}
    \CommentTok{// push the parsed object out to the readable consumer}
    \KeywordTok{this}\NormalTok{.}\FunctionTok{push}\NormalTok{(obj);}
  \NormalTok{\}}
  \NormalTok{cb();}
\NormalTok{\};}

\KeywordTok{JSONParseStream.prototype}\NormalTok{._}\FunctionTok{flush} \NormalTok{= }\KeywordTok{function}\NormalTok{(cb) \{}
  \CommentTok{// Just handle any leftover}
  \KeywordTok{var} \NormalTok{rem = }\KeywordTok{this}\NormalTok{._}\FunctionTok{buffer}\NormalTok{.}\FunctionTok{trim}\NormalTok{();}
  \KeywordTok{if} \NormalTok{(rem) \{}
    \KeywordTok{try} \NormalTok{\{}
      \KeywordTok{var} \NormalTok{obj = }\KeywordTok{JSON}\NormalTok{.}\FunctionTok{parse}\NormalTok{(rem);}
    \NormalTok{\} }\KeywordTok{catch} \NormalTok{(er) \{}
      \KeywordTok{this}\NormalTok{.}\FunctionTok{emit}\NormalTok{(}\CharTok{'error'}\NormalTok{, er);}
      \KeywordTok{return}\NormalTok{;}
    \NormalTok{\}}
    \CommentTok{// push the parsed object out to the readable consumer}
    \KeywordTok{this}\NormalTok{.}\FunctionTok{push}\NormalTok{(obj);}
  \NormalTok{\}}
  \NormalTok{cb();}
\NormalTok{\};}
\end{Highlighting}
\end{Shaded}

The state objects contain other useful information for debugging the
state of streams in your programs. It is safe to look at them, but
beyond setting option flags in the constructor, it is \textbf{not} safe
to modify them.
