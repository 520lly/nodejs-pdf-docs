\section{Stream}

\begin{Shaded}
\begin{Highlighting}[]
\DataTypeTok{Stability}\NormalTok{: }\DecValTok{2} \NormalTok{- Unstable}
\end{Highlighting}
\end{Shaded}

A stream is an abstract interface implemented by various objects in
Node. For example a request to an HTTP server is a stream, as is stdout.
Streams are readable, writable, or both. All streams are instances of
\href{events.html\#events\_class\_events\_eventemitter}{EventEmitter}

You can load up the Stream base class by doing
\texttt{require('stream')}.

\subsection{Readable Stream}

A \texttt{Readable Stream} has the following methods, members, and
events.

\subsubsection{Event: `data'}

\texttt{function (data) \{ \}}

The \texttt{'data'} event emits either a \texttt{Buffer} (by default) or
a string if \texttt{setEncoding()} was used.

Note that the \textbf{data will be lost} if there is no listener when a
\texttt{Readable Stream} emits a \texttt{'data'} event.

\subsubsection{Event: `end'}

\texttt{function () \{ \}}

Emitted when the stream has received an EOF (FIN in TCP terminology).
Indicates that no more \texttt{'data'} events will happen. If the stream
is also writable, it may be possible to continue writing.

\subsubsection{Event: `error'}

\texttt{function (exception) \{ \}}

Emitted if there was an error receiving data.

\subsubsection{Event: `close'}

\texttt{function () \{ \}}

Emitted when the underlying resource (for example, the backing file
descriptor) has been closed. Not all streams will emit this.

\subsubsection{stream.readable}

A boolean that is \texttt{true} by default, but turns \texttt{false}
after an \texttt{'error'} occurred, the stream came to an
\texttt{'end'}, or \texttt{destroy()} was called.

\subsubsection{stream.setEncoding({[}encoding{]})}

Makes the \texttt{'data'} event emit a string instead of a
\texttt{Buffer}. \texttt{encoding} can be \texttt{'utf8'},
\texttt{'utf16le'} (\texttt{'ucs2'}), \texttt{'ascii'}, or
\texttt{'hex'}. Defaults to \texttt{'utf8'}.

\subsubsection{stream.pause()}

Issues an advisory signal to the underlying communication layer,
requesting that no further data be sent until \texttt{resume()} is
called.

Note that, due to the advisory nature, certain streams will not be
paused immediately, and so \texttt{'data'} events may be emitted for
some indeterminate period of time even after \texttt{pause()} is called.
You may wish to buffer such \texttt{'data'} events.

\subsubsection{stream.resume()}

Resumes the incoming \texttt{'data'} events after a \texttt{pause()}.

\subsubsection{stream.destroy()}

Closes the underlying file descriptor. Stream will not emit any more
events.

\subsubsection{stream.pipe(destination, {[}options{]})}

This is a \texttt{Stream.prototype} method available on all
\texttt{Stream}s.

Connects this read stream to \texttt{destination} WriteStream. Incoming
data on this stream gets written to \texttt{destination}. The
destination and source streams are kept in sync by pausing and resuming
as necessary.

This function returns the \texttt{destination} stream.

Emulating the Unix \texttt{cat} command:

\begin{Shaded}
\begin{Highlighting}[]
\KeywordTok{process.stdin}\NormalTok{.}\FunctionTok{resume}\NormalTok{();}
\KeywordTok{process.stdin}\NormalTok{.}\FunctionTok{pipe}\NormalTok{(}\KeywordTok{process}\NormalTok{.}\FunctionTok{stdout}\NormalTok{);}
\end{Highlighting}
\end{Shaded}

By default \texttt{end()} is called on the destination when the source
stream emits \texttt{end}, so that \texttt{destination} is no longer
writable. Pass \texttt{\{ end: false \}} as \texttt{options} to keep the
destination stream open.

This keeps \texttt{process.stdout} open so that ``Goodbye'' can be
written at the end.

\begin{Shaded}
\begin{Highlighting}[]
\KeywordTok{process.stdin}\NormalTok{.}\FunctionTok{resume}\NormalTok{();}

\KeywordTok{process.stdin}\NormalTok{.}\FunctionTok{pipe}\NormalTok{(}\KeywordTok{process}\NormalTok{.}\FunctionTok{stdout}\NormalTok{, \{ }\DataTypeTok{end}\NormalTok{: }\KeywordTok{false} \NormalTok{\});}

\KeywordTok{process.stdin}\NormalTok{.}\FunctionTok{on}\NormalTok{(}\StringTok{"end"}\NormalTok{, }\KeywordTok{function}\NormalTok{() \{}
  \KeywordTok{process.stdout}\NormalTok{.}\FunctionTok{write}\NormalTok{(}\StringTok{"Goodbye}\CharTok{\textbackslash{}n}\StringTok{"}\NormalTok{);}
\NormalTok{\});}
\end{Highlighting}
\end{Shaded}

\subsection{Writable Stream}

A \texttt{Writable Stream} has the following methods, members, and
events.

\subsubsection{Event: `drain'}

\texttt{function () \{ \}}

After a \texttt{write()} method returned \texttt{false}, this event is
emitted to indicate that it is safe to write again.

\subsubsection{Event: `error'}

\texttt{function (exception) \{ \}}

Emitted on error with the exception \texttt{exception}.

\subsubsection{Event: `close'}

\texttt{function () \{ \}}

Emitted when the underlying file descriptor has been closed.

\subsubsection{Event: `pipe'}

\texttt{function (src) \{ \}}

Emitted when the stream is passed to a readable stream's pipe method.

\subsubsection{stream.writable}

A boolean that is \texttt{true} by default, but turns \texttt{false}
after an \texttt{'error'} occurred or \texttt{end()} /
\texttt{destroy()} was called.

\subsubsection{stream.write(string, {[}encoding{]}, {[}fd{]})}

Writes \texttt{string} with the given \texttt{encoding} to the stream.
Returns \texttt{true} if the string has been flushed to the kernel
buffer. Returns \texttt{false} to indicate that the kernel buffer is
full, and the data will be sent out in the future. The \texttt{'drain'}
event will indicate when the kernel buffer is empty again. The
\texttt{encoding} defaults to \texttt{'utf8'}.

If the optional \texttt{fd} parameter is specified, it is interpreted as
an integral file descriptor to be sent over the stream. This is only
supported for UNIX streams, and is silently ignored otherwise. When
writing a file descriptor in this manner, closing the descriptor before
the stream drains risks sending an invalid (closed) FD.

\subsubsection{stream.write(buffer)}

Same as the above except with a raw buffer.

\subsubsection{stream.end()}

Terminates the stream with EOF or FIN. This call will allow queued write
data to be sent before closing the stream.

\subsubsection{stream.end(string, encoding)}

Sends \texttt{string} with the given \texttt{encoding} and terminates
the stream with EOF or FIN. This is useful to reduce the number of
packets sent.

\subsubsection{stream.end(buffer)}

Same as above but with a \texttt{buffer}.

\subsubsection{stream.destroy()}

Closes the underlying file descriptor. Stream will not emit any more
events. Any queued write data will not be sent.

\subsubsection{stream.destroySoon()}

After the write queue is drained, close the file descriptor.
\texttt{destroySoon()} can still destroy straight away, as long as there
is no data left in the queue for writes.
