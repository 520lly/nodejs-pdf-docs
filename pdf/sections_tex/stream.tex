\section{Stream}

\begin{Shaded}
\begin{Highlighting}[]
\DataTypeTok{Stability}\NormalTok{: }\DecValTok{2} \NormalTok{- Unstable}
\end{Highlighting}
\end{Shaded}

A stream is an abstract interface implemented by various objects in
Node. For example a request to an HTTP server is a stream, as is stdout.
Streams are readable, writable, or both. All streams are instances of
\href{events.html\#events\_class\_events\_eventemitter}{EventEmitter}

You can load the Stream base classes by doing
\texttt{require('stream')}. There are base classes provided for Readable
streams, Writable streams, Duplex streams, and Transform streams.

\subsection{Compatibility}

In earlier versions of Node, the Readable stream interface was simpler,
but also less powerful and less useful.

\begin{itemize}
\item
  Rather than waiting for you to call the \texttt{read()} method,
  \texttt{'data'} events would start emitting immediately. If you needed
  to do some I/O to decide how to handle data, then you had to store the
  chunks in some kind of buffer so that they would not be lost.
\item
  The \texttt{pause()} method was advisory, rather than guaranteed. This
  meant that you still had to be prepared to receive \texttt{'data'}
  events even when the stream was in a paused state.
\end{itemize}

In Node v0.10, the Readable class described below was added. For
backwards compatibility with older Node programs, Readable streams
switch into ``old mode'' when a \texttt{'data'} event handler is added,
or when the \texttt{pause()} or \texttt{resume()} methods are called.
The effect is that, even if you are not using the new \texttt{read()}
method and \texttt{'readable'} event, you no longer have to worry about
losing \texttt{'data'} chunks.

Most programs will continue to function normally. However, this
introduces an edge case in the following conditions:

\begin{itemize}
\item
  No \texttt{'data'} event handler is added.
\item
  The \texttt{pause()} and \texttt{resume()} methods are never called.
\end{itemize}

For example, consider the following code:

\begin{Shaded}
\begin{Highlighting}[]
\CommentTok{// WARNING!  BROKEN!}
\KeywordTok{net}\NormalTok{.}\FunctionTok{createServer}\NormalTok{(}\KeywordTok{function}\NormalTok{(socket) \{}

  \CommentTok{// we add an 'end' method, but never consume the data}
  \KeywordTok{socket}\NormalTok{.}\FunctionTok{on}\NormalTok{(}\CharTok{'end'}\NormalTok{, }\KeywordTok{function}\NormalTok{() \{}
    \CommentTok{// It will never get here.}
    \KeywordTok{socket}\NormalTok{.}\FunctionTok{end}\NormalTok{(}\CharTok{'I got your message (but didnt read it)\textbackslash{}n'}\NormalTok{);}
  \NormalTok{\});}

\NormalTok{\}).}\FunctionTok{listen}\NormalTok{(}\DecValTok{1337}\NormalTok{);}
\end{Highlighting}
\end{Shaded}

In versions of node prior to v0.10, the incoming message data would be
simply discarded. However, in Node v0.10 and beyond, the socket will
remain paused forever.

The workaround in this situation is to call the \texttt{resume()} method
to trigger ``old mode'' behavior:

\begin{Shaded}
\begin{Highlighting}[]
\CommentTok{// Workaround}
\KeywordTok{net}\NormalTok{.}\FunctionTok{createServer}\NormalTok{(}\KeywordTok{function}\NormalTok{(socket) \{}

  \KeywordTok{socket}\NormalTok{.}\FunctionTok{on}\NormalTok{(}\CharTok{'end'}\NormalTok{, }\KeywordTok{function}\NormalTok{() \{}
    \KeywordTok{socket}\NormalTok{.}\FunctionTok{end}\NormalTok{(}\CharTok{'I got your message (but didnt read it)\textbackslash{}n'}\NormalTok{);}
  \NormalTok{\});}

  \CommentTok{// start the flow of data, discarding it.}
  \KeywordTok{socket}\NormalTok{.}\FunctionTok{resume}\NormalTok{();}

\NormalTok{\}).}\FunctionTok{listen}\NormalTok{(}\DecValTok{1337}\NormalTok{);}
\end{Highlighting}
\end{Shaded}

In addition to new Readable streams switching into old-mode, pre-v0.10
style streams can be wrapped in a Readable class using the
\texttt{wrap()} method.

\subsection{Class: stream.Readable}

A \texttt{Readable Stream} has the following methods, members, and
events.

Note that \texttt{stream.Readable} is an abstract class designed to be
extended with an underlying implementation of the
\texttt{\_read(size, cb)} method. (See below.)

\subsubsection{new stream.Readable({[}options{]})}

\begin{itemize}
\item
  \texttt{options} \{Object\}
\item
  \texttt{bufferSize} \{Number\} The size of the chunks to consume from
  the underlying resource. Default=16kb
\item
  \texttt{lowWaterMark} \{Number\} The minimum number of bytes to store
  in the internal buffer before emitting \texttt{readable}. Default=0
\item
  \texttt{highWaterMark} \{Number\} The maximum number of bytes to store
  in the internal buffer before ceasing to read from the underlying
  resource. Default=16kb
\item
  \texttt{encoding} \{String\} If specified, then buffers will be
  decoded to strings using the specified encoding. Default=null
\end{itemize}

In classes that extend the Readable class, make sure to call the
constructor so that the buffering settings can be properly initialized.

\subsubsection{readable.\_read(size, callback)}

\begin{itemize}
\item
  \texttt{size} \{Number\} Number of bytes to read asynchronously
\item
  \texttt{callback} \{Function\} Called with an error or with data
\end{itemize}

All Readable stream implementations must provide a \texttt{\_read}
method to fetch data from the underlying resource.

Note: \textbf{This function MUST NOT be called directly.} It should be
implemented by child classes, and called by the internal Readable class
methods only.

Call the callback using the standard \texttt{callback(error, data)}
pattern. When no more data can be fetched, call
\texttt{callback(null, null)} to signal the EOF.

This method is prefixed with an underscore because it is internal to the
class that defines it, and should not be called directly by user
programs. However, you \textbf{are} expected to override this method in
your own extension classes.

\subsubsection{readable.wrap(stream)}

\begin{itemize}
\item
  \texttt{stream} \{Stream\} An ``old style'' readable stream
\end{itemize}

If you are using an older Node library that emits \texttt{'data'} events
and has a \texttt{pause()} method that is advisory only, then you can
use the \texttt{wrap()} method to create a Readable stream that uses the
old stream as its data source.

For example:

\begin{Shaded}
\begin{Highlighting}[]
\KeywordTok{var} \NormalTok{OldReader = require(}\CharTok{'./old-api-module.js'}\NormalTok{).}\FunctionTok{OldReader}\NormalTok{;}
\KeywordTok{var} \NormalTok{oreader = }\KeywordTok{new} \NormalTok{OldReader;}
\KeywordTok{var} \NormalTok{Readable = require(}\CharTok{'stream'}\NormalTok{).}\FunctionTok{Readable}\NormalTok{;}
\KeywordTok{var} \NormalTok{myReader = }\KeywordTok{new} \NormalTok{Readable().}\FunctionTok{wrap}\NormalTok{(oreader);}

\KeywordTok{myReader}\NormalTok{.}\FunctionTok{on}\NormalTok{(}\CharTok{'readable'}\NormalTok{, }\KeywordTok{function}\NormalTok{() \{}
  \KeywordTok{myReader}\NormalTok{.}\FunctionTok{read}\NormalTok{(); }\CommentTok{// etc.}
\NormalTok{\});}
\end{Highlighting}
\end{Shaded}

\subsubsection{Event: `readable'}

When there is data ready to be consumed, this event will fire. The
number of bytes that are required to be considered ``readable'' depends
on the \texttt{lowWaterMark} option set in the constructor.

When this event emits, call the \texttt{read()} method to consume the
data.

\subsubsection{Event: `end'}

Emitted when the stream has received an EOF (FIN in TCP terminology).
Indicates that no more \texttt{'data'} events will happen. If the stream
is also writable, it may be possible to continue writing.

\subsubsection{Event: `data'}

The \texttt{'data'} event emits either a \texttt{Buffer} (by default) or
a string if \texttt{setEncoding()} was used.

Note that adding a \texttt{'data'} event listener will switch the
Readable stream into ``old mode'', where data is emitted as soon as it
is available, rather than waiting for you to call \texttt{read()} to
consume it.

\subsubsection{Event: `error'}

Emitted if there was an error receiving data.

\subsubsection{Event: `close'}

Emitted when the underlying resource (for example, the backing file
descriptor) has been closed. Not all streams will emit this.

\subsubsection{readable.setEncoding(encoding)}

Makes the \texttt{'data'} event emit a string instead of a
\texttt{Buffer}. \texttt{encoding} can be \texttt{'utf8'},
\texttt{'utf16le'} (\texttt{'ucs2'}), \texttt{'ascii'}, or
\texttt{'hex'}.

The encoding can also be set by specifying an \texttt{encoding} field to
the constructor.

\subsubsection{readable.read({[}size{]})}

\begin{itemize}
\item
  \texttt{size} \{Number \textbar{} null\} Optional number of bytes to
  read.
\item
  Return: \{Buffer \textbar{} String \textbar{} null\}
\end{itemize}

Call this method to consume data once the \texttt{'readable'} event is
emitted.

The \texttt{size} argument will set a minimum number of bytes that you
are interested in. If not set, then the entire content of the internal
buffer is returned.

If there is no data to consume, or if there are fewer bytes in the
internal buffer than the \texttt{size} argument, then \texttt{null} is
returned, and a future \texttt{'readable'} event will be emitted when
more is available.

Note that calling \texttt{stream.read(0)} will always return
\texttt{null}, and will trigger a refresh of the internal buffer, but
otherwise be a no-op.

\subsubsection{readable.pipe(destination, {[}options{]})}

\begin{itemize}
\item
  \texttt{destination} \{Writable Stream\}
\item
  \texttt{options} \{Object\} Optional
\item
  \texttt{end} \{Boolean\} Default=true
\end{itemize}

Connects this readable stream to \texttt{destination} WriteStream.
Incoming data on this stream gets written to \texttt{destination}.
Properly manages back-pressure so that a slow destination will not be
overwhelmed by a fast readable stream.

This function returns the \texttt{destination} stream.

For example, emulating the Unix \texttt{cat} command:

\begin{Shaded}
\begin{Highlighting}[]
\KeywordTok{process.stdin}\NormalTok{.}\FunctionTok{pipe}\NormalTok{(}\KeywordTok{process}\NormalTok{.}\FunctionTok{stdout}\NormalTok{);}
\end{Highlighting}
\end{Shaded}

By default \texttt{end()} is called on the destination when the source
stream emits \texttt{end}, so that \texttt{destination} is no longer
writable. Pass \texttt{\{ end: false \}} as \texttt{options} to keep the
destination stream open.

This keeps \texttt{writer} open so that ``Goodbye'' can be written at
the end.

\begin{Shaded}
\begin{Highlighting}[]
\KeywordTok{reader}\NormalTok{.}\FunctionTok{pipe}\NormalTok{(writer, \{ }\DataTypeTok{end}\NormalTok{: }\KeywordTok{false} \NormalTok{\});}
\KeywordTok{reader}\NormalTok{.}\FunctionTok{on}\NormalTok{(}\StringTok{"end"}\NormalTok{, }\KeywordTok{function}\NormalTok{() \{}
  \KeywordTok{writer}\NormalTok{.}\FunctionTok{end}\NormalTok{(}\StringTok{"Goodbye}\CharTok{\textbackslash{}n}\StringTok{"}\NormalTok{);}
\NormalTok{\});}
\end{Highlighting}
\end{Shaded}

Note that \texttt{process.stderr} and \texttt{process.stdout} are never
closed until the process exits, regardless of the specified options.

\subsubsection{readable.unpipe({[}destination{]})}

\begin{itemize}
\item
  \texttt{destination} \{Writable Stream\} Optional
\end{itemize}

Undo a previously established \texttt{pipe()}. If no destination is
provided, then all previously established pipes are removed.

\subsubsection{readable.pause()}

Switches the readable stream into ``old mode'', where data is emitted
using a \texttt{'data'} event rather than being buffered for consumption
via the \texttt{read()} method.

Ceases the flow of data. No \texttt{'data'} events are emitted while the
stream is in a paused state.

\subsubsection{readable.resume()}

Switches the readable stream into ``old mode'', where data is emitted
using a \texttt{'data'} event rather than being buffered for consumption
via the \texttt{read()} method.

Resumes the incoming \texttt{'data'} events after a \texttt{pause()}.

\subsection{Class: stream.Writable}

A \texttt{Writable} Stream has the following methods, members, and
events.

Note that \texttt{stream.Writable} is an abstract class designed to be
extended with an underlying implementation of the
\texttt{\_write(chunk, cb)} method. (See below.)

\subsubsection{new stream.Writable({[}options{]})}

\begin{itemize}
\item
  \texttt{options} \{Object\}
\item
  \texttt{highWaterMark} \{Number\} Buffer level when \texttt{write()}
  starts returning false. Default=16kb
\item
  \texttt{lowWaterMark} \{Number\} The buffer level when
  \texttt{'drain'} is emitted. Default=0
\item
  \texttt{decodeStrings} \{Boolean\} Whether or not to decode strings
  into Buffers before passing them to \texttt{\_write()}. Default=true
\end{itemize}

In classes that extend the Writable class, make sure to call the
constructor so that the buffering settings can be properly initialized.

\subsubsection{writable.\_write(chunk, callback)}

\begin{itemize}
\item
  \texttt{chunk} \{Buffer \textbar{} Array\} The data to be written
\item
  \texttt{callback} \{Function\} Called with an error, or null when
  finished
\end{itemize}

All Writable stream implementations must provide a \texttt{\_write}
method to send data to the underlying resource.

Note: \textbf{This function MUST NOT be called directly.} It should be
implemented by child classes, and called by the internal Writable class
methods only.

Call the callback using the standard \texttt{callback(error)} pattern to
signal that the write completed successfully or with an error.

If the \texttt{decodeStrings} flag is set in the constructor options,
then \texttt{chunk} will be an array rather than a Buffer. This is to
support implementations that have an optimized handling for certain
string data encodings.

This method is prefixed with an underscore because it is internal to the
class that defines it, and should not be called directly by user
programs. However, you \textbf{are} expected to override this method in
your own extension classes.

\subsubsection{writable.write(chunk, {[}encoding{]}, {[}callback{]})}

\begin{itemize}
\item
  \texttt{chunk} \{Buffer \textbar{} String\} Data to be written
\item
  \texttt{encoding} \{String\} Optional. If \texttt{chunk} is a string,
  then encoding defaults to \texttt{'utf8'}
\item
  \texttt{callback} \{Function\} Optional. Called when this chunk is
  successfully written.
\item
  Returns \{Boolean\}
\end{itemize}

Writes \texttt{chunk} to the stream. Returns \texttt{true} if the data
has been flushed to the underlying resource. Returns \texttt{false} to
indicate that the buffer is full, and the data will be sent out in the
future. The \texttt{'drain'} event will indicate when the buffer is
empty again.

The specifics of when \texttt{write()} will return false, and when a
subsequent \texttt{'drain'} event will be emitted, are determined by the
\texttt{highWaterMark} and \texttt{lowWaterMark} options provided to the
constructor.

\subsubsection{writable.end({[}chunk{]}, {[}encoding{]})}

\begin{itemize}
\item
  \texttt{chunk} \{Buffer \textbar{} String\} Optional final data to be
  written
\item
  \texttt{encoding} \{String\} Optional. If \texttt{chunk} is a string,
  then encoding defaults to \texttt{'utf8'}
\end{itemize}

Call this method to signal the end of the data being written to the
stream.

\subsubsection{Event: `drain'}

Emitted when the stream's write queue empties and it's safe to write
without buffering again. Listen for it when \texttt{stream.write()}
returns \texttt{false}.

\subsubsection{Event: `close'}

Emitted when the underlying resource (for example, the backing file
descriptor) has been closed. Not all streams will emit this.

\subsubsection{Event: `finish'}

When \texttt{end()} is called and there are no more chunks to write,
this event is emitted.

\subsubsection{Event: `pipe'}

\begin{itemize}
\item
  \texttt{source} \{Readable Stream\}
\end{itemize}

Emitted when the stream is passed to a readable stream's pipe method.

\subsubsection{Event `unpipe'}

\begin{itemize}
\item
  \texttt{source} \{Readable Stream\}
\end{itemize}

Emitted when a previously established \texttt{pipe()} is removed using
the source Readable stream's \texttt{unpipe()} method.

\subsection{Class: stream.Duplex}

A ``duplex'' stream is one that is both Readable and Writable, such as a
TCP socket connection.

Note that \texttt{stream.Duplex} is an abstract class designed to be
extended with an underlying implementation of the
\texttt{\_read(size, cb)} and \texttt{\_write(chunk, callback)} methods
as you would with a Readable or Writable stream class.

Since JavaScript doesn't have multiple prototypal inheritance, this
class prototypally inherits from Readable, and then parasitically from
Writable. It is thus up to the user to implement both the lowlevel
\texttt{\_read(n,cb)} method as well as the lowlevel
\texttt{\_write(chunk,cb)} method on extension duplex classes.

\subsubsection{new stream.Duplex(options)}

\begin{itemize}
\item
  \texttt{options} \{Object\} Passed to both Writable and Readable
  constructors. Also has the following fields:
\item
  \texttt{allowHalfOpen} \{Boolean\} Default=true. If set to
  \texttt{false}, then the stream will automatically end the readable
  side when the writable side ends and vice versa.
\end{itemize}

In classes that extend the Duplex class, make sure to call the
constructor so that the buffering settings can be properly initialized.

\subsection{Class: stream.Transform}

A ``transform'' stream is a duplex stream where the output is causally
connected in some way to the input, such as a zlib stream or a crypto
stream.

There is no requirement that the output be the same size as the input,
the same number of chunks, or arrive at the same time. For example, a
Hash stream will only ever have a single chunk of output which is
provided when the input is ended. A zlib stream will either produce much
smaller or much larger than its input.

Rather than implement the \texttt{\_read()} and \texttt{\_write()}
methods, Transform classes must implement the \texttt{\_transform()}
method, and may optionally also implement the \texttt{\_flush()} method.
(See below.)

\subsubsection{new stream.Transform({[}options{]})}

\begin{itemize}
\item
  \texttt{options} \{Object\} Passed to both Writable and Readable
  constructors.
\end{itemize}

In classes that extend the Transform class, make sure to call the
constructor so that the buffering settings can be properly initialized.

\subsubsection{transform.\_transform(chunk, outputFn, callback)}

\begin{itemize}
\item
  \texttt{chunk} \{Buffer\} The chunk to be transformed.
\item
  \texttt{outputFn} \{Function\} Call this function with any output data
  to be passed to the readable interface.
\item
  \texttt{callback} \{Function\} Call this function (optionally with an
  error argument) when you are done processing the supplied chunk.
\end{itemize}

All Transform stream implementations must provide a \texttt{\_transform}
method to accept input and produce output.

Note: \textbf{This function MUST NOT be called directly.} It should be
implemented by child classes, and called by the internal Transform class
methods only.

\texttt{\_transform} should do whatever has to be done in this specific
Transform class, to handle the bytes being written, and pass them off to
the readable portion of the interface. Do asynchronous I/O, process
things, and so on.

Call the callback function only when the current chunk is completely
consumed. Note that this may mean that you call the \texttt{outputFn}
zero or more times, depending on how much data you want to output as a
result of this chunk.

This method is prefixed with an underscore because it is internal to the
class that defines it, and should not be called directly by user
programs. However, you \textbf{are} expected to override this method in
your own extension classes.

\subsubsection{transform.\_flush(outputFn, callback)}

\begin{itemize}
\item
  \texttt{outputFn} \{Function\} Call this function with any output data
  to be passed to the readable interface.
\item
  \texttt{callback} \{Function\} Call this function (optionally with an
  error argument) when you are done flushing any remaining data.
\end{itemize}

Note: \textbf{This function MUST NOT be called directly.} It MAY be
implemented by child classes, and if so, will be called by the internal
Transform class methods only.

In some cases, your transform operation may need to emit a bit more data
at the end of the stream. For example, a \texttt{Zlib} compression
stream will store up some internal state so that it can optimally
compress the output. At the end, however, it needs to do the best it can
with what is left, so that the data will be complete.

In those cases, you can implement a \texttt{\_flush} method, which will
be called at the very end, after all the written data is consumed, but
before emitting \texttt{end} to signal the end of the readable side.
Just like with \texttt{\_transform}, call \texttt{outputFn} zero or more
times, as appropriate, and call \texttt{callback} when the flush
operation is complete.

This method is prefixed with an underscore because it is internal to the
class that defines it, and should not be called directly by user
programs. However, you \textbf{are} expected to override this method in
your own extension classes.

\subsection{Class: stream.PassThrough}

This is a trivial implementation of a \texttt{Transform} stream that
simply passes the input bytes across to the output. Its purpose is
mainly for examples and testing, but there are occasionally use cases
where it can come in handy.
