\section{HTTP}

\begin{Shaded}
\begin{Highlighting}[]
\DataTypeTok{Stability}\NormalTok{: }\DecValTok{3} \NormalTok{- Stable}
\end{Highlighting}
\end{Shaded}

To use the HTTP server and client one must \texttt{require('http')}.

The HTTP interfaces in Node are designed to support many features of the
protocol which have been traditionally difficult to use. In particular,
large, possibly chunk-encoded, messages. The interface is careful to
never buffer entire requests or responses--the user is able to stream
data.

HTTP message headers are represented by an object like this:

\begin{Shaded}
\begin{Highlighting}[]
\NormalTok{\{ }\CharTok{'content-length'}\NormalTok{: }\CharTok{'123'}\NormalTok{,}
  \CharTok{'content-type'}\NormalTok{: }\CharTok{'text/plain'}\NormalTok{,}
  \CharTok{'connection'}\NormalTok{: }\CharTok{'keep-alive'}\NormalTok{,}
  \CharTok{'accept'}\NormalTok{: }\CharTok{'*/*'} \NormalTok{\}}
\end{Highlighting}
\end{Shaded}

Keys are lowercased. Values are not modified.

In order to support the full spectrum of possible HTTP applications,
Node's HTTP API is very low-level. It deals with stream handling and
message parsing only. It parses a message into headers and body but it
does not parse the actual headers or the body.

\subsection{http.STATUS\_CODES}

\begin{itemize}
\item
  \{Object\}
\end{itemize}

A collection of all the standard HTTP response status codes, and the
short description of each. For example,
\texttt{http.STATUS\_CODES{[}404{]} === 'Not Found'}.

\subsection{http.createServer({[}requestListener{]})}

Returns a new web server object.

The \texttt{requestListener} is a function which is automatically added
to the \texttt{'request'} event.

\subsection{http.createClient({[}port{]}, {[}host{]})}

This function is \textbf{deprecated}; please use
\hyperref[http_http_request_options_callback]{http.request()}
instead. Constructs a new HTTP client. \texttt{port} and \texttt{host}
refer to the server to be connected to.

\subsection{Class: http.Server}

This is an
\href{events.html\#events\_class\_events\_eventemitter}{EventEmitter}
with the following events:

\subsubsection{Event: `request'}

\texttt{function (request, response) \{ \}}

Emitted each time there is a request. Note that there may be multiple
requests per connection (in the case of keep-alive connections).
\texttt{request} is an instance of \texttt{http.ServerRequest} and
\texttt{response} is an instance of \texttt{http.ServerResponse}

\subsubsection{Event: `connection'}

\texttt{function (socket) \{ \}}

When a new TCP stream is established. \texttt{socket} is an object of
type \texttt{net.Socket}. Usually users will not want to access this
event. The \texttt{socket} can also be accessed at
\texttt{request.connection}.

\subsubsection{Event: `close'}

\texttt{function () \{ \}}

Emitted when the server closes.

\subsubsection{Event: `checkContinue'}

\texttt{function (request, response) \{ \}}

Emitted each time a request with an http Expect: 100-continue is
received. If this event isn't listened for, the server will
automatically respond with a 100 Continue as appropriate.

Handling this event involves calling \texttt{response.writeContinue} if
the client should continue to send the request body, or generating an
appropriate HTTP response (e.g., 400 Bad Request) if the client should
not continue to send the request body.

Note that when this event is emitted and handled, the \texttt{request}
event will not be emitted.

\subsubsection{Event: `connect'}

\texttt{function (request, socket, head) \{ \}}

Emitted each time a client requests a http CONNECT method. If this event
isn't listened for, then clients requesting a CONNECT method will have
their connections closed.

\begin{itemize}
\item
  \texttt{request} is the arguments for the http request, as it is in
  the request event.
\item
  \texttt{socket} is the network socket between the server and client.
\item
  \texttt{head} is an instance of Buffer, the first packet of the
  tunneling stream, this may be empty.
\end{itemize}

After this event is emitted, the request's socket will not have a
\texttt{data} event listener, meaning you will need to bind to it in
order to handle data sent to the server on that socket.

\subsubsection{Event: `upgrade'}

\texttt{function (request, socket, head) \{ \}}

Emitted each time a client requests a http upgrade. If this event isn't
listened for, then clients requesting an upgrade will have their
connections closed.

\begin{itemize}
\item
  \texttt{request} is the arguments for the http request, as it is in
  the request event.
\item
  \texttt{socket} is the network socket between the server and client.
\item
  \texttt{head} is an instance of Buffer, the first packet of the
  upgraded stream, this may be empty.
\end{itemize}

After this event is emitted, the request's socket will not have a
\texttt{data} event listener, meaning you will need to bind to it in
order to handle data sent to the server on that socket.

\subsubsection{Event: `clientError'}

\texttt{function (exception) \{ \}}

If a client connection emits an `error' event - it will forwarded here.

\subsubsection{server.listen(port, {[}hostname{]}, {[}backlog{]},
{[}callback{]})}

Begin accepting connections on the specified port and hostname. If the
hostname is omitted, the server will accept connections directed to any
IPv4 address (\texttt{INADDR\_ANY}).

To listen to a unix socket, supply a filename instead of port and
hostname.

Backlog is the maximum length of the queue of pending connections. The
actual length will be determined by your OS through sysctl settings such
as \texttt{tcp\_max\_syn\_backlog} and \texttt{somaxconn} on linux. The
default value of this parameter is 511 (not 512).

This function is asynchronous. The last parameter \texttt{callback} will
be added as a listener for the
\href{net.html\#net\_event\_listening}{`listening'} event. See also
\href{net.html\#net\_server\_listen\_port\_host\_backlog\_listeninglistener}{net.Server.listen(port)}.

\subsubsection{server.listen(path, {[}callback{]})}

Start a UNIX socket server listening for connections on the given
\texttt{path}.

This function is asynchronous. The last parameter \texttt{callback} will
be added as a listener for the
\href{net.html\#net\_event\_listening}{`listening'} event. See also
\href{net.html\#net\_server\_listen\_path\_listeninglistener}{net.Server.listen(path)}.

\subsubsection{server.listen(handle, {[}listeningListener{]})}

\begin{itemize}
\item
  \texttt{handle} \{Object\}
\item
  \texttt{listeningListener} \{Function\}
\end{itemize}

The \texttt{handle} object can be set to either a server or socket
(anything with an underlying \texttt{\_handle} member), or a
\texttt{\{fd: \textless{}n\textgreater{}\}} object.

This will cause the server to accept connections on the specified
handle, but it is presumed that the file descriptor or handle has
already been bound to a port or domain socket.

Listening on a file descriptor is not supported on Windows.

This function is asynchronous. The last parameter \texttt{callback} will
be added as a listener for the
\href{net.html\#event\_listening\_}{`listening'} event. See also
\href{net.html\#server.listen}{net.Server.listen()}.

\subsubsection{server.close({[}cb{]})}

Stops the server from accepting new connections. See
\href{net.html\#net\_server\_close\_cb}{net.Server.close()}.

\subsubsection{server.maxHeadersCount}

Limits maximum incoming headers count, equal to 1000 by default. If set
to 0 - no limit will be applied.

\subsection{Class: http.ServerRequest}

This object is created internally by a HTTP server -- not by the user --
and passed as the first argument to a \texttt{'request'} listener.

The request implements the
\href{stream.html\#stream\_readable\_stream}{Readable Stream} interface.
This is an
\href{events.html\#events\_class\_events\_eventemitter}{EventEmitter}
with the following events:

\subsubsection{Event: `data'}

\texttt{function (chunk) \{ \}}

Emitted when a piece of the message body is received. The chunk is a
string if an encoding has been set with \texttt{request.setEncoding()},
otherwise it's a \href{buffer.html\#buffer\_buffer}{Buffer}.

Note that the \textbf{data will be lost} if there is no listener when a
\texttt{ServerRequest} emits a \texttt{'data'} event.

\subsubsection{Event: `end'}

\texttt{function () \{ \}}

Emitted exactly once for each request. After that, no more
\texttt{'data'} events will be emitted on the request.

\subsubsection{Event: `close'}

\texttt{function () \{ \}}

Indicates that the underlaying connection was terminated before
\texttt{response.end()} was called or able to flush.

Just like \texttt{'end'}, this event occurs only once per request, and
no more \texttt{'data'} events will fire afterwards.

Note: \texttt{'close'} can fire after \texttt{'end'}, but not vice
versa.

\subsubsection{request.method}

The request method as a string. Read only. Example: \texttt{'GET'},
\texttt{'DELETE'}.

\subsubsection{request.url}

Request URL string. This contains only the URL that is present in the
actual HTTP request. If the request is:

\begin{Shaded}
\begin{Highlighting}[]
\NormalTok{GET /status?name=ryan HTTP/}\FloatTok{1.1}\NormalTok{\textbackslash{}r\textbackslash{}n}
\DataTypeTok{Accept}\NormalTok{: text/plain\textbackslash{}r\textbackslash{}n}
\NormalTok{\textbackslash{}r\textbackslash{}n}
\end{Highlighting}
\end{Shaded}

Then \texttt{request.url} will be:

\begin{Shaded}
\begin{Highlighting}[]
\CharTok{'/status?name=ryan'}
\end{Highlighting}
\end{Shaded}

If you would like to parse the URL into its parts, you can use
\texttt{require('url').parse(request.url)}. Example:

\begin{Shaded}
\begin{Highlighting}[]
\NormalTok{node> require(}\CharTok{'url'}\NormalTok{).}\FunctionTok{parse}\NormalTok{(}\CharTok{'/status?name=ryan'}\NormalTok{)}
\NormalTok{\{ }\DataTypeTok{href}\NormalTok{: }\CharTok{'/status?name=ryan'}\NormalTok{,}
  \DataTypeTok{search}\NormalTok{: }\CharTok{'?name=ryan'}\NormalTok{,}
  \DataTypeTok{query}\NormalTok{: }\CharTok{'name=ryan'}\NormalTok{,}
  \DataTypeTok{pathname}\NormalTok{: }\CharTok{'/status'} \NormalTok{\}}
\end{Highlighting}
\end{Shaded}

If you would like to extract the params from the query string, you can
use the \texttt{require('querystring').parse} function, or pass
\texttt{true} as the second argument to \texttt{require('url').parse}.
Example:

\begin{Shaded}
\begin{Highlighting}[]
\NormalTok{node> require(}\CharTok{'url'}\NormalTok{).}\FunctionTok{parse}\NormalTok{(}\CharTok{'/status?name=ryan'}\NormalTok{, }\KeywordTok{true}\NormalTok{)}
\NormalTok{\{ }\DataTypeTok{href}\NormalTok{: }\CharTok{'/status?name=ryan'}\NormalTok{,}
  \DataTypeTok{search}\NormalTok{: }\CharTok{'?name=ryan'}\NormalTok{,}
  \DataTypeTok{query}\NormalTok{: \{ }\DataTypeTok{name}\NormalTok{: }\CharTok{'ryan'} \NormalTok{\},}
  \DataTypeTok{pathname}\NormalTok{: }\CharTok{'/status'} \NormalTok{\}}
\end{Highlighting}
\end{Shaded}

\subsubsection{request.headers}

Read only map of header names and values. Header names are lower-cased.
Example:

\begin{Shaded}
\begin{Highlighting}[]
\CommentTok{// Prints something like:}
\CommentTok{//}
\CommentTok{// \{ 'user-agent': 'curl/7.22.0',}
\CommentTok{//   host: '127.0.0.1:8000',}
\CommentTok{//   accept: '*/*' \}}
\KeywordTok{console}\NormalTok{.}\FunctionTok{log}\NormalTok{(}\KeywordTok{request}\NormalTok{.}\FunctionTok{headers}\NormalTok{);}
\end{Highlighting}
\end{Shaded}

\subsubsection{request.trailers}

Read only; HTTP trailers (if present). Only populated after the `end'
event.

\subsubsection{request.httpVersion}

The HTTP protocol version as a string. Read only. Examples:
\texttt{'1.1'}, \texttt{'1.0'}. Also \texttt{request.httpVersionMajor}
is the first integer and \texttt{request.httpVersionMinor} is the
second.

\subsubsection{request.setEncoding({[}encoding{]})}

Set the encoding for the request body. See
\href{stream.html\#stream\_stream\_setencoding\_encoding}{stream.setEncoding()}
for more information.

\subsubsection{request.pause()}

Pauses request from emitting events. Useful to throttle back an upload.

\subsubsection{request.resume()}

Resumes a paused request.

\subsubsection{request.connection}

The \texttt{net.Socket} object associated with the connection.

With HTTPS support, use request.connection.verifyPeer() and
request.connection.getPeerCertificate() to obtain the client's
authentication details.

\subsection{Class: http.ServerResponse}

This object is created internally by a HTTP server--not by the user. It
is passed as the second parameter to the \texttt{'request'} event.

The response implements the
\href{stream.html\#stream\_writable\_stream}{Writable Stream} interface.
This is an
\href{events.html\#events\_class\_events\_eventemitter}{EventEmitter}
with the following events:

\subsubsection{Event: `end'}

\texttt{function () \{ \}}

Emitted when the response has been sent. More specifically, this event
is emitted when the last segment of the response headers and body have
been handed off to the operating system for transmission over the
network. It does not imply that the client has received anything yet.

After this event, no more events will be emitted on the response object.

\subsubsection{Event: `close'}

\texttt{function () \{ \}}

Indicates that the underlaying connection was terminated before
\texttt{response.end()} was called or able to flush.

\subsubsection{response.writeContinue()}

Sends a HTTP/1.1 100 Continue message to the client, indicating that the
request body should be sent. See the
\hyperref[http_event_checkcontinue]{`checkContinue'} event on
\texttt{Server}.

\subsubsection{response.writeHead(statusCode, {[}reasonPhrase{]},
{[}headers{]})}

Sends a response header to the request. The status code is a 3-digit
HTTP status code, like \texttt{404}. The last argument,
\texttt{headers}, are the response headers. Optionally one can give a
human-readable \texttt{reasonPhrase} as the second argument.

Example:

\begin{Shaded}
\begin{Highlighting}[]
\KeywordTok{var} \NormalTok{body = }\CharTok{'hello world'}\NormalTok{;}
\KeywordTok{response}\NormalTok{.}\FunctionTok{writeHead}\NormalTok{(}\DecValTok{200}\NormalTok{, \{}
  \CharTok{'Content-Length'}\NormalTok{: }\KeywordTok{body}\NormalTok{.}\FunctionTok{length}\NormalTok{,}
  \CharTok{'Content-Type'}\NormalTok{: }\CharTok{'text/plain'} \NormalTok{\});}
\end{Highlighting}
\end{Shaded}

This method must only be called once on a message and it must be called
before \texttt{response.end()} is called.

If you call \texttt{response.write()} or \texttt{response.end()} before
calling this, the implicit/mutable headers will be calculated and call
this function for you.

Note: that Content-Length is given in bytes not characters. The above
example works because the string \texttt{'hello world'} contains only
single byte characters. If the body contains higher coded characters
then \texttt{Buffer.byteLength()} should be used to determine the number
of bytes in a given encoding. And Node does not check whether
Content-Length and the length of the body which has been transmitted are
equal or not.

\subsubsection{response.statusCode}

When using implicit headers (not calling \texttt{response.writeHead()}
explicitly), this property controls the status code that will be sent to
the client when the headers get flushed.

Example:

\begin{Shaded}
\begin{Highlighting}[]
\KeywordTok{response}\NormalTok{.}\FunctionTok{statusCode} \NormalTok{= }\DecValTok{404}\NormalTok{;}
\end{Highlighting}
\end{Shaded}

After response header was sent to the client, this property indicates
the status code which was sent out.

\subsubsection{response.setHeader(name, value)}

Sets a single header value for implicit headers. If this header already
exists in the to-be-sent headers, its value will be replaced. Use an
array of strings here if you need to send multiple headers with the same
name.

Example:

\begin{Shaded}
\begin{Highlighting}[]
\KeywordTok{response}\NormalTok{.}\FunctionTok{setHeader}\NormalTok{(}\StringTok{"Content-Type"}\NormalTok{, }\StringTok{"text/html"}\NormalTok{);}
\end{Highlighting}
\end{Shaded}

or

\begin{Shaded}
\begin{Highlighting}[]
\KeywordTok{response}\NormalTok{.}\FunctionTok{setHeader}\NormalTok{(}\StringTok{"Set-Cookie"}\NormalTok{, [}\StringTok{"type=ninja"}\NormalTok{, }\StringTok{"language=javascript"}\NormalTok{]);}
\end{Highlighting}
\end{Shaded}

\subsubsection{response.sendDate}

When true, the Date header will be automatically generated and sent in
the response if it is not already present in the headers. Defaults to
true.

This should only be disabled for testing; HTTP requires the Date header
in responses.

\subsubsection{response.getHeader(name)}

Reads out a header that's already been queued but not sent to the
client. Note that the name is case insensitive. This can only be called
before headers get implicitly flushed.

Example:

\begin{Shaded}
\begin{Highlighting}[]
\KeywordTok{var} \NormalTok{contentType = }\KeywordTok{response}\NormalTok{.}\FunctionTok{getHeader}\NormalTok{(}\CharTok{'content-type'}\NormalTok{);}
\end{Highlighting}
\end{Shaded}

\subsubsection{response.removeHeader(name)}

Removes a header that's queued for implicit sending.

Example:

\begin{Shaded}
\begin{Highlighting}[]
\KeywordTok{response}\NormalTok{.}\FunctionTok{removeHeader}\NormalTok{(}\StringTok{"Content-Encoding"}\NormalTok{);}
\end{Highlighting}
\end{Shaded}

\subsubsection{response.write(chunk, {[}encoding{]})}

If this method is called and \texttt{response.writeHead()} has not been
called, it will switch to implicit header mode and flush the implicit
headers.

This sends a chunk of the response body. This method may be called
multiple times to provide successive parts of the body.

\texttt{chunk} can be a string or a buffer. If \texttt{chunk} is a
string, the second parameter specifies how to encode it into a byte
stream. By default the \texttt{encoding} is \texttt{'utf8'}.

\textbf{Note}: This is the raw HTTP body and has nothing to do with
higher-level multi-part body encodings that may be used.

The first time \texttt{response.write()} is called, it will send the
buffered header information and the first body to the client. The second
time \texttt{response.write()} is called, Node assumes you're going to
be streaming data, and sends that separately. That is, the response is
buffered up to the first chunk of body.

Returns \texttt{true} if the entire data was flushed successfully to the
kernel buffer. Returns \texttt{false} if all or part of the data was
queued in user memory. \texttt{'drain'} will be emitted when the buffer
is again free.

\subsubsection{response.addTrailers(headers)}

This method adds HTTP trailing headers (a header but at the end of the
message) to the response.

Trailers will \textbf{only} be emitted if chunked encoding is used for
the response; if it is not (e.g., if the request was HTTP/1.0), they
will be silently discarded.

Note that HTTP requires the \texttt{Trailer} header to be sent if you
intend to emit trailers, with a list of the header fields in its value.
E.g.,

\begin{Shaded}
\begin{Highlighting}[]
\KeywordTok{response}\NormalTok{.}\FunctionTok{writeHead}\NormalTok{(}\DecValTok{200}\NormalTok{, \{ }\CharTok{'Content-Type'}\NormalTok{: }\CharTok{'text/plain'}\NormalTok{,}
                          \CharTok{'Trailer'}\NormalTok{: }\CharTok{'Content-MD5'} \NormalTok{\});}
\KeywordTok{response}\NormalTok{.}\FunctionTok{write}\NormalTok{(fileData);}
\KeywordTok{response}\NormalTok{.}\FunctionTok{addTrailers}\NormalTok{(\{}\CharTok{'Content-MD5'}\NormalTok{: }\StringTok{"7895bf4b8828b55ceaf47747b4bca667"}\NormalTok{\});}
\KeywordTok{response}\NormalTok{.}\FunctionTok{end}\NormalTok{();}
\end{Highlighting}
\end{Shaded}

\subsubsection{response.end({[}data{]}, {[}encoding{]})}

This method signals to the server that all of the response headers and
body have been sent; that server should consider this message complete.
The method, \texttt{response.end()}, MUST be called on each response.

If \texttt{data} is specified, it is equivalent to calling
\texttt{response.write(data, encoding)} followed by
\texttt{response.end()}.

\subsection{http.request(options, callback)}

Node maintains several connections per server to make HTTP requests.
This function allows one to transparently issue requests.

\texttt{options} can be an object or a string. If \texttt{options} is a
string, it is automatically parsed with
\href{url.html\#url\_url\_parse\_urlstr\_parsequerystring\_slashesdenotehost}{url.parse()}.

Options:

\begin{itemize}
\item
  \texttt{host}: A domain name or IP address of the server to issue the
  request to. Defaults to \texttt{'localhost'}.
\item
  \texttt{hostname}: To support \texttt{url.parse()} \texttt{hostname}
  is preferred over \texttt{host}
\item
  \texttt{port}: Port of remote server. Defaults to 80.
\item
  \texttt{localAddress}: Local interface to bind for network
  connections.
\item
  \texttt{socketPath}: Unix Domain Socket (use one of host:port or
  socketPath)
\item
  \texttt{method}: A string specifying the HTTP request method. Defaults
  to \texttt{'GET'}.
\item
  \texttt{path}: Request path. Defaults to \texttt{'/'}. Should include
  query string if any. E.G. \texttt{'/index.html?page=12'}
\item
  \texttt{headers}: An object containing request headers.
\item
  \texttt{auth}: Basic authentication i.e. \texttt{'user:password'} to
  compute an Authorization header.
\item
  \texttt{agent}: Controls \hyperref[http_class_http_agent]{Agent}
  behavior. When an Agent is used request will default to
  \texttt{Connection: keep-alive}. Possible values:
\item
  \texttt{undefined} (default): use
  \hyperref[http_http_globalagent]{global Agent} for this host and
  port.
\item
  \texttt{Agent} object: explicitly use the passed in \texttt{Agent}.
\item
  \texttt{false}: opts out of connection pooling with an Agent, defaults
  request to \texttt{Connection: close}.
\end{itemize}

\texttt{http.request()} returns an instance of the
\texttt{http.ClientRequest} class. The \texttt{ClientRequest} instance
is a writable stream. If one needs to upload a file with a POST request,
then write to the \texttt{ClientRequest} object.

Example:

\begin{Shaded}
\begin{Highlighting}[]
\KeywordTok{var} \NormalTok{options = \{}
  \DataTypeTok{host}\NormalTok{: }\CharTok{'www.google.com'}\NormalTok{,}
  \DataTypeTok{port}\NormalTok{: }\DecValTok{80}\NormalTok{,}
  \DataTypeTok{path}\NormalTok{: }\CharTok{'/upload'}\NormalTok{,}
  \DataTypeTok{method}\NormalTok{: }\CharTok{'POST'}
\NormalTok{\};}

\KeywordTok{var} \NormalTok{req = }\KeywordTok{http}\NormalTok{.}\FunctionTok{request}\NormalTok{(options, }\KeywordTok{function}\NormalTok{(res) \{}
  \KeywordTok{console}\NormalTok{.}\FunctionTok{log}\NormalTok{(}\CharTok{'STATUS: '} \NormalTok{+ }\KeywordTok{res}\NormalTok{.}\FunctionTok{statusCode}\NormalTok{);}
  \KeywordTok{console}\NormalTok{.}\FunctionTok{log}\NormalTok{(}\CharTok{'HEADERS: '} \NormalTok{+ }\KeywordTok{JSON}\NormalTok{.}\FunctionTok{stringify}\NormalTok{(}\KeywordTok{res}\NormalTok{.}\FunctionTok{headers}\NormalTok{));}
  \KeywordTok{res}\NormalTok{.}\FunctionTok{setEncoding}\NormalTok{(}\CharTok{'utf8'}\NormalTok{);}
  \KeywordTok{res}\NormalTok{.}\FunctionTok{on}\NormalTok{(}\CharTok{'data'}\NormalTok{, }\KeywordTok{function} \NormalTok{(chunk) \{}
    \KeywordTok{console}\NormalTok{.}\FunctionTok{log}\NormalTok{(}\CharTok{'BODY: '} \NormalTok{+ chunk);}
  \NormalTok{\});}
\NormalTok{\});}

\KeywordTok{req}\NormalTok{.}\FunctionTok{on}\NormalTok{(}\CharTok{'error'}\NormalTok{, }\KeywordTok{function}\NormalTok{(e) \{}
  \KeywordTok{console}\NormalTok{.}\FunctionTok{log}\NormalTok{(}\CharTok{'problem with request: '} \NormalTok{+ }\KeywordTok{e}\NormalTok{.}\FunctionTok{message}\NormalTok{);}
\NormalTok{\});}

\CommentTok{// write data to request body}
\KeywordTok{req}\NormalTok{.}\FunctionTok{write}\NormalTok{(}\CharTok{'data\textbackslash{}n'}\NormalTok{);}
\KeywordTok{req}\NormalTok{.}\FunctionTok{write}\NormalTok{(}\CharTok{'data\textbackslash{}n'}\NormalTok{);}
\KeywordTok{req}\NormalTok{.}\FunctionTok{end}\NormalTok{();}
\end{Highlighting}
\end{Shaded}

Note that in the example \texttt{req.end()} was called. With
\texttt{http.request()} one must always call \texttt{req.end()} to
signify that you're done with the request - even if there is no data
being written to the request body.

If any error is encountered during the request (be that with DNS
resolution, TCP level errors, or actual HTTP parse errors) an
\texttt{'error'} event is emitted on the returned request object.

There are a few special headers that should be noted.

\begin{itemize}
\item
  Sending a `Connection: keep-alive' will notify Node that the
  connection to the server should be persisted until the next request.
\item
  Sending a `Content-length' header will disable the default chunked
  encoding.
\item
  Sending an `Expect' header will immediately send the request headers.
  Usually, when sending `Expect: 100-continue', you should both set a
  timeout and listen for the \texttt{continue} event. See RFC2616
  Section 8.2.3 for more information.
\item
  Sending an Authorization header will override using the \texttt{auth}
  option to compute basic authentication.
\end{itemize}

\subsection{http.get(options, callback)}

Since most requests are GET requests without bodies, Node provides this
convenience method. The only difference between this method and
\texttt{http.request()} is that it sets the method to GET and calls
\texttt{req.end()} automatically.

Example:

\begin{Shaded}
\begin{Highlighting}[]
\KeywordTok{http}\NormalTok{.}\FunctionTok{get}\NormalTok{(}\StringTok{"http://www.google.com/index.html"}\NormalTok{, }\KeywordTok{function}\NormalTok{(res) \{}
  \KeywordTok{console}\NormalTok{.}\FunctionTok{log}\NormalTok{(}\StringTok{"Got response: "} \NormalTok{+ }\KeywordTok{res}\NormalTok{.}\FunctionTok{statusCode}\NormalTok{);}
\NormalTok{\}).}\FunctionTok{on}\NormalTok{(}\CharTok{'error'}\NormalTok{, }\KeywordTok{function}\NormalTok{(e) \{}
  \KeywordTok{console}\NormalTok{.}\FunctionTok{log}\NormalTok{(}\StringTok{"Got error: "} \NormalTok{+ }\KeywordTok{e}\NormalTok{.}\FunctionTok{message}\NormalTok{);}
\NormalTok{\});}
\end{Highlighting}
\end{Shaded}

\subsection{Class: http.Agent}

In node 0.5.3+ there is a new implementation of the HTTP Agent which is
used for pooling sockets used in HTTP client requests.

Previously, a single agent instance helped pool for a single host+port.
The current implementation now holds sockets for any number of hosts.

The current HTTP Agent also defaults client requests to using
Connection:keep-alive. If no pending HTTP requests are waiting on a
socket to become free the socket is closed. This means that node's pool
has the benefit of keep-alive when under load but still does not require
developers to manually close the HTTP clients using keep-alive.

Sockets are removed from the agent's pool when the socket emits either a
``close'' event or a special ``agentRemove'' event. This means that if
you intend to keep one HTTP request open for a long time and don't want
it to stay in the pool you can do something along the lines of:

\begin{Shaded}
\begin{Highlighting}[]
\KeywordTok{http}\NormalTok{.}\FunctionTok{get}\NormalTok{(options, }\KeywordTok{function}\NormalTok{(res) \{}
  \CommentTok{// Do stuff}
\NormalTok{\}).}\FunctionTok{on}\NormalTok{(}\StringTok{"socket"}\NormalTok{, }\KeywordTok{function} \NormalTok{(socket) \{}
  \KeywordTok{socket}\NormalTok{.}\FunctionTok{emit}\NormalTok{(}\StringTok{"agentRemove"}\NormalTok{);}
\NormalTok{\});}
\end{Highlighting}
\end{Shaded}

Alternatively, you could just opt out of pooling entirely using
\texttt{agent:false}:

\begin{Shaded}
\begin{Highlighting}[]
\KeywordTok{http}\NormalTok{.}\FunctionTok{get}\NormalTok{(\{}\DataTypeTok{host}\NormalTok{:}\CharTok{'localhost'}\NormalTok{, }\DataTypeTok{port}\NormalTok{:}\DecValTok{80}\NormalTok{, }\DataTypeTok{path}\NormalTok{:}\CharTok{'/'}\NormalTok{, }\DataTypeTok{agent}\NormalTok{:}\KeywordTok{false}\NormalTok{\}, }\KeywordTok{function} \NormalTok{(res) \{}
  \CommentTok{// Do stuff}
\NormalTok{\})}
\end{Highlighting}
\end{Shaded}

\subsubsection{agent.maxSockets}

By default set to 5. Determines how many concurrent sockets the agent
can have open per host.

\subsubsection{agent.sockets}

An object which contains arrays of sockets currently in use by the
Agent. Do not modify.

\subsubsection{agent.requests}

An object which contains queues of requests that have not yet been
assigned to sockets. Do not modify.

\subsection{http.globalAgent}

Global instance of Agent which is used as the default for all http
client requests.

\subsection{Class: http.ClientRequest}

This object is created internally and returned from
\texttt{http.request()}. It represents an \emph{in-progress} request
whose header has already been queued. The header is still mutable using
the \texttt{setHeader(name, value)}, \texttt{getHeader(name)},
\texttt{removeHeader(name)} API. The actual header will be sent along
with the first data chunk or when closing the connection.

To get the response, add a listener for \texttt{'response'} to the
request object. \texttt{'response'} will be emitted from the request
object when the response headers have been received. The
\texttt{'response'} event is executed with one argument which is an
instance of \texttt{http.ClientResponse}.

During the \texttt{'response'} event, one can add listeners to the
response object; particularly to listen for the \texttt{'data'} event.
Note that the \texttt{'response'} event is called before any part of the
response body is received, so there is no need to worry about racing to
catch the first part of the body. As long as a listener for
\texttt{'data'} is added during the \texttt{'response'} event, the
entire body will be caught.

\begin{Shaded}
\begin{Highlighting}[]
\CommentTok{// Good}
\KeywordTok{request}\NormalTok{.}\FunctionTok{on}\NormalTok{(}\CharTok{'response'}\NormalTok{, }\KeywordTok{function} \NormalTok{(response) \{}
  \KeywordTok{response}\NormalTok{.}\FunctionTok{on}\NormalTok{(}\CharTok{'data'}\NormalTok{, }\KeywordTok{function} \NormalTok{(chunk) \{}
    \KeywordTok{console}\NormalTok{.}\FunctionTok{log}\NormalTok{(}\CharTok{'BODY: '} \NormalTok{+ chunk);}
  \NormalTok{\});}
\NormalTok{\});}

\CommentTok{// Bad - misses all or part of the body}
\KeywordTok{request}\NormalTok{.}\FunctionTok{on}\NormalTok{(}\CharTok{'response'}\NormalTok{, }\KeywordTok{function} \NormalTok{(response) \{}
  \NormalTok{setTimeout(}\KeywordTok{function} \NormalTok{() \{}
    \KeywordTok{response}\NormalTok{.}\FunctionTok{on}\NormalTok{(}\CharTok{'data'}\NormalTok{, }\KeywordTok{function} \NormalTok{(chunk) \{}
      \KeywordTok{console}\NormalTok{.}\FunctionTok{log}\NormalTok{(}\CharTok{'BODY: '} \NormalTok{+ chunk);}
    \NormalTok{\});}
  \NormalTok{\}, }\DecValTok{10}\NormalTok{);}
\NormalTok{\});}
\end{Highlighting}
\end{Shaded}

Note: Node does not check whether Content-Length and the length of the
body which has been transmitted are equal or not.

The request implements the
\href{stream.html\#stream\_writable\_stream}{Writable Stream} interface.
This is an
\href{events.html\#events\_class\_events\_eventemitter}{EventEmitter}
with the following events:

\subsubsection{Event `response'}

\texttt{function (response) \{ \}}

Emitted when a response is received to this request. This event is
emitted only once. The \texttt{response} argument will be an instance of
\texttt{http.ClientResponse}.

Options:

\begin{itemize}
\item
  \texttt{host}: A domain name or IP address of the server to issue the
  request to.
\item
  \texttt{port}: Port of remote server.
\item
  \texttt{socketPath}: Unix Domain Socket (use one of host:port or
  socketPath)
\end{itemize}

\subsubsection{Event: `socket'}

\texttt{function (socket) \{ \}}

Emitted after a socket is assigned to this request.

\subsubsection{Event: `connect'}

\texttt{function (response, socket, head) \{ \}}

Emitted each time a server responds to a request with a CONNECT method.
If this event isn't being listened for, clients receiving a CONNECT
method will have their connections closed.

A client server pair that show you how to listen for the
\texttt{connect} event.

\begin{Shaded}
\begin{Highlighting}[]
\KeywordTok{var} \NormalTok{http = require(}\CharTok{'http'}\NormalTok{);}
\KeywordTok{var} \NormalTok{net = require(}\CharTok{'net'}\NormalTok{);}
\KeywordTok{var} \NormalTok{url = require(}\CharTok{'url'}\NormalTok{);}

\CommentTok{// Create an HTTP tunneling proxy}
\KeywordTok{var} \NormalTok{proxy = }\KeywordTok{http}\NormalTok{.}\FunctionTok{createServer}\NormalTok{(}\KeywordTok{function} \NormalTok{(req, res) \{}
  \KeywordTok{res}\NormalTok{.}\FunctionTok{writeHead}\NormalTok{(}\DecValTok{200}\NormalTok{, \{}\CharTok{'Content-Type'}\NormalTok{: }\CharTok{'text/plain'}\NormalTok{\});}
  \KeywordTok{res}\NormalTok{.}\FunctionTok{end}\NormalTok{(}\CharTok{'okay'}\NormalTok{);}
\NormalTok{\});}
\KeywordTok{proxy}\NormalTok{.}\FunctionTok{on}\NormalTok{(}\CharTok{'connect'}\NormalTok{, }\KeywordTok{function}\NormalTok{(req, cltSocket, head) \{}
  \CommentTok{// connect to an origin server}
  \KeywordTok{var} \NormalTok{srvUrl = }\KeywordTok{url}\NormalTok{.}\FunctionTok{parse}\NormalTok{(}\CharTok{'http://'} \NormalTok{+ }\KeywordTok{req}\NormalTok{.}\FunctionTok{url}\NormalTok{);}
  \KeywordTok{var} \NormalTok{srvSocket = }\KeywordTok{net}\NormalTok{.}\FunctionTok{connect}\NormalTok{(}\KeywordTok{srvUrl}\NormalTok{.}\FunctionTok{port}\NormalTok{, }\KeywordTok{srvUrl}\NormalTok{.}\FunctionTok{hostname}\NormalTok{, }\KeywordTok{function}\NormalTok{() \{}
    \KeywordTok{cltSocket}\NormalTok{.}\FunctionTok{write}\NormalTok{(}\CharTok{'HTTP/1.1 200 Connection Established\textbackslash{}r\textbackslash{}n'} \NormalTok{+}
                    \CharTok{'Proxy-agent: Node-Proxy\textbackslash{}r\textbackslash{}n'} \NormalTok{+}
                    \CharTok{'\textbackslash{}r\textbackslash{}n'}\NormalTok{);}
    \KeywordTok{srvSocket}\NormalTok{.}\FunctionTok{write}\NormalTok{(head);}
    \KeywordTok{srvSocket}\NormalTok{.}\FunctionTok{pipe}\NormalTok{(cltSocket);}
    \KeywordTok{cltSocket}\NormalTok{.}\FunctionTok{pipe}\NormalTok{(srvSocket);}
  \NormalTok{\});}
\NormalTok{\});}

\CommentTok{// now that proxy is running}
\KeywordTok{proxy}\NormalTok{.}\FunctionTok{listen}\NormalTok{(}\DecValTok{1337}\NormalTok{, }\CharTok{'127.0.0.1'}\NormalTok{, }\KeywordTok{function}\NormalTok{() \{}

  \CommentTok{// make a request to a tunneling proxy}
  \KeywordTok{var} \NormalTok{options = \{}
    \DataTypeTok{port}\NormalTok{: }\DecValTok{1337}\NormalTok{,}
    \DataTypeTok{host}\NormalTok{: }\CharTok{'127.0.0.1'}\NormalTok{,}
    \DataTypeTok{method}\NormalTok{: }\CharTok{'CONNECT'}\NormalTok{,}
    \DataTypeTok{path}\NormalTok{: }\CharTok{'www.google.com:80'}
  \NormalTok{\};}

  \KeywordTok{var} \NormalTok{req = }\KeywordTok{http}\NormalTok{.}\FunctionTok{request}\NormalTok{(options);}
  \KeywordTok{req}\NormalTok{.}\FunctionTok{end}\NormalTok{();}

  \KeywordTok{req}\NormalTok{.}\FunctionTok{on}\NormalTok{(}\CharTok{'connect'}\NormalTok{, }\KeywordTok{function}\NormalTok{(res, socket, head) \{}
    \KeywordTok{console}\NormalTok{.}\FunctionTok{log}\NormalTok{(}\CharTok{'got connected!'}\NormalTok{);}

    \CommentTok{// make a request over an HTTP tunnel}
    \KeywordTok{socket}\NormalTok{.}\FunctionTok{write}\NormalTok{(}\CharTok{'GET / HTTP/1.1\textbackslash{}r\textbackslash{}n'} \NormalTok{+}
                 \CharTok{'Host: www.google.com:80\textbackslash{}r\textbackslash{}n'} \NormalTok{+}
                 \CharTok{'Connection: close\textbackslash{}r\textbackslash{}n'} \NormalTok{+}
                 \CharTok{'\textbackslash{}r\textbackslash{}n'}\NormalTok{);}
    \KeywordTok{socket}\NormalTok{.}\FunctionTok{on}\NormalTok{(}\CharTok{'data'}\NormalTok{, }\KeywordTok{function}\NormalTok{(chunk) \{}
      \KeywordTok{console}\NormalTok{.}\FunctionTok{log}\NormalTok{(}\KeywordTok{chunk}\NormalTok{.}\FunctionTok{toString}\NormalTok{());}
    \NormalTok{\});}
    \KeywordTok{socket}\NormalTok{.}\FunctionTok{on}\NormalTok{(}\CharTok{'end'}\NormalTok{, }\KeywordTok{function}\NormalTok{() \{}
      \KeywordTok{proxy}\NormalTok{.}\FunctionTok{close}\NormalTok{();}
    \NormalTok{\});}
  \NormalTok{\});}
\NormalTok{\});}
\end{Highlighting}
\end{Shaded}

\subsubsection{Event: `upgrade'}

\texttt{function (response, socket, head) \{ \}}

Emitted each time a server responds to a request with an upgrade. If
this event isn't being listened for, clients receiving an upgrade header
will have their connections closed.

A client server pair that show you how to listen for the
\texttt{upgrade} event.

\begin{Shaded}
\begin{Highlighting}[]
\KeywordTok{var} \NormalTok{http = require(}\CharTok{'http'}\NormalTok{);}

\CommentTok{// Create an HTTP server}
\KeywordTok{var} \NormalTok{srv = }\KeywordTok{http}\NormalTok{.}\FunctionTok{createServer}\NormalTok{(}\KeywordTok{function} \NormalTok{(req, res) \{}
  \KeywordTok{res}\NormalTok{.}\FunctionTok{writeHead}\NormalTok{(}\DecValTok{200}\NormalTok{, \{}\CharTok{'Content-Type'}\NormalTok{: }\CharTok{'text/plain'}\NormalTok{\});}
  \KeywordTok{res}\NormalTok{.}\FunctionTok{end}\NormalTok{(}\CharTok{'okay'}\NormalTok{);}
\NormalTok{\});}
\KeywordTok{srv}\NormalTok{.}\FunctionTok{on}\NormalTok{(}\CharTok{'upgrade'}\NormalTok{, }\KeywordTok{function}\NormalTok{(req, socket, head) \{}
  \KeywordTok{socket}\NormalTok{.}\FunctionTok{write}\NormalTok{(}\CharTok{'HTTP/1.1 101 Web Socket Protocol Handshake\textbackslash{}r\textbackslash{}n'} \NormalTok{+}
               \CharTok{'Upgrade: WebSocket\textbackslash{}r\textbackslash{}n'} \NormalTok{+}
               \CharTok{'Connection: Upgrade\textbackslash{}r\textbackslash{}n'} \NormalTok{+}
               \CharTok{'\textbackslash{}r\textbackslash{}n'}\NormalTok{);}

  \KeywordTok{socket}\NormalTok{.}\FunctionTok{pipe}\NormalTok{(socket); }\CommentTok{// echo back}
\NormalTok{\});}

\CommentTok{// now that server is running}
\KeywordTok{srv}\NormalTok{.}\FunctionTok{listen}\NormalTok{(}\DecValTok{1337}\NormalTok{, }\CharTok{'127.0.0.1'}\NormalTok{, }\KeywordTok{function}\NormalTok{() \{}

  \CommentTok{// make a request}
  \KeywordTok{var} \NormalTok{options = \{}
    \DataTypeTok{port}\NormalTok{: }\DecValTok{1337}\NormalTok{,}
    \DataTypeTok{host}\NormalTok{: }\CharTok{'127.0.0.1'}\NormalTok{,}
    \DataTypeTok{headers}\NormalTok{: \{}
      \CharTok{'Connection'}\NormalTok{: }\CharTok{'Upgrade'}\NormalTok{,}
      \CharTok{'Upgrade'}\NormalTok{: }\CharTok{'websocket'}
    \NormalTok{\}}
  \NormalTok{\};}

  \KeywordTok{var} \NormalTok{req = }\KeywordTok{http}\NormalTok{.}\FunctionTok{request}\NormalTok{(options);}
  \KeywordTok{req}\NormalTok{.}\FunctionTok{end}\NormalTok{();}

  \KeywordTok{req}\NormalTok{.}\FunctionTok{on}\NormalTok{(}\CharTok{'upgrade'}\NormalTok{, }\KeywordTok{function}\NormalTok{(res, socket, upgradeHead) \{}
    \KeywordTok{console}\NormalTok{.}\FunctionTok{log}\NormalTok{(}\CharTok{'got upgraded!'}\NormalTok{);}
    \KeywordTok{socket}\NormalTok{.}\FunctionTok{end}\NormalTok{();}
    \KeywordTok{process}\NormalTok{.}\FunctionTok{exit}\NormalTok{(}\DecValTok{0}\NormalTok{);}
  \NormalTok{\});}
\NormalTok{\});}
\end{Highlighting}
\end{Shaded}

\subsubsection{Event: `continue'}

\texttt{function () \{ \}}

Emitted when the server sends a `100 Continue' HTTP response, usually
because the request contained `Expect: 100-continue'. This is an
instruction that the client should send the request body.

\subsubsection{request.write(chunk, {[}encoding{]})}

Sends a chunk of the body. By calling this method many times, the user
can stream a request body to a server--in that case it is suggested to
use the \texttt{{[}'Transfer-Encoding', 'chunked'{]}} header line when
creating the request.

The \texttt{chunk} argument should be a
\href{buffer.html\#buffer\_buffer}{Buffer} or a string.

The \texttt{encoding} argument is optional and only applies when
\texttt{chunk} is a string. Defaults to \texttt{'utf8'}.

\subsubsection{request.end({[}data{]}, {[}encoding{]})}

Finishes sending the request. If any parts of the body are unsent, it
will flush them to the stream. If the request is chunked, this will send
the terminating
\texttt{'0\textbackslash{}r\textbackslash{}n\textbackslash{}r\textbackslash{}n'}.

If \texttt{data} is specified, it is equivalent to calling
\texttt{request.write(data, encoding)} followed by
\texttt{request.end()}.

\subsubsection{request.abort()}

Aborts a request. (New since v0.3.8.)

\subsubsection{request.setTimeout(timeout, {[}callback{]})}

Once a socket is assigned to this request and is connected
\href{net.html\#net\_socket\_settimeout\_timeout\_callback}{socket.setTimeout()}
will be called.

\subsubsection{request.setNoDelay({[}noDelay{]})}

Once a socket is assigned to this request and is connected
\href{net.html\#net\_socket\_setnodelay\_nodelay}{socket.setNoDelay()}
will be called.

\subsubsection{request.setSocketKeepAlive({[}enable{]},
{[}initialDelay{]})}

Once a socket is assigned to this request and is connected
\href{net.html\#net\_socket\_setkeepalive\_enable\_initialdelay}{socket.setKeepAlive()}
will be called.

\subsection{http.ClientResponse}

This object is created when making a request with
\texttt{http.request()}. It is passed to the \texttt{'response'} event
of the request object.

The response implements the
\href{stream.html\#stream\_readable\_stream}{Readable Stream} interface.
This is an
\href{events.html\#events\_class\_events\_eventemitter}{EventEmitter}
with the following events:

\subsubsection{Event: `data'}

\texttt{function (chunk) \{ \}}

Emitted when a piece of the message body is received.

Note that the \textbf{data will be lost} if there is no listener when a
\texttt{ClientResponse} emits a \texttt{'data'} event.

\subsubsection{Event: `end'}

\texttt{function () \{ \}}

Emitted exactly once for each message. No arguments. After emitted no
other events will be emitted on the response.

\subsubsection{Event: `close'}

\texttt{function (err) \{ \}}

Indicates that the underlaying connection was terminated before
\texttt{end} event was emitted. See
\hyperref[http_class_http_serverrequest]{http.ServerRequest}'s
\texttt{'close'} event for more information.

\subsubsection{response.statusCode}

The 3-digit HTTP response status code. E.G. \texttt{404}.

\subsubsection{response.httpVersion}

The HTTP version of the connected-to server. Probably either
\texttt{'1.1'} or \texttt{'1.0'}. Also
\texttt{response.httpVersionMajor} is the first integer and
\texttt{response.httpVersionMinor} is the second.

\subsubsection{response.headers}

The response headers object.

\subsubsection{response.trailers}

The response trailers object. Only populated after the `end' event.

\subsubsection{response.setEncoding({[}encoding{]})}

Set the encoding for the response body. See
\href{stream.html\#stream\_stream\_setencoding\_encoding}{stream.setEncoding()}
for more information.

\subsubsection{response.pause()}

Pauses response from emitting events. Useful to throttle back a
download.

\subsubsection{response.resume()}

Resumes a paused response.
