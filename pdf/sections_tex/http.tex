\section{HTTP}\label{http}

\begin{Shaded}
\begin{Highlighting}[]
\NormalTok{Stability: }\DecValTok{3} \NormalTok{- Stable}
\end{Highlighting}
\end{Shaded}

To use the HTTP server and client one must \texttt{require('http')}.

The HTTP interfaces in Node are designed to support many features of the
protocol which have been traditionally difficult to use. In particular,
large, possibly chunk-encoded, messages. The interface is careful to
never buffer entire requests or responses--the user is able to stream
data.

HTTP message headers are represented by an object like this:

\begin{Shaded}
\begin{Highlighting}[]
\NormalTok{\{ }\StringTok{'content-length'}\NormalTok{: }\StringTok{'123'}\NormalTok{,}
  \StringTok{'content-type'}\NormalTok{: }\StringTok{'text/plain'}\NormalTok{,}
  \StringTok{'connection'}\NormalTok{: }\StringTok{'keep-alive'}\NormalTok{,}
  \StringTok{'host'}\NormalTok{: }\StringTok{'mysite.com'}\NormalTok{,}
  \StringTok{'accept'}\NormalTok{: }\StringTok{'*/*'} \NormalTok{\}}
\end{Highlighting}
\end{Shaded}

Keys are lowercased. Values are not modified.

In order to support the full spectrum of possible HTTP applications,
Node's HTTP API is very low-level. It deals with stream handling and
message parsing only. It parses a message into headers and body but it
does not parse the actual headers or the body.

Defined headers that allow multiple values are concatenated with a
\texttt{,} character, except for the \texttt{set-cookie} and
\texttt{cookie} headers which are represented as an array of values.
Headers such as \texttt{content-length} which can only have a single
value are parsed accordingly, and only a single value is represented on
the parsed object.

The raw headers as they were received are retained in the
\texttt{rawHeaders} property, which is an array of
\texttt{{[}key, value, key2, value2, ...{]}}. For example, the previous
message header object might have a \texttt{rawHeaders} list like the
following:

\begin{Shaded}
\begin{Highlighting}[]
\NormalTok{[ }\StringTok{'ConTent-Length'}\NormalTok{, }\StringTok{'123456'}\NormalTok{,}
  \StringTok{'content-LENGTH'}\NormalTok{, }\StringTok{'123'}\NormalTok{,}
  \StringTok{'content-type'}\NormalTok{, }\StringTok{'text/plain'}\NormalTok{,}
  \StringTok{'CONNECTION'}\NormalTok{, }\StringTok{'keep-alive'}\NormalTok{,}
  \StringTok{'Host'}\NormalTok{, }\StringTok{'mysite.com'}\NormalTok{,}
  \StringTok{'accepT'}\NormalTok{, }\StringTok{'*/*'} \NormalTok{]}
\end{Highlighting}
\end{Shaded}

\subsection{http.METHODS}\label{http.methods}

\begin{itemize}
\itemsep1pt\parskip0pt\parsep0pt
\item
  \{Array\}
\end{itemize}

A list of the HTTP methods that are supported by the parser.

\subsection{http.STATUS\_CODES}\label{http.statusux5fcodes}

\begin{itemize}
\itemsep1pt\parskip0pt\parsep0pt
\item
  \{Object\}
\end{itemize}

A collection of all the standard HTTP response status codes, and the
short description of each. For example,
\texttt{http.STATUS\_CODES{[}404{]} === 'Not Found'}.

\subsection{http.createServer({[}requestListener{]})}\label{http.createserverrequestlistener}

Returns a new web server object.

The \texttt{requestListener} is a function which is automatically added
to the \texttt{'request'} event.

\subsection{http.createClient({[}port{]},
{[}host{]})}\label{http.createclientport-host}

This function is \textbf{deprecated}; please use
\hyperref[httpux5fhttpux5frequestux5foptionsux5fcallback]{http.request()}
instead. Constructs a new HTTP client. \texttt{port} and \texttt{host}
refer to the server to be connected to.

\subsection{Class: http.Server}\label{class-http.server}

This is an
\href{events.html\#events_class_events_eventemitter}{EventEmitter} with
the following events:

\subsubsection{Event: `request'}\label{event-request}

\texttt{function (request, response) \{ \}}

Emitted each time there is a request. Note that there may be multiple
requests per connection (in the case of keep-alive connections).
\texttt{request} is an instance of
\hyperref[httpux5fhttpux5fincomingmessage]{http.IncomingMessage} and
\texttt{response} is an instance of
\hyperref[httpux5fclassux5fhttpux5fserverresponse]{http.ServerResponse}.

\subsubsection{Event: `connection'}\label{event-connection}

\texttt{function (socket) \{ \}}

When a new TCP stream is established. \texttt{socket} is an object of
type \texttt{net.Socket}. Usually users will not want to access this
event. In particular, the socket will not emit \texttt{readable} events
because of how the protocol parser attaches to the socket. The
\texttt{socket} can also be accessed at \texttt{request.connection}.

\subsubsection{Event: `close'}\label{event-close}

\texttt{function () \{ \}}

Emitted when the server closes.

\subsubsection{Event: `checkContinue'}\label{event-checkcontinue}

\texttt{function (request, response) \{ \}}

Emitted each time a request with an http Expect: 100-continue is
received. If this event isn't listened for, the server will
automatically respond with a 100 Continue as appropriate.

Handling this event involves calling
\hyperref[httpux5fresponseux5fwritecontinue]{response.writeContinue()}
if the client should continue to send the request body, or generating an
appropriate HTTP response (e.g., 400 Bad Request) if the client should
not continue to send the request body.

Note that when this event is emitted and handled, the \texttt{request}
event will not be emitted.

\subsubsection{Event: `connect'}\label{event-connect}

\texttt{function (request, socket, head) \{ \}}

Emitted each time a client requests a http CONNECT method. If this event
isn't listened for, then clients requesting a CONNECT method will have
their connections closed.

\begin{itemize}
\itemsep1pt\parskip0pt\parsep0pt
\item
  \texttt{request} is the arguments for the http request, as it is in
  the request event.
\item
  \texttt{socket} is the network socket between the server and client.
\item
  \texttt{head} is an instance of Buffer, the first packet of the
  tunneling stream, this may be empty.
\end{itemize}

After this event is emitted, the request's socket will not have a
\texttt{data} event listener, meaning you will need to bind to it in
order to handle data sent to the server on that socket.

\subsubsection{Event: `upgrade'}\label{event-upgrade}

\texttt{function (request, socket, head) \{ \}}

Emitted each time a client requests a http upgrade. If this event isn't
listened for, then clients requesting an upgrade will have their
connections closed.

\begin{itemize}
\itemsep1pt\parskip0pt\parsep0pt
\item
  \texttt{request} is the arguments for the http request, as it is in
  the request event.
\item
  \texttt{socket} is the network socket between the server and client.
\item
  \texttt{head} is an instance of Buffer, the first packet of the
  upgraded stream, this may be empty.
\end{itemize}

After this event is emitted, the request's socket will not have a
\texttt{data} event listener, meaning you will need to bind to it in
order to handle data sent to the server on that socket.

\subsubsection{Event: `clientError'}\label{event-clienterror}

\texttt{function (exception, socket) \{ \}}

If a client connection emits an `error' event - it will forwarded here.

\texttt{socket} is the \texttt{net.Socket} object that the error
originated from.

\subsubsection{server.listen(port, {[}hostname{]}, {[}backlog{]},
{[}callback{]})}\label{server.listenport-hostname-backlog-callback}

Begin accepting connections on the specified port and hostname. If the
hostname is omitted, the server will accept connections directed to any
IPv4 address (\texttt{INADDR\_ANY}).

To listen to a unix socket, supply a filename instead of port and
hostname.

Backlog is the maximum length of the queue of pending connections. The
actual length will be determined by your OS through sysctl settings such
as \texttt{tcp\_max\_syn\_backlog} and \texttt{somaxconn} on linux. The
default value of this parameter is 511 (not 512).

This function is asynchronous. The last parameter \texttt{callback} will
be added as a listener for the
\href{net.html\#net_event_listening}{`listening'} event. See also
\href{net.html\#net_server_listen_port_host_backlog_callback}{net.Server.listen(port)}.

\subsubsection{server.listen(path,
{[}callback{]})}\label{server.listenpath-callback}

Start a UNIX socket server listening for connections on the given
\texttt{path}.

This function is asynchronous. The last parameter \texttt{callback} will
be added as a listener for the
\href{net.html\#net_event_listening}{`listening'} event. See also
\href{net.html\#net_server_listen_path_callback}{net.Server.listen(path)}.

\subsubsection{server.listen(handle,
{[}callback{]})}\label{server.listenhandle-callback}

\begin{itemize}
\itemsep1pt\parskip0pt\parsep0pt
\item
  \texttt{handle} \{Object\}
\item
  \texttt{callback} \{Function\}
\end{itemize}

The \texttt{handle} object can be set to either a server or socket
(anything with an underlying \texttt{\_handle} member), or a
\texttt{\{fd: \textless{}n\textgreater{}\}} object.

This will cause the server to accept connections on the specified
handle, but it is presumed that the file descriptor or handle has
already been bound to a port or domain socket.

Listening on a file descriptor is not supported on Windows.

This function is asynchronous. The last parameter \texttt{callback} will
be added as a listener for the
\href{net.html\#event_listening_}{`listening'} event. See also
\href{net.html\#net_server_listen_handle_callback}{net.Server.listen()}.

\subsubsection{server.close({[}callback{]})}\label{server.closecallback}

Stops the server from accepting new connections. See
\href{net.html\#net_server_close_callback}{net.Server.close()}.

\subsubsection{server.maxHeadersCount}\label{server.maxheaderscount}

Limits maximum incoming headers count, equal to 1000 by default. If set
to 0 - no limit will be applied.

\subsubsection{server.setTimeout(msecs,
callback)}\label{server.settimeoutmsecs-callback}

\begin{itemize}
\itemsep1pt\parskip0pt\parsep0pt
\item
  \texttt{msecs} \{Number\}
\item
  \texttt{callback} \{Function\}
\end{itemize}

Sets the timeout value for sockets, and emits a \texttt{'timeout'} event
on the Server object, passing the socket as an argument, if a timeout
occurs.

If there is a \texttt{'timeout'} event listener on the Server object,
then it will be called with the timed-out socket as an argument.

By default, the Server's timeout value is 2 minutes, and sockets are
destroyed automatically if they time out. However, if you assign a
callback to the Server's \texttt{'timeout'} event, then you are
responsible for handling socket timeouts.

\subsubsection{server.timeout}\label{server.timeout}

\begin{itemize}
\itemsep1pt\parskip0pt\parsep0pt
\item
  \{Number\} Default = 120000 (2 minutes)
\end{itemize}

The number of milliseconds of inactivity before a socket is presumed to
have timed out.

Note that the socket timeout logic is set up on connection, so changing
this value only affects \emph{new} connections to the server, not any
existing connections.

Set to 0 to disable any kind of automatic timeout behavior on incoming
connections.

\subsection{Class: http.ServerResponse}\label{class-http.serverresponse}

This object is created internally by a HTTP server--not by the user. It
is passed as the second parameter to the \texttt{'request'} event.

The response implements the
\href{stream.html\#stream_class_stream_writable}{Writable Stream}
interface. This is an
\href{events.html\#events_class_events_eventemitter}{EventEmitter} with
the following events:

\subsubsection{Event: `close'}\label{event-close-1}

\texttt{function () \{ \}}

Indicates that the underlying connection was terminated before
\hyperref[httpux5fresponseux5fendux5fdataux5fencoding]{response.end()}
was called or able to flush.

\subsubsection{Event: `finish'}\label{event-finish}

\texttt{function () \{ \}}

Emitted when the response has been sent. More specifically, this event
is emitted when the last segment of the response headers and body have
been handed off to the operating system for transmission over the
network. It does not imply that the client has received anything yet.

After this event, no more events will be emitted on the response object.

\subsubsection{response.writeContinue()}\label{response.writecontinue}

Sends a HTTP/1.1 100 Continue message to the client, indicating that the
request body should be sent. See the
\hyperref[httpux5feventux5fcheckcontinue]{`checkContinue'} event on
\texttt{Server}.

\subsubsection{response.writeHead(statusCode, {[}statusMessage{]},
{[}headers{]})}\label{response.writeheadstatuscode-statusmessage-headers}

Sends a response header to the request. The status code is a 3-digit
HTTP status code, like \texttt{404}. The last argument,
\texttt{headers}, are the response headers. Optionally one can give a
human-readable \texttt{statusMessage} as the second argument.

Example:

\begin{Shaded}
\begin{Highlighting}[]
\KeywordTok{var} \NormalTok{body = }\StringTok{'hello world'}\NormalTok{;}
\OtherTok{response}\NormalTok{.}\FunctionTok{writeHead}\NormalTok{(}\DecValTok{200}\NormalTok{, \{}
  \StringTok{'Content-Length'}\NormalTok{: }\OtherTok{body}\NormalTok{.}\FunctionTok{length}\NormalTok{,}
  \StringTok{'Content-Type'}\NormalTok{: }\StringTok{'text/plain'} \NormalTok{\});}
\end{Highlighting}
\end{Shaded}

This method must only be called once on a message and it must be called
before
\hyperref[httpux5fresponseux5fendux5fdataux5fencoding]{response.end()}
is called.

If you call
\hyperref[httpux5fresponseux5fwriteux5fchunkux5fencoding]{response.write()}
or
\hyperref[httpux5fresponseux5fendux5fdataux5fencoding]{response.end()}
before calling this, the implicit/mutable headers will be calculated and
call this function for you.

Note: that Content-Length is given in bytes not characters. The above
example works because the string \texttt{'hello world'} contains only
single byte characters. If the body contains higher coded characters
then \texttt{Buffer.byteLength()} should be used to determine the number
of bytes in a given encoding. And Node does not check whether
Content-Length and the length of the body which has been transmitted are
equal or not.

\subsubsection{response.setTimeout(msecs,
callback)}\label{response.settimeoutmsecs-callback}

\begin{itemize}
\itemsep1pt\parskip0pt\parsep0pt
\item
  \texttt{msecs} \{Number\}
\item
  \texttt{callback} \{Function\}
\end{itemize}

Sets the Socket's timeout value to \texttt{msecs}. If a callback is
provided, then it is added as a listener on the \texttt{'timeout'} event
on the response object.

If no \texttt{'timeout'} listener is added to the request, the response,
or the server, then sockets are destroyed when they time out. If you
assign a handler on the request, the response, or the server's
\texttt{'timeout'} events, then it is your responsibility to handle
timed out sockets.

\subsubsection{response.statusCode}\label{response.statuscode}

When using implicit headers (not calling
\hyperref[httpux5fresponseux5fwriteheadux5fstatuscodeux5freasonphraseux5fheaders]{response.writeHead()}
explicitly), this property controls the status code that will be sent to
the client when the headers get flushed.

Example:

\begin{Shaded}
\begin{Highlighting}[]
\OtherTok{response}\NormalTok{.}\FunctionTok{statusCode} \NormalTok{= }\DecValTok{404}\NormalTok{;}
\end{Highlighting}
\end{Shaded}

After response header was sent to the client, this property indicates
the status code which was sent out.

\subsubsection{response.statusMessage}\label{response.statusmessage}

When using implicit headers (not calling \texttt{response.writeHead()}
explicitly), this property controls the status message that will be sent
to the client when the headers get flushed. If this is left as
\texttt{undefined} then the standard message for the status code will be
used.

Example:

\begin{Shaded}
\begin{Highlighting}[]
\OtherTok{response}\NormalTok{.}\FunctionTok{statusMessage} \NormalTok{= }\StringTok{'Not found'}\NormalTok{;}
\end{Highlighting}
\end{Shaded}

After response header was sent to the client, this property indicates
the status message which was sent out.

\subsubsection{response.setHeader(name,
value)}\label{response.setheadername-value}

Sets a single header value for implicit headers. If this header already
exists in the to-be-sent headers, its value will be replaced. Use an
array of strings here if you need to send multiple headers with the same
name.

Example:

\begin{Shaded}
\begin{Highlighting}[]
\OtherTok{response}\NormalTok{.}\FunctionTok{setHeader}\NormalTok{(}\StringTok{"Content-Type"}\NormalTok{, }\StringTok{"text/html"}\NormalTok{);}
\end{Highlighting}
\end{Shaded}

or

\begin{Shaded}
\begin{Highlighting}[]
\OtherTok{response}\NormalTok{.}\FunctionTok{setHeader}\NormalTok{(}\StringTok{"Set-Cookie"}\NormalTok{, [}\StringTok{"type=ninja"}\NormalTok{, }\StringTok{"language=javascript"}\NormalTok{]);}
\end{Highlighting}
\end{Shaded}

\subsubsection{response.headersSent}\label{response.headerssent}

Boolean (read-only). True if headers were sent, false otherwise.

\subsubsection{response.sendDate}\label{response.senddate}

When true, the Date header will be automatically generated and sent in
the response if it is not already present in the headers. Defaults to
true.

This should only be disabled for testing; HTTP requires the Date header
in responses.

\subsubsection{response.getHeader(name)}\label{response.getheadername}

Reads out a header that's already been queued but not sent to the
client. Note that the name is case insensitive. This can only be called
before headers get implicitly flushed.

Example:

\begin{Shaded}
\begin{Highlighting}[]
\KeywordTok{var} \NormalTok{contentType = }\OtherTok{response}\NormalTok{.}\FunctionTok{getHeader}\NormalTok{(}\StringTok{'content-type'}\NormalTok{);}
\end{Highlighting}
\end{Shaded}

\subsubsection{response.removeHeader(name)}\label{response.removeheadername}

Removes a header that's queued for implicit sending.

Example:

\begin{Shaded}
\begin{Highlighting}[]
\OtherTok{response}\NormalTok{.}\FunctionTok{removeHeader}\NormalTok{(}\StringTok{"Content-Encoding"}\NormalTok{);}
\end{Highlighting}
\end{Shaded}

\subsubsection{response.write(chunk,
{[}encoding{]})}\label{response.writechunk-encoding}

If this method is called and
\hyperref[httpux5fresponseux5fwriteheadux5fstatuscodeux5freasonphraseux5fheaders]{response.writeHead()}
has not been called, it will switch to implicit header mode and flush
the implicit headers.

This sends a chunk of the response body. This method may be called
multiple times to provide successive parts of the body.

\texttt{chunk} can be a string or a buffer. If \texttt{chunk} is a
string, the second parameter specifies how to encode it into a byte
stream. By default the \texttt{encoding} is \texttt{'utf8'}.

\textbf{Note}: This is the raw HTTP body and has nothing to do with
higher-level multi-part body encodings that may be used.

The first time \texttt{response.write()} is called, it will send the
buffered header information and the first body to the client. The second
time \texttt{response.write()} is called, Node assumes you're going to
be streaming data, and sends that separately. That is, the response is
buffered up to the first chunk of body.

Returns \texttt{true} if the entire data was flushed successfully to the
kernel buffer. Returns \texttt{false} if all or part of the data was
queued in user memory. \texttt{'drain'} will be emitted when the buffer
is again free.

\subsubsection{response.addTrailers(headers)}\label{response.addtrailersheaders}

This method adds HTTP trailing headers (a header but at the end of the
message) to the response.

Trailers will \textbf{only} be emitted if chunked encoding is used for
the response; if it is not (e.g., if the request was HTTP/1.0), they
will be silently discarded.

Note that HTTP requires the \texttt{Trailer} header to be sent if you
intend to emit trailers, with a list of the header fields in its value.
E.g.,

\begin{Shaded}
\begin{Highlighting}[]
\OtherTok{response}\NormalTok{.}\FunctionTok{writeHead}\NormalTok{(}\DecValTok{200}\NormalTok{, \{ }\StringTok{'Content-Type'}\NormalTok{: }\StringTok{'text/plain'}\NormalTok{,}
                          \StringTok{'Trailer'}\NormalTok{: }\StringTok{'Content-MD5'} \NormalTok{\});}
\OtherTok{response}\NormalTok{.}\FunctionTok{write}\NormalTok{(fileData);}
\OtherTok{response}\NormalTok{.}\FunctionTok{addTrailers}\NormalTok{(\{}\StringTok{'Content-MD5'}\NormalTok{: }\StringTok{"7895bf4b8828b55ceaf47747b4bca667"}\NormalTok{\});}
\OtherTok{response}\NormalTok{.}\FunctionTok{end}\NormalTok{();}
\end{Highlighting}
\end{Shaded}

\subsubsection{response.end({[}data{]},
{[}encoding{]})}\label{response.enddata-encoding}

This method signals to the server that all of the response headers and
body have been sent; that server should consider this message complete.
The method, \texttt{response.end()}, MUST be called on each response.

If \texttt{data} is specified, it is equivalent to calling
\texttt{response.write(data, encoding)} followed by
\texttt{response.end()}.

\subsection{http.request(options,
{[}callback{]})}\label{http.requestoptions-callback}

Node maintains several connections per server to make HTTP requests.
This function allows one to transparently issue requests.

\texttt{options} can be an object or a string. If \texttt{options} is a
string, it is automatically parsed with
\href{url.html\#url_url_parse_urlstr_parsequerystring_slashesdenotehost}{url.parse()}.

Options:

\begin{itemize}
\itemsep1pt\parskip0pt\parsep0pt
\item
  \texttt{host}: A domain name or IP address of the server to issue the
  request to. Defaults to \texttt{'localhost'}.
\item
  \texttt{hostname}: To support \texttt{url.parse()} \texttt{hostname}
  is preferred over \texttt{host}
\item
  \texttt{port}: Port of remote server. Defaults to 80.
\item
  \texttt{localAddress}: Local interface to bind for network
  connections.
\item
  \texttt{socketPath}: Unix Domain Socket (use one of host:port or
  socketPath)
\item
  \texttt{method}: A string specifying the HTTP request method. Defaults
  to \texttt{'GET'}.
\item
  \texttt{path}: Request path. Defaults to \texttt{'/'}. Should include
  query string if any. E.G. \texttt{'/index.html?page=12'}. An exception
  is thrown when the request path contains illegal characters.
  Currently, only spaces are rejected but that may change in the future.
\item
  \texttt{headers}: An object containing request headers.
\item
  \texttt{auth}: Basic authentication i.e. \texttt{'user:password'} to
  compute an Authorization header.
\item
  \texttt{agent}: Controls
  \hyperref[httpux5fclassux5fhttpux5fagent]{Agent} behavior. When an
  Agent is used request will default to \texttt{Connection: keep-alive}.
  Possible values:
\item
  \texttt{undefined} (default): use
  \hyperref[httpux5fhttpux5fglobalagent]{global Agent} for this host and
  port.
\item
  \texttt{Agent} object: explicitly use the passed in \texttt{Agent}.
\item
  \texttt{false}: opts out of connection pooling with an Agent, defaults
  request to \texttt{Connection: close}.
\item
  \texttt{keepAlive}: \{Boolean\} Keep sockets around in a pool to be
  used by other requests in the future. Default = \texttt{false}
\item
  \texttt{keepAliveMsecs}: \{Integer\} When using HTTP KeepAlive, how
  often to send TCP KeepAlive packets over sockets being kept alive.
  Default = \texttt{1000}. Only relevant if \texttt{keepAlive} is set to
  \texttt{true}.
\end{itemize}

The optional \texttt{callback} parameter will be added as a one time
listener for the \hyperref[httpux5feventux5fresponse]{`response'} event.

\texttt{http.request()} returns an instance of the
\hyperref[httpux5fclassux5fhttpux5fclientrequest]{http.ClientRequest}
class. The \texttt{ClientRequest} instance is a writable stream. If one
needs to upload a file with a POST request, then write to the
\texttt{ClientRequest} object.

Example:

\begin{Shaded}
\begin{Highlighting}[]
\KeywordTok{var} \NormalTok{options = \{}
  \DataTypeTok{hostname}\NormalTok{: }\StringTok{'www.google.com'}\NormalTok{,}
  \DataTypeTok{port}\NormalTok{: }\DecValTok{80}\NormalTok{,}
  \DataTypeTok{path}\NormalTok{: }\StringTok{'/upload'}\NormalTok{,}
  \DataTypeTok{method}\NormalTok{: }\StringTok{'POST'}
\NormalTok{\};}

\KeywordTok{var} \NormalTok{req = }\OtherTok{http}\NormalTok{.}\FunctionTok{request}\NormalTok{(options, }\KeywordTok{function}\NormalTok{(res) \{}
  \OtherTok{console}\NormalTok{.}\FunctionTok{log}\NormalTok{(}\StringTok{'STATUS: '} \NormalTok{+ }\OtherTok{res}\NormalTok{.}\FunctionTok{statusCode}\NormalTok{);}
  \OtherTok{console}\NormalTok{.}\FunctionTok{log}\NormalTok{(}\StringTok{'HEADERS: '} \NormalTok{+ }\OtherTok{JSON}\NormalTok{.}\FunctionTok{stringify}\NormalTok{(}\OtherTok{res}\NormalTok{.}\FunctionTok{headers}\NormalTok{));}
  \OtherTok{res}\NormalTok{.}\FunctionTok{setEncoding}\NormalTok{(}\StringTok{'utf8'}\NormalTok{);}
  \OtherTok{res}\NormalTok{.}\FunctionTok{on}\NormalTok{(}\StringTok{'data'}\NormalTok{, }\KeywordTok{function} \NormalTok{(chunk) \{}
    \OtherTok{console}\NormalTok{.}\FunctionTok{log}\NormalTok{(}\StringTok{'BODY: '} \NormalTok{+ chunk);}
  \NormalTok{\});}
\NormalTok{\});}

\OtherTok{req}\NormalTok{.}\FunctionTok{on}\NormalTok{(}\StringTok{'error'}\NormalTok{, }\KeywordTok{function}\NormalTok{(e) \{}
  \OtherTok{console}\NormalTok{.}\FunctionTok{log}\NormalTok{(}\StringTok{'problem with request: '} \NormalTok{+ }\OtherTok{e}\NormalTok{.}\FunctionTok{message}\NormalTok{);}
\NormalTok{\});}

\CommentTok{// write data to request body}
\OtherTok{req}\NormalTok{.}\FunctionTok{write}\NormalTok{(}\StringTok{'data}\CharTok{\textbackslash{}n}\StringTok{'}\NormalTok{);}
\OtherTok{req}\NormalTok{.}\FunctionTok{write}\NormalTok{(}\StringTok{'data}\CharTok{\textbackslash{}n}\StringTok{'}\NormalTok{);}
\OtherTok{req}\NormalTok{.}\FunctionTok{end}\NormalTok{();}
\end{Highlighting}
\end{Shaded}

Note that in the example \texttt{req.end()} was called. With
\texttt{http.request()} one must always call \texttt{req.end()} to
signify that you're done with the request - even if there is no data
being written to the request body.

If any error is encountered during the request (be that with DNS
resolution, TCP level errors, or actual HTTP parse errors) an
\texttt{'error'} event is emitted on the returned request object.

There are a few special headers that should be noted.

\begin{itemize}
\item
  Sending a `Connection: keep-alive' will notify Node that the
  connection to the server should be persisted until the next request.
\item
  Sending a `Content-length' header will disable the default chunked
  encoding.
\item
  Sending an `Expect' header will immediately send the request headers.
  Usually, when sending `Expect: 100-continue', you should both set a
  timeout and listen for the \texttt{continue} event. See RFC2616
  Section 8.2.3 for more information.
\item
  Sending an Authorization header will override using the \texttt{auth}
  option to compute basic authentication.
\end{itemize}

\subsection{http.get(options,
{[}callback{]})}\label{http.getoptions-callback}

Since most requests are GET requests without bodies, Node provides this
convenience method. The only difference between this method and
\texttt{http.request()} is that it sets the method to GET and calls
\texttt{req.end()} automatically.

Example:

\begin{Shaded}
\begin{Highlighting}[]
\OtherTok{http}\NormalTok{.}\FunctionTok{get}\NormalTok{(}\StringTok{"http://www.google.com/index.html"}\NormalTok{, }\KeywordTok{function}\NormalTok{(res) \{}
  \OtherTok{console}\NormalTok{.}\FunctionTok{log}\NormalTok{(}\StringTok{"Got response: "} \NormalTok{+ }\OtherTok{res}\NormalTok{.}\FunctionTok{statusCode}\NormalTok{);}
\NormalTok{\}).}\FunctionTok{on}\NormalTok{(}\StringTok{'error'}\NormalTok{, }\KeywordTok{function}\NormalTok{(e) \{}
  \OtherTok{console}\NormalTok{.}\FunctionTok{log}\NormalTok{(}\StringTok{"Got error: "} \NormalTok{+ }\OtherTok{e}\NormalTok{.}\FunctionTok{message}\NormalTok{);}
\NormalTok{\});}
\end{Highlighting}
\end{Shaded}

\subsection{Class: http.Agent}\label{class-http.agent}

The HTTP Agent is used for pooling sockets used in HTTP client requests.

The HTTP Agent also defaults client requests to using
Connection:keep-alive. If no pending HTTP requests are waiting on a
socket to become free the socket is closed. This means that Node's pool
has the benefit of keep-alive when under load but still does not require
developers to manually close the HTTP clients using KeepAlive.

If you opt into using HTTP KeepAlive, you can create an Agent object
with that flag set to \texttt{true}. (See the
\hyperref[httpux5fnewux5fagentux5foptions]{constructor options} below.)
Then, the Agent will keep unused sockets in a pool for later use. They
will be explicitly marked so as to not keep the Node process running.
However, it is still a good idea to explicitly
\hyperref[httpux5fagentux5fdestroy]{\texttt{destroy()}} KeepAlive agents
when they are no longer in use, so that the Sockets will be shut down.

Sockets are removed from the agent's pool when the socket emits either a
``close'' event or a special ``agentRemove'' event. This means that if
you intend to keep one HTTP request open for a long time and don't want
it to stay in the pool you can do something along the lines of:

\begin{Shaded}
\begin{Highlighting}[]
\OtherTok{http}\NormalTok{.}\FunctionTok{get}\NormalTok{(options, }\KeywordTok{function}\NormalTok{(res) \{}
  \CommentTok{// Do stuff}
\NormalTok{\}).}\FunctionTok{on}\NormalTok{(}\StringTok{"socket"}\NormalTok{, }\KeywordTok{function} \NormalTok{(socket) \{}
  \OtherTok{socket}\NormalTok{.}\FunctionTok{emit}\NormalTok{(}\StringTok{"agentRemove"}\NormalTok{);}
\NormalTok{\});}
\end{Highlighting}
\end{Shaded}

Alternatively, you could just opt out of pooling entirely using
\texttt{agent:false}:

\begin{Shaded}
\begin{Highlighting}[]
\OtherTok{http}\NormalTok{.}\FunctionTok{get}\NormalTok{(\{}
  \DataTypeTok{hostname}\NormalTok{: }\StringTok{'localhost'}\NormalTok{,}
  \DataTypeTok{port}\NormalTok{: }\DecValTok{80}\NormalTok{,}
  \DataTypeTok{path}\NormalTok{: }\StringTok{'/'}\NormalTok{,}
  \DataTypeTok{agent}\NormalTok{: }\KeywordTok{false}  \CommentTok{// create a new agent just for this one request}
\NormalTok{\}, }\KeywordTok{function} \NormalTok{(res) \{}
  \CommentTok{// Do stuff with response}
\NormalTok{\})}
\end{Highlighting}
\end{Shaded}

\subsubsection{new Agent({[}options{]})}\label{new-agentoptions}

\begin{itemize}
\itemsep1pt\parskip0pt\parsep0pt
\item
  \texttt{options} \{Object\} Set of configurable options to set on the
  agent. Can have the following fields:
\item
  \texttt{keepAlive} \{Boolean\} Keep sockets around in a pool to be
  used by other requests in the future. Default = \texttt{false}
\item
  \texttt{keepAliveMsecs} \{Integer\} When using HTTP KeepAlive, how
  often to send TCP KeepAlive packets over sockets being kept alive.
  Default = \texttt{1000}. Only relevant if \texttt{keepAlive} is set to
  \texttt{true}.
\item
  \texttt{maxSockets} \{Number\} Maximum number of sockets to allow per
  host. Default = \texttt{Infinity}.
\item
  \texttt{maxFreeSockets} \{Number\} Maximum number of sockets to leave
  open in a free state. Only relevant if \texttt{keepAlive} is set to
  \texttt{true}. Default = \texttt{256}.
\end{itemize}

The default \texttt{http.globalAgent} that is used by
\texttt{http.request} has all of these values set to their respective
defaults.

To configure any of them, you must create your own \texttt{Agent}
object.

\begin{Shaded}
\begin{Highlighting}[]
\KeywordTok{var} \NormalTok{http = }\FunctionTok{require}\NormalTok{(}\StringTok{'http'}\NormalTok{);}
\KeywordTok{var} \NormalTok{keepAliveAgent = }\KeywordTok{new} \OtherTok{http}\NormalTok{.}\FunctionTok{Agent}\NormalTok{(\{ }\DataTypeTok{keepAlive}\NormalTok{: }\KeywordTok{true} \NormalTok{\});}
\OtherTok{options}\NormalTok{.}\FunctionTok{agent} \NormalTok{= keepAliveAgent;}
\OtherTok{http}\NormalTok{.}\FunctionTok{request}\NormalTok{(options, onResponseCallback);}
\end{Highlighting}
\end{Shaded}

\subsubsection{agent.maxSockets}\label{agent.maxsockets}

By default set to Infinity. Determines how many concurrent sockets the
agent can have open per origin. Origin is either a `host:port' or
`host:port:localAddress' combination.

\subsubsection{agent.maxFreeSockets}\label{agent.maxfreesockets}

By default set to 256. For Agents supporting HTTP KeepAlive, this sets
the maximum number of sockets that will be left open in the free state.

\subsubsection{agent.sockets}\label{agent.sockets}

An object which contains arrays of sockets currently in use by the
Agent. Do not modify.

\subsubsection{agent.freeSockets}\label{agent.freesockets}

An object which contains arrays of sockets currently awaiting use by the
Agent when HTTP KeepAlive is used. Do not modify.

\subsubsection{agent.requests}\label{agent.requests}

An object which contains queues of requests that have not yet been
assigned to sockets. Do not modify.

\subsubsection{agent.destroy()}\label{agent.destroy}

Destroy any sockets that are currently in use by the agent.

It is usually not necessary to do this. However, if you are using an
agent with KeepAlive enabled, then it is best to explicitly shut down
the agent when you know that it will no longer be used. Otherwise,
sockets may hang open for quite a long time before the server terminates
them.

\subsubsection{agent.getName(options)}\label{agent.getnameoptions}

Get a unique name for a set of request options, to determine whether a
connection can be reused. In the http agent, this returns
\texttt{host:port:localAddress}. In the https agent, the name includes
the CA, cert, ciphers, and other HTTPS/TLS-specific options that
determine socket reusability.

\subsection{http.globalAgent}\label{http.globalagent}

Global instance of Agent which is used as the default for all http
client requests.

\subsection{Class: http.ClientRequest}\label{class-http.clientrequest}

This object is created internally and returned from
\texttt{http.request()}. It represents an \emph{in-progress} request
whose header has already been queued. The header is still mutable using
the \texttt{setHeader(name, value)}, \texttt{getHeader(name)},
\texttt{removeHeader(name)} API. The actual header will be sent along
with the first data chunk or when closing the connection.

To get the response, add a listener for \texttt{'response'} to the
request object. \texttt{'response'} will be emitted from the request
object when the response headers have been received. The
\texttt{'response'} event is executed with one argument which is an
instance of
\hyperref[httpux5fhttpux5fincomingmessage]{http.IncomingMessage}.

During the \texttt{'response'} event, one can add listeners to the
response object; particularly to listen for the \texttt{'data'} event.

If no \texttt{'response'} handler is added, then the response will be
entirely discarded. However, if you add a \texttt{'response'} event
handler, then you \textbf{must} consume the data from the response
object, either by calling \texttt{response.read()} whenever there is a
\texttt{'readable'} event, or by adding a \texttt{'data'} handler, or by
calling the \texttt{.resume()} method. Until the data is consumed, the
\texttt{'end'} event will not fire. Also, until the data is read it will
consume memory that can eventually lead to a `process out of memory'
error.

Note: Node does not check whether Content-Length and the length of the
body which has been transmitted are equal or not.

The request implements the
\href{stream.html\#stream_class_stream_writable}{Writable Stream}
interface. This is an
\href{events.html\#events_class_events_eventemitter}{EventEmitter} with
the following events:

\subsubsection{Event: `response'}\label{event-response}

\texttt{function (response) \{ \}}

Emitted when a response is received to this request. This event is
emitted only once. The \texttt{response} argument will be an instance of
\hyperref[httpux5fhttpux5fincomingmessage]{http.IncomingMessage}.

Options:

\begin{itemize}
\itemsep1pt\parskip0pt\parsep0pt
\item
  \texttt{host}: A domain name or IP address of the server to issue the
  request to.
\item
  \texttt{port}: Port of remote server.
\item
  \texttt{socketPath}: Unix Domain Socket (use one of host:port or
  socketPath)
\end{itemize}

\subsubsection{Event: `socket'}\label{event-socket}

\texttt{function (socket) \{ \}}

Emitted after a socket is assigned to this request.

\subsubsection{Event: `connect'}\label{event-connect-1}

\texttt{function (response, socket, head) \{ \}}

Emitted each time a server responds to a request with a CONNECT method.
If this event isn't being listened for, clients receiving a CONNECT
method will have their connections closed.

A client server pair that show you how to listen for the
\texttt{connect} event.

\begin{Shaded}
\begin{Highlighting}[]
\KeywordTok{var} \NormalTok{http = }\FunctionTok{require}\NormalTok{(}\StringTok{'http'}\NormalTok{);}
\KeywordTok{var} \NormalTok{net = }\FunctionTok{require}\NormalTok{(}\StringTok{'net'}\NormalTok{);}
\KeywordTok{var} \NormalTok{url = }\FunctionTok{require}\NormalTok{(}\StringTok{'url'}\NormalTok{);}

\CommentTok{// Create an HTTP tunneling proxy}
\KeywordTok{var} \NormalTok{proxy = }\OtherTok{http}\NormalTok{.}\FunctionTok{createServer}\NormalTok{(}\KeywordTok{function} \NormalTok{(req, res) \{}
  \OtherTok{res}\NormalTok{.}\FunctionTok{writeHead}\NormalTok{(}\DecValTok{200}\NormalTok{, \{}\StringTok{'Content-Type'}\NormalTok{: }\StringTok{'text/plain'}\NormalTok{\});}
  \OtherTok{res}\NormalTok{.}\FunctionTok{end}\NormalTok{(}\StringTok{'okay'}\NormalTok{);}
\NormalTok{\});}
\OtherTok{proxy}\NormalTok{.}\FunctionTok{on}\NormalTok{(}\StringTok{'connect'}\NormalTok{, }\KeywordTok{function}\NormalTok{(req, cltSocket, head) \{}
  \CommentTok{// connect to an origin server}
  \KeywordTok{var} \NormalTok{srvUrl = }\OtherTok{url}\NormalTok{.}\FunctionTok{parse}\NormalTok{(}\StringTok{'http://'} \NormalTok{+ }\OtherTok{req}\NormalTok{.}\FunctionTok{url}\NormalTok{);}
  \KeywordTok{var} \NormalTok{srvSocket = }\OtherTok{net}\NormalTok{.}\FunctionTok{connect}\NormalTok{(}\OtherTok{srvUrl}\NormalTok{.}\FunctionTok{port}\NormalTok{, }\OtherTok{srvUrl}\NormalTok{.}\FunctionTok{hostname}\NormalTok{, }\KeywordTok{function}\NormalTok{() \{}
    \OtherTok{cltSocket}\NormalTok{.}\FunctionTok{write}\NormalTok{(}\StringTok{'HTTP/1.1 200 Connection Established}\CharTok{\textbackslash{}r\textbackslash{}n}\StringTok{'} \NormalTok{+}
                    \StringTok{'Proxy-agent: Node-Proxy}\CharTok{\textbackslash{}r\textbackslash{}n}\StringTok{'} \NormalTok{+}
                    \StringTok{'}\CharTok{\textbackslash{}r\textbackslash{}n}\StringTok{'}\NormalTok{);}
    \OtherTok{srvSocket}\NormalTok{.}\FunctionTok{write}\NormalTok{(head);}
    \OtherTok{srvSocket}\NormalTok{.}\FunctionTok{pipe}\NormalTok{(cltSocket);}
    \OtherTok{cltSocket}\NormalTok{.}\FunctionTok{pipe}\NormalTok{(srvSocket);}
  \NormalTok{\});}
\NormalTok{\});}

\CommentTok{// now that proxy is running}
\OtherTok{proxy}\NormalTok{.}\FunctionTok{listen}\NormalTok{(}\DecValTok{1337}\NormalTok{, }\StringTok{'127.0.0.1'}\NormalTok{, }\KeywordTok{function}\NormalTok{() \{}

  \CommentTok{// make a request to a tunneling proxy}
  \KeywordTok{var} \NormalTok{options = \{}
    \DataTypeTok{port}\NormalTok{: }\DecValTok{1337}\NormalTok{,}
    \DataTypeTok{hostname}\NormalTok{: }\StringTok{'127.0.0.1'}\NormalTok{,}
    \DataTypeTok{method}\NormalTok{: }\StringTok{'CONNECT'}\NormalTok{,}
    \DataTypeTok{path}\NormalTok{: }\StringTok{'www.google.com:80'}
  \NormalTok{\};}

  \KeywordTok{var} \NormalTok{req = }\OtherTok{http}\NormalTok{.}\FunctionTok{request}\NormalTok{(options);}
  \OtherTok{req}\NormalTok{.}\FunctionTok{end}\NormalTok{();}

  \OtherTok{req}\NormalTok{.}\FunctionTok{on}\NormalTok{(}\StringTok{'connect'}\NormalTok{, }\KeywordTok{function}\NormalTok{(res, socket, head) \{}
    \OtherTok{console}\NormalTok{.}\FunctionTok{log}\NormalTok{(}\StringTok{'got connected!'}\NormalTok{);}

    \CommentTok{// make a request over an HTTP tunnel}
    \OtherTok{socket}\NormalTok{.}\FunctionTok{write}\NormalTok{(}\StringTok{'GET / HTTP/1.1}\CharTok{\textbackslash{}r\textbackslash{}n}\StringTok{'} \NormalTok{+}
                 \StringTok{'Host: www.google.com:80}\CharTok{\textbackslash{}r\textbackslash{}n}\StringTok{'} \NormalTok{+}
                 \StringTok{'Connection: close}\CharTok{\textbackslash{}r\textbackslash{}n}\StringTok{'} \NormalTok{+}
                 \StringTok{'}\CharTok{\textbackslash{}r\textbackslash{}n}\StringTok{'}\NormalTok{);}
    \OtherTok{socket}\NormalTok{.}\FunctionTok{on}\NormalTok{(}\StringTok{'data'}\NormalTok{, }\KeywordTok{function}\NormalTok{(chunk) \{}
      \OtherTok{console}\NormalTok{.}\FunctionTok{log}\NormalTok{(}\OtherTok{chunk}\NormalTok{.}\FunctionTok{toString}\NormalTok{());}
    \NormalTok{\});}
    \OtherTok{socket}\NormalTok{.}\FunctionTok{on}\NormalTok{(}\StringTok{'end'}\NormalTok{, }\KeywordTok{function}\NormalTok{() \{}
      \OtherTok{proxy}\NormalTok{.}\FunctionTok{close}\NormalTok{();}
    \NormalTok{\});}
  \NormalTok{\});}
\NormalTok{\});}
\end{Highlighting}
\end{Shaded}

\subsubsection{Event: `upgrade'}\label{event-upgrade-1}

\texttt{function (response, socket, head) \{ \}}

Emitted each time a server responds to a request with an upgrade. If
this event isn't being listened for, clients receiving an upgrade header
will have their connections closed.

A client server pair that show you how to listen for the
\texttt{upgrade} event.

\begin{Shaded}
\begin{Highlighting}[]
\KeywordTok{var} \NormalTok{http = }\FunctionTok{require}\NormalTok{(}\StringTok{'http'}\NormalTok{);}

\CommentTok{// Create an HTTP server}
\KeywordTok{var} \NormalTok{srv = }\OtherTok{http}\NormalTok{.}\FunctionTok{createServer}\NormalTok{(}\KeywordTok{function} \NormalTok{(req, res) \{}
  \OtherTok{res}\NormalTok{.}\FunctionTok{writeHead}\NormalTok{(}\DecValTok{200}\NormalTok{, \{}\StringTok{'Content-Type'}\NormalTok{: }\StringTok{'text/plain'}\NormalTok{\});}
  \OtherTok{res}\NormalTok{.}\FunctionTok{end}\NormalTok{(}\StringTok{'okay'}\NormalTok{);}
\NormalTok{\});}
\OtherTok{srv}\NormalTok{.}\FunctionTok{on}\NormalTok{(}\StringTok{'upgrade'}\NormalTok{, }\KeywordTok{function}\NormalTok{(req, socket, head) \{}
  \OtherTok{socket}\NormalTok{.}\FunctionTok{write}\NormalTok{(}\StringTok{'HTTP/1.1 101 Web Socket Protocol Handshake}\CharTok{\textbackslash{}r\textbackslash{}n}\StringTok{'} \NormalTok{+}
               \StringTok{'Upgrade: WebSocket}\CharTok{\textbackslash{}r\textbackslash{}n}\StringTok{'} \NormalTok{+}
               \StringTok{'Connection: Upgrade}\CharTok{\textbackslash{}r\textbackslash{}n}\StringTok{'} \NormalTok{+}
               \StringTok{'}\CharTok{\textbackslash{}r\textbackslash{}n}\StringTok{'}\NormalTok{);}

  \OtherTok{socket}\NormalTok{.}\FunctionTok{pipe}\NormalTok{(socket); }\CommentTok{// echo back}
\NormalTok{\});}

\CommentTok{// now that server is running}
\OtherTok{srv}\NormalTok{.}\FunctionTok{listen}\NormalTok{(}\DecValTok{1337}\NormalTok{, }\StringTok{'127.0.0.1'}\NormalTok{, }\KeywordTok{function}\NormalTok{() \{}

  \CommentTok{// make a request}
  \KeywordTok{var} \NormalTok{options = \{}
    \DataTypeTok{port}\NormalTok{: }\DecValTok{1337}\NormalTok{,}
    \DataTypeTok{hostname}\NormalTok{: }\StringTok{'127.0.0.1'}\NormalTok{,}
    \DataTypeTok{headers}\NormalTok{: \{}
      \StringTok{'Connection'}\NormalTok{: }\StringTok{'Upgrade'}\NormalTok{,}
      \StringTok{'Upgrade'}\NormalTok{: }\StringTok{'websocket'}
    \NormalTok{\}}
  \NormalTok{\};}

  \KeywordTok{var} \NormalTok{req = }\OtherTok{http}\NormalTok{.}\FunctionTok{request}\NormalTok{(options);}
  \OtherTok{req}\NormalTok{.}\FunctionTok{end}\NormalTok{();}

  \OtherTok{req}\NormalTok{.}\FunctionTok{on}\NormalTok{(}\StringTok{'upgrade'}\NormalTok{, }\KeywordTok{function}\NormalTok{(res, socket, upgradeHead) \{}
    \OtherTok{console}\NormalTok{.}\FunctionTok{log}\NormalTok{(}\StringTok{'got upgraded!'}\NormalTok{);}
    \OtherTok{socket}\NormalTok{.}\FunctionTok{end}\NormalTok{();}
    \OtherTok{process}\NormalTok{.}\FunctionTok{exit}\NormalTok{(}\DecValTok{0}\NormalTok{);}
  \NormalTok{\});}
\NormalTok{\});}
\end{Highlighting}
\end{Shaded}

\subsubsection{Event: `continue'}\label{event-continue}

\texttt{function () \{ \}}

Emitted when the server sends a `100 Continue' HTTP response, usually
because the request contained `Expect: 100-continue'. This is an
instruction that the client should send the request body.

\subsubsection{request.flush()}\label{request.flush}

Flush the request headers.

For efficiency reasons, node.js normally buffers the request headers
until you call \texttt{request.end()} or write the first chunk of
request data. It then tries hard to pack the request headers and data
into a single TCP packet.

That's usually what you want (it saves a TCP round-trip) but not when
the first data isn't sent until possibly much later.
\texttt{request.flush()} lets you bypass the optimization and kickstart
the request.

\subsubsection{request.write(chunk,
{[}encoding{]})}\label{request.writechunk-encoding}

Sends a chunk of the body. By calling this method many times, the user
can stream a request body to a server--in that case it is suggested to
use the \texttt{{[}'Transfer-Encoding', 'chunked'{]}} header line when
creating the request.

The \texttt{chunk} argument should be a
\href{buffer.html\#buffer_buffer}{Buffer} or a string.

The \texttt{encoding} argument is optional and only applies when
\texttt{chunk} is a string. Defaults to \texttt{'utf8'}.

\subsubsection{request.end({[}data{]},
{[}encoding{]})}\label{request.enddata-encoding}

Finishes sending the request. If any parts of the body are unsent, it
will flush them to the stream. If the request is chunked, this will send
the terminating
\texttt{'0\textbackslash{}r\textbackslash{}n\textbackslash{}r\textbackslash{}n'}.

If \texttt{data} is specified, it is equivalent to calling
\texttt{request.write(data, encoding)} followed by
\texttt{request.end()}.

\subsubsection{request.abort()}\label{request.abort}

Aborts a request. (New since v0.3.8.)

\subsubsection{request.setTimeout(timeout,
{[}callback{]})}\label{request.settimeouttimeout-callback}

Once a socket is assigned to this request and is connected
\href{net.html\#net_socket_settimeout_timeout_callback}{socket.setTimeout()}
will be called.

\subsubsection{request.setNoDelay({[}noDelay{]})}\label{request.setnodelaynodelay}

Once a socket is assigned to this request and is connected
\href{net.html\#net_socket_setnodelay_nodelay}{socket.setNoDelay()} will
be called.

\subsubsection{request.setSocketKeepAlive({[}enable{]},
{[}initialDelay{]})}\label{request.setsocketkeepaliveenable-initialdelay}

Once a socket is assigned to this request and is connected
\href{net.html\#net_socket_setkeepalive_enable_initialdelay}{socket.setKeepAlive()}
will be called.

\subsection{http.IncomingMessage}\label{http.incomingmessage}

An \texttt{IncomingMessage} object is created by
\hyperref[httpux5fclassux5fhttpux5fserver]{http.Server} or
\hyperref[httpux5fclassux5fhttpux5fclientrequest]{http.ClientRequest}
and passed as the first argument to the \texttt{'request'} and
\texttt{'response'} event respectively. It may be used to access
response status, headers and data.

It implements the
\href{stream.html\#stream_class_stream_readable}{Readable Stream}
interface, as well as the following additional events, methods, and
properties.

\subsubsection{Event: `close'}\label{event-close-2}

\texttt{function () \{ \}}

Indicates that the underlaying connection was closed. Just like
\texttt{'end'}, this event occurs only once per response.

\subsubsection{message.httpVersion}\label{message.httpversion}

In case of server request, the HTTP version sent by the client. In the
case of client response, the HTTP version of the connected-to server.
Probably either \texttt{'1.1'} or \texttt{'1.0'}.

Also \texttt{response.httpVersionMajor} is the first integer and
\texttt{response.httpVersionMinor} is the second.

\subsubsection{message.headers}\label{message.headers}

The request/response headers object.

Read only map of header names and values. Header names are lower-cased.
Example:

\begin{Shaded}
\begin{Highlighting}[]
\CommentTok{// Prints something like:}
\CommentTok{//}
\CommentTok{// \{ 'user-agent': 'curl/7.22.0',}
\CommentTok{//   host: '127.0.0.1:8000',}
\CommentTok{//   accept: '*/*' \}}
\OtherTok{console}\NormalTok{.}\FunctionTok{log}\NormalTok{(}\OtherTok{request}\NormalTok{.}\FunctionTok{headers}\NormalTok{);}
\end{Highlighting}
\end{Shaded}

\subsubsection{message.rawHeaders}\label{message.rawheaders}

The raw request/response headers list exactly as they were received.

Note that the keys and values are in the same list. It is \emph{not} a
list of tuples. So, the even-numbered offsets are key values, and the
odd-numbered offsets are the associated values.

Header names are not lowercased, and duplicates are not merged.

\begin{Shaded}
\begin{Highlighting}[]
\CommentTok{// Prints something like:}
\CommentTok{//}
\CommentTok{// [ 'user-agent',}
\CommentTok{//   'this is invalid because there can be only one',}
\CommentTok{//   'User-Agent',}
\CommentTok{//   'curl/7.22.0',}
\CommentTok{//   'Host',}
\CommentTok{//   '127.0.0.1:8000',}
\CommentTok{//   'ACCEPT',}
\CommentTok{//   '*/*' ]}
\OtherTok{console}\NormalTok{.}\FunctionTok{log}\NormalTok{(}\OtherTok{request}\NormalTok{.}\FunctionTok{rawHeaders}\NormalTok{);}
\end{Highlighting}
\end{Shaded}

\subsubsection{message.trailers}\label{message.trailers}

The request/response trailers object. Only populated at the `end' event.

\subsubsection{message.rawTrailers}\label{message.rawtrailers}

The raw request/response trailer keys and values exactly as they were
received. Only populated at the `end' event.

\subsubsection{message.setTimeout(msecs,
callback)}\label{message.settimeoutmsecs-callback}

\begin{itemize}
\itemsep1pt\parskip0pt\parsep0pt
\item
  \texttt{msecs} \{Number\}
\item
  \texttt{callback} \{Function\}
\end{itemize}

Calls \texttt{message.connection.setTimeout(msecs, callback)}.

\subsubsection{message.method}\label{message.method}

\textbf{Only valid for request obtained from
\hyperref[httpux5fclassux5fhttpux5fserver]{http.Server}.}

The request method as a string. Read only. Example: \texttt{'GET'},
\texttt{'DELETE'}.

\subsubsection{message.url}\label{message.url}

\textbf{Only valid for request obtained from
\hyperref[httpux5fclassux5fhttpux5fserver]{http.Server}.}

Request URL string. This contains only the URL that is present in the
actual HTTP request. If the request is:

\begin{Shaded}
\begin{Highlighting}[]
\NormalTok{GET /status?name=ryan HTTP/}\FloatTok{1.1}\NormalTok{\textbackslash{}r\textbackslash{}n}
\NormalTok{Accept: text/plain\textbackslash{}r\textbackslash{}n}
\NormalTok{\textbackslash{}r\textbackslash{}n}
\end{Highlighting}
\end{Shaded}

Then \texttt{request.url} will be:

\begin{Shaded}
\begin{Highlighting}[]
\StringTok{'/status?name=ryan'}
\end{Highlighting}
\end{Shaded}

If you would like to parse the URL into its parts, you can use
\texttt{require('url').parse(request.url)}. Example:

\begin{Shaded}
\begin{Highlighting}[]
\NormalTok{node> }\FunctionTok{require}\NormalTok{(}\StringTok{'url'}\NormalTok{).}\FunctionTok{parse}\NormalTok{(}\StringTok{'/status?name=ryan'}\NormalTok{)}
\NormalTok{\{ }\DataTypeTok{href}\NormalTok{: }\StringTok{'/status?name=ryan'}\NormalTok{,}
  \DataTypeTok{search}\NormalTok{: }\StringTok{'?name=ryan'}\NormalTok{,}
  \DataTypeTok{query}\NormalTok{: }\StringTok{'name=ryan'}\NormalTok{,}
  \DataTypeTok{pathname}\NormalTok{: }\StringTok{'/status'} \NormalTok{\}}
\end{Highlighting}
\end{Shaded}

If you would like to extract the params from the query string, you can
use the \texttt{require('querystring').parse} function, or pass
\texttt{true} as the second argument to \texttt{require('url').parse}.
Example:

\begin{Shaded}
\begin{Highlighting}[]
\NormalTok{node> }\FunctionTok{require}\NormalTok{(}\StringTok{'url'}\NormalTok{).}\FunctionTok{parse}\NormalTok{(}\StringTok{'/status?name=ryan'}\NormalTok{, }\KeywordTok{true}\NormalTok{)}
\NormalTok{\{ }\DataTypeTok{href}\NormalTok{: }\StringTok{'/status?name=ryan'}\NormalTok{,}
  \DataTypeTok{search}\NormalTok{: }\StringTok{'?name=ryan'}\NormalTok{,}
  \DataTypeTok{query}\NormalTok{: \{ }\DataTypeTok{name}\NormalTok{: }\StringTok{'ryan'} \NormalTok{\},}
  \DataTypeTok{pathname}\NormalTok{: }\StringTok{'/status'} \NormalTok{\}}
\end{Highlighting}
\end{Shaded}

\subsubsection{message.statusCode}\label{message.statuscode}

\textbf{Only valid for response obtained from
\texttt{http.ClientRequest}.}

The 3-digit HTTP response status code. E.G. \texttt{404}.

\subsubsection{message.statusMessage}\label{message.statusmessage}

\textbf{Only valid for response obtained from
\texttt{http.ClientRequest}.}

The HTTP response status message (reason phrase). E.G. \texttt{OK} or
\texttt{Internal Server Error}.

\subsubsection{message.socket}\label{message.socket}

The \texttt{net.Socket} object associated with the connection.

With HTTPS support, use request.connection.verifyPeer() and
request.connection.getPeerCertificate() to obtain the client's
authentication details.
