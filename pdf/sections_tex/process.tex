\section{process}

The \texttt{process} object is a global object and can be accessed from
anywhere. It is an instance of
\href{events.html\#events\_class\_events\_eventemitter}{EventEmitter}.

\subsection{Event: `exit'}

Emitted when the process is about to exit. This is a good hook to
perform constant time checks of the module's state (like for unit
tests). The main event loop will no longer be run after the `exit'
callback finishes, so timers may not be scheduled.

Example of listening for \texttt{exit}:

\begin{Shaded}
\begin{Highlighting}[]
\KeywordTok{process}\NormalTok{.}\FunctionTok{on}\NormalTok{(}\CharTok{'exit'}\NormalTok{, }\KeywordTok{function}\NormalTok{() \{}
  \NormalTok{setTimeout(}\KeywordTok{function}\NormalTok{() \{}
    \KeywordTok{console}\NormalTok{.}\FunctionTok{log}\NormalTok{(}\CharTok{'This will not run'}\NormalTok{);}
  \NormalTok{\}, }\DecValTok{0}\NormalTok{);}
  \KeywordTok{console}\NormalTok{.}\FunctionTok{log}\NormalTok{(}\CharTok{'About to exit.'}\NormalTok{);}
\NormalTok{\});}
\end{Highlighting}
\end{Shaded}

\subsection{Event: `uncaughtException'}

Emitted when an exception bubbles all the way back to the event loop. If
a listener is added for this exception, the default action (which is to
print a stack trace and exit) will not occur.

Example of listening for \texttt{uncaughtException}:

\begin{Shaded}
\begin{Highlighting}[]
\KeywordTok{process}\NormalTok{.}\FunctionTok{on}\NormalTok{(}\CharTok{'uncaughtException'}\NormalTok{, }\KeywordTok{function}\NormalTok{(err) \{}
  \KeywordTok{console}\NormalTok{.}\FunctionTok{log}\NormalTok{(}\CharTok{'Caught exception: '} \NormalTok{+ err);}
\NormalTok{\});}

\NormalTok{setTimeout(}\KeywordTok{function}\NormalTok{() \{}
  \KeywordTok{console}\NormalTok{.}\FunctionTok{log}\NormalTok{(}\CharTok{'This will still run.'}\NormalTok{);}
\NormalTok{\}, }\DecValTok{500}\NormalTok{);}

\CommentTok{// Intentionally cause an exception, but don't catch it.}
\NormalTok{nonexistentFunc();}
\KeywordTok{console}\NormalTok{.}\FunctionTok{log}\NormalTok{(}\CharTok{'This will not run.'}\NormalTok{);}
\end{Highlighting}
\end{Shaded}

Note that \texttt{uncaughtException} is a very crude mechanism for
exception handling and may be removed in the future.

Don't use it, use \href{domain.html}{domains} instead. If you do use it,
restart your application after every unhandled exception!

Do \emph{not} use it as the node.js equivalent of
\texttt{On Error Resume Next}. An unhandled exception means your
application - and by extension node.js itself - is in an undefined
state. Blindly resuming means \emph{anything} could happen.

Think of resuming as pulling the power cord when you are upgrading your
system. Nine out of ten times nothing happens - but the 10th time, your
system is bust.

You have been warned.

\subsection{Signal Events}

Emitted when the processes receives a signal. See sigaction(2) for a
list of standard POSIX signal names such as SIGINT, SIGUSR1, etc.

Example of listening for \texttt{SIGINT}:

\begin{Shaded}
\begin{Highlighting}[]
\CommentTok{// Start reading from stdin so we don't exit.}
\KeywordTok{process.stdin}\NormalTok{.}\FunctionTok{resume}\NormalTok{();}

\KeywordTok{process}\NormalTok{.}\FunctionTok{on}\NormalTok{(}\CharTok{'SIGINT'}\NormalTok{, }\KeywordTok{function}\NormalTok{() \{}
  \KeywordTok{console}\NormalTok{.}\FunctionTok{log}\NormalTok{(}\CharTok{'Got SIGINT.  Press Control-D to exit.'}\NormalTok{);}
\NormalTok{\});}
\end{Highlighting}
\end{Shaded}

An easy way to send the \texttt{SIGINT} signal is with
\texttt{Control-C} in most terminal programs.

\subsection{process.stdout}

A \texttt{Writable Stream} to \texttt{stdout}.

Example: the definition of \texttt{console.log}

\begin{Shaded}
\begin{Highlighting}[]
\KeywordTok{console}\NormalTok{.}\FunctionTok{log} \NormalTok{= }\KeywordTok{function}\NormalTok{(d) \{}
  \KeywordTok{process.stdout}\NormalTok{.}\FunctionTok{write}\NormalTok{(d + }\CharTok{'\textbackslash{}n'}\NormalTok{);}
\NormalTok{\};}
\end{Highlighting}
\end{Shaded}

\texttt{process.stderr} and \texttt{process.stdout} are unlike other
streams in Node in that writes to them are usually blocking. They are
blocking in the case that they refer to regular files or TTY file
descriptors. In the case they refer to pipes, they are non-blocking like
other streams.

\subsection{process.stderr}

A writable stream to stderr.

\texttt{process.stderr} and \texttt{process.stdout} are unlike other
streams in Node in that writes to them are usually blocking. They are
blocking in the case that they refer to regular files or TTY file
descriptors. In the case they refer to pipes, they are non-blocking like
other streams.

\subsection{process.stdin}

A \texttt{Readable Stream} for stdin. The stdin stream is paused by
default, so one must call \texttt{process.stdin.resume()} to read from
it.

Example of opening standard input and listening for both events:

\begin{Shaded}
\begin{Highlighting}[]
\KeywordTok{process.stdin}\NormalTok{.}\FunctionTok{resume}\NormalTok{();}
\KeywordTok{process.stdin}\NormalTok{.}\FunctionTok{setEncoding}\NormalTok{(}\CharTok{'utf8'}\NormalTok{);}

\KeywordTok{process.stdin}\NormalTok{.}\FunctionTok{on}\NormalTok{(}\CharTok{'data'}\NormalTok{, }\KeywordTok{function}\NormalTok{(chunk) \{}
  \KeywordTok{process.stdout}\NormalTok{.}\FunctionTok{write}\NormalTok{(}\CharTok{'data: '} \NormalTok{+ chunk);}
\NormalTok{\});}

\KeywordTok{process.stdin}\NormalTok{.}\FunctionTok{on}\NormalTok{(}\CharTok{'end'}\NormalTok{, }\KeywordTok{function}\NormalTok{() \{}
  \KeywordTok{process.stdout}\NormalTok{.}\FunctionTok{write}\NormalTok{(}\CharTok{'end'}\NormalTok{);}
\NormalTok{\});}
\end{Highlighting}
\end{Shaded}

\subsection{process.argv}

An array containing the command line arguments. The first element will
be `node', the second element will be the name of the JavaScript file.
The next elements will be any additional command line arguments.

\begin{Shaded}
\begin{Highlighting}[]
\CommentTok{// print process.argv}
\KeywordTok{process.argv}\NormalTok{.}\FunctionTok{forEach}\NormalTok{(}\KeywordTok{function}\NormalTok{(val, index, array) \{}
  \KeywordTok{console}\NormalTok{.}\FunctionTok{log}\NormalTok{(index + }\CharTok{': '} \NormalTok{+ val);}
\NormalTok{\});}
\end{Highlighting}
\end{Shaded}

This will generate:

\begin{Shaded}
\begin{Highlighting}[]
\NormalTok{$ node process}\FloatTok{-2.}\NormalTok{js one two=three four}
\DecValTok{0}\NormalTok{: node}
\DecValTok{1}\NormalTok{: }\OtherTok{/}\FloatTok{Users}\OtherTok{/}\NormalTok{mjr/work/node/process}\FloatTok{-2.}\NormalTok{js}
\DecValTok{2}\NormalTok{: one}
\DecValTok{3}\NormalTok{: two=three}
\DecValTok{4}\NormalTok{: four}
\end{Highlighting}
\end{Shaded}

\subsection{process.execPath}

This is the absolute pathname of the executable that started the
process.

Example:

\begin{Shaded}
\begin{Highlighting}[]
\NormalTok{/usr/local/bin/node}
\end{Highlighting}
\end{Shaded}

\subsection{process.abort()}

This causes node to emit an abort. This will cause node to exit and
generate a core file.

\subsection{process.chdir(directory)}

Changes the current working directory of the process or throws an
exception if that fails.

\begin{Shaded}
\begin{Highlighting}[]
\KeywordTok{console}\NormalTok{.}\FunctionTok{log}\NormalTok{(}\CharTok{'Starting directory: '} \NormalTok{+ }\KeywordTok{process}\NormalTok{.}\FunctionTok{cwd}\NormalTok{());}
\KeywordTok{try} \NormalTok{\{}
  \KeywordTok{process}\NormalTok{.}\FunctionTok{chdir}\NormalTok{(}\CharTok{'/tmp'}\NormalTok{);}
  \KeywordTok{console}\NormalTok{.}\FunctionTok{log}\NormalTok{(}\CharTok{'New directory: '} \NormalTok{+ }\KeywordTok{process}\NormalTok{.}\FunctionTok{cwd}\NormalTok{());}
\NormalTok{\}}
\KeywordTok{catch} \NormalTok{(err) \{}
  \KeywordTok{console}\NormalTok{.}\FunctionTok{log}\NormalTok{(}\CharTok{'chdir: '} \NormalTok{+ err);}
\NormalTok{\}}
\end{Highlighting}
\end{Shaded}

\subsection{process.cwd()}

Returns the current working directory of the process.

\begin{Shaded}
\begin{Highlighting}[]
\KeywordTok{console}\NormalTok{.}\FunctionTok{log}\NormalTok{(}\CharTok{'Current directory: '} \NormalTok{+ }\KeywordTok{process}\NormalTok{.}\FunctionTok{cwd}\NormalTok{());}
\end{Highlighting}
\end{Shaded}

\subsection{process.env}

An object containing the user environment. See environ(7).

\subsection{process.exit({[}code{]})}

Ends the process with the specified \texttt{code}. If omitted, exit uses
the `success' code \texttt{0}.

To exit with a `failure' code:

\begin{Shaded}
\begin{Highlighting}[]
\KeywordTok{process}\NormalTok{.}\FunctionTok{exit}\NormalTok{(}\DecValTok{1}\NormalTok{);}
\end{Highlighting}
\end{Shaded}

The shell that executed node should see the exit code as 1.

\subsection{process.getgid()}

Note: this function is only available on POSIX platforms (i.e.~not
Windows)

Gets the group identity of the process. (See getgid(2).) This is the
numerical group id, not the group name.

\begin{Shaded}
\begin{Highlighting}[]
\KeywordTok{if} \NormalTok{(}\KeywordTok{process}\NormalTok{.}\FunctionTok{getgid}\NormalTok{) \{}
  \KeywordTok{console}\NormalTok{.}\FunctionTok{log}\NormalTok{(}\CharTok{'Current gid: '} \NormalTok{+ }\KeywordTok{process}\NormalTok{.}\FunctionTok{getgid}\NormalTok{());}
\NormalTok{\}}
\end{Highlighting}
\end{Shaded}

\subsection{process.setgid(id)}

Note: this function is only available on POSIX platforms (i.e.~not
Windows)

Sets the group identity of the process. (See setgid(2).) This accepts
either a numerical ID or a groupname string. If a groupname is
specified, this method blocks while resolving it to a numerical ID.

\begin{Shaded}
\begin{Highlighting}[]
\KeywordTok{if} \NormalTok{(}\KeywordTok{process}\NormalTok{.}\FunctionTok{getgid} \NormalTok{&& }\KeywordTok{process}\NormalTok{.}\FunctionTok{setgid}\NormalTok{) \{}
  \KeywordTok{console}\NormalTok{.}\FunctionTok{log}\NormalTok{(}\CharTok{'Current gid: '} \NormalTok{+ }\KeywordTok{process}\NormalTok{.}\FunctionTok{getgid}\NormalTok{());}
  \KeywordTok{try} \NormalTok{\{}
    \KeywordTok{process}\NormalTok{.}\FunctionTok{setgid}\NormalTok{(}\DecValTok{501}\NormalTok{);}
    \KeywordTok{console}\NormalTok{.}\FunctionTok{log}\NormalTok{(}\CharTok{'New gid: '} \NormalTok{+ }\KeywordTok{process}\NormalTok{.}\FunctionTok{getgid}\NormalTok{());}
  \NormalTok{\}}
  \KeywordTok{catch} \NormalTok{(err) \{}
    \KeywordTok{console}\NormalTok{.}\FunctionTok{log}\NormalTok{(}\CharTok{'Failed to set gid: '} \NormalTok{+ err);}
  \NormalTok{\}}
\NormalTok{\}}
\end{Highlighting}
\end{Shaded}

\subsection{process.getuid()}

Note: this function is only available on POSIX platforms (i.e.~not
Windows)

Gets the user identity of the process. (See getuid(2).) This is the
numerical userid, not the username.

\begin{Shaded}
\begin{Highlighting}[]
\KeywordTok{if} \NormalTok{(}\KeywordTok{process}\NormalTok{.}\FunctionTok{getuid}\NormalTok{) \{}
  \KeywordTok{console}\NormalTok{.}\FunctionTok{log}\NormalTok{(}\CharTok{'Current uid: '} \NormalTok{+ }\KeywordTok{process}\NormalTok{.}\FunctionTok{getuid}\NormalTok{());}
\NormalTok{\}}
\end{Highlighting}
\end{Shaded}

\subsection{process.setuid(id)}

Note: this function is only available on POSIX platforms (i.e.~not
Windows)

Sets the user identity of the process. (See setuid(2).) This accepts
either a numerical ID or a username string. If a username is specified,
this method blocks while resolving it to a numerical ID.

\begin{Shaded}
\begin{Highlighting}[]
\KeywordTok{if} \NormalTok{(}\KeywordTok{process}\NormalTok{.}\FunctionTok{getuid} \NormalTok{&& }\KeywordTok{process}\NormalTok{.}\FunctionTok{setuid}\NormalTok{) \{}
  \KeywordTok{console}\NormalTok{.}\FunctionTok{log}\NormalTok{(}\CharTok{'Current uid: '} \NormalTok{+ }\KeywordTok{process}\NormalTok{.}\FunctionTok{getuid}\NormalTok{());}
  \KeywordTok{try} \NormalTok{\{}
    \KeywordTok{process}\NormalTok{.}\FunctionTok{setuid}\NormalTok{(}\DecValTok{501}\NormalTok{);}
    \KeywordTok{console}\NormalTok{.}\FunctionTok{log}\NormalTok{(}\CharTok{'New uid: '} \NormalTok{+ }\KeywordTok{process}\NormalTok{.}\FunctionTok{getuid}\NormalTok{());}
  \NormalTok{\}}
  \KeywordTok{catch} \NormalTok{(err) \{}
    \KeywordTok{console}\NormalTok{.}\FunctionTok{log}\NormalTok{(}\CharTok{'Failed to set uid: '} \NormalTok{+ err);}
  \NormalTok{\}}
\NormalTok{\}}
\end{Highlighting}
\end{Shaded}

\subsection{process.version}

A compiled-in property that exposes \texttt{NODE\_VERSION}.

\begin{Shaded}
\begin{Highlighting}[]
\KeywordTok{console}\NormalTok{.}\FunctionTok{log}\NormalTok{(}\CharTok{'Version: '} \NormalTok{+ }\KeywordTok{process}\NormalTok{.}\FunctionTok{version}\NormalTok{);}
\end{Highlighting}
\end{Shaded}

\subsection{process.versions}

A property exposing version strings of node and its dependencies.

\begin{Shaded}
\begin{Highlighting}[]
\KeywordTok{console}\NormalTok{.}\FunctionTok{log}\NormalTok{(}\KeywordTok{process}\NormalTok{.}\FunctionTok{versions}\NormalTok{);}
\end{Highlighting}
\end{Shaded}

Will output:

\begin{Shaded}
\begin{Highlighting}[]
\NormalTok{\{ }\DataTypeTok{node}\NormalTok{: }\CharTok{'0.4.12'}\NormalTok{,}
  \DataTypeTok{v8}\NormalTok{: }\CharTok{'3.1.8.26'}\NormalTok{,}
  \DataTypeTok{ares}\NormalTok{: }\CharTok{'1.7.4'}\NormalTok{,}
  \DataTypeTok{ev}\NormalTok{: }\CharTok{'4.4'}\NormalTok{,}
  \DataTypeTok{openssl}\NormalTok{: }\CharTok{'1.0.0e-fips'} \NormalTok{\}}
\end{Highlighting}
\end{Shaded}

\subsection{process.config}

An Object containing the JavaScript representation of the configure
options that were used to compile the current node executable. This is
the same as the ``config.gypi'' file that was produced when running the
\texttt{./configure} script.

An example of the possible output looks like:

\begin{Shaded}
\begin{Highlighting}[]
\NormalTok{\{ }\DataTypeTok{target_defaults}\NormalTok{:}
   \NormalTok{\{ }\DataTypeTok{cflags}\NormalTok{: [],}
     \DataTypeTok{default_configuration}\NormalTok{: }\CharTok{'Release'}\NormalTok{,}
     \DataTypeTok{defines}\NormalTok{: [],}
     \DataTypeTok{include_dirs}\NormalTok{: [],}
     \DataTypeTok{libraries}\NormalTok{: [] \},}
  \DataTypeTok{variables}\NormalTok{:}
   \NormalTok{\{ }\DataTypeTok{host_arch}\NormalTok{: }\CharTok{'x64'}\NormalTok{,}
     \DataTypeTok{node_install_npm}\NormalTok{: }\CharTok{'true'}\NormalTok{,}
     \DataTypeTok{node_prefix}\NormalTok{: }\CharTok{''}\NormalTok{,}
     \DataTypeTok{node_shared_v8}\NormalTok{: }\CharTok{'false'}\NormalTok{,}
     \DataTypeTok{node_shared_zlib}\NormalTok{: }\CharTok{'false'}\NormalTok{,}
     \DataTypeTok{node_use_dtrace}\NormalTok{: }\CharTok{'false'}\NormalTok{,}
     \DataTypeTok{node_use_openssl}\NormalTok{: }\CharTok{'true'}\NormalTok{,}
     \DataTypeTok{node_shared_openssl}\NormalTok{: }\CharTok{'false'}\NormalTok{,}
     \DataTypeTok{strict_aliasing}\NormalTok{: }\CharTok{'true'}\NormalTok{,}
     \DataTypeTok{target_arch}\NormalTok{: }\CharTok{'x64'}\NormalTok{,}
     \DataTypeTok{v8_use_snapshot}\NormalTok{: }\CharTok{'true'} \NormalTok{\} \}}
\end{Highlighting}
\end{Shaded}

\subsection{process.kill(pid, {[}signal{]})}

Send a signal to a process. \texttt{pid} is the process id and
\texttt{signal} is the string describing the signal to send. Signal
names are strings like `SIGINT' or `SIGUSR1'. If omitted, the signal
will be `SIGTERM'. See kill(2) for more information.

Note that just because the name of this function is
\texttt{process.kill}, it is really just a signal sender, like the
\texttt{kill} system call. The signal sent may do something other than
kill the target process.

Example of sending a signal to yourself:

\begin{Shaded}
\begin{Highlighting}[]
\KeywordTok{process}\NormalTok{.}\FunctionTok{on}\NormalTok{(}\CharTok{'SIGHUP'}\NormalTok{, }\KeywordTok{function}\NormalTok{() \{}
  \KeywordTok{console}\NormalTok{.}\FunctionTok{log}\NormalTok{(}\CharTok{'Got SIGHUP signal.'}\NormalTok{);}
\NormalTok{\});}

\NormalTok{setTimeout(}\KeywordTok{function}\NormalTok{() \{}
  \KeywordTok{console}\NormalTok{.}\FunctionTok{log}\NormalTok{(}\CharTok{'Exiting.'}\NormalTok{);}
  \KeywordTok{process}\NormalTok{.}\FunctionTok{exit}\NormalTok{(}\DecValTok{0}\NormalTok{);}
\NormalTok{\}, }\DecValTok{100}\NormalTok{);}

\KeywordTok{process}\NormalTok{.}\FunctionTok{kill}\NormalTok{(}\KeywordTok{process}\NormalTok{.}\FunctionTok{pid}\NormalTok{, }\CharTok{'SIGHUP'}\NormalTok{);}
\end{Highlighting}
\end{Shaded}

\subsection{process.pid}

The PID of the process.

\begin{Shaded}
\begin{Highlighting}[]
\KeywordTok{console}\NormalTok{.}\FunctionTok{log}\NormalTok{(}\CharTok{'This process is pid '} \NormalTok{+ }\KeywordTok{process}\NormalTok{.}\FunctionTok{pid}\NormalTok{);}
\end{Highlighting}
\end{Shaded}

\subsection{process.title}

Getter/setter to set what is displayed in `ps'.

\subsection{process.arch}

What processor architecture you're running on: \texttt{'arm'},
\texttt{'ia32'}, or \texttt{'x64'}.

\begin{Shaded}
\begin{Highlighting}[]
\KeywordTok{console}\NormalTok{.}\FunctionTok{log}\NormalTok{(}\CharTok{'This processor architecture is '} \NormalTok{+ }\KeywordTok{process}\NormalTok{.}\FunctionTok{arch}\NormalTok{);}
\end{Highlighting}
\end{Shaded}

\subsection{process.platform}

What platform you're running on: \texttt{'darwin'}, \texttt{'freebsd'},
\texttt{'linux'}, \texttt{'sunos'} or \texttt{'win32'}

\begin{Shaded}
\begin{Highlighting}[]
\KeywordTok{console}\NormalTok{.}\FunctionTok{log}\NormalTok{(}\CharTok{'This platform is '} \NormalTok{+ }\KeywordTok{process}\NormalTok{.}\FunctionTok{platform}\NormalTok{);}
\end{Highlighting}
\end{Shaded}

\subsection{process.memoryUsage()}

Returns an object describing the memory usage of the Node process
measured in bytes.

\begin{Shaded}
\begin{Highlighting}[]
\KeywordTok{var} \NormalTok{util = require(}\CharTok{'util'}\NormalTok{);}

\KeywordTok{console}\NormalTok{.}\FunctionTok{log}\NormalTok{(}\KeywordTok{util}\NormalTok{.}\FunctionTok{inspect}\NormalTok{(}\KeywordTok{process}\NormalTok{.}\FunctionTok{memoryUsage}\NormalTok{()));}
\end{Highlighting}
\end{Shaded}

This will generate:

\begin{Shaded}
\begin{Highlighting}[]
\NormalTok{\{ }\DataTypeTok{rss}\NormalTok{: }\DecValTok{4935680}\NormalTok{,}
  \DataTypeTok{heapTotal}\NormalTok{: }\DecValTok{1826816}\NormalTok{,}
  \DataTypeTok{heapUsed}\NormalTok{: }\DecValTok{650472} \NormalTok{\}}
\end{Highlighting}
\end{Shaded}

\texttt{heapTotal} and \texttt{heapUsed} refer to V8's memory usage.

\subsection{process.nextTick(callback)}

On the next loop around the event loop call this callback. This is
\emph{not} a simple alias to \texttt{setTimeout(fn, 0)}, it's much more
efficient. It typically runs before any other I/O events fire, but there
are some exceptions. See \texttt{process.maxTickDepth} below.

\begin{Shaded}
\begin{Highlighting}[]
\KeywordTok{process}\NormalTok{.}\FunctionTok{nextTick}\NormalTok{(}\KeywordTok{function}\NormalTok{() \{}
  \KeywordTok{console}\NormalTok{.}\FunctionTok{log}\NormalTok{(}\CharTok{'nextTick callback'}\NormalTok{);}
\NormalTok{\});}
\end{Highlighting}
\end{Shaded}

This is important in developing APIs where you want to give the user the
chance to assign event handlers after an object has been constructed,
but before any I/O has occurred.

\begin{Shaded}
\begin{Highlighting}[]
\KeywordTok{function} \NormalTok{MyThing(options) \{}
  \KeywordTok{this}\NormalTok{.}\FunctionTok{setupOptions}\NormalTok{(options);}

  \KeywordTok{process}\NormalTok{.}\FunctionTok{nextTick}\NormalTok{(}\KeywordTok{function}\NormalTok{() \{}
    \KeywordTok{this}\NormalTok{.}\FunctionTok{startDoingStuff}\NormalTok{();}
  \NormalTok{\}.}\FunctionTok{bind}\NormalTok{(}\KeywordTok{this}\NormalTok{));}
\NormalTok{\}}

\KeywordTok{var} \NormalTok{thing = }\KeywordTok{new} \NormalTok{MyThing();}
\KeywordTok{thing}\NormalTok{.}\FunctionTok{getReadyForStuff}\NormalTok{();}

\CommentTok{// thing.startDoingStuff() gets called now, not before.}
\end{Highlighting}
\end{Shaded}

It is very important for APIs to be either 100\% synchronous or 100\%
asynchronous. Consider this example:

\begin{Shaded}
\begin{Highlighting}[]
\CommentTok{// WARNING!  DO NOT USE!  BAD UNSAFE HAZARD!}
\KeywordTok{function} \NormalTok{maybeSync(arg, cb) \{}
  \KeywordTok{if} \NormalTok{(arg) \{}
    \NormalTok{cb();}
    \KeywordTok{return}\NormalTok{;}
  \NormalTok{\}}

  \KeywordTok{fs}\NormalTok{.}\FunctionTok{stat}\NormalTok{(}\CharTok{'file'}\NormalTok{, cb);}
\NormalTok{\}}
\end{Highlighting}
\end{Shaded}

This API is hazardous. If you do this:

\begin{Shaded}
\begin{Highlighting}[]
\NormalTok{maybeSync(}\KeywordTok{true}\NormalTok{, }\KeywordTok{function}\NormalTok{() \{}
  \NormalTok{foo();}
\NormalTok{\});}
\NormalTok{bar();}
\end{Highlighting}
\end{Shaded}

then it's not clear whether \texttt{foo()} or \texttt{bar()} will be
called first.

This approach is much better:

\begin{Shaded}
\begin{Highlighting}[]
\KeywordTok{function} \NormalTok{definitelyAsync(arg, cb) \{}
  \KeywordTok{if} \NormalTok{(arg) \{}
    \KeywordTok{process}\NormalTok{.}\FunctionTok{nextTick}\NormalTok{(cb);}
    \KeywordTok{return}\NormalTok{;}
  \NormalTok{\}}

  \KeywordTok{fs}\NormalTok{.}\FunctionTok{stat}\NormalTok{(}\CharTok{'file'}\NormalTok{, cb);}
\NormalTok{\}}
\end{Highlighting}
\end{Shaded}

\subsection{process.maxTickDepth}

\begin{itemize}
\item
  \{Number\} Default = 1000
\end{itemize}

Callbacks passed to \texttt{process.nextTick} will \emph{usually} be
called at the end of the current flow of execution, and are thus
approximately as fast as calling a function synchronously. Left
unchecked, this would starve the event loop, preventing any I/O from
occurring.

Consider this code:

\begin{Shaded}
\begin{Highlighting}[]
\KeywordTok{process}\NormalTok{.}\FunctionTok{nextTick}\NormalTok{(}\KeywordTok{function} \NormalTok{foo() \{}
  \KeywordTok{process}\NormalTok{.}\FunctionTok{nextTick}\NormalTok{(foo);}
\NormalTok{\});}
\end{Highlighting}
\end{Shaded}

In order to avoid the situation where Node is blocked by an infinite
loop of recursive series of nextTick calls, it defers to allow some I/O
to be done every so often.

The \texttt{process.maxTickDepth} value is the maximum depth of
nextTick-calling nextTick-callbacks that will be evaluated before
allowing other forms of I/O to occur.

\subsection{process.umask({[}mask{]})}

Sets or reads the process's file mode creation mask. Child processes
inherit the mask from the parent process. Returns the old mask if
\texttt{mask} argument is given, otherwise returns the current mask.

\begin{Shaded}
\begin{Highlighting}[]
\KeywordTok{var} \NormalTok{oldmask, newmask = }\DecValTok{0644}\NormalTok{;}

\NormalTok{oldmask = }\KeywordTok{process}\NormalTok{.}\FunctionTok{umask}\NormalTok{(newmask);}
\KeywordTok{console}\NormalTok{.}\FunctionTok{log}\NormalTok{(}\CharTok{'Changed umask from: '} \NormalTok{+ }\KeywordTok{oldmask}\NormalTok{.}\FunctionTok{toString}\NormalTok{(}\DecValTok{8}\NormalTok{) +}
            \CharTok{' to '} \NormalTok{+ }\KeywordTok{newmask}\NormalTok{.}\FunctionTok{toString}\NormalTok{(}\DecValTok{8}\NormalTok{));}
\end{Highlighting}
\end{Shaded}

\subsection{process.uptime()}

Number of seconds Node has been running.

\subsection{process.hrtime()}

Returns the current high-resolution real time in a
\texttt{{[}seconds, nanoseconds{]}} tuple Array. It is relative to an
arbitrary time in the past. It is not related to the time of day and
therefore not subject to clock drift. The primary use is for measuring
performance between intervals.

You may pass in the result of a previous call to
\texttt{process.hrtime()} to get a diff reading, useful for benchmarks
and measuring intervals:

\begin{Shaded}
\begin{Highlighting}[]
\KeywordTok{var} \NormalTok{t = }\KeywordTok{process}\NormalTok{.}\FunctionTok{hrtime}\NormalTok{();}
\CommentTok{// [ 1800216, 927643717 ]}

\NormalTok{setTimeout(}\KeywordTok{function}\NormalTok{() \{}
  \NormalTok{t = }\KeywordTok{process}\NormalTok{.}\FunctionTok{hrtime}\NormalTok{(t);}
  \CommentTok{// [ 1, 6962306 ]}

  \KeywordTok{console}\NormalTok{.}\FunctionTok{log}\NormalTok{(}\CharTok{'benchmark took %d seconds and %d nanoseconds'}\NormalTok{, t[}\DecValTok{0}\NormalTok{], t[}\DecValTok{1}\NormalTok{]);}
  \CommentTok{// benchmark took 1 seconds and 6962306 nanoseconds}
\NormalTok{\}, }\DecValTok{1000}\NormalTok{);}
\end{Highlighting}
\end{Shaded}

