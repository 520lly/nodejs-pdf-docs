\section{Addons}\label{addons}

Addons are dynamically linked shared objects. They can provide glue to C
and C++ libraries. The API (at the moment) is rather complex, involving
knowledge of several libraries:

\begin{itemize}
\item
  V8 JavaScript, a C++ library. Used for interfacing with JavaScript:
  creating objects, calling functions, etc. Documented mostly in the
  \texttt{v8.h} header file (\texttt{deps/v8/include/v8.h} in the Node
  source tree), which is also available
  \href{http://izs.me/v8-docs/main.html}{online}.
\item
  \href{https://github.com/joyent/libuv}{libuv}, C event loop library.
  Anytime one needs to wait for a file descriptor to become readable,
  wait for a timer, or wait for a signal to be received one will need to
  interface with libuv. That is, if you perform any I/O, libuv will need
  to be used.
\item
  Internal Node libraries. Most importantly is the
  \texttt{node::ObjectWrap} class which you will likely want to derive
  from.
\item
  Others. Look in \texttt{deps/} for what else is available.
\end{itemize}

Node statically compiles all its dependencies into the executable. When
compiling your module, you don't need to worry about linking to any of
these libraries.

All of the following examples are available for
\href{https://github.com/rvagg/node-addon-examples}{download} and may be
used as a starting-point for your own Addon.

\subsection{Hello world}\label{hello-world}

To get started let's make a small Addon which is the C++ equivalent of
the following JavaScript code:

\begin{Shaded}
\begin{Highlighting}[]
\OtherTok{module}\NormalTok{.}\OtherTok{exports}\NormalTok{.}\FunctionTok{hello} \NormalTok{= }\KeywordTok{function}\NormalTok{() \{ }\KeywordTok{return} \StringTok{'world'}\NormalTok{; \};}
\end{Highlighting}
\end{Shaded}

First we create a file \texttt{hello.cc}:

\begin{Shaded}
\begin{Highlighting}[]
\CommentTok{// hello.cc}
\NormalTok{#include <}\OtherTok{node}\NormalTok{.}\FunctionTok{h}\NormalTok{>}

\NormalTok{using namespace v8;}

\KeywordTok{void} \FunctionTok{Method}\NormalTok{(}\KeywordTok{const} \NormalTok{FunctionCallbackInfo<Value>& args) \{}
  \NormalTok{Isolate* isolate = }\DataTypeTok{Isolate}\NormalTok{::}\FunctionTok{GetCurrent}\NormalTok{();}
  \NormalTok{HandleScope }\FunctionTok{scope}\NormalTok{(isolate);}
  \OtherTok{args}\NormalTok{.}\FunctionTok{GetReturnValue}\NormalTok{().}\FunctionTok{Set}\NormalTok{(}\DataTypeTok{String}\NormalTok{::}\FunctionTok{NewFromUtf8}\NormalTok{(isolate, }\StringTok{"world"}\NormalTok{));}
\NormalTok{\}}

\KeywordTok{void} \FunctionTok{init}\NormalTok{(Handle<Object> exports) \{}
  \FunctionTok{NODE_SET_METHOD}\NormalTok{(exports, }\StringTok{"hello"}\NormalTok{, Method);}
\NormalTok{\}}

\FunctionTok{NODE_MODULE}\NormalTok{(addon, init)}
\end{Highlighting}
\end{Shaded}

Note that all Node addons must export an initialization function:

\begin{Shaded}
\begin{Highlighting}[]
\KeywordTok{void} \FunctionTok{Initialize} \NormalTok{(Handle<Object> exports);}
\FunctionTok{NODE_MODULE}\NormalTok{(module_name, Initialize)}
\end{Highlighting}
\end{Shaded}

There is no semi-colon after \texttt{NODE\_MODULE} as it's not a
function (see \texttt{node.h}).

The \texttt{module\_name} needs to match the filename of the final
binary (minus the .node suffix).

The source code needs to be built into \texttt{addon.node}, the binary
Addon. To do this we create a file called \texttt{binding.gyp} which
describes the configuration to build your module in a JSON-like format.
This file gets compiled by
\href{https://github.com/TooTallNate/node-gyp}{node-gyp}.

\begin{Shaded}
\begin{Highlighting}[]
\NormalTok{\{}
  \StringTok{"targets"}\NormalTok{: [}
    \NormalTok{\{}
      \StringTok{"target_name"}\NormalTok{: }\StringTok{"addon"}\NormalTok{,}
      \StringTok{"sources"}\NormalTok{: [ }\StringTok{"hello.cc"} \NormalTok{]}
    \NormalTok{\}}
  \NormalTok{]}
\NormalTok{\}}
\end{Highlighting}
\end{Shaded}

The next step is to generate the appropriate project build files for the
current platform. Use \texttt{node-gyp configure} for that.

Now you will have either a \texttt{Makefile} (on Unix platforms) or a
\texttt{vcxproj} file (on Windows) in the \texttt{build/} directory.
Next invoke the \texttt{node-gyp build} command.

Now you have your compiled \texttt{.node} bindings file! The compiled
bindings end up in \texttt{build/Release/}.

You can now use the binary addon in a Node project \texttt{hello.js} by
pointing \texttt{require} to the recently built \texttt{hello.node}
module:

\begin{Shaded}
\begin{Highlighting}[]
\CommentTok{// hello.js}
\KeywordTok{var} \NormalTok{addon = }\FunctionTok{require}\NormalTok{(}\StringTok{'./build/Release/addon'}\NormalTok{);}

\OtherTok{console}\NormalTok{.}\FunctionTok{log}\NormalTok{(}\OtherTok{addon}\NormalTok{.}\FunctionTok{hello}\NormalTok{()); }\CommentTok{// 'world'}
\end{Highlighting}
\end{Shaded}

Please see patterns below for further information or
\url{https://github.com/arturadib/node-qt} for an example in production.

\subsection{Addon patterns}\label{addon-patterns}

Below are some addon patterns to help you get started. Consult the
online \href{http://izs.me/v8-docs/main.html}{v8 reference} for help
with the various v8 calls, and v8's
\href{http://code.google.com/apis/v8/embed.html}{Embedder's Guide} for
an explanation of several concepts used such as handles, scopes,
function templates, etc.

In order to use these examples you need to compile them using
\texttt{node-gyp}. Create the following \texttt{binding.gyp} file:

\begin{Shaded}
\begin{Highlighting}[]
\NormalTok{\{}
  \StringTok{"targets"}\NormalTok{: [}
    \NormalTok{\{}
      \StringTok{"target_name"}\NormalTok{: }\StringTok{"addon"}\NormalTok{,}
      \StringTok{"sources"}\NormalTok{: [ }\StringTok{"addon.cc"} \NormalTok{]}
    \NormalTok{\}}
  \NormalTok{]}
\NormalTok{\}}
\end{Highlighting}
\end{Shaded}

In cases where there is more than one \texttt{.cc} file, simply add the
file name to the \texttt{sources} array, e.g.:

\begin{Shaded}
\begin{Highlighting}[]
\StringTok{"sources"}\NormalTok{: [}\StringTok{"addon.cc"}\NormalTok{, }\StringTok{"myexample.cc"}\NormalTok{]}
\end{Highlighting}
\end{Shaded}

Now that you have your \texttt{binding.gyp} ready, you can configure and
build the addon:

\begin{Shaded}
\begin{Highlighting}[]
\NormalTok{$ node-gyp configure build}
\end{Highlighting}
\end{Shaded}

\subsubsection{Function arguments}\label{function-arguments}

The following pattern illustrates how to read arguments from JavaScript
function calls and return a result. This is the main and only needed
source \texttt{addon.cc}:

\begin{Shaded}
\begin{Highlighting}[]
\CommentTok{// addon.cc}
\NormalTok{#include <}\OtherTok{node}\NormalTok{.}\FunctionTok{h}\NormalTok{>}

\NormalTok{using namespace v8;}

\KeywordTok{void} \FunctionTok{Add}\NormalTok{(}\KeywordTok{const} \NormalTok{FunctionCallbackInfo<Value>& args) \{}
  \NormalTok{Isolate* isolate = }\DataTypeTok{Isolate}\NormalTok{::}\FunctionTok{GetCurrent}\NormalTok{();}
  \NormalTok{HandleScope }\FunctionTok{scope}\NormalTok{(isolate);}

  \KeywordTok{if} \NormalTok{(}\OtherTok{args}\NormalTok{.}\FunctionTok{Length}\NormalTok{() < }\DecValTok{2}\NormalTok{) \{}
    \NormalTok{isolate->}\FunctionTok{ThrowException}\NormalTok{(}\DataTypeTok{Exception}\NormalTok{::}\FunctionTok{TypeError}\NormalTok{(}
        \DataTypeTok{String}\NormalTok{::}\FunctionTok{NewFromUtf8}\NormalTok{(isolate, }\StringTok{"Wrong number of arguments"}\NormalTok{)));}
    \KeywordTok{return}\NormalTok{;}
  \NormalTok{\}}

  \KeywordTok{if} \NormalTok{(!args[}\DecValTok{0}\NormalTok{]->}\FunctionTok{IsNumber}\NormalTok{() || !args[}\DecValTok{1}\NormalTok{]->}\FunctionTok{IsNumber}\NormalTok{()) \{}
    \NormalTok{isolate->}\FunctionTok{ThrowException}\NormalTok{(}\DataTypeTok{Exception}\NormalTok{::}\FunctionTok{TypeError}\NormalTok{(}
        \DataTypeTok{String}\NormalTok{::}\FunctionTok{NewFromUtf8}\NormalTok{(isolate, }\StringTok{"Wrong arguments"}\NormalTok{)));}
    \KeywordTok{return}\NormalTok{;}
  \NormalTok{\}}

  \NormalTok{Local<Number> num = }\DataTypeTok{Number}\NormalTok{::}\FunctionTok{New}\NormalTok{(args[}\DecValTok{0}\NormalTok{]->}\FunctionTok{NumberValue}\NormalTok{() +}
      \NormalTok{args[}\DecValTok{1}\NormalTok{]->}\FunctionTok{NumberValue}\NormalTok{());}

  \OtherTok{args}\NormalTok{.}\FunctionTok{GetReturnValue}\NormalTok{().}\FunctionTok{Set}\NormalTok{(num);}
\NormalTok{\}}

\KeywordTok{void} \FunctionTok{Init}\NormalTok{(Handle<Object> exports) \{}
  \FunctionTok{NODE_SET_METHOD}\NormalTok{(exports, }\StringTok{"add"}\NormalTok{, Add);}
\NormalTok{\}}

\FunctionTok{NODE_MODULE}\NormalTok{(addon, Init)}
\end{Highlighting}
\end{Shaded}

You can test it with the following JavaScript snippet:

\begin{Shaded}
\begin{Highlighting}[]
\CommentTok{// test.js}
\KeywordTok{var} \NormalTok{addon = }\FunctionTok{require}\NormalTok{(}\StringTok{'./build/Release/addon'}\NormalTok{);}

\OtherTok{console}\NormalTok{.}\FunctionTok{log}\NormalTok{( }\StringTok{'This should be eight:'}\NormalTok{, }\OtherTok{addon}\NormalTok{.}\FunctionTok{add}\NormalTok{(}\DecValTok{3}\NormalTok{,}\DecValTok{5}\NormalTok{) );}
\end{Highlighting}
\end{Shaded}

\subsubsection{Callbacks}\label{callbacks}

You can pass JavaScript functions to a C++ function and execute them
from there. Here's \texttt{addon.cc}:

\begin{Shaded}
\begin{Highlighting}[]
\CommentTok{// addon.cc}
\NormalTok{#include <}\OtherTok{node}\NormalTok{.}\FunctionTok{h}\NormalTok{>}

\NormalTok{using namespace v8;}

\KeywordTok{void} \FunctionTok{RunCallback}\NormalTok{(}\KeywordTok{const} \NormalTok{FunctionCallbackInfo<Value>& args) \{}
  \NormalTok{Isolate* isolate = }\DataTypeTok{Isolate}\NormalTok{::}\FunctionTok{GetCurrent}\NormalTok{();}
  \NormalTok{HandleScope }\FunctionTok{scope}\NormalTok{(isolate);}

  \NormalTok{Local<Function> cb = Local<Function>::}\FunctionTok{Cast}\NormalTok{(args[}\DecValTok{0}\NormalTok{]);}
  \KeywordTok{const} \NormalTok{unsigned argc = }\DecValTok{1}\NormalTok{;}
  \NormalTok{Local<Value> argv[argc] = \{ }\DataTypeTok{String}\NormalTok{::}\FunctionTok{NewFromUtf8}\NormalTok{(isolate, }\StringTok{"hello world"}\NormalTok{) \};}
  \NormalTok{cb->}\FunctionTok{Call}\NormalTok{(}\DataTypeTok{Context}\NormalTok{::}\FunctionTok{GetCurrent}\NormalTok{()->}\FunctionTok{Global}\NormalTok{(), argc, argv);}
\NormalTok{\}}

\KeywordTok{void} \FunctionTok{Init}\NormalTok{(Handle<Object> exports, Handle<Object> module) \{}
  \FunctionTok{NODE_SET_METHOD}\NormalTok{(module, }\StringTok{"exports"}\NormalTok{, RunCallback);}
\NormalTok{\}}

\FunctionTok{NODE_MODULE}\NormalTok{(addon, Init)}
\end{Highlighting}
\end{Shaded}

Note that this example uses a two-argument form of \texttt{Init()} that
receives the full \texttt{module} object as the second argument. This
allows the addon to completely overwrite \texttt{exports} with a single
function instead of adding the function as a property of
\texttt{exports}.

To test it run the following JavaScript snippet:

\begin{Shaded}
\begin{Highlighting}[]
\CommentTok{// test.js}
\KeywordTok{var} \NormalTok{addon = }\FunctionTok{require}\NormalTok{(}\StringTok{'./build/Release/addon'}\NormalTok{);}

\FunctionTok{addon}\NormalTok{(}\KeywordTok{function}\NormalTok{(msg)\{}
  \OtherTok{console}\NormalTok{.}\FunctionTok{log}\NormalTok{(msg); }\CommentTok{// 'hello world'}
\NormalTok{\});}
\end{Highlighting}
\end{Shaded}

\subsubsection{Object factory}\label{object-factory}

You can create and return new objects from within a C++ function with
this \texttt{addon.cc} pattern, which returns an object with property
\texttt{msg} that echoes the string passed to \texttt{createObject()}:

\begin{Shaded}
\begin{Highlighting}[]
\CommentTok{// addon.cc}
\NormalTok{#include <}\OtherTok{node}\NormalTok{.}\FunctionTok{h}\NormalTok{>}

\NormalTok{using namespace v8;}

\KeywordTok{void} \FunctionTok{CreateObject}\NormalTok{(}\KeywordTok{const} \NormalTok{FunctionCallbackInfo<Value>& args) \{}
  \NormalTok{Isolate* isolate = }\DataTypeTok{Isolate}\NormalTok{::}\FunctionTok{GetCurrent}\NormalTok{();}
  \NormalTok{HandleScope }\FunctionTok{scope}\NormalTok{(isolate);}

  \NormalTok{Local<Object> obj = }\DataTypeTok{Object}\NormalTok{::}\FunctionTok{New}\NormalTok{();}
  \NormalTok{obj->}\FunctionTok{Set}\NormalTok{(}\DataTypeTok{String}\NormalTok{::}\FunctionTok{NewFromUtf8}\NormalTok{(isolate, }\StringTok{"msg"}\NormalTok{), args[}\DecValTok{0}\NormalTok{]->}\FunctionTok{ToString}\NormalTok{());}

  \OtherTok{args}\NormalTok{.}\FunctionTok{GetReturnValue}\NormalTok{().}\FunctionTok{Set}\NormalTok{(obj);}
\NormalTok{\}}

\KeywordTok{void} \FunctionTok{Init}\NormalTok{(Handle<Object> exports, Handle<Object> module) \{}
  \FunctionTok{NODE_SET_METHOD}\NormalTok{(module, }\StringTok{"exports"}\NormalTok{, CreateObject);}
\NormalTok{\}}

\FunctionTok{NODE_MODULE}\NormalTok{(addon, Init)}
\end{Highlighting}
\end{Shaded}

To test it in JavaScript:

\begin{Shaded}
\begin{Highlighting}[]
\CommentTok{// test.js}
\KeywordTok{var} \NormalTok{addon = }\FunctionTok{require}\NormalTok{(}\StringTok{'./build/Release/addon'}\NormalTok{);}

\KeywordTok{var} \NormalTok{obj1 = }\FunctionTok{addon}\NormalTok{(}\StringTok{'hello'}\NormalTok{);}
\KeywordTok{var} \NormalTok{obj2 = }\FunctionTok{addon}\NormalTok{(}\StringTok{'world'}\NormalTok{);}
\OtherTok{console}\NormalTok{.}\FunctionTok{log}\NormalTok{(}\OtherTok{obj1}\NormalTok{.}\FunctionTok{msg}\NormalTok{+}\StringTok{' '}\NormalTok{+}\OtherTok{obj2}\NormalTok{.}\FunctionTok{msg}\NormalTok{); }\CommentTok{// 'hello world'}
\end{Highlighting}
\end{Shaded}

\subsubsection{Function factory}\label{function-factory}

This pattern illustrates how to create and return a JavaScript function
that wraps a C++ function:

\begin{Shaded}
\begin{Highlighting}[]
\CommentTok{// addon.cc}
\NormalTok{#include <}\OtherTok{node}\NormalTok{.}\FunctionTok{h}\NormalTok{>}

\NormalTok{using namespace v8;}

\KeywordTok{void} \FunctionTok{MyFunction}\NormalTok{(}\KeywordTok{const} \NormalTok{FunctionCallbackInfo<Value>& args) \{}
  \NormalTok{Isolate* isolate = }\DataTypeTok{Isolate}\NormalTok{::}\FunctionTok{GetCurrent}\NormalTok{();}
  \NormalTok{HandleScope }\FunctionTok{scope}\NormalTok{(isolate);}
  \OtherTok{args}\NormalTok{.}\FunctionTok{GetReturnValue}\NormalTok{().}\FunctionTok{Set}\NormalTok{(}\DataTypeTok{String}\NormalTok{::}\FunctionTok{NewFromUtf8}\NormalTok{(isolate, }\StringTok{"hello world"}\NormalTok{));}
\NormalTok{\}}

\KeywordTok{void} \FunctionTok{CreateFunction}\NormalTok{(}\KeywordTok{const} \NormalTok{FunctionCallbackInfo<Value>& args) \{}
  \NormalTok{Isolate* isolate = }\DataTypeTok{Isolate}\NormalTok{::}\FunctionTok{GetCurrent}\NormalTok{();}
  \NormalTok{HandleScope }\FunctionTok{scope}\NormalTok{(isolate);}

  \NormalTok{Local<FunctionTemplate> tpl = }\DataTypeTok{FunctionTemplate}\NormalTok{::}\FunctionTok{New}\NormalTok{(MyFunction);}
  \NormalTok{Local<Function> fn = tpl->}\FunctionTok{GetFunction}\NormalTok{();}

  \CommentTok{// omit this to make it anonymous}
  \NormalTok{fn->}\FunctionTok{SetName}\NormalTok{(}\DataTypeTok{String}\NormalTok{::}\FunctionTok{NewFromUtf8}\NormalTok{(isolate, }\StringTok{"theFunction"}\NormalTok{));}

  \OtherTok{args}\NormalTok{.}\FunctionTok{GetReturnValue}\NormalTok{().}\FunctionTok{Set}\NormalTok{(fn);}
\NormalTok{\}}

\KeywordTok{void} \FunctionTok{Init}\NormalTok{(Handle<Object> exports, Handle<Object> module) \{}
  \FunctionTok{NODE_SET_METHOD}\NormalTok{(module, }\StringTok{"exports"}\NormalTok{, CreateFunction);}
\NormalTok{\}}

\FunctionTok{NODE_MODULE}\NormalTok{(addon, Init)}
\end{Highlighting}
\end{Shaded}

To test:

\begin{Shaded}
\begin{Highlighting}[]
\CommentTok{// test.js}
\KeywordTok{var} \NormalTok{addon = }\FunctionTok{require}\NormalTok{(}\StringTok{'./build/Release/addon'}\NormalTok{);}

\KeywordTok{var} \NormalTok{fn = }\FunctionTok{addon}\NormalTok{();}
\OtherTok{console}\NormalTok{.}\FunctionTok{log}\NormalTok{(}\FunctionTok{fn}\NormalTok{()); }\CommentTok{// 'hello world'}
\end{Highlighting}
\end{Shaded}

\subsubsection{Wrapping C++ objects}\label{wrapping-c-objects}

Here we will create a wrapper for a C++ object/class \texttt{MyObject}
that can be instantiated in JavaScript through the \texttt{new}
operator. First prepare the main module \texttt{addon.cc}:

\begin{Shaded}
\begin{Highlighting}[]
\CommentTok{// addon.cc}
\NormalTok{#include <}\OtherTok{node}\NormalTok{.}\FunctionTok{h}\NormalTok{>}
\NormalTok{#include }\StringTok{"myobject.h"}

\NormalTok{using namespace v8;}

\KeywordTok{void} \FunctionTok{InitAll}\NormalTok{(Handle<Object> exports) \{}
  \DataTypeTok{MyObject}\NormalTok{::}\FunctionTok{Init}\NormalTok{(exports);}
\NormalTok{\}}

\FunctionTok{NODE_MODULE}\NormalTok{(addon, InitAll)}
\end{Highlighting}
\end{Shaded}

Then in \texttt{myobject.h} make your wrapper inherit from
\texttt{node::ObjectWrap}:

\begin{Shaded}
\begin{Highlighting}[]
\CommentTok{// myobject.h}
\NormalTok{#ifndef MYOBJECT_H}
\NormalTok{#define MYOBJECT_H}

\NormalTok{#include <}\OtherTok{node}\NormalTok{.}\FunctionTok{h}\NormalTok{>}
\NormalTok{#include <}\OtherTok{node_object_wrap}\NormalTok{.}\FunctionTok{h}\NormalTok{>}

\KeywordTok{class} \NormalTok{MyObject : }\KeywordTok{public} \NormalTok{node::ObjectWrap \{}
 \DataTypeTok{public}\NormalTok{:}
  \KeywordTok{static} \KeywordTok{void} \FunctionTok{Init}\NormalTok{(}\DataTypeTok{v8}\NormalTok{::Handle<}\DataTypeTok{v8}\NormalTok{::Object> exports);}

 \DataTypeTok{private}\NormalTok{:}
  \NormalTok{explicit }\FunctionTok{MyObject}\NormalTok{(double value = }\DecValTok{0}\NormalTok{);}
  \NormalTok{~}\FunctionTok{MyObject}\NormalTok{();}

  \KeywordTok{static} \KeywordTok{void} \FunctionTok{New}\NormalTok{(}\KeywordTok{const} \DataTypeTok{v8}\NormalTok{::FunctionCallbackInfo<}\DataTypeTok{v8}\NormalTok{::Value>& args);}
  \KeywordTok{static} \KeywordTok{void} \FunctionTok{PlusOne}\NormalTok{(}\KeywordTok{const} \DataTypeTok{v8}\NormalTok{::FunctionCallbackInfo<}\DataTypeTok{v8}\NormalTok{::Value>& args);}
  \KeywordTok{static} \DataTypeTok{v8}\NormalTok{::Persistent<}\DataTypeTok{v8}\NormalTok{::Function> constructor;}
  \NormalTok{double value_;}
\NormalTok{\};}

\NormalTok{#endif}
\end{Highlighting}
\end{Shaded}

And in \texttt{myobject.cc} implement the various methods that you want
to expose. Here we expose the method \texttt{plusOne} by adding it to
the constructor's prototype:

\begin{Shaded}
\begin{Highlighting}[]
\CommentTok{// myobject.cc}
\NormalTok{#include }\StringTok{"myobject.h"}

\NormalTok{using namespace v8;}

\NormalTok{Persistent<Function> MyObject::constructor;}

\NormalTok{MyObject::}\FunctionTok{MyObject}\NormalTok{(double value) : }\FunctionTok{value_}\NormalTok{(value) \{}
\NormalTok{\}}

\NormalTok{MyObject::~}\FunctionTok{MyObject}\NormalTok{() \{}
\NormalTok{\}}

\KeywordTok{void} \NormalTok{MyObject::}\FunctionTok{Init}\NormalTok{(Handle<Object> exports) \{}
  \NormalTok{Isolate* isolate = }\DataTypeTok{Isolate}\NormalTok{::}\FunctionTok{GetCurrent}\NormalTok{();}

  \CommentTok{// Prepare constructor template}
  \NormalTok{Local<FunctionTemplate> tpl = }\DataTypeTok{FunctionTemplate}\NormalTok{::}\FunctionTok{New}\NormalTok{(New);}
  \NormalTok{tpl->}\FunctionTok{SetClassName}\NormalTok{(}\DataTypeTok{String}\NormalTok{::}\FunctionTok{NewFromUtf8}\NormalTok{(isolate, }\StringTok{"MyObject"}\NormalTok{));}
  \NormalTok{tpl->}\FunctionTok{InstanceTemplate}\NormalTok{()->}\FunctionTok{SetInternalFieldCount}\NormalTok{(}\DecValTok{1}\NormalTok{);}

  \CommentTok{// Prototype}
  \FunctionTok{NODE_SET_PROTOTYPE_METHOD}\NormalTok{(tpl, }\StringTok{"plusOne"}\NormalTok{, PlusOne);}

  \OtherTok{constructor}\NormalTok{.}\FunctionTok{Reset}\NormalTok{(isolate, tpl->}\FunctionTok{GetFunction}\NormalTok{());}
  \NormalTok{exports->}\FunctionTok{Set}\NormalTok{(}\DataTypeTok{String}\NormalTok{::}\FunctionTok{NewFromUtf8}\NormalTok{(isolate, }\StringTok{"MyObject"}\NormalTok{),}
               \NormalTok{tpl->}\FunctionTok{GetFunction}\NormalTok{());}
\NormalTok{\}}

\KeywordTok{void} \NormalTok{MyObject::}\FunctionTok{New}\NormalTok{(}\KeywordTok{const} \NormalTok{FunctionCallbackInfo<Value>& args) \{}
  \NormalTok{Isolate* isolate = }\DataTypeTok{Isolate}\NormalTok{::}\FunctionTok{GetCurrent}\NormalTok{();}
  \NormalTok{HandleScope }\FunctionTok{scope}\NormalTok{(isolate);}

  \KeywordTok{if} \NormalTok{(}\OtherTok{args}\NormalTok{.}\FunctionTok{IsConstructCall}\NormalTok{()) \{}
    \CommentTok{// Invoked as constructor: `new MyObject(...)`}
    \NormalTok{double value = args[}\DecValTok{0}\NormalTok{]->}\FunctionTok{IsUndefined}\NormalTok{() ? }\DecValTok{0} \NormalTok{: args[}\DecValTok{0}\NormalTok{]->}\FunctionTok{NumberValue}\NormalTok{();}
    \NormalTok{MyObject* obj = }\KeywordTok{new} \FunctionTok{MyObject}\NormalTok{(value);}
    \NormalTok{obj->}\FunctionTok{Wrap}\NormalTok{(}\OtherTok{args}\NormalTok{.}\FunctionTok{This}\NormalTok{());}
    \OtherTok{args}\NormalTok{.}\FunctionTok{GetReturnValue}\NormalTok{().}\FunctionTok{Set}\NormalTok{(}\OtherTok{args}\NormalTok{.}\FunctionTok{This}\NormalTok{());}
  \NormalTok{\} }\KeywordTok{else} \NormalTok{\{}
    \CommentTok{// Invoked as plain function `MyObject(...)`, turn into construct call.}
    \KeywordTok{const} \NormalTok{int argc = }\DecValTok{1}\NormalTok{;}
    \NormalTok{Local<Value> argv[argc] = \{ args[}\DecValTok{0}\NormalTok{] \};}
    \NormalTok{Local<Function> cons = Local<Function>::}\FunctionTok{New}\NormalTok{(isolate, constructor);}
    \OtherTok{args}\NormalTok{.}\FunctionTok{GetReturnValue}\NormalTok{().}\FunctionTok{Set}\NormalTok{(cons->}\FunctionTok{NewInstance}\NormalTok{(argc, argv));}
  \NormalTok{\}}
\NormalTok{\}}

\KeywordTok{void} \NormalTok{MyObject::}\FunctionTok{PlusOne}\NormalTok{(}\KeywordTok{const} \NormalTok{FunctionCallbackInfo<Value>& args) \{}
  \NormalTok{Isolate* isolate = }\DataTypeTok{Isolate}\NormalTok{::}\FunctionTok{GetCurrent}\NormalTok{();}
  \NormalTok{HandleScope }\FunctionTok{scope}\NormalTok{(isolate);}

  \NormalTok{MyObject* obj = }\DataTypeTok{ObjectWrap}\NormalTok{::Unwrap<MyObject>(}\OtherTok{args}\NormalTok{.}\FunctionTok{This}\NormalTok{());}
  \NormalTok{obj->value_ += }\DecValTok{1}\NormalTok{;}

  \OtherTok{args}\NormalTok{.}\FunctionTok{GetReturnValue}\NormalTok{().}\FunctionTok{Set}\NormalTok{(}\DataTypeTok{Number}\NormalTok{::}\FunctionTok{New}\NormalTok{(obj->value_));}
\NormalTok{\}}
\end{Highlighting}
\end{Shaded}

Test it with:

\begin{Shaded}
\begin{Highlighting}[]
\CommentTok{// test.js}
\KeywordTok{var} \NormalTok{addon = }\FunctionTok{require}\NormalTok{(}\StringTok{'./build/Release/addon'}\NormalTok{);}

\KeywordTok{var} \NormalTok{obj = }\KeywordTok{new} \OtherTok{addon}\NormalTok{.}\FunctionTok{MyObject}\NormalTok{(}\DecValTok{10}\NormalTok{);}
\OtherTok{console}\NormalTok{.}\FunctionTok{log}\NormalTok{( }\OtherTok{obj}\NormalTok{.}\FunctionTok{plusOne}\NormalTok{() ); }\CommentTok{// 11}
\OtherTok{console}\NormalTok{.}\FunctionTok{log}\NormalTok{( }\OtherTok{obj}\NormalTok{.}\FunctionTok{plusOne}\NormalTok{() ); }\CommentTok{// 12}
\OtherTok{console}\NormalTok{.}\FunctionTok{log}\NormalTok{( }\OtherTok{obj}\NormalTok{.}\FunctionTok{plusOne}\NormalTok{() ); }\CommentTok{// 13}
\end{Highlighting}
\end{Shaded}

\subsubsection{Factory of wrapped
objects}\label{factory-of-wrapped-objects}

This is useful when you want to be able to create native objects without
explicitly instantiating them with the \texttt{new} operator in
JavaScript, e.g.

\begin{Shaded}
\begin{Highlighting}[]
\KeywordTok{var} \NormalTok{obj = }\OtherTok{addon}\NormalTok{.}\FunctionTok{createObject}\NormalTok{();}
\CommentTok{// instead of:}
\CommentTok{// var obj = new addon.Object();}
\end{Highlighting}
\end{Shaded}

Let's register our \texttt{createObject} method in \texttt{addon.cc}:

\begin{Shaded}
\begin{Highlighting}[]
\CommentTok{// addon.cc}
\NormalTok{#include <}\OtherTok{node}\NormalTok{.}\FunctionTok{h}\NormalTok{>}
\NormalTok{#include }\StringTok{"myobject.h"}

\NormalTok{using namespace v8;}

\KeywordTok{void} \FunctionTok{CreateObject}\NormalTok{(}\KeywordTok{const} \NormalTok{FunctionCallbackInfo<Value>& args) \{}
  \NormalTok{Isolate* isolate = }\DataTypeTok{Isolate}\NormalTok{::}\FunctionTok{GetCurrent}\NormalTok{();}
  \NormalTok{HandleScope }\FunctionTok{scope}\NormalTok{(isolate);}
  \DataTypeTok{MyObject}\NormalTok{::}\FunctionTok{NewInstance}\NormalTok{(args);}
\NormalTok{\}}

\KeywordTok{void} \FunctionTok{InitAll}\NormalTok{(Handle<Object> exports, Handle<Object> module) \{}
  \DataTypeTok{MyObject}\NormalTok{::}\FunctionTok{Init}\NormalTok{();}

  \FunctionTok{NODE_SET_METHOD}\NormalTok{(module, }\StringTok{"exports"}\NormalTok{, CreateObject);}
\NormalTok{\}}

\FunctionTok{NODE_MODULE}\NormalTok{(addon, InitAll)}
\end{Highlighting}
\end{Shaded}

In \texttt{myobject.h} we now introduce the static method
\texttt{NewInstance} that takes care of instantiating the object
(i.e.~it does the job of \texttt{new} in JavaScript):

\begin{Shaded}
\begin{Highlighting}[]
\CommentTok{// myobject.h}
\NormalTok{#ifndef MYOBJECT_H}
\NormalTok{#define MYOBJECT_H}

\NormalTok{#include <}\OtherTok{node}\NormalTok{.}\FunctionTok{h}\NormalTok{>}
\NormalTok{#include <}\OtherTok{node_object_wrap}\NormalTok{.}\FunctionTok{h}\NormalTok{>}

\KeywordTok{class} \NormalTok{MyObject : }\KeywordTok{public} \NormalTok{node::ObjectWrap \{}
 \DataTypeTok{public}\NormalTok{:}
  \KeywordTok{static} \KeywordTok{void} \FunctionTok{Init}\NormalTok{();}
  \KeywordTok{static} \KeywordTok{void} \FunctionTok{NewInstance}\NormalTok{(}\KeywordTok{const} \DataTypeTok{v8}\NormalTok{::FunctionCallbackInfo<}\DataTypeTok{v8}\NormalTok{::Value>& args);}

 \DataTypeTok{private}\NormalTok{:}
  \NormalTok{explicit }\FunctionTok{MyObject}\NormalTok{(double value = }\DecValTok{0}\NormalTok{);}
  \NormalTok{~}\FunctionTok{MyObject}\NormalTok{();}

  \KeywordTok{static} \KeywordTok{void} \FunctionTok{New}\NormalTok{(}\KeywordTok{const} \DataTypeTok{v8}\NormalTok{::FunctionCallbackInfo<}\DataTypeTok{v8}\NormalTok{::Value>& args);}
  \KeywordTok{static} \KeywordTok{void} \FunctionTok{PlusOne}\NormalTok{(}\KeywordTok{const} \DataTypeTok{v8}\NormalTok{::FunctionCallbackInfo<}\DataTypeTok{v8}\NormalTok{::Value>& args);}
  \KeywordTok{static} \DataTypeTok{v8}\NormalTok{::Persistent<}\DataTypeTok{v8}\NormalTok{::Function> constructor;}
  \NormalTok{double value_;}
\NormalTok{\};}

\NormalTok{#endif}
\end{Highlighting}
\end{Shaded}

The implementation is similar to the above in \texttt{myobject.cc}:

\begin{Shaded}
\begin{Highlighting}[]
\CommentTok{// myobject.cc}
\NormalTok{#include <}\OtherTok{node}\NormalTok{.}\FunctionTok{h}\NormalTok{>}
\NormalTok{#include }\StringTok{"myobject.h"}

\NormalTok{using namespace v8;}

\NormalTok{Persistent<Function> MyObject::constructor;}

\NormalTok{MyObject::}\FunctionTok{MyObject}\NormalTok{(double value) : }\FunctionTok{value_}\NormalTok{(value) \{}
\NormalTok{\}}

\NormalTok{MyObject::~}\FunctionTok{MyObject}\NormalTok{() \{}
\NormalTok{\}}

\KeywordTok{void} \NormalTok{MyObject::}\FunctionTok{Init}\NormalTok{() \{}
  \NormalTok{Isolate* isolate = }\DataTypeTok{Isolate}\NormalTok{::}\FunctionTok{GetCurrent}\NormalTok{();}
  \CommentTok{// Prepare constructor template}
  \NormalTok{Local<FunctionTemplate> tpl = }\DataTypeTok{FunctionTemplate}\NormalTok{::}\FunctionTok{New}\NormalTok{(New);}
  \NormalTok{tpl->}\FunctionTok{SetClassName}\NormalTok{(}\DataTypeTok{String}\NormalTok{::}\FunctionTok{NewFromUtf8}\NormalTok{(isolate, }\StringTok{"MyObject"}\NormalTok{));}
  \NormalTok{tpl->}\FunctionTok{InstanceTemplate}\NormalTok{()->}\FunctionTok{SetInternalFieldCount}\NormalTok{(}\DecValTok{1}\NormalTok{);}

  \CommentTok{// Prototype}
  \FunctionTok{NODE_SET_PROTOTYPE_METHOD}\NormalTok{(tpl, }\StringTok{"plusOne"}\NormalTok{, PlusOne);}

  \OtherTok{constructor}\NormalTok{.}\FunctionTok{Reset}\NormalTok{(isolate, tpl->}\FunctionTok{GetFunction}\NormalTok{());}
\NormalTok{\}}

\KeywordTok{void} \NormalTok{MyObject::}\FunctionTok{New}\NormalTok{(}\KeywordTok{const} \NormalTok{FunctionCallbackInfo<Value>& args) \{}
  \NormalTok{Isolate* isolate = }\DataTypeTok{Isolate}\NormalTok{::}\FunctionTok{GetCurrent}\NormalTok{();}
  \NormalTok{HandleScope }\FunctionTok{scope}\NormalTok{(isolate);}

  \KeywordTok{if} \NormalTok{(}\OtherTok{args}\NormalTok{.}\FunctionTok{IsConstructCall}\NormalTok{()) \{}
    \CommentTok{// Invoked as constructor: `new MyObject(...)`}
    \NormalTok{double value = args[}\DecValTok{0}\NormalTok{]->}\FunctionTok{IsUndefined}\NormalTok{() ? }\DecValTok{0} \NormalTok{: args[}\DecValTok{0}\NormalTok{]->}\FunctionTok{NumberValue}\NormalTok{();}
    \NormalTok{MyObject* obj = }\KeywordTok{new} \FunctionTok{MyObject}\NormalTok{(value);}
    \NormalTok{obj->}\FunctionTok{Wrap}\NormalTok{(}\OtherTok{args}\NormalTok{.}\FunctionTok{This}\NormalTok{());}
    \OtherTok{args}\NormalTok{.}\FunctionTok{GetReturnValue}\NormalTok{().}\FunctionTok{Set}\NormalTok{(}\OtherTok{args}\NormalTok{.}\FunctionTok{This}\NormalTok{());}
  \NormalTok{\} }\KeywordTok{else} \NormalTok{\{}
    \CommentTok{// Invoked as plain function `MyObject(...)`, turn into construct call.}
    \KeywordTok{const} \NormalTok{int argc = }\DecValTok{1}\NormalTok{;}
    \NormalTok{Local<Value> argv[argc] = \{ args[}\DecValTok{0}\NormalTok{] \};}
    \NormalTok{Local<Function> cons = Local<Function>::}\FunctionTok{New}\NormalTok{(isolate, constructor);}
    \OtherTok{args}\NormalTok{.}\FunctionTok{GetReturnValue}\NormalTok{().}\FunctionTok{Set}\NormalTok{(cons->}\FunctionTok{NewInstance}\NormalTok{(argc, argv));}
  \NormalTok{\}}
\NormalTok{\}}

\KeywordTok{void} \NormalTok{MyObject::}\FunctionTok{NewInstance}\NormalTok{(}\KeywordTok{const} \NormalTok{FunctionCallbackInfo<Value>& args) \{}
  \NormalTok{Isolate* isolate = }\DataTypeTok{Isolate}\NormalTok{::}\FunctionTok{GetCurrent}\NormalTok{();}
  \NormalTok{HandleScope }\FunctionTok{scope}\NormalTok{(isolate);}

  \KeywordTok{const} \NormalTok{unsigned argc = }\DecValTok{1}\NormalTok{;}
  \NormalTok{Handle<Value> argv[argc] = \{ args[}\DecValTok{0}\NormalTok{] \};}
  \NormalTok{Local<Function> cons = Local<Function>::}\FunctionTok{New}\NormalTok{(isolate, constructor);}
  \NormalTok{Local<Object> instance = cons->}\FunctionTok{NewInstance}\NormalTok{(argc, argv);}

  \OtherTok{args}\NormalTok{.}\FunctionTok{GetReturnValue}\NormalTok{().}\FunctionTok{Set}\NormalTok{(instance);}
\NormalTok{\}}

\KeywordTok{void} \NormalTok{MyObject::}\FunctionTok{PlusOne}\NormalTok{(}\KeywordTok{const} \NormalTok{FunctionCallbackInfo<Value>& args) \{}
  \NormalTok{Isolate* isolate = }\DataTypeTok{Isolate}\NormalTok{::}\FunctionTok{GetCurrent}\NormalTok{();}
  \NormalTok{HandleScope }\FunctionTok{scope}\NormalTok{(isolate);}

  \NormalTok{MyObject* obj = }\DataTypeTok{ObjectWrap}\NormalTok{::Unwrap<MyObject>(}\OtherTok{args}\NormalTok{.}\FunctionTok{This}\NormalTok{());}
  \NormalTok{obj->value_ += }\DecValTok{1}\NormalTok{;}

  \OtherTok{args}\NormalTok{.}\FunctionTok{GetReturnValue}\NormalTok{().}\FunctionTok{Set}\NormalTok{(}\DataTypeTok{Number}\NormalTok{::}\FunctionTok{New}\NormalTok{(obj->value_));}
\NormalTok{\}}
\end{Highlighting}
\end{Shaded}

Test it with:

\begin{Shaded}
\begin{Highlighting}[]
\CommentTok{// test.js}
\KeywordTok{var} \NormalTok{createObject = }\FunctionTok{require}\NormalTok{(}\StringTok{'./build/Release/addon'}\NormalTok{);}

\KeywordTok{var} \NormalTok{obj = }\FunctionTok{createObject}\NormalTok{(}\DecValTok{10}\NormalTok{);}
\OtherTok{console}\NormalTok{.}\FunctionTok{log}\NormalTok{( }\OtherTok{obj}\NormalTok{.}\FunctionTok{plusOne}\NormalTok{() ); }\CommentTok{// 11}
\OtherTok{console}\NormalTok{.}\FunctionTok{log}\NormalTok{( }\OtherTok{obj}\NormalTok{.}\FunctionTok{plusOne}\NormalTok{() ); }\CommentTok{// 12}
\OtherTok{console}\NormalTok{.}\FunctionTok{log}\NormalTok{( }\OtherTok{obj}\NormalTok{.}\FunctionTok{plusOne}\NormalTok{() ); }\CommentTok{// 13}

\KeywordTok{var} \NormalTok{obj2 = }\FunctionTok{createObject}\NormalTok{(}\DecValTok{20}\NormalTok{);}
\OtherTok{console}\NormalTok{.}\FunctionTok{log}\NormalTok{( }\OtherTok{obj2}\NormalTok{.}\FunctionTok{plusOne}\NormalTok{() ); }\CommentTok{// 21}
\OtherTok{console}\NormalTok{.}\FunctionTok{log}\NormalTok{( }\OtherTok{obj2}\NormalTok{.}\FunctionTok{plusOne}\NormalTok{() ); }\CommentTok{// 22}
\OtherTok{console}\NormalTok{.}\FunctionTok{log}\NormalTok{( }\OtherTok{obj2}\NormalTok{.}\FunctionTok{plusOne}\NormalTok{() ); }\CommentTok{// 23}
\end{Highlighting}
\end{Shaded}

\subsubsection{Passing wrapped objects
around}\label{passing-wrapped-objects-around}

In addition to wrapping and returning C++ objects, you can pass them
around by unwrapping them with Node's \texttt{node::ObjectWrap::Unwrap}
helper function. In the following \texttt{addon.cc} we introduce a
function \texttt{add()} that can take on two \texttt{MyObject} objects:

\begin{Shaded}
\begin{Highlighting}[]
\CommentTok{// addon.cc}
\NormalTok{#include <}\OtherTok{node}\NormalTok{.}\FunctionTok{h}\NormalTok{>}
\NormalTok{#include <}\OtherTok{node_object_wrap}\NormalTok{.}\FunctionTok{h}\NormalTok{>}
\NormalTok{#include }\StringTok{"myobject.h"}

\NormalTok{using namespace v8;}

\KeywordTok{void} \FunctionTok{CreateObject}\NormalTok{(}\KeywordTok{const} \NormalTok{FunctionCallbackInfo<Value>& args) \{}
  \NormalTok{Isolate* isolate = }\DataTypeTok{Isolate}\NormalTok{::}\FunctionTok{GetCurrent}\NormalTok{();}
  \NormalTok{HandleScope }\FunctionTok{scope}\NormalTok{(isolate);}
  \DataTypeTok{MyObject}\NormalTok{::}\FunctionTok{NewInstance}\NormalTok{(args);}
\NormalTok{\}}

\KeywordTok{void} \FunctionTok{Add}\NormalTok{(}\KeywordTok{const} \NormalTok{FunctionCallbackInfo<Value>& args) \{}
  \NormalTok{Isolate* isolate = }\DataTypeTok{Isolate}\NormalTok{::}\FunctionTok{GetCurrent}\NormalTok{();}
  \NormalTok{HandleScope }\FunctionTok{scope}\NormalTok{(isolate);}

  \NormalTok{MyObject* obj1 = }\DataTypeTok{node}\NormalTok{::}\DataTypeTok{ObjectWrap}\NormalTok{::Unwrap<MyObject>(}
      \NormalTok{args[}\DecValTok{0}\NormalTok{]->}\FunctionTok{ToObject}\NormalTok{());}
  \NormalTok{MyObject* obj2 = }\DataTypeTok{node}\NormalTok{::}\DataTypeTok{ObjectWrap}\NormalTok{::Unwrap<MyObject>(}
      \NormalTok{args[}\DecValTok{1}\NormalTok{]->}\FunctionTok{ToObject}\NormalTok{());}

  \NormalTok{double sum = obj1->}\FunctionTok{value}\NormalTok{() + obj2->}\FunctionTok{value}\NormalTok{();}
  \OtherTok{args}\NormalTok{.}\FunctionTok{GetReturnValue}\NormalTok{().}\FunctionTok{Set}\NormalTok{(}\DataTypeTok{Number}\NormalTok{::}\FunctionTok{New}\NormalTok{(sum));}
\NormalTok{\}}

\KeywordTok{void} \FunctionTok{InitAll}\NormalTok{(Handle<Object> exports) \{}
  \DataTypeTok{MyObject}\NormalTok{::}\FunctionTok{Init}\NormalTok{();}

  \FunctionTok{NODE_SET_METHOD}\NormalTok{(exports, }\StringTok{"createObject"}\NormalTok{, CreateObject);}
  \FunctionTok{NODE_SET_METHOD}\NormalTok{(exports, }\StringTok{"add"}\NormalTok{, Add);}
\NormalTok{\}}

\FunctionTok{NODE_MODULE}\NormalTok{(addon, InitAll)}
\end{Highlighting}
\end{Shaded}

To make things interesting we introduce a public method in
\texttt{myobject.h} so we can probe private values after unwrapping the
object:

\begin{Shaded}
\begin{Highlighting}[]
\CommentTok{// myobject.h}
\NormalTok{#ifndef MYOBJECT_H}
\NormalTok{#define MYOBJECT_H}

\NormalTok{#include <}\OtherTok{node}\NormalTok{.}\FunctionTok{h}\NormalTok{>}
\NormalTok{#include <}\OtherTok{node_object_wrap}\NormalTok{.}\FunctionTok{h}\NormalTok{>}

\KeywordTok{class} \NormalTok{MyObject : }\KeywordTok{public} \NormalTok{node::ObjectWrap \{}
 \DataTypeTok{public}\NormalTok{:}
  \KeywordTok{static} \KeywordTok{void} \FunctionTok{Init}\NormalTok{();}
  \KeywordTok{static} \KeywordTok{void} \FunctionTok{NewInstance}\NormalTok{(}\KeywordTok{const} \DataTypeTok{v8}\NormalTok{::FunctionCallbackInfo<}\DataTypeTok{v8}\NormalTok{::Value>& args);}
  \NormalTok{inline double }\FunctionTok{value}\NormalTok{() }\KeywordTok{const} \NormalTok{\{ }\KeywordTok{return} \NormalTok{value_; \}}

 \DataTypeTok{private}\NormalTok{:}
  \NormalTok{explicit }\FunctionTok{MyObject}\NormalTok{(double value = }\DecValTok{0}\NormalTok{);}
  \NormalTok{~}\FunctionTok{MyObject}\NormalTok{();}

  \KeywordTok{static} \KeywordTok{void} \FunctionTok{New}\NormalTok{(}\KeywordTok{const} \DataTypeTok{v8}\NormalTok{::FunctionCallbackInfo<}\DataTypeTok{v8}\NormalTok{::Value>& args);}
  \KeywordTok{static} \DataTypeTok{v8}\NormalTok{::Persistent<}\DataTypeTok{v8}\NormalTok{::Function> constructor;}
  \NormalTok{double value_;}
\NormalTok{\};}

\NormalTok{#endif}
\end{Highlighting}
\end{Shaded}

The implementation of \texttt{myobject.cc} is similar as before:

\begin{Shaded}
\begin{Highlighting}[]
\CommentTok{// myobject.cc}
\NormalTok{#include <}\OtherTok{node}\NormalTok{.}\FunctionTok{h}\NormalTok{>}
\NormalTok{#include }\StringTok{"myobject.h"}

\NormalTok{using namespace v8;}

\NormalTok{Persistent<Function> MyObject::constructor;}

\NormalTok{MyObject::}\FunctionTok{MyObject}\NormalTok{(double value) : }\FunctionTok{value_}\NormalTok{(value) \{}
\NormalTok{\}}

\NormalTok{MyObject::~}\FunctionTok{MyObject}\NormalTok{() \{}
\NormalTok{\}}

\KeywordTok{void} \NormalTok{MyObject::}\FunctionTok{Init}\NormalTok{() \{}
  \NormalTok{Isolate* isolate = }\DataTypeTok{Isolate}\NormalTok{::}\FunctionTok{GetCurrent}\NormalTok{();}

  \CommentTok{// Prepare constructor template}
  \NormalTok{Local<FunctionTemplate> tpl = }\DataTypeTok{FunctionTemplate}\NormalTok{::}\FunctionTok{New}\NormalTok{(New);}
  \NormalTok{tpl->}\FunctionTok{SetClassName}\NormalTok{(}\DataTypeTok{String}\NormalTok{::}\FunctionTok{NewFromUtf8}\NormalTok{(isolate, }\StringTok{"MyObject"}\NormalTok{));}
  \NormalTok{tpl->}\FunctionTok{InstanceTemplate}\NormalTok{()->}\FunctionTok{SetInternalFieldCount}\NormalTok{(}\DecValTok{1}\NormalTok{);}

  \OtherTok{constructor}\NormalTok{.}\FunctionTok{Reset}\NormalTok{(isolate, tpl->}\FunctionTok{GetFunction}\NormalTok{());}
\NormalTok{\}}

\KeywordTok{void} \NormalTok{MyObject::}\FunctionTok{New}\NormalTok{(}\KeywordTok{const} \NormalTok{FunctionCallbackInfo<Value>& args) \{}
  \NormalTok{Isolate* isolate = }\DataTypeTok{Isolate}\NormalTok{::}\FunctionTok{GetCurrent}\NormalTok{();}
  \NormalTok{HandleScope }\FunctionTok{scope}\NormalTok{(isolate);}

  \KeywordTok{if} \NormalTok{(}\OtherTok{args}\NormalTok{.}\FunctionTok{IsConstructCall}\NormalTok{()) \{}
    \CommentTok{// Invoked as constructor: `new MyObject(...)`}
    \NormalTok{double value = args[}\DecValTok{0}\NormalTok{]->}\FunctionTok{IsUndefined}\NormalTok{() ? }\DecValTok{0} \NormalTok{: args[}\DecValTok{0}\NormalTok{]->}\FunctionTok{NumberValue}\NormalTok{();}
    \NormalTok{MyObject* obj = }\KeywordTok{new} \FunctionTok{MyObject}\NormalTok{(value);}
    \NormalTok{obj->}\FunctionTok{Wrap}\NormalTok{(}\OtherTok{args}\NormalTok{.}\FunctionTok{This}\NormalTok{());}
    \OtherTok{args}\NormalTok{.}\FunctionTok{GetReturnValue}\NormalTok{().}\FunctionTok{Set}\NormalTok{(}\OtherTok{args}\NormalTok{.}\FunctionTok{This}\NormalTok{());}
  \NormalTok{\} }\KeywordTok{else} \NormalTok{\{}
    \CommentTok{// Invoked as plain function `MyObject(...)`, turn into construct call.}
    \KeywordTok{const} \NormalTok{int argc = }\DecValTok{1}\NormalTok{;}
    \NormalTok{Local<Value> argv[argc] = \{ args[}\DecValTok{0}\NormalTok{] \};}
    \NormalTok{Local<Function> cons = Local<Function>::}\FunctionTok{New}\NormalTok{(isolate, constructor);}
    \OtherTok{args}\NormalTok{.}\FunctionTok{GetReturnValue}\NormalTok{().}\FunctionTok{Set}\NormalTok{(cons->}\FunctionTok{NewInstance}\NormalTok{(argc, argv));}
  \NormalTok{\}}
\NormalTok{\}}

\KeywordTok{void} \NormalTok{MyObject::}\FunctionTok{NewInstance}\NormalTok{(}\KeywordTok{const} \NormalTok{FunctionCallbackInfo<Value>& args) \{}
  \NormalTok{Isolate* isolate = }\DataTypeTok{Isolate}\NormalTok{::}\FunctionTok{GetCurrent}\NormalTok{();}
  \NormalTok{HandleScope }\FunctionTok{scope}\NormalTok{(isolate);}

  \KeywordTok{const} \NormalTok{unsigned argc = }\DecValTok{1}\NormalTok{;}
  \NormalTok{Handle<Value> argv[argc] = \{ args[}\DecValTok{0}\NormalTok{] \};}
  \NormalTok{Local<Function> cons = Local<Function>::}\FunctionTok{New}\NormalTok{(isolate, constructor);}
  \NormalTok{Local<Object> instance = cons->}\FunctionTok{NewInstance}\NormalTok{(argc, argv);}

  \OtherTok{args}\NormalTok{.}\FunctionTok{GetReturnValue}\NormalTok{().}\FunctionTok{Set}\NormalTok{(instance);}
\NormalTok{\}}
\end{Highlighting}
\end{Shaded}

Test it with:

\begin{Shaded}
\begin{Highlighting}[]
\CommentTok{// test.js}
\KeywordTok{var} \NormalTok{addon = }\FunctionTok{require}\NormalTok{(}\StringTok{'./build/Release/addon'}\NormalTok{);}

\KeywordTok{var} \NormalTok{obj1 = }\OtherTok{addon}\NormalTok{.}\FunctionTok{createObject}\NormalTok{(}\DecValTok{10}\NormalTok{);}
\KeywordTok{var} \NormalTok{obj2 = }\OtherTok{addon}\NormalTok{.}\FunctionTok{createObject}\NormalTok{(}\DecValTok{20}\NormalTok{);}
\KeywordTok{var} \NormalTok{result = }\OtherTok{addon}\NormalTok{.}\FunctionTok{add}\NormalTok{(obj1, obj2);}

\OtherTok{console}\NormalTok{.}\FunctionTok{log}\NormalTok{(result); }\CommentTok{// 30}
\end{Highlighting}
\end{Shaded}

