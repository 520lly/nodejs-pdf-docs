\section{Addons}

Addons are dynamically linked shared objects. They can provide glue to C
and C++ libraries. The API (at the moment) is rather complex, involving
knowledge of several libraries:

\begin{itemize}
\item
  V8 JavaScript, a C++ library. Used for interfacing with JavaScript:
  creating objects, calling functions, etc. Documented mostly in the
  \texttt{v8.h} header file (\texttt{deps/v8/include/v8.h} in the Node
  source tree), which is also available
  \href{http://izs.me/v8-docs/main.html}{online}.
\item
  \href{https://github.com/joyent/libuv}{libuv}, C event loop library.
  Anytime one needs to wait for a file descriptor to become readable,
  wait for a timer, or wait for a signal to received one will need to
  interface with libuv. That is, if you perform any I/O, libuv will need
  to be used.
\item
  Internal Node libraries. Most importantly is the
  \texttt{node::ObjectWrap} class which you will likely want to derive
  from.
\item
  Others. Look in \texttt{deps/} for what else is available.
\end{itemize}

Node statically compiles all its dependencies into the executable. When
compiling your module, you don't need to worry about linking to any of
these libraries.

\subsection{Hello world}

To get started let's make a small Addon which is the C++ equivalent of
the following JavaScript code:

\begin{Shaded}
\begin{Highlighting}[]
\KeywordTok{exports}\NormalTok{.}\FunctionTok{hello} \NormalTok{= }\KeywordTok{function}\NormalTok{() \{ }\KeywordTok{return} \CharTok{'world'}\NormalTok{; \};}
\end{Highlighting}
\end{Shaded}

First we create a file \texttt{hello.cc}:

\begin{Shaded}
\begin{Highlighting}[]
\NormalTok{#include <}\KeywordTok{node}\NormalTok{.}\FunctionTok{h}\NormalTok{>}
\NormalTok{#include <}\KeywordTok{v8}\NormalTok{.}\FunctionTok{h}\NormalTok{>}

\NormalTok{using namespace v8;}

\NormalTok{Handle<Value> Method(}\KeywordTok{const} \NormalTok{Arguments& args) \{}
  \NormalTok{HandleScope scope;}
  \KeywordTok{return} \KeywordTok{scope}\NormalTok{.}\FunctionTok{Close}\NormalTok{(}\DataTypeTok{String}\NormalTok{::New(}\StringTok{"world"}\NormalTok{));}
\NormalTok{\}}

\KeywordTok{void} \NormalTok{init(Handle<}\KeywordTok{Object}\NormalTok{> target) \{}
  \NormalTok{target->Set(}\DataTypeTok{String}\NormalTok{::NewSymbol(}\StringTok{"hello"}\NormalTok{),}
      \DataTypeTok{FunctionTemplate}\NormalTok{::New(Method)->GetFunction());}
\NormalTok{\}}
\NormalTok{NODE_MODULE(hello, init)}
\end{Highlighting}
\end{Shaded}

Note that all Node addons must export an initialization function:

\begin{Shaded}
\begin{Highlighting}[]
\KeywordTok{void} \NormalTok{Initialize (Handle<}\KeywordTok{Object}\NormalTok{> target);}
\NormalTok{NODE_MODULE(module_name, Initialize)}
\end{Highlighting}
\end{Shaded}

There is no semi-colon after \texttt{NODE\_MODULE} as it's not a
function (see \texttt{node.h}).

The \texttt{module\_name} needs to match the filename of the final
binary (minus the .node suffix).

The source code needs to be built into \texttt{hello.node}, the binary
Addon. To do this we create a file called \texttt{binding.gyp} which
describes the configuration to build your module in a JSON-like format.
This file gets compiled by
\href{https://github.com/TooTallNate/node-gyp}{node-gyp}.

\begin{Shaded}
\begin{Highlighting}[]
\NormalTok{\{}
  \StringTok{"targets"}\NormalTok{: [}
    \NormalTok{\{}
      \StringTok{"target_name"}\NormalTok{: }\StringTok{"hello"}\NormalTok{,}
      \StringTok{"sources"}\NormalTok{: [ }\StringTok{"hello.cc"} \NormalTok{]}
    \NormalTok{\}}
  \NormalTok{]}
\NormalTok{\}}
\end{Highlighting}
\end{Shaded}

The next step is to generate the appropriate project build files for the
current platform. Use \texttt{node-gyp configure} for that.

Now you will have either a \texttt{Makefile} (on Unix platforms) or a
\texttt{vcxproj} file (on Windows) in the \texttt{build/} directory.
Next invoke the \texttt{node-gyp build} command.

Now you have your compiled \texttt{.node} bindings file! The compiled
bindings end up in \texttt{build/Release/}.

You can now use the binary addon in a Node project \texttt{hello.js} by
pointing \texttt{require} to the recently built \texttt{hello.node}
module:

\begin{Shaded}
\begin{Highlighting}[]
\KeywordTok{var} \NormalTok{addon = require(}\CharTok{'./build/Release/hello'}\NormalTok{);}

\KeywordTok{console}\NormalTok{.}\FunctionTok{log}\NormalTok{(}\KeywordTok{addon}\NormalTok{.}\FunctionTok{hello}\NormalTok{()); }\CommentTok{// 'world'}
\end{Highlighting}
\end{Shaded}

Please see patterns below for further information or
\url{https://github.com/arturadib/node-qt} for an example in production.

\subsection{Addon patterns}

Below are some addon patterns to help you get started. Consult the
online \href{http://izs.me/v8-docs/main.html}{v8 reference} for help
with the various v8 calls, and v8's
\href{http://code.google.com/apis/v8/embed.html}{Embedder's Guide} for
an explanation of several concepts used such as handles, scopes,
function templates, etc.

In order to use these examples you need to compile them using
\texttt{node-gyp}. Create the following \texttt{binding.gyp} file:

\begin{Shaded}
\begin{Highlighting}[]
\NormalTok{\{}
  \StringTok{"targets"}\NormalTok{: [}
    \NormalTok{\{}
      \StringTok{"target_name"}\NormalTok{: }\StringTok{"addon"}\NormalTok{,}
      \StringTok{"sources"}\NormalTok{: [ }\StringTok{"addon.cc"} \NormalTok{]}
    \NormalTok{\}}
  \NormalTok{]}
\NormalTok{\}}
\end{Highlighting}
\end{Shaded}

In cases where there is more than one \texttt{.cc} file, simply add the
file name to the \texttt{sources} array, e.g.:

\begin{Shaded}
\begin{Highlighting}[]
\StringTok{"sources"}\NormalTok{: [}\StringTok{"addon.cc"}\NormalTok{, }\StringTok{"myexample.cc"}\NormalTok{]}
\end{Highlighting}
\end{Shaded}

Now that you have your \texttt{binding.gyp} ready, you can configure and
build the addon:

\begin{Shaded}
\begin{Highlighting}[]
\NormalTok{$ node-gyp configure build}
\end{Highlighting}
\end{Shaded}

\subsubsection{Function arguments}

The following pattern illustrates how to read arguments from JavaScript
function calls and return a result. This is the main and only needed
source \texttt{addon.cc}:

\begin{Shaded}
\begin{Highlighting}[]
\NormalTok{#define BUILDING_NODE_EXTENSION}
\NormalTok{#include <}\KeywordTok{node}\NormalTok{.}\FunctionTok{h}\NormalTok{>}

\NormalTok{using namespace v8;}

\NormalTok{Handle<Value> Add(}\KeywordTok{const} \NormalTok{Arguments& args) \{}
  \NormalTok{HandleScope scope;}

  \KeywordTok{if} \NormalTok{(}\KeywordTok{args}\NormalTok{.}\FunctionTok{Length}\NormalTok{() < }\DecValTok{2}\NormalTok{) \{}
    \NormalTok{ThrowException(}\DataTypeTok{Exception}\NormalTok{::TypeError(}\DataTypeTok{String}\NormalTok{::New(}\StringTok{"Wrong number of arguments"}\NormalTok{)));}
    \KeywordTok{return} \KeywordTok{scope}\NormalTok{.}\FunctionTok{Close}\NormalTok{(Undefined());}
  \NormalTok{\}}

  \KeywordTok{if} \NormalTok{(!args[}\DecValTok{0}\NormalTok{]->IsNumber() \textbar{}\textbar{} !args[}\DecValTok{1}\NormalTok{]->IsNumber()) \{}
    \NormalTok{ThrowException(}\DataTypeTok{Exception}\NormalTok{::TypeError(}\DataTypeTok{String}\NormalTok{::New(}\StringTok{"Wrong arguments"}\NormalTok{)));}
    \KeywordTok{return} \KeywordTok{scope}\NormalTok{.}\FunctionTok{Close}\NormalTok{(Undefined());}
  \NormalTok{\}}

  \NormalTok{Local<}\FunctionTok{Number}\NormalTok{> num = }\DataTypeTok{Number}\NormalTok{::New(args[}\DecValTok{0}\NormalTok{]->NumberValue() +}
      \NormalTok{args[}\DecValTok{1}\NormalTok{]->NumberValue());}
  \KeywordTok{return} \KeywordTok{scope}\NormalTok{.}\FunctionTok{Close}\NormalTok{(num);}
\NormalTok{\}}

\KeywordTok{void} \NormalTok{Init(Handle<}\KeywordTok{Object}\NormalTok{> target) \{}
  \NormalTok{target->Set(}\DataTypeTok{String}\NormalTok{::NewSymbol(}\StringTok{"add"}\NormalTok{),}
      \DataTypeTok{FunctionTemplate}\NormalTok{::New(Add)->GetFunction());}
\NormalTok{\}}

\NormalTok{NODE_MODULE(addon, Init)}
\end{Highlighting}
\end{Shaded}

You can test it with the following JavaScript snippet:

\begin{Shaded}
\begin{Highlighting}[]
\KeywordTok{var} \NormalTok{addon = require(}\CharTok{'./build/Release/addon'}\NormalTok{);}

\KeywordTok{console}\NormalTok{.}\FunctionTok{log}\NormalTok{( }\CharTok{'This should be eight:'}\NormalTok{, }\KeywordTok{addon}\NormalTok{.}\FunctionTok{add}\NormalTok{(}\DecValTok{3}\NormalTok{,}\DecValTok{5}\NormalTok{) );}
\end{Highlighting}
\end{Shaded}

\subsubsection{Callbacks}

You can pass JavaScript functions to a C++ function and execute them
from there. Here's \texttt{addon.cc}:

\begin{Shaded}
\begin{Highlighting}[]
\NormalTok{#define BUILDING_NODE_EXTENSION}
\NormalTok{#include <}\KeywordTok{node}\NormalTok{.}\FunctionTok{h}\NormalTok{>}

\NormalTok{using namespace v8;}

\NormalTok{Handle<Value> RunCallback(}\KeywordTok{const} \NormalTok{Arguments& args) \{}
  \NormalTok{HandleScope scope;}

  \NormalTok{Local<}\KeywordTok{Function}\NormalTok{> cb = Local<}\KeywordTok{Function}\NormalTok{>::Cast(args[}\DecValTok{0}\NormalTok{]);}
  \KeywordTok{const} \NormalTok{unsigned argc = }\DecValTok{1}\NormalTok{;}
  \NormalTok{Local<Value> argv[argc] = \{ Local<Value>::New(}\DataTypeTok{String}\NormalTok{::New(}\StringTok{"hello world"}\NormalTok{)) \};}
  \NormalTok{cb->Call(}\DataTypeTok{Context}\NormalTok{::GetCurrent()->Global(), argc, argv);}

  \KeywordTok{return} \KeywordTok{scope}\NormalTok{.}\FunctionTok{Close}\NormalTok{(Undefined());}
\NormalTok{\}}

\KeywordTok{void} \NormalTok{Init(Handle<}\KeywordTok{Object}\NormalTok{> target) \{}
  \NormalTok{target->Set(}\DataTypeTok{String}\NormalTok{::NewSymbol(}\StringTok{"runCallback"}\NormalTok{),}
      \DataTypeTok{FunctionTemplate}\NormalTok{::New(RunCallback)->GetFunction());}
\NormalTok{\}}

\NormalTok{NODE_MODULE(addon, Init)}
\end{Highlighting}
\end{Shaded}

To test it run the following JavaScript snippet:

\begin{Shaded}
\begin{Highlighting}[]
\KeywordTok{var} \NormalTok{addon = require(}\CharTok{'./build/Release/addon'}\NormalTok{);}

\KeywordTok{addon}\NormalTok{.}\FunctionTok{runCallback}\NormalTok{(}\KeywordTok{function}\NormalTok{(msg)\{}
  \KeywordTok{console}\NormalTok{.}\FunctionTok{log}\NormalTok{(msg); }\CommentTok{// 'hello world'}
\NormalTok{\});}
\end{Highlighting}
\end{Shaded}

\subsubsection{Object factory}

You can create and return new objects from within a C++ function with
this \texttt{addon.cc} pattern, which returns an object with property
\texttt{msg} that echoes the string passed to \texttt{createObject()}:

\begin{Shaded}
\begin{Highlighting}[]
\NormalTok{#define BUILDING_NODE_EXTENSION}
\NormalTok{#include <}\KeywordTok{node}\NormalTok{.}\FunctionTok{h}\NormalTok{>}

\NormalTok{using namespace v8;}

\NormalTok{Handle<Value> CreateObject(}\KeywordTok{const} \NormalTok{Arguments& args) \{}
  \NormalTok{HandleScope scope;}

  \NormalTok{Local<}\KeywordTok{Object}\NormalTok{> obj = }\DataTypeTok{Object}\NormalTok{::New();}
  \NormalTok{obj->Set(}\DataTypeTok{String}\NormalTok{::NewSymbol(}\StringTok{"msg"}\NormalTok{), args[}\DecValTok{0}\NormalTok{]->ToString());}

  \KeywordTok{return} \KeywordTok{scope}\NormalTok{.}\FunctionTok{Close}\NormalTok{(obj);}
\NormalTok{\}}

\KeywordTok{void} \NormalTok{Init(Handle<}\KeywordTok{Object}\NormalTok{> target) \{}
  \NormalTok{target->Set(}\DataTypeTok{String}\NormalTok{::NewSymbol(}\StringTok{"createObject"}\NormalTok{),}
      \DataTypeTok{FunctionTemplate}\NormalTok{::New(CreateObject)->GetFunction());}
\NormalTok{\}}

\NormalTok{NODE_MODULE(addon, Init)}
\end{Highlighting}
\end{Shaded}

To test it in JavaScript:

\begin{Shaded}
\begin{Highlighting}[]
\KeywordTok{var} \NormalTok{addon = require(}\CharTok{'./build/Release/addon'}\NormalTok{);}

\KeywordTok{var} \NormalTok{obj1 = }\KeywordTok{addon}\NormalTok{.}\FunctionTok{createObject}\NormalTok{(}\CharTok{'hello'}\NormalTok{);}
\KeywordTok{var} \NormalTok{obj2 = }\KeywordTok{addon}\NormalTok{.}\FunctionTok{createObject}\NormalTok{(}\CharTok{'world'}\NormalTok{);}
\KeywordTok{console}\NormalTok{.}\FunctionTok{log}\NormalTok{(}\KeywordTok{obj1}\NormalTok{.}\FunctionTok{msg}\NormalTok{+}\CharTok{' '}\NormalTok{+}\KeywordTok{obj2}\NormalTok{.}\FunctionTok{msg}\NormalTok{); }\CommentTok{// 'hello world'}
\end{Highlighting}
\end{Shaded}

\subsubsection{Function factory}

This pattern illustrates how to create and return a JavaScript function
that wraps a C++ function:

\begin{Shaded}
\begin{Highlighting}[]
\NormalTok{#define BUILDING_NODE_EXTENSION}
\NormalTok{#include <}\KeywordTok{node}\NormalTok{.}\FunctionTok{h}\NormalTok{>}

\NormalTok{using namespace v8;}

\NormalTok{Handle<Value> MyFunction(}\KeywordTok{const} \NormalTok{Arguments& args) \{}
  \NormalTok{HandleScope scope;}
  \KeywordTok{return} \KeywordTok{scope}\NormalTok{.}\FunctionTok{Close}\NormalTok{(}\DataTypeTok{String}\NormalTok{::New(}\StringTok{"hello world"}\NormalTok{));}
\NormalTok{\}}

\NormalTok{Handle<Value> CreateFunction(}\KeywordTok{const} \NormalTok{Arguments& args) \{}
  \NormalTok{HandleScope scope;}

  \NormalTok{Local<FunctionTemplate> tpl = }\DataTypeTok{FunctionTemplate}\NormalTok{::New(MyFunction);}
  \NormalTok{Local<}\KeywordTok{Function}\NormalTok{> fn = tpl->GetFunction();}
  \NormalTok{fn->SetName(}\DataTypeTok{String}\NormalTok{::NewSymbol(}\StringTok{"theFunction"}\NormalTok{)); }\CommentTok{// omit this to make it anonymous}

  \KeywordTok{return} \KeywordTok{scope}\NormalTok{.}\FunctionTok{Close}\NormalTok{(fn);}
\NormalTok{\}}

\KeywordTok{void} \NormalTok{Init(Handle<}\KeywordTok{Object}\NormalTok{> target) \{}
  \NormalTok{target->Set(}\DataTypeTok{String}\NormalTok{::NewSymbol(}\StringTok{"createFunction"}\NormalTok{),}
      \DataTypeTok{FunctionTemplate}\NormalTok{::New(CreateFunction)->GetFunction());}
\NormalTok{\}}

\NormalTok{NODE_MODULE(addon, Init)}
\end{Highlighting}
\end{Shaded}

To test:

\begin{Shaded}
\begin{Highlighting}[]
\KeywordTok{var} \NormalTok{addon = require(}\CharTok{'./build/Release/addon'}\NormalTok{);}

\KeywordTok{var} \NormalTok{fn = }\KeywordTok{addon}\NormalTok{.}\FunctionTok{createFunction}\NormalTok{();}
\KeywordTok{console}\NormalTok{.}\FunctionTok{log}\NormalTok{(fn()); }\CommentTok{// 'hello world'}
\end{Highlighting}
\end{Shaded}

\subsubsection{Wrapping C++ objects}

Here we will create a wrapper for a C++ object/class \texttt{MyObject}
that can be instantiated in JavaScript through the \texttt{new}
operator. First prepare the main module \texttt{addon.cc}:

\begin{Shaded}
\begin{Highlighting}[]
\NormalTok{#define BUILDING_NODE_EXTENSION}
\NormalTok{#include <}\KeywordTok{node}\NormalTok{.}\FunctionTok{h}\NormalTok{>}
\NormalTok{#include }\StringTok{"myobject.h"}

\NormalTok{using namespace v8;}

\KeywordTok{void} \NormalTok{InitAll(Handle<}\KeywordTok{Object}\NormalTok{> target) \{}
  \DataTypeTok{MyObject}\NormalTok{::Init(target);}
\NormalTok{\}}

\NormalTok{NODE_MODULE(addon, InitAll)}
\end{Highlighting}
\end{Shaded}

Then in \texttt{myobject.h} make your wrapper inherit from
\texttt{node::ObjectWrap}:

\begin{Shaded}
\begin{Highlighting}[]
\NormalTok{#ifndef MYOBJECT_H}
\NormalTok{#define MYOBJECT_H}

\NormalTok{#include <}\KeywordTok{node}\NormalTok{.}\FunctionTok{h}\NormalTok{>}

\NormalTok{class }\DataTypeTok{MyObject }\NormalTok{: public }\DataTypeTok{node}\NormalTok{::ObjectWrap \{}
 \DataTypeTok{public}\NormalTok{:}
  \NormalTok{static }\KeywordTok{void} \NormalTok{Init(}\DataTypeTok{v8}\NormalTok{::Handle<}\DataTypeTok{v8}\NormalTok{::}\KeywordTok{Object}\NormalTok{> target);}

 \DataTypeTok{private}\NormalTok{:}
  \NormalTok{MyObject();}
  \NormalTok{~MyObject();}

  \NormalTok{static }\DataTypeTok{v8}\NormalTok{::Handle<}\DataTypeTok{v8}\NormalTok{::Value> New(}\KeywordTok{const} \DataTypeTok{v8}\NormalTok{::Arguments& args);}
  \NormalTok{static }\DataTypeTok{v8}\NormalTok{::Handle<}\DataTypeTok{v8}\NormalTok{::Value> PlusOne(}\KeywordTok{const} \DataTypeTok{v8}\NormalTok{::Arguments& args);}
  \NormalTok{double counter_;}
\NormalTok{\};}

\NormalTok{#endif}
\end{Highlighting}
\end{Shaded}

And in \texttt{myobject.cc} implement the various methods that you want
to expose. Here we expose the method \texttt{plusOne} by adding it to
the constructor's prototype:

\begin{Shaded}
\begin{Highlighting}[]
\NormalTok{#define BUILDING_NODE_EXTENSION}
\NormalTok{#include <}\KeywordTok{node}\NormalTok{.}\FunctionTok{h}\NormalTok{>}
\NormalTok{#include }\StringTok{"myobject.h"}

\NormalTok{using namespace v8;}

\DataTypeTok{MyObject}\NormalTok{::MyObject() \{\};}
\DataTypeTok{MyObject}\NormalTok{::~MyObject() \{\};}

\KeywordTok{void} \DataTypeTok{MyObject}\NormalTok{::Init(Handle<}\KeywordTok{Object}\NormalTok{> target) \{}
  \CommentTok{// Prepare constructor template}
  \NormalTok{Local<FunctionTemplate> tpl = }\DataTypeTok{FunctionTemplate}\NormalTok{::New(New);}
  \NormalTok{tpl->SetClassName(}\DataTypeTok{String}\NormalTok{::NewSymbol(}\StringTok{"MyObject"}\NormalTok{));}
  \NormalTok{tpl->InstanceTemplate()->SetInternalFieldCount(}\DecValTok{1}\NormalTok{);}
  \CommentTok{// Prototype}
  \NormalTok{tpl->PrototypeTemplate()->Set(}\DataTypeTok{String}\NormalTok{::NewSymbol(}\StringTok{"plusOne"}\NormalTok{),}
      \DataTypeTok{FunctionTemplate}\NormalTok{::New(PlusOne)->GetFunction());}

  \NormalTok{Persistent<}\KeywordTok{Function}\NormalTok{> constructor = Persistent<}\KeywordTok{Function}\NormalTok{>::New(tpl->GetFunction());}
  \NormalTok{target->Set(}\DataTypeTok{String}\NormalTok{::NewSymbol(}\StringTok{"MyObject"}\NormalTok{), constructor);}
\NormalTok{\}}

\NormalTok{Handle<Value> }\DataTypeTok{MyObject}\NormalTok{::New(}\KeywordTok{const} \NormalTok{Arguments& args) \{}
  \NormalTok{HandleScope scope;}

  \NormalTok{MyObject* obj = }\KeywordTok{new} \NormalTok{MyObject();}
  \NormalTok{obj->counter_ = args[}\DecValTok{0}\NormalTok{]->IsUndefined() ? }\DecValTok{0} \NormalTok{: args[}\DecValTok{0}\NormalTok{]->NumberValue();}
  \NormalTok{obj->Wrap(}\KeywordTok{args}\NormalTok{.}\FunctionTok{This}\NormalTok{());}

  \KeywordTok{return} \KeywordTok{args}\NormalTok{.}\FunctionTok{This}\NormalTok{();}
\NormalTok{\}}

\NormalTok{Handle<Value> }\DataTypeTok{MyObject}\NormalTok{::PlusOne(}\KeywordTok{const} \NormalTok{Arguments& args) \{}
  \NormalTok{HandleScope scope;}

  \NormalTok{MyObject* obj = }\DataTypeTok{ObjectWrap}\NormalTok{::Unwrap<MyObject>(}\KeywordTok{args}\NormalTok{.}\FunctionTok{This}\NormalTok{());}
  \NormalTok{obj->counter_ += }\DecValTok{1}\NormalTok{;}

  \KeywordTok{return} \KeywordTok{scope}\NormalTok{.}\FunctionTok{Close}\NormalTok{(}\DataTypeTok{Number}\NormalTok{::New(obj->counter_));}
\NormalTok{\}}
\end{Highlighting}
\end{Shaded}

Test it with:

\begin{Shaded}
\begin{Highlighting}[]
\KeywordTok{var} \NormalTok{addon = require(}\CharTok{'./build/Release/addon'}\NormalTok{);}

\KeywordTok{var} \NormalTok{obj = }\KeywordTok{new} \KeywordTok{addon}\NormalTok{.}\FunctionTok{MyObject}\NormalTok{(}\DecValTok{10}\NormalTok{);}
\KeywordTok{console}\NormalTok{.}\FunctionTok{log}\NormalTok{( }\KeywordTok{obj}\NormalTok{.}\FunctionTok{plusOne}\NormalTok{() ); }\CommentTok{// 11}
\KeywordTok{console}\NormalTok{.}\FunctionTok{log}\NormalTok{( }\KeywordTok{obj}\NormalTok{.}\FunctionTok{plusOne}\NormalTok{() ); }\CommentTok{// 12}
\KeywordTok{console}\NormalTok{.}\FunctionTok{log}\NormalTok{( }\KeywordTok{obj}\NormalTok{.}\FunctionTok{plusOne}\NormalTok{() ); }\CommentTok{// 13}
\end{Highlighting}
\end{Shaded}

\subsubsection{Factory of wrapped objects}

This is useful when you want to be able to create native objects without
explicitly instantiating them with the \texttt{new} operator in
JavaScript, e.g.

\begin{Shaded}
\begin{Highlighting}[]
\KeywordTok{var} \NormalTok{obj = }\KeywordTok{addon}\NormalTok{.}\FunctionTok{createObject}\NormalTok{();}
\CommentTok{// instead of:}
\CommentTok{// var obj = new addon.Object();}
\end{Highlighting}
\end{Shaded}

Let's register our \texttt{createObject} method in \texttt{addon.cc}:

\begin{Shaded}
\begin{Highlighting}[]
\NormalTok{#define BUILDING_NODE_EXTENSION}
\NormalTok{#include <}\KeywordTok{node}\NormalTok{.}\FunctionTok{h}\NormalTok{>}
\NormalTok{#include }\StringTok{"myobject.h"}

\NormalTok{using namespace v8;}

\NormalTok{Handle<Value> CreateObject(}\KeywordTok{const} \NormalTok{Arguments& args) \{}
  \NormalTok{HandleScope scope;}
  \KeywordTok{return} \KeywordTok{scope}\NormalTok{.}\FunctionTok{Close}\NormalTok{(}\DataTypeTok{MyObject}\NormalTok{::NewInstance(args));}
\NormalTok{\}}

\KeywordTok{void} \NormalTok{InitAll(Handle<}\KeywordTok{Object}\NormalTok{> target) \{}
  \DataTypeTok{MyObject}\NormalTok{::Init();}

  \NormalTok{target->Set(}\DataTypeTok{String}\NormalTok{::NewSymbol(}\StringTok{"createObject"}\NormalTok{),}
      \DataTypeTok{FunctionTemplate}\NormalTok{::New(CreateObject)->GetFunction());}
\NormalTok{\}}

\NormalTok{NODE_MODULE(addon, InitAll)}
\end{Highlighting}
\end{Shaded}

In \texttt{myobject.h} we now introduce the static method
\texttt{NewInstance} that takes care of instantiating the object
(i.e.~it does the job of \texttt{new} in JavaScript):

\begin{Shaded}
\begin{Highlighting}[]
\NormalTok{#define BUILDING_NODE_EXTENSION}
\NormalTok{#ifndef MYOBJECT_H}
\NormalTok{#define MYOBJECT_H}

\NormalTok{#include <}\KeywordTok{node}\NormalTok{.}\FunctionTok{h}\NormalTok{>}

\NormalTok{class }\DataTypeTok{MyObject }\NormalTok{: public }\DataTypeTok{node}\NormalTok{::ObjectWrap \{}
 \DataTypeTok{public}\NormalTok{:}
  \NormalTok{static }\KeywordTok{void} \NormalTok{Init();}
  \NormalTok{static }\DataTypeTok{v8}\NormalTok{::Handle<}\DataTypeTok{v8}\NormalTok{::Value> NewInstance(}\KeywordTok{const} \DataTypeTok{v8}\NormalTok{::Arguments& args);}

 \DataTypeTok{private}\NormalTok{:}
  \NormalTok{MyObject();}
  \NormalTok{~MyObject();}

  \NormalTok{static }\DataTypeTok{v8}\NormalTok{::Persistent<}\DataTypeTok{v8}\NormalTok{::}\KeywordTok{Function}\NormalTok{> constructor;}
  \NormalTok{static }\DataTypeTok{v8}\NormalTok{::Handle<}\DataTypeTok{v8}\NormalTok{::Value> New(}\KeywordTok{const} \DataTypeTok{v8}\NormalTok{::Arguments& args);}
  \NormalTok{static }\DataTypeTok{v8}\NormalTok{::Handle<}\DataTypeTok{v8}\NormalTok{::Value> PlusOne(}\KeywordTok{const} \DataTypeTok{v8}\NormalTok{::Arguments& args);}
  \NormalTok{double counter_;}
\NormalTok{\};}

\NormalTok{#endif}
\end{Highlighting}
\end{Shaded}

The implementation is similar to the above in \texttt{myobject.cc}:

\begin{Shaded}
\begin{Highlighting}[]
\NormalTok{#define BUILDING_NODE_EXTENSION}
\NormalTok{#include <}\KeywordTok{node}\NormalTok{.}\FunctionTok{h}\NormalTok{>}
\NormalTok{#include }\StringTok{"myobject.h"}

\NormalTok{using namespace v8;}

\DataTypeTok{MyObject}\NormalTok{::MyObject() \{\};}
\DataTypeTok{MyObject}\NormalTok{::~MyObject() \{\};}

\NormalTok{Persistent<}\KeywordTok{Function}\NormalTok{> }\DataTypeTok{MyObject}\NormalTok{::constructor;}

\KeywordTok{void} \DataTypeTok{MyObject}\NormalTok{::Init() \{}
  \CommentTok{// Prepare constructor template}
  \NormalTok{Local<FunctionTemplate> tpl = }\DataTypeTok{FunctionTemplate}\NormalTok{::New(New);}
  \NormalTok{tpl->SetClassName(}\DataTypeTok{String}\NormalTok{::NewSymbol(}\StringTok{"MyObject"}\NormalTok{));}
  \NormalTok{tpl->InstanceTemplate()->SetInternalFieldCount(}\DecValTok{1}\NormalTok{);}
  \CommentTok{// Prototype}
  \NormalTok{tpl->PrototypeTemplate()->Set(}\DataTypeTok{String}\NormalTok{::NewSymbol(}\StringTok{"plusOne"}\NormalTok{),}
      \DataTypeTok{FunctionTemplate}\NormalTok{::New(PlusOne)->GetFunction());}

  \NormalTok{constructor = Persistent<}\KeywordTok{Function}\NormalTok{>::New(tpl->GetFunction());}
\NormalTok{\}}

\NormalTok{Handle<Value> }\DataTypeTok{MyObject}\NormalTok{::New(}\KeywordTok{const} \NormalTok{Arguments& args) \{}
  \NormalTok{HandleScope scope;}

  \NormalTok{MyObject* obj = }\KeywordTok{new} \NormalTok{MyObject();}
  \NormalTok{obj->counter_ = args[}\DecValTok{0}\NormalTok{]->IsUndefined() ? }\DecValTok{0} \NormalTok{: args[}\DecValTok{0}\NormalTok{]->NumberValue();}
  \NormalTok{obj->Wrap(}\KeywordTok{args}\NormalTok{.}\FunctionTok{This}\NormalTok{());}

  \KeywordTok{return} \KeywordTok{args}\NormalTok{.}\FunctionTok{This}\NormalTok{();}
\NormalTok{\}}

\NormalTok{Handle<Value> }\DataTypeTok{MyObject}\NormalTok{::NewInstance(}\KeywordTok{const} \NormalTok{Arguments& args) \{}
  \NormalTok{HandleScope scope;}

  \KeywordTok{const} \NormalTok{unsigned argc = }\DecValTok{1}\NormalTok{;}
  \NormalTok{Handle<Value> argv[argc] = \{ args[}\DecValTok{0}\NormalTok{] \};}
  \NormalTok{Local<}\KeywordTok{Object}\NormalTok{> instance = constructor->NewInstance(argc, argv);}

  \KeywordTok{return} \KeywordTok{scope}\NormalTok{.}\FunctionTok{Close}\NormalTok{(instance);}
\NormalTok{\}}

\NormalTok{Handle<Value> }\DataTypeTok{MyObject}\NormalTok{::PlusOne(}\KeywordTok{const} \NormalTok{Arguments& args) \{}
  \NormalTok{HandleScope scope;}

  \NormalTok{MyObject* obj = }\DataTypeTok{ObjectWrap}\NormalTok{::Unwrap<MyObject>(}\KeywordTok{args}\NormalTok{.}\FunctionTok{This}\NormalTok{());}
  \NormalTok{obj->counter_ += }\DecValTok{1}\NormalTok{;}

  \KeywordTok{return} \KeywordTok{scope}\NormalTok{.}\FunctionTok{Close}\NormalTok{(}\DataTypeTok{Number}\NormalTok{::New(obj->counter_));}
\NormalTok{\}}
\end{Highlighting}
\end{Shaded}

Test it with:

\begin{Shaded}
\begin{Highlighting}[]
\KeywordTok{var} \NormalTok{addon = require(}\CharTok{'./build/Release/addon'}\NormalTok{);}

\KeywordTok{var} \NormalTok{obj = }\KeywordTok{addon}\NormalTok{.}\FunctionTok{createObject}\NormalTok{(}\DecValTok{10}\NormalTok{);}
\KeywordTok{console}\NormalTok{.}\FunctionTok{log}\NormalTok{( }\KeywordTok{obj}\NormalTok{.}\FunctionTok{plusOne}\NormalTok{() ); }\CommentTok{// 11}
\KeywordTok{console}\NormalTok{.}\FunctionTok{log}\NormalTok{( }\KeywordTok{obj}\NormalTok{.}\FunctionTok{plusOne}\NormalTok{() ); }\CommentTok{// 12}
\KeywordTok{console}\NormalTok{.}\FunctionTok{log}\NormalTok{( }\KeywordTok{obj}\NormalTok{.}\FunctionTok{plusOne}\NormalTok{() ); }\CommentTok{// 13}

\KeywordTok{var} \NormalTok{obj2 = }\KeywordTok{addon}\NormalTok{.}\FunctionTok{createObject}\NormalTok{(}\DecValTok{20}\NormalTok{);}
\KeywordTok{console}\NormalTok{.}\FunctionTok{log}\NormalTok{( }\KeywordTok{obj2}\NormalTok{.}\FunctionTok{plusOne}\NormalTok{() ); }\CommentTok{// 21}
\KeywordTok{console}\NormalTok{.}\FunctionTok{log}\NormalTok{( }\KeywordTok{obj2}\NormalTok{.}\FunctionTok{plusOne}\NormalTok{() ); }\CommentTok{// 22}
\KeywordTok{console}\NormalTok{.}\FunctionTok{log}\NormalTok{( }\KeywordTok{obj2}\NormalTok{.}\FunctionTok{plusOne}\NormalTok{() ); }\CommentTok{// 23}
\end{Highlighting}
\end{Shaded}

\subsubsection{Passing wrapped objects around}

In addition to wrapping and returning C++ objects, you can pass them
around by unwrapping them with Node's \texttt{node::ObjectWrap::Unwrap}
helper function. In the following \texttt{addon.cc} we introduce a
function \texttt{add()} that can take on two \texttt{MyObject} objects:

\begin{Shaded}
\begin{Highlighting}[]
\NormalTok{#define BUILDING_NODE_EXTENSION}
\NormalTok{#include <}\KeywordTok{node}\NormalTok{.}\FunctionTok{h}\NormalTok{>}
\NormalTok{#include }\StringTok{"myobject.h"}

\NormalTok{using namespace v8;}

\NormalTok{Handle<Value> CreateObject(}\KeywordTok{const} \NormalTok{Arguments& args) \{}
  \NormalTok{HandleScope scope;}
  \KeywordTok{return} \KeywordTok{scope}\NormalTok{.}\FunctionTok{Close}\NormalTok{(}\DataTypeTok{MyObject}\NormalTok{::NewInstance(args));}
\NormalTok{\}}

\NormalTok{Handle<Value> Add(}\KeywordTok{const} \NormalTok{Arguments& args) \{}
  \NormalTok{HandleScope scope;}

  \NormalTok{MyObject* obj1 = }\DataTypeTok{node}\NormalTok{::}\DataTypeTok{ObjectWrap}\NormalTok{::Unwrap<MyObject>(}
      \NormalTok{args[}\DecValTok{0}\NormalTok{]->ToObject());}
  \NormalTok{MyObject* obj2 = }\DataTypeTok{node}\NormalTok{::}\DataTypeTok{ObjectWrap}\NormalTok{::Unwrap<MyObject>(}
      \NormalTok{args[}\DecValTok{1}\NormalTok{]->ToObject());}

  \NormalTok{double sum = obj1->Val() + obj2->Val();}
  \KeywordTok{return} \KeywordTok{scope}\NormalTok{.}\FunctionTok{Close}\NormalTok{(}\DataTypeTok{Number}\NormalTok{::New(sum));}
\NormalTok{\}}

\KeywordTok{void} \NormalTok{InitAll(Handle<}\KeywordTok{Object}\NormalTok{> target) \{}
  \DataTypeTok{MyObject}\NormalTok{::Init();}

  \NormalTok{target->Set(}\DataTypeTok{String}\NormalTok{::NewSymbol(}\StringTok{"createObject"}\NormalTok{),}
      \DataTypeTok{FunctionTemplate}\NormalTok{::New(CreateObject)->GetFunction());}

  \NormalTok{target->Set(}\DataTypeTok{String}\NormalTok{::NewSymbol(}\StringTok{"add"}\NormalTok{),}
      \DataTypeTok{FunctionTemplate}\NormalTok{::New(Add)->GetFunction());}
\NormalTok{\}}

\NormalTok{NODE_MODULE(addon, InitAll)}
\end{Highlighting}
\end{Shaded}

To make things interesting we introduce a public method in
\texttt{myobject.h} so we can probe private values after unwrapping the
object:

\begin{Shaded}
\begin{Highlighting}[]
\NormalTok{#define BUILDING_NODE_EXTENSION}
\NormalTok{#ifndef MYOBJECT_H}
\NormalTok{#define MYOBJECT_H}

\NormalTok{#include <}\KeywordTok{node}\NormalTok{.}\FunctionTok{h}\NormalTok{>}

\NormalTok{class }\DataTypeTok{MyObject }\NormalTok{: public }\DataTypeTok{node}\NormalTok{::ObjectWrap \{}
 \DataTypeTok{public}\NormalTok{:}
  \NormalTok{static }\KeywordTok{void} \NormalTok{Init();}
  \NormalTok{static }\DataTypeTok{v8}\NormalTok{::Handle<}\DataTypeTok{v8}\NormalTok{::Value> NewInstance(}\KeywordTok{const} \DataTypeTok{v8}\NormalTok{::Arguments& args);}
  \NormalTok{double Val() }\KeywordTok{const} \NormalTok{\{ }\KeywordTok{return} \NormalTok{val_; \}}

 \DataTypeTok{private}\NormalTok{:}
  \NormalTok{MyObject();}
  \NormalTok{~MyObject();}

  \NormalTok{static }\DataTypeTok{v8}\NormalTok{::Persistent<}\DataTypeTok{v8}\NormalTok{::}\KeywordTok{Function}\NormalTok{> constructor;}
  \NormalTok{static }\DataTypeTok{v8}\NormalTok{::Handle<}\DataTypeTok{v8}\NormalTok{::Value> New(}\KeywordTok{const} \DataTypeTok{v8}\NormalTok{::Arguments& args);}
  \NormalTok{double val_;}
\NormalTok{\};}

\NormalTok{#endif}
\end{Highlighting}
\end{Shaded}

The implementation of \texttt{myobject.cc} is similar as before:

\begin{Shaded}
\begin{Highlighting}[]
\NormalTok{#define BUILDING_NODE_EXTENSION}
\NormalTok{#include <}\KeywordTok{node}\NormalTok{.}\FunctionTok{h}\NormalTok{>}
\NormalTok{#include }\StringTok{"myobject.h"}

\NormalTok{using namespace v8;}

\DataTypeTok{MyObject}\NormalTok{::MyObject() \{\};}
\DataTypeTok{MyObject}\NormalTok{::~MyObject() \{\};}

\NormalTok{Persistent<}\KeywordTok{Function}\NormalTok{> }\DataTypeTok{MyObject}\NormalTok{::constructor;}

\KeywordTok{void} \DataTypeTok{MyObject}\NormalTok{::Init() \{}
  \CommentTok{// Prepare constructor template}
  \NormalTok{Local<FunctionTemplate> tpl = }\DataTypeTok{FunctionTemplate}\NormalTok{::New(New);}
  \NormalTok{tpl->SetClassName(}\DataTypeTok{String}\NormalTok{::NewSymbol(}\StringTok{"MyObject"}\NormalTok{));}
  \NormalTok{tpl->InstanceTemplate()->SetInternalFieldCount(}\DecValTok{1}\NormalTok{);}

  \NormalTok{constructor = Persistent<}\KeywordTok{Function}\NormalTok{>::New(tpl->GetFunction());}
\NormalTok{\}}

\NormalTok{Handle<Value> }\DataTypeTok{MyObject}\NormalTok{::New(}\KeywordTok{const} \NormalTok{Arguments& args) \{}
  \NormalTok{HandleScope scope;}

  \NormalTok{MyObject* obj = }\KeywordTok{new} \NormalTok{MyObject();}
  \NormalTok{obj->val_ = args[}\DecValTok{0}\NormalTok{]->IsUndefined() ? }\DecValTok{0} \NormalTok{: args[}\DecValTok{0}\NormalTok{]->NumberValue();}
  \NormalTok{obj->Wrap(}\KeywordTok{args}\NormalTok{.}\FunctionTok{This}\NormalTok{());}

  \KeywordTok{return} \KeywordTok{args}\NormalTok{.}\FunctionTok{This}\NormalTok{();}
\NormalTok{\}}

\NormalTok{Handle<Value> }\DataTypeTok{MyObject}\NormalTok{::NewInstance(}\KeywordTok{const} \NormalTok{Arguments& args) \{}
  \NormalTok{HandleScope scope;}

  \KeywordTok{const} \NormalTok{unsigned argc = }\DecValTok{1}\NormalTok{;}
  \NormalTok{Handle<Value> argv[argc] = \{ args[}\DecValTok{0}\NormalTok{] \};}
  \NormalTok{Local<}\KeywordTok{Object}\NormalTok{> instance = constructor->NewInstance(argc, argv);}

  \KeywordTok{return} \KeywordTok{scope}\NormalTok{.}\FunctionTok{Close}\NormalTok{(instance);}
\NormalTok{\}}
\end{Highlighting}
\end{Shaded}

Test it with:

\begin{Shaded}
\begin{Highlighting}[]
\KeywordTok{var} \NormalTok{addon = require(}\CharTok{'./build/Release/addon'}\NormalTok{);}

\KeywordTok{var} \NormalTok{obj1 = }\KeywordTok{addon}\NormalTok{.}\FunctionTok{createObject}\NormalTok{(}\DecValTok{10}\NormalTok{);}
\KeywordTok{var} \NormalTok{obj2 = }\KeywordTok{addon}\NormalTok{.}\FunctionTok{createObject}\NormalTok{(}\DecValTok{20}\NormalTok{);}
\KeywordTok{var} \NormalTok{result = }\KeywordTok{addon}\NormalTok{.}\FunctionTok{add}\NormalTok{(obj1, obj2);}

\KeywordTok{console}\NormalTok{.}\FunctionTok{log}\NormalTok{(result); }\CommentTok{// 30}
\end{Highlighting}
\end{Shaded}

