\section{Zlib}\label{zlib}

\begin{Shaded}
\begin{Highlighting}[]
\NormalTok{Stability: }\DecValTok{3} \NormalTok{- Stable}
\end{Highlighting}
\end{Shaded}

You can access this module with:

\begin{Shaded}
\begin{Highlighting}[]
\KeywordTok{var} \NormalTok{zlib = }\FunctionTok{require}\NormalTok{(}\StringTok{'zlib'}\NormalTok{);}
\end{Highlighting}
\end{Shaded}

This provides bindings to Gzip/Gunzip, Deflate/Inflate, and
DeflateRaw/InflateRaw classes. Each class takes the same options, and is
a readable/writable Stream.

\subsection{Examples}\label{examples}

Compressing or decompressing a file can be done by piping an
fs.ReadStream into a zlib stream, then into an fs.WriteStream.

\begin{Shaded}
\begin{Highlighting}[]
\KeywordTok{var} \NormalTok{gzip = }\OtherTok{zlib}\NormalTok{.}\FunctionTok{createGzip}\NormalTok{();}
\KeywordTok{var} \NormalTok{fs = }\FunctionTok{require}\NormalTok{(}\StringTok{'fs'}\NormalTok{);}
\KeywordTok{var} \NormalTok{inp = }\OtherTok{fs}\NormalTok{.}\FunctionTok{createReadStream}\NormalTok{(}\StringTok{'input.txt'}\NormalTok{);}
\KeywordTok{var} \NormalTok{out = }\OtherTok{fs}\NormalTok{.}\FunctionTok{createWriteStream}\NormalTok{(}\StringTok{'input.txt.gz'}\NormalTok{);}

\OtherTok{inp}\NormalTok{.}\FunctionTok{pipe}\NormalTok{(gzip).}\FunctionTok{pipe}\NormalTok{(out);}
\end{Highlighting}
\end{Shaded}

Compressing or decompressing data in one step can be done by using the
convenience methods.

\begin{Shaded}
\begin{Highlighting}[]
\KeywordTok{var} \NormalTok{input = }\StringTok{'.................................'}\NormalTok{;}
\OtherTok{zlib}\NormalTok{.}\FunctionTok{deflate}\NormalTok{(input, }\KeywordTok{function}\NormalTok{(err, buffer) \{}
  \KeywordTok{if} \NormalTok{(!err) \{}
    \OtherTok{console}\NormalTok{.}\FunctionTok{log}\NormalTok{(}\OtherTok{buffer}\NormalTok{.}\FunctionTok{toString}\NormalTok{(}\StringTok{'base64'}\NormalTok{));}
  \NormalTok{\}}
\NormalTok{\});}

\KeywordTok{var} \NormalTok{buffer = }\KeywordTok{new} \FunctionTok{Buffer}\NormalTok{(}\StringTok{'eJzT0yMAAGTvBe8='}\NormalTok{, }\StringTok{'base64'}\NormalTok{);}
\OtherTok{zlib}\NormalTok{.}\FunctionTok{unzip}\NormalTok{(buffer, }\KeywordTok{function}\NormalTok{(err, buffer) \{}
  \KeywordTok{if} \NormalTok{(!err) \{}
    \OtherTok{console}\NormalTok{.}\FunctionTok{log}\NormalTok{(}\OtherTok{buffer}\NormalTok{.}\FunctionTok{toString}\NormalTok{());}
  \NormalTok{\}}
\NormalTok{\});}
\end{Highlighting}
\end{Shaded}

To use this module in an HTTP client or server, use the
\href{http://www.w3.org/Protocols/rfc2616/rfc2616-sec14.html\#sec14.3}{accept-encoding}
on requests, and the
\href{http://www.w3.org/Protocols/rfc2616/rfc2616-sec14.html\#sec14.11}{content-encoding}
header on responses.

\textbf{Note: these examples are drastically simplified to show the
basic concept.} Zlib encoding can be expensive, and the results ought to
be cached. See \hyperref[zlibux5fmemoryux5fusageux5ftuning]{Memory Usage
Tuning} below for more information on the speed/memory/compression
tradeoffs involved in zlib usage.

\begin{Shaded}
\begin{Highlighting}[]
\CommentTok{// client request example}
\KeywordTok{var} \NormalTok{zlib = }\FunctionTok{require}\NormalTok{(}\StringTok{'zlib'}\NormalTok{);}
\KeywordTok{var} \NormalTok{http = }\FunctionTok{require}\NormalTok{(}\StringTok{'http'}\NormalTok{);}
\KeywordTok{var} \NormalTok{fs = }\FunctionTok{require}\NormalTok{(}\StringTok{'fs'}\NormalTok{);}
\KeywordTok{var} \NormalTok{request = }\OtherTok{http}\NormalTok{.}\FunctionTok{get}\NormalTok{(\{ }\DataTypeTok{host}\NormalTok{: }\StringTok{'izs.me'}\NormalTok{,}
                         \DataTypeTok{path}\NormalTok{: }\StringTok{'/'}\NormalTok{,}
                         \DataTypeTok{port}\NormalTok{: }\DecValTok{80}\NormalTok{,}
                         \DataTypeTok{headers}\NormalTok{: \{ }\StringTok{'accept-encoding'}\NormalTok{: }\StringTok{'gzip,deflate'} \NormalTok{\} \});}
\OtherTok{request}\NormalTok{.}\FunctionTok{on}\NormalTok{(}\StringTok{'response'}\NormalTok{, }\KeywordTok{function}\NormalTok{(response) \{}
  \KeywordTok{var} \NormalTok{output = }\OtherTok{fs}\NormalTok{.}\FunctionTok{createWriteStream}\NormalTok{(}\StringTok{'izs.me_index.html'}\NormalTok{);}

  \KeywordTok{switch} \NormalTok{(}\OtherTok{response}\NormalTok{.}\FunctionTok{headers}\NormalTok{[}\StringTok{'content-encoding'}\NormalTok{]) \{}
    \CommentTok{// or, just use zlib.createUnzip() to handle both cases}
    \KeywordTok{case} \StringTok{'gzip'}\NormalTok{:}
      \OtherTok{response}\NormalTok{.}\FunctionTok{pipe}\NormalTok{(}\OtherTok{zlib}\NormalTok{.}\FunctionTok{createGunzip}\NormalTok{()).}\FunctionTok{pipe}\NormalTok{(output);}
      \KeywordTok{break}\NormalTok{;}
    \KeywordTok{case} \StringTok{'deflate'}\NormalTok{:}
      \OtherTok{response}\NormalTok{.}\FunctionTok{pipe}\NormalTok{(}\OtherTok{zlib}\NormalTok{.}\FunctionTok{createInflate}\NormalTok{()).}\FunctionTok{pipe}\NormalTok{(output);}
      \KeywordTok{break}\NormalTok{;}
    \KeywordTok{default}\NormalTok{:}
      \OtherTok{response}\NormalTok{.}\FunctionTok{pipe}\NormalTok{(output);}
      \KeywordTok{break}\NormalTok{;}
  \NormalTok{\}}
\NormalTok{\});}

\CommentTok{// server example}
\CommentTok{// Running a gzip operation on every request is quite expensive.}
\CommentTok{// It would be much more efficient to cache the compressed buffer.}
\KeywordTok{var} \NormalTok{zlib = }\FunctionTok{require}\NormalTok{(}\StringTok{'zlib'}\NormalTok{);}
\KeywordTok{var} \NormalTok{http = }\FunctionTok{require}\NormalTok{(}\StringTok{'http'}\NormalTok{);}
\KeywordTok{var} \NormalTok{fs = }\FunctionTok{require}\NormalTok{(}\StringTok{'fs'}\NormalTok{);}
\OtherTok{http}\NormalTok{.}\FunctionTok{createServer}\NormalTok{(}\KeywordTok{function}\NormalTok{(request, response) \{}
  \KeywordTok{var} \NormalTok{raw = }\OtherTok{fs}\NormalTok{.}\FunctionTok{createReadStream}\NormalTok{(}\StringTok{'index.html'}\NormalTok{);}
  \KeywordTok{var} \NormalTok{acceptEncoding = }\OtherTok{request}\NormalTok{.}\FunctionTok{headers}\NormalTok{[}\StringTok{'accept-encoding'}\NormalTok{];}
  \KeywordTok{if} \NormalTok{(!acceptEncoding) \{}
    \NormalTok{acceptEncoding = }\StringTok{''}\NormalTok{;}
  \NormalTok{\}}

  \CommentTok{// Note: this is not a conformant accept-encoding parser.}
  \CommentTok{// See http://www.w3.org/Protocols/rfc2616/rfc2616-sec14.html#sec14.3}
  \KeywordTok{if} \NormalTok{(}\OtherTok{acceptEncoding}\NormalTok{.}\FunctionTok{match}\NormalTok{(}\OtherTok{/}\FloatTok{\textbackslash{}b}\OtherTok{deflate}\FloatTok{\textbackslash{}b}\OtherTok{/}\NormalTok{)) \{}
    \OtherTok{response}\NormalTok{.}\FunctionTok{writeHead}\NormalTok{(}\DecValTok{200}\NormalTok{, \{ }\StringTok{'content-encoding'}\NormalTok{: }\StringTok{'deflate'} \NormalTok{\});}
    \OtherTok{raw}\NormalTok{.}\FunctionTok{pipe}\NormalTok{(}\OtherTok{zlib}\NormalTok{.}\FunctionTok{createDeflate}\NormalTok{()).}\FunctionTok{pipe}\NormalTok{(response);}
  \NormalTok{\} }\KeywordTok{else} \KeywordTok{if} \NormalTok{(}\OtherTok{acceptEncoding}\NormalTok{.}\FunctionTok{match}\NormalTok{(}\OtherTok{/}\FloatTok{\textbackslash{}b}\OtherTok{gzip}\FloatTok{\textbackslash{}b}\OtherTok{/}\NormalTok{)) \{}
    \OtherTok{response}\NormalTok{.}\FunctionTok{writeHead}\NormalTok{(}\DecValTok{200}\NormalTok{, \{ }\StringTok{'content-encoding'}\NormalTok{: }\StringTok{'gzip'} \NormalTok{\});}
    \OtherTok{raw}\NormalTok{.}\FunctionTok{pipe}\NormalTok{(}\OtherTok{zlib}\NormalTok{.}\FunctionTok{createGzip}\NormalTok{()).}\FunctionTok{pipe}\NormalTok{(response);}
  \NormalTok{\} }\KeywordTok{else} \NormalTok{\{}
    \OtherTok{response}\NormalTok{.}\FunctionTok{writeHead}\NormalTok{(}\DecValTok{200}\NormalTok{, \{\});}
    \OtherTok{raw}\NormalTok{.}\FunctionTok{pipe}\NormalTok{(response);}
  \NormalTok{\}}
\NormalTok{\}).}\FunctionTok{listen}\NormalTok{(}\DecValTok{1337}\NormalTok{);}
\end{Highlighting}
\end{Shaded}

\subsection{zlib.createGzip({[}options{]})}\label{zlib.creategzipoptions}

Returns a new \hyperref[zlibux5fclassux5fzlibux5fgzip]{Gzip} object with
an \hyperref[zlibux5foptions]{options}.

\subsection{zlib.createGunzip({[}options{]})}\label{zlib.creategunzipoptions}

Returns a new \hyperref[zlibux5fclassux5fzlibux5fgunzip]{Gunzip} object
with an \hyperref[zlibux5foptions]{options}.

\subsection{zlib.createDeflate({[}options{]})}\label{zlib.createdeflateoptions}

Returns a new \hyperref[zlibux5fclassux5fzlibux5fdeflate]{Deflate}
object with an \hyperref[zlibux5foptions]{options}.

\subsection{zlib.createInflate({[}options{]})}\label{zlib.createinflateoptions}

Returns a new \hyperref[zlibux5fclassux5fzlibux5finflate]{Inflate}
object with an \hyperref[zlibux5foptions]{options}.

\subsection{zlib.createDeflateRaw({[}options{]})}\label{zlib.createdeflaterawoptions}

Returns a new \hyperref[zlibux5fclassux5fzlibux5fdeflateraw]{DeflateRaw}
object with an \hyperref[zlibux5foptions]{options}.

\subsection{zlib.createInflateRaw({[}options{]})}\label{zlib.createinflaterawoptions}

Returns a new \hyperref[zlibux5fclassux5fzlibux5finflateraw]{InflateRaw}
object with an \hyperref[zlibux5foptions]{options}.

\subsection{zlib.createUnzip({[}options{]})}\label{zlib.createunzipoptions}

Returns a new \hyperref[zlibux5fclassux5fzlibux5funzip]{Unzip} object
with an \hyperref[zlibux5foptions]{options}.

\subsection{Class: zlib.Zlib}\label{class-zlib.zlib}

Not exported by the \texttt{zlib} module. It is documented here because
it is the base class of the compressor/decompressor classes.

\subsubsection{zlib.flush({[}kind{]},
callback)}\label{zlib.flushkind-callback}

\texttt{kind} defaults to \texttt{zlib.Z\_FULL\_FLUSH}.

Flush pending data. Don't call this frivolously, premature flushes
negatively impact the effectiveness of the compression algorithm.

\subsubsection{zlib.params(level, strategy,
callback)}\label{zlib.paramslevel-strategy-callback}

Dynamically update the compression level and compression strategy. Only
applicable to deflate algorithm.

\subsubsection{zlib.reset()}\label{zlib.reset}

Reset the compressor/decompressor to factory defaults. Only applicable
to the inflate and deflate algorithms.

\subsection{Class: zlib.Gzip}\label{class-zlib.gzip}

Compress data using gzip.

\subsection{Class: zlib.Gunzip}\label{class-zlib.gunzip}

Decompress a gzip stream.

\subsection{Class: zlib.Deflate}\label{class-zlib.deflate}

Compress data using deflate.

\subsection{Class: zlib.Inflate}\label{class-zlib.inflate}

Decompress a deflate stream.

\subsection{Class: zlib.DeflateRaw}\label{class-zlib.deflateraw}

Compress data using deflate, and do not append a zlib header.

\subsection{Class: zlib.InflateRaw}\label{class-zlib.inflateraw}

Decompress a raw deflate stream.

\subsection{Class: zlib.Unzip}\label{class-zlib.unzip}

Decompress either a Gzip- or Deflate-compressed stream by auto-detecting
the header.

\subsection{Convenience Methods}\label{convenience-methods}

All of these take a string or buffer as the first argument, an optional
second argument to supply options to the zlib classes and will call the
supplied callback with \texttt{callback(error, result)}.

Every method has a \texttt{*Sync} counterpart, which accept the same
arguments, but without a callback.

\subsection{zlib.deflate(buf, {[}options{]},
callback)}\label{zlib.deflatebuf-options-callback}

\subsection{zlib.deflateSync(buf,
{[}options{]})}\label{zlib.deflatesyncbuf-options}

Compress a string with Deflate.

\subsection{zlib.deflateRaw(buf, {[}options{]},
callback)}\label{zlib.deflaterawbuf-options-callback}

\subsection{zlib.deflateRawSync(buf,
{[}options{]})}\label{zlib.deflaterawsyncbuf-options}

Compress a string with DeflateRaw.

\subsection{zlib.gzip(buf, {[}options{]},
callback)}\label{zlib.gzipbuf-options-callback}

\subsection{zlib.gzipSync(buf,
{[}options{]})}\label{zlib.gzipsyncbuf-options}

Compress a string with Gzip.

\subsection{zlib.gunzip(buf, {[}options{]},
callback)}\label{zlib.gunzipbuf-options-callback}

\subsection{zlib.gunzipSync(buf,
{[}options{]})}\label{zlib.gunzipsyncbuf-options}

Decompress a raw Buffer with Gunzip.

\subsection{zlib.inflate(buf, {[}options{]},
callback)}\label{zlib.inflatebuf-options-callback}

\subsection{zlib.inflateSync(buf,
{[}options{]})}\label{zlib.inflatesyncbuf-options}

Decompress a raw Buffer with Inflate.

\subsection{zlib.inflateRaw(buf, {[}options{]},
callback)}\label{zlib.inflaterawbuf-options-callback}

\subsection{zlib.inflateRawSync(buf,
{[}options{]})}\label{zlib.inflaterawsyncbuf-options}

Decompress a raw Buffer with InflateRaw.

\subsection{zlib.unzip(buf, {[}options{]},
callback)}\label{zlib.unzipbuf-options-callback}

\subsection{zlib.unzipSync(buf,
{[}options{]})}\label{zlib.unzipsyncbuf-options}

Decompress a raw Buffer with Unzip.

\subsection{Options}\label{options}

Each class takes an options object. All options are optional.

Note that some options are only relevant when compressing, and are
ignored by the decompression classes.

\begin{itemize}
\itemsep1pt\parskip0pt\parsep0pt
\item
  flush (default: \texttt{zlib.Z\_NO\_FLUSH})
\item
  chunkSize (default: 16*1024)
\item
  windowBits
\item
  level (compression only)
\item
  memLevel (compression only)
\item
  strategy (compression only)
\item
  dictionary (deflate/inflate only, empty dictionary by default)
\end{itemize}

See the description of \texttt{deflateInit2} and \texttt{inflateInit2}
at \url{http://zlib.net/manual.html\#Advanced} for more information on
these.

\subsection{Memory Usage Tuning}\label{memory-usage-tuning}

From \texttt{zlib/zconf.h}, modified to node's usage:

The memory requirements for deflate are (in bytes):

\begin{Shaded}
\begin{Highlighting}[]
\NormalTok{(}\DecValTok{1} \NormalTok{<< (windowBits}\DecValTok{+2}\NormalTok{)) +  (}\DecValTok{1} \NormalTok{<< (memLevel}\DecValTok{+9}\NormalTok{))}
\end{Highlighting}
\end{Shaded}

that is: 128K for windowBits=15 + 128K for memLevel = 8 (default values)
plus a few kilobytes for small objects.

For example, if you want to reduce the default memory requirements from
256K to 128K, set the options to:

\begin{Shaded}
\begin{Highlighting}[]
\NormalTok{\{ }\DataTypeTok{windowBits}\NormalTok{: }\DecValTok{14}\NormalTok{, }\DataTypeTok{memLevel}\NormalTok{: }\DecValTok{7} \NormalTok{\}}
\end{Highlighting}
\end{Shaded}

Of course this will generally degrade compression (there's no free
lunch).

The memory requirements for inflate are (in bytes)

\begin{Shaded}
\begin{Highlighting}[]
\DecValTok{1} \NormalTok{<< windowBits}
\end{Highlighting}
\end{Shaded}

that is, 32K for windowBits=15 (default value) plus a few kilobytes for
small objects.

This is in addition to a single internal output slab buffer of size
\texttt{chunkSize}, which defaults to 16K.

The speed of zlib compression is affected most dramatically by the
\texttt{level} setting. A higher level will result in better
compression, but will take longer to complete. A lower level will result
in less compression, but will be much faster.

In general, greater memory usage options will mean that node has to make
fewer calls to zlib, since it'll be able to process more data in a
single \texttt{write} operation. So, this is another factor that affects
the speed, at the cost of memory usage.

\subsection{Constants}\label{constants}

All of the constants defined in zlib.h are also defined on
\texttt{require('zlib')}. In the normal course of operations, you will
not need to ever set any of these. They are documented here so that
their presence is not surprising. This section is taken almost directly
from the \href{http://zlib.net/manual.html\#Constants}{zlib
documentation}. See \url{http://zlib.net/manual.html\#Constants} for
more details.

Allowed flush values.

\begin{itemize}
\itemsep1pt\parskip0pt\parsep0pt
\item
  \texttt{zlib.Z\_NO\_FLUSH}
\item
  \texttt{zlib.Z\_PARTIAL\_FLUSH}
\item
  \texttt{zlib.Z\_SYNC\_FLUSH}
\item
  \texttt{zlib.Z\_FULL\_FLUSH}
\item
  \texttt{zlib.Z\_FINISH}
\item
  \texttt{zlib.Z\_BLOCK}
\item
  \texttt{zlib.Z\_TREES}
\end{itemize}

Return codes for the compression/decompression functions. Negative
values are errors, positive values are used for special but normal
events.

\begin{itemize}
\itemsep1pt\parskip0pt\parsep0pt
\item
  \texttt{zlib.Z\_OK}
\item
  \texttt{zlib.Z\_STREAM\_END}
\item
  \texttt{zlib.Z\_NEED\_DICT}
\item
  \texttt{zlib.Z\_ERRNO}
\item
  \texttt{zlib.Z\_STREAM\_ERROR}
\item
  \texttt{zlib.Z\_DATA\_ERROR}
\item
  \texttt{zlib.Z\_MEM\_ERROR}
\item
  \texttt{zlib.Z\_BUF\_ERROR}
\item
  \texttt{zlib.Z\_VERSION\_ERROR}
\end{itemize}

Compression levels.

\begin{itemize}
\itemsep1pt\parskip0pt\parsep0pt
\item
  \texttt{zlib.Z\_NO\_COMPRESSION}
\item
  \texttt{zlib.Z\_BEST\_SPEED}
\item
  \texttt{zlib.Z\_BEST\_COMPRESSION}
\item
  \texttt{zlib.Z\_DEFAULT\_COMPRESSION}
\end{itemize}

Compression strategy.

\begin{itemize}
\itemsep1pt\parskip0pt\parsep0pt
\item
  \texttt{zlib.Z\_FILTERED}
\item
  \texttt{zlib.Z\_HUFFMAN\_ONLY}
\item
  \texttt{zlib.Z\_RLE}
\item
  \texttt{zlib.Z\_FIXED}
\item
  \texttt{zlib.Z\_DEFAULT\_STRATEGY}
\end{itemize}

Possible values of the data\_type field.

\begin{itemize}
\itemsep1pt\parskip0pt\parsep0pt
\item
  \texttt{zlib.Z\_BINARY}
\item
  \texttt{zlib.Z\_TEXT}
\item
  \texttt{zlib.Z\_ASCII}
\item
  \texttt{zlib.Z\_UNKNOWN}
\end{itemize}

The deflate compression method (the only one supported in this version).

\begin{itemize}
\itemsep1pt\parskip0pt\parsep0pt
\item
  \texttt{zlib.Z\_DEFLATED}
\end{itemize}

For initializing zalloc, zfree, opaque.

\begin{itemize}
\itemsep1pt\parskip0pt\parsep0pt
\item
  \texttt{zlib.Z\_NULL}
\end{itemize}
