\section{REPL}\label{repl}

\begin{Shaded}
\begin{Highlighting}[]
\NormalTok{Stability: }\DecValTok{3} \NormalTok{- Stable}
\end{Highlighting}
\end{Shaded}

A Read-Eval-Print-Loop (REPL) is available both as a standalone program
and easily includable in other programs. The REPL provides a way to
interactively run JavaScript and see the results. It can be used for
debugging, testing, or just trying things out.

By executing \texttt{node} without any arguments from the command-line
you will be dropped into the REPL. It has simplistic emacs line-editing.

\begin{Shaded}
\begin{Highlighting}[]
\NormalTok{mjr:~$ node}
\NormalTok{Type }\StringTok{'.help'} \KeywordTok{for} \OtherTok{options}\NormalTok{.}
\NormalTok{> a = [ }\DecValTok{1}\NormalTok{, }\DecValTok{2}\NormalTok{, }\DecValTok{3}\NormalTok{];}
\NormalTok{[ }\DecValTok{1}\NormalTok{, }\DecValTok{2}\NormalTok{, }\DecValTok{3} \NormalTok{]}
\NormalTok{> }\OtherTok{a}\NormalTok{.}\FunctionTok{forEach}\NormalTok{(}\KeywordTok{function} \NormalTok{(v) \{}
\NormalTok{...   }\OtherTok{console}\NormalTok{.}\FunctionTok{log}\NormalTok{(v);}
\NormalTok{...   \});}
\DecValTok{1}
\DecValTok{2}
\DecValTok{3}
\end{Highlighting}
\end{Shaded}

For advanced line-editors, start node with the environmental variable
\texttt{NODE\_NO\_READLINE=1}. This will start the main and debugger
REPL in canonical terminal settings which will allow you to use with
\texttt{rlwrap}.

For example, you could add this to your bashrc file:

\begin{Shaded}
\begin{Highlighting}[]
\NormalTok{alias node=}\StringTok{"env NODE_NO_READLINE=1 rlwrap node"}
\end{Highlighting}
\end{Shaded}

\subsection{repl.start(options)}\label{repl.startoptions}

Returns and starts a \texttt{REPLServer} instance. Accepts an
``options'' Object that takes the following values:

\begin{itemize}
\item
  \texttt{prompt} - the prompt and \texttt{stream} for all I/O. Defaults
  to \texttt{\textgreater{}}.
\item
  \texttt{input} - the readable stream to listen to. Defaults to
  \texttt{process.stdin}.
\item
  \texttt{output} - the writable stream to write readline data to.
  Defaults to \texttt{process.stdout}.
\item
  \texttt{terminal} - pass \texttt{true} if the \texttt{stream} should
  be treated like a TTY, and have ANSI/VT100 escape codes written to it.
  Defaults to checking \texttt{isTTY} on the \texttt{output} stream upon
  instantiation.
\item
  \texttt{eval} - function that will be used to eval each given line.
  Defaults to an async wrapper for \texttt{eval()}. See below for an
  example of a custom \texttt{eval}.
\item
  \texttt{useColors} - a boolean which specifies whether or not the
  \texttt{writer} function should output colors. If a different
  \texttt{writer} function is set then this does nothing. Defaults to
  the repl's \texttt{terminal} value.
\item
  \texttt{useGlobal} - if set to \texttt{true}, then the repl will use
  the \texttt{global} object, instead of running scripts in a separate
  context. Defaults to \texttt{false}.
\item
  \texttt{ignoreUndefined} - if set to \texttt{true}, then the repl will
  not output the return value of command if it's \texttt{undefined}.
  Defaults to \texttt{false}.
\item
  \texttt{writer} - the function to invoke for each command that gets
  evaluated which returns the formatting (including coloring) to
  display. Defaults to \texttt{util.inspect}.
\end{itemize}

You can use your own \texttt{eval} function if it has following
signature:

\begin{Shaded}
\begin{Highlighting}[]
\KeywordTok{function} \FunctionTok{eval}\NormalTok{(cmd, context, filename, callback) \{}
  \FunctionTok{callback}\NormalTok{(}\KeywordTok{null}\NormalTok{, result);}
\NormalTok{\}}
\end{Highlighting}
\end{Shaded}

Multiple REPLs may be started against the same running instance of node.
Each will share the same global object but will have unique I/O.

Here is an example that starts a REPL on stdin, a Unix socket, and a TCP
socket:

\begin{Shaded}
\begin{Highlighting}[]
\KeywordTok{var} \NormalTok{net = }\FunctionTok{require}\NormalTok{(}\StringTok{"net"}\NormalTok{),}
    \NormalTok{repl = }\FunctionTok{require}\NormalTok{(}\StringTok{"repl"}\NormalTok{);}

\NormalTok{connections = }\DecValTok{0}\NormalTok{;}

\OtherTok{repl}\NormalTok{.}\FunctionTok{start}\NormalTok{(\{}
  \DataTypeTok{prompt}\NormalTok{: }\StringTok{"node via stdin> "}\NormalTok{,}
  \DataTypeTok{input}\NormalTok{: }\OtherTok{process}\NormalTok{.}\FunctionTok{stdin}\NormalTok{,}
  \DataTypeTok{output}\NormalTok{: }\OtherTok{process}\NormalTok{.}\FunctionTok{stdout}
\NormalTok{\});}

\OtherTok{net}\NormalTok{.}\FunctionTok{createServer}\NormalTok{(}\KeywordTok{function} \NormalTok{(socket) \{}
  \NormalTok{connections += }\DecValTok{1}\NormalTok{;}
  \OtherTok{repl}\NormalTok{.}\FunctionTok{start}\NormalTok{(\{}
    \DataTypeTok{prompt}\NormalTok{: }\StringTok{"node via Unix socket> "}\NormalTok{,}
    \DataTypeTok{input}\NormalTok{: socket,}
    \DataTypeTok{output}\NormalTok{: socket}
  \NormalTok{\}).}\FunctionTok{on}\NormalTok{(}\StringTok{'exit'}\NormalTok{, }\KeywordTok{function}\NormalTok{() \{}
    \OtherTok{socket}\NormalTok{.}\FunctionTok{end}\NormalTok{();}
  \NormalTok{\})}
\NormalTok{\}).}\FunctionTok{listen}\NormalTok{(}\StringTok{"/tmp/node-repl-sock"}\NormalTok{);}

\OtherTok{net}\NormalTok{.}\FunctionTok{createServer}\NormalTok{(}\KeywordTok{function} \NormalTok{(socket) \{}
  \NormalTok{connections += }\DecValTok{1}\NormalTok{;}
  \OtherTok{repl}\NormalTok{.}\FunctionTok{start}\NormalTok{(\{}
    \DataTypeTok{prompt}\NormalTok{: }\StringTok{"node via TCP socket> "}\NormalTok{,}
    \DataTypeTok{input}\NormalTok{: socket,}
    \DataTypeTok{output}\NormalTok{: socket}
  \NormalTok{\}).}\FunctionTok{on}\NormalTok{(}\StringTok{'exit'}\NormalTok{, }\KeywordTok{function}\NormalTok{() \{}
    \OtherTok{socket}\NormalTok{.}\FunctionTok{end}\NormalTok{();}
  \NormalTok{\});}
\NormalTok{\}).}\FunctionTok{listen}\NormalTok{(}\DecValTok{5001}\NormalTok{);}
\end{Highlighting}
\end{Shaded}

Running this program from the command line will start a REPL on stdin.
Other REPL clients may connect through the Unix socket or TCP socket.
\texttt{telnet} is useful for connecting to TCP sockets, and
\texttt{socat} can be used to connect to both Unix and TCP sockets.

By starting a REPL from a Unix socket-based server instead of stdin, you
can connect to a long-running node process without restarting it.

For an example of running a ``full-featured'' (\texttt{terminal}) REPL
over a \texttt{net.Server} and \texttt{net.Socket} instance, see:
https://gist.github.com/2209310

For an example of running a REPL instance over \texttt{curl(1)}, see:
https://gist.github.com/2053342

\subsubsection{Event: `exit'}\label{event-exit}

\texttt{function () \{\}}

Emitted when the user exits the REPL in any of the defined ways. Namely,
typing \texttt{.exit} at the repl, pressing Ctrl+C twice to signal
SIGINT, or pressing Ctrl+D to signal ``end'' on the \texttt{input}
stream.

Example of listening for \texttt{exit}:

\begin{Shaded}
\begin{Highlighting}[]
\OtherTok{r}\NormalTok{.}\FunctionTok{on}\NormalTok{(}\StringTok{'exit'}\NormalTok{, }\KeywordTok{function} \NormalTok{() \{}
  \OtherTok{console}\NormalTok{.}\FunctionTok{log}\NormalTok{(}\StringTok{'Got "exit" event from repl!'}\NormalTok{);}
  \OtherTok{process}\NormalTok{.}\FunctionTok{exit}\NormalTok{();}
\NormalTok{\});}
\end{Highlighting}
\end{Shaded}

\subsubsection{Event: `reset'}\label{event-reset}

\texttt{function (context) \{\}}

Emitted when the REPL's context is reset. This happens when you type
\texttt{.clear}. If you start the repl with
\texttt{\{ useGlobal: true \}} then this event will never be emitted.

Example of listening for \texttt{reset}:

\begin{Shaded}
\begin{Highlighting}[]
\CommentTok{// Extend the initial repl context.}
\NormalTok{r = }\OtherTok{repl}\NormalTok{.}\FunctionTok{start}\NormalTok{(\{ }\OtherTok{options} \NormalTok{... \});}
\OtherTok{someExtension}\NormalTok{.}\FunctionTok{extend}\NormalTok{(}\OtherTok{r}\NormalTok{.}\FunctionTok{context}\NormalTok{);}

\CommentTok{// When a new context is created extend it as well.}
\OtherTok{r}\NormalTok{.}\FunctionTok{on}\NormalTok{(}\StringTok{'reset'}\NormalTok{, }\KeywordTok{function} \NormalTok{(context) \{}
  \OtherTok{console}\NormalTok{.}\FunctionTok{log}\NormalTok{(}\StringTok{'repl has a new context'}\NormalTok{);}
  \OtherTok{someExtension}\NormalTok{.}\FunctionTok{extend}\NormalTok{(context);}
\NormalTok{\});}
\end{Highlighting}
\end{Shaded}

\subsection{REPL Features}\label{repl-features}

Inside the REPL, Control+D will exit. Multi-line expressions can be
input. Tab completion is supported for both global and local variables.

The special variable \texttt{\_} (underscore) contains the result of the
last expression.

\begin{Shaded}
\begin{Highlighting}[]
\NormalTok{> [ }\StringTok{"a"}\NormalTok{, }\StringTok{"b"}\NormalTok{, }\StringTok{"c"} \NormalTok{]}
\NormalTok{[ }\StringTok{'a'}\NormalTok{, }\StringTok{'b'}\NormalTok{, }\StringTok{'c'} \NormalTok{]}
\NormalTok{> }\OtherTok{_}\NormalTok{.}\FunctionTok{length}
\DecValTok{3}
\NormalTok{> _ += }\DecValTok{1}
\DecValTok{4}
\end{Highlighting}
\end{Shaded}

The REPL provides access to any variables in the global scope. You can
expose a variable to the REPL explicitly by assigning it to the
\texttt{context} object associated with each \texttt{REPLServer}. For
example:

\begin{Shaded}
\begin{Highlighting}[]
\CommentTok{// repl_test.js}
\KeywordTok{var} \NormalTok{repl = }\FunctionTok{require}\NormalTok{(}\StringTok{"repl"}\NormalTok{),}
    \NormalTok{msg = }\StringTok{"message"}\NormalTok{;}

\OtherTok{repl}\NormalTok{.}\FunctionTok{start}\NormalTok{(}\StringTok{"> "}\NormalTok{).}\OtherTok{context}\NormalTok{.}\FunctionTok{m} \NormalTok{= msg;}
\end{Highlighting}
\end{Shaded}

Things in the \texttt{context} object appear as local within the REPL:

\begin{Shaded}
\begin{Highlighting}[]
\NormalTok{mjr:~$ node }\OtherTok{repl_test}\NormalTok{.}\FunctionTok{js}
\NormalTok{> m}
\StringTok{'message'}
\end{Highlighting}
\end{Shaded}

There are a few special REPL commands:

\begin{itemize}
\itemsep1pt\parskip0pt\parsep0pt
\item
  \texttt{.break} - While inputting a multi-line expression, sometimes
  you get lost or just don't care about completing it. \texttt{.break}
  will start over.
\item
  \texttt{.clear} - Resets the \texttt{context} object to an empty
  object and clears any multi-line expression.
\item
  \texttt{.exit} - Close the I/O stream, which will cause the REPL to
  exit.
\item
  \texttt{.help} - Show this list of special commands.
\item
  \texttt{.save} - Save the current REPL session to a file
  \textgreater{}.save ./file/to/save.js
\item
  \texttt{.load} - Load a file into the current REPL session.
  \textgreater{}.load ./file/to/load.js
\end{itemize}

The following key combinations in the REPL have these special effects:

\begin{itemize}
\itemsep1pt\parskip0pt\parsep0pt
\item
  \texttt{\textless{}ctrl\textgreater{}C} - Similar to the
  \texttt{.break} keyword. Terminates the current command. Press twice
  on a blank line to forcibly exit.
\item
  \texttt{\textless{}ctrl\textgreater{}D} - Similar to the
  \texttt{.exit} keyword.
\end{itemize}
