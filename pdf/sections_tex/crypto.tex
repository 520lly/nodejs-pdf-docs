\section{Crypto}

\begin{Shaded}
\begin{Highlighting}[]
\DataTypeTok{Stability}\NormalTok{: }\DecValTok{3} \NormalTok{- Stable}
\end{Highlighting}
\end{Shaded}

Use \texttt{require('crypto')} to access this module.

The crypto module requires OpenSSL to be available on the underlying
platform. It offers a way of encapsulating secure credentials to be used
as part of a secure HTTPS net or http connection.

It also offers a set of wrappers for OpenSSL's hash, hmac, cipher,
decipher, sign and verify methods.

\subsection{crypto.createCredentials(details)}

Creates a credentials object, with the optional details being a
dictionary with keys:

\begin{itemize}
\item
  \texttt{pfx} : A string or buffer holding the PFX or PKCS12 encoded
  private key, certificate and CA certificates
\item
  \texttt{key} : A string holding the PEM encoded private key
\item
  \texttt{passphrase} : A string of passphrase for the private key or
  pfx
\item
  \texttt{cert} : A string holding the PEM encoded certificate
\item
  \texttt{ca} : Either a string or list of strings of PEM encoded CA
  certificates to trust.
\item
  \texttt{crl} : Either a string or list of strings of PEM encoded CRLs
  (Certificate Revocation List)
\item
  \texttt{ciphers}: A string describing the ciphers to use or exclude.
  Consult
  \url{http://www.openssl.org/docs/apps/ciphers.html#CIPHER_LIST_FORMAT}
  for details on the format.
\end{itemize}

If no `ca' details are given, then node.js will use the default publicly
trusted list of CAs as given in
\url{http://mxr.mozilla.org/mozilla/source/security/nss/lib/ckfw/builtins/certdata.txt}.

\subsection{crypto.createHash(algorithm)}

Creates and returns a hash object, a cryptographic hash with the given
algorithm which can be used to generate hash digests.

\texttt{algorithm} is dependent on the available algorithms supported by
the version of OpenSSL on the platform. Examples are \texttt{'sha1'},
\texttt{'md5'}, \texttt{'sha256'}, \texttt{'sha512'}, etc. On recent
releases, \texttt{openssl list-message-digest-algorithms} will display
the available digest algorithms.

Example: this program that takes the sha1 sum of a file

\begin{Shaded}
\begin{Highlighting}[]
\KeywordTok{var} \NormalTok{filename = }\KeywordTok{process}\NormalTok{.}\FunctionTok{argv}\NormalTok{[}\DecValTok{2}\NormalTok{];}
\KeywordTok{var} \NormalTok{crypto = require(}\CharTok{'crypto'}\NormalTok{);}
\KeywordTok{var} \NormalTok{fs = require(}\CharTok{'fs'}\NormalTok{);}

\KeywordTok{var} \NormalTok{shasum = }\KeywordTok{crypto}\NormalTok{.}\FunctionTok{createHash}\NormalTok{(}\CharTok{'sha1'}\NormalTok{);}

\KeywordTok{var} \NormalTok{s = }\KeywordTok{fs}\NormalTok{.}\FunctionTok{ReadStream}\NormalTok{(filename);}
\KeywordTok{s}\NormalTok{.}\FunctionTok{on}\NormalTok{(}\CharTok{'data'}\NormalTok{, }\KeywordTok{function}\NormalTok{(d) \{}
  \KeywordTok{shasum}\NormalTok{.}\FunctionTok{update}\NormalTok{(d);}
\NormalTok{\});}

\KeywordTok{s}\NormalTok{.}\FunctionTok{on}\NormalTok{(}\CharTok{'end'}\NormalTok{, }\KeywordTok{function}\NormalTok{() \{}
  \KeywordTok{var} \NormalTok{d = }\KeywordTok{shasum}\NormalTok{.}\FunctionTok{digest}\NormalTok{(}\CharTok{'hex'}\NormalTok{);}
  \KeywordTok{console}\NormalTok{.}\FunctionTok{log}\NormalTok{(d + }\CharTok{'  '} \NormalTok{+ filename);}
\NormalTok{\});}
\end{Highlighting}
\end{Shaded}

\subsection{Class: Hash}

The class for creating hash digests of data.

Returned by \texttt{crypto.createHash}.

\subsubsection{hash.update(data, {[}input\_encoding{]})}

Updates the hash content with the given \texttt{data}, the encoding of
which is given in \texttt{input\_encoding} and can be \texttt{'buffer'},
\texttt{'utf8'}, \texttt{'ascii'} or \texttt{'binary'}. Defaults to
\texttt{'binary'}. This can be called many times with new data as it is
streamed.

\subsubsection{hash.digest({[}encoding{]})}

Calculates the digest of all of the passed data to be hashed. The
\texttt{encoding} can be \texttt{'buffer'}, \texttt{'hex'},
\texttt{'binary'} or \texttt{'base64'}. Defaults to \texttt{'binary'}.

Note: \texttt{hash} object can not be used after \texttt{digest()}
method been called.

\subsection{crypto.createHmac(algorithm, key)}

Creates and returns a hmac object, a cryptographic hmac with the given
algorithm and key.

\texttt{algorithm} is dependent on the available algorithms supported by
OpenSSL - see createHash above. \texttt{key} is the hmac key to be used.

\subsection{Class: Hmac}

Class for creating cryptographic hmac content.

Returned by \texttt{crypto.createHmac}.

\subsubsection{hmac.update(data)}

Update the hmac content with the given \texttt{data}. This can be called
many times with new data as it is streamed.

\subsubsection{hmac.digest({[}encoding{]})}

Calculates the digest of all of the passed data to the hmac. The
\texttt{encoding} can be \texttt{'buffer'}, \texttt{'hex'},
\texttt{'binary'} or \texttt{'base64'}. Defaults to \texttt{'binary'}.

Note: \texttt{hmac} object can not be used after \texttt{digest()}
method been called.

\subsection{crypto.createCipher(algorithm, password)}

Creates and returns a cipher object, with the given algorithm and
password.

\texttt{algorithm} is dependent on OpenSSL, examples are
\texttt{'aes192'}, etc. On recent releases,
\texttt{openssl list-cipher-algorithms} will display the available
cipher algorithms. \texttt{password} is used to derive key and IV, which
must be a \texttt{'binary'} encoded string or a
\href{buffer.html}{buffer}.

\subsection{crypto.createCipheriv(algorithm, key, iv)}

Creates and returns a cipher object, with the given algorithm, key and
iv.

\texttt{algorithm} is the same as the argument to
\texttt{createCipher()}. \texttt{key} is the raw key used by the
algorithm. \texttt{iv} is an
\href{http://en.wikipedia.org/wiki/Initialization\_vector}{initialization
vector}.

\texttt{key} and \texttt{iv} must be \texttt{'binary'} encoded strings
or \href{buffer.html}{buffers}.

\subsection{Class: Cipher}

Class for encrypting data.

Returned by \texttt{crypto.createCipher} and
\texttt{crypto.createCipheriv}.

\subsubsection{cipher.update(data, {[}input\_encoding{]},
{[}output\_encoding{]})}

Updates the cipher with \texttt{data}, the encoding of which is given in
\texttt{input\_encoding} and can be \texttt{'buffer'}, \texttt{'utf8'},
\texttt{'ascii'} or \texttt{'binary'}. Defaults to \texttt{'binary'}.

The \texttt{output\_encoding} specifies the output format of the
enciphered data, and can be \texttt{'buffer'}, \texttt{'binary'},
\texttt{'base64'} or \texttt{'hex'}. Defaults to \texttt{'binary'}.

Returns the enciphered contents, and can be called many times with new
data as it is streamed.

\subsubsection{cipher.final({[}output\_encoding{]})}

Returns any remaining enciphered contents, with
\texttt{output\_encoding} being one of: \texttt{'buffer'},
\texttt{'binary'}, \texttt{'base64'} or \texttt{'hex'}. Defaults to
\texttt{'binary'}.

Note: \texttt{cipher} object can not be used after \texttt{final()}
method been called.

\subsubsection{cipher.setAutoPadding(auto\_padding=true)}

You can disable automatic padding of the input data to block size. If
\texttt{auto\_padding} is false, the length of the entire input data
must be a multiple of the cipher's block size or \texttt{final} will
fail. Useful for non-standard padding, e.g.~using \texttt{0x0} instead
of PKCS padding. You must call this before \texttt{cipher.final}.

\subsection{crypto.createDecipher(algorithm, password)}

Creates and returns a decipher object, with the given algorithm and key.
This is the mirror of the
\hyperref[crypto_crypto_createcipher_algorithm_password]{createCipher()}
above.

\subsection{crypto.createDecipheriv(algorithm, key, iv)}

Creates and returns a decipher object, with the given algorithm, key and
iv. This is the mirror of the
\hyperref[crypto_crypto_createcipheriv_algorithm_key_iv]{createCipheriv()}
above.

\subsection{Class: Decipher}

Class for decrypting data.

Returned by \texttt{crypto.createDecipher} and
\texttt{crypto.createDecipheriv}.

\subsubsection{decipher.update(data, {[}input\_encoding{]},
{[}output\_encoding{]})}

Updates the decipher with \texttt{data}, which is encoded in
\texttt{'buffer'}, \texttt{'binary'}, \texttt{'base64'} or
\texttt{'hex'}. Defaults to \texttt{'binary'}.

The \texttt{output\_decoding} specifies in what format to return the
deciphered plaintext: \texttt{'buffer'}, \texttt{'binary'},
\texttt{'ascii'} or \texttt{'utf8'}. Defaults to \texttt{'binary'}.

\subsubsection{decipher.final({[}output\_encoding{]})}

Returns any remaining plaintext which is deciphered, with
\texttt{output\_encoding} being one of: \texttt{'buffer'},
\texttt{'binary'}, \texttt{'ascii'} or \texttt{'utf8'}. Defaults to
\texttt{'binary'}.

Note: \texttt{decipher} object can not be used after \texttt{final()}
method been called.

\subsubsection{decipher.setAutoPadding(auto\_padding=true)}

You can disable auto padding if the data has been encrypted without
standard block padding to prevent \texttt{decipher.final} from checking
and removing it. Can only work if the input data's length is a multiple
of the ciphers block size. You must call this before streaming data to
\texttt{decipher.update}.

\subsection{crypto.createSign(algorithm)}

Creates and returns a signing object, with the given algorithm. On
recent OpenSSL releases, \texttt{openssl list-public-key-algorithms}
will display the available signing algorithms. Examples are
\texttt{'RSA-SHA256'}.

\subsection{Class: Signer}

Class for generating signatures.

Returned by \texttt{crypto.createSign}.

\subsubsection{signer.update(data)}

Updates the signer object with data. This can be called many times with
new data as it is streamed.

\subsubsection{signer.sign(private\_key, {[}output\_format{]})}

Calculates the signature on all the updated data passed through the
signer. \texttt{private\_key} is a string containing the PEM encoded
private key for signing.

Returns the signature in \texttt{output\_format} which can be
\texttt{'buffer'}, \texttt{'binary'}, \texttt{'hex'} or
\texttt{'base64'}. Defaults to \texttt{'binary'}.

Note: \texttt{signer} object can not be used after \texttt{sign()}
method been called.

\subsection{crypto.createVerify(algorithm)}

Creates and returns a verification object, with the given algorithm.
This is the mirror of the signing object above.

\subsection{Class: Verify}

Class for verifying signatures.

Returned by \texttt{crypto.createVerify}.

\subsubsection{verifier.update(data)}

Updates the verifier object with data. This can be called many times
with new data as it is streamed.

\subsubsection{verifier.verify(object, signature,
{[}signature\_format{]})}

Verifies the signed data by using the \texttt{object} and
\texttt{signature}. \texttt{object} is a string containing a PEM encoded
object, which can be one of RSA public key, DSA public key, or X.509
certificate. \texttt{signature} is the previously calculated signature
for the data, in the \texttt{signature\_format} which can be
\texttt{'buffer'}, \texttt{'binary'}, \texttt{'hex'} or
\texttt{'base64'}. Defaults to \texttt{'binary'}.

Returns true or false depending on the validity of the signature for the
data and public key.

Note: \texttt{verifier} object can not be used after \texttt{verify()}
method been called.

\subsection{crypto.createDiffieHellman(prime\_length)}

Creates a Diffie-Hellman key exchange object and generates a prime of
the given bit length. The generator used is \texttt{2}.

\subsection{crypto.createDiffieHellman(prime, {[}encoding{]})}

Creates a Diffie-Hellman key exchange object using the supplied prime.
The generator used is \texttt{2}. Encoding can be \texttt{'buffer'},
\texttt{'binary'}, \texttt{'hex'}, or \texttt{'base64'}. Defaults to
\texttt{'binary'}.

\subsection{Class: DiffieHellman}

The class for creating Diffie-Hellman key exchanges.

Returned by \texttt{crypto.createDiffieHellman}.

\subsubsection{diffieHellman.generateKeys({[}encoding{]})}

Generates private and public Diffie-Hellman key values, and returns the
public key in the specified encoding. This key should be transferred to
the other party. Encoding can be \texttt{'binary'}, \texttt{'hex'}, or
\texttt{'base64'}. Defaults to \texttt{'binary'}.

\subsubsection{diffieHellman.computeSecret(other\_public\_key,
{[}input\_encoding{]}, {[}output\_encoding{]})}

Computes the shared secret using \texttt{other\_public\_key} as the
other party's public key and returns the computed shared secret.
Supplied key is interpreted using specified \texttt{input\_encoding},
and secret is encoded using specified \texttt{output\_encoding}.
Encodings can be \texttt{'buffer'}, \texttt{'binary'}, \texttt{'hex'},
or \texttt{'base64'}. The input encoding defaults to \texttt{'binary'}.
If no output encoding is given, the input encoding is used as output
encoding.

\subsubsection{diffieHellman.getPrime({[}encoding{]})}

Returns the Diffie-Hellman prime in the specified encoding, which can be
\texttt{'buffer'}, \texttt{'binary'}, \texttt{'hex'}, or
\texttt{'base64'}. Defaults to \texttt{'binary'}.

\subsubsection{diffieHellman.getGenerator({[}encoding{]})}

Returns the Diffie-Hellman prime in the specified encoding, which can be
\texttt{'buffer'}, \texttt{'binary'}, \texttt{'hex'}, or
\texttt{'base64'}. Defaults to \texttt{'binary'}.

\subsubsection{diffieHellman.getPublicKey({[}encoding{]})}

Returns the Diffie-Hellman public key in the specified encoding, which
can be \texttt{'binary'}, \texttt{'hex'}, or \texttt{'base64'}. Defaults
to \texttt{'binary'}.

\subsubsection{diffieHellman.getPrivateKey({[}encoding{]})}

Returns the Diffie-Hellman private key in the specified encoding, which
can be \texttt{'buffer'}, \texttt{'binary'}, \texttt{'hex'}, or
\texttt{'base64'}. Defaults to \texttt{'binary'}.

\subsubsection{diffieHellman.setPublicKey(public\_key, {[}encoding{]})}

Sets the Diffie-Hellman public key. Key encoding can be
\texttt{'buffer', ``'binary'}, \texttt{'hex'} or \texttt{'base64'}.
Defaults to \texttt{'binary'}.

\subsubsection{diffieHellman.setPrivateKey(public\_key, {[}encoding{]})}

Sets the Diffie-Hellman private key. Key encoding can be
\texttt{'buffer'}, \texttt{'binary'}, \texttt{'hex'} or
\texttt{'base64'}. Defaults to \texttt{'binary'}.

\subsection{crypto.getDiffieHellman(group\_name)}

Creates a predefined Diffie-Hellman key exchange object. The supported
groups are: \texttt{'modp1'}, \texttt{'modp2'}, \texttt{'modp5'}
(defined in \href{http://www.rfc-editor.org/rfc/rfc2412.txt}{RFC 2412})
and \texttt{'modp14'}, \texttt{'modp15'}, \texttt{'modp16'},
\texttt{'modp17'}, \texttt{'modp18'} (defined in
\href{http://www.rfc-editor.org/rfc/rfc3526.txt}{RFC 3526}). The
returned object mimics the interface of objects created by
\hyperref[crypto_crypto_creatediffiehellman_prime_encoding]{crypto.createDiffieHellman()}
above, but will not allow to change the keys (with
\hyperref[crypto_diffiehellman_setpublickey_public_key_encoding]{diffieHellman.setPublicKey()}
for example). The advantage of using this routine is that the parties
don't have to generate nor exchange group modulus beforehand, saving
both processor and communication time.

Example (obtaining a shared secret):

\begin{Shaded}
\begin{Highlighting}[]
\KeywordTok{var} \NormalTok{crypto = require(}\CharTok{'crypto'}\NormalTok{);}
\KeywordTok{var} \NormalTok{alice = }\KeywordTok{crypto}\NormalTok{.}\FunctionTok{getDiffieHellman}\NormalTok{(}\CharTok{'modp5'}\NormalTok{);}
\KeywordTok{var} \NormalTok{bob = }\KeywordTok{crypto}\NormalTok{.}\FunctionTok{getDiffieHellman}\NormalTok{(}\CharTok{'modp5'}\NormalTok{);}

\KeywordTok{alice}\NormalTok{.}\FunctionTok{generateKeys}\NormalTok{();}
\KeywordTok{bob}\NormalTok{.}\FunctionTok{generateKeys}\NormalTok{();}

\KeywordTok{var} \NormalTok{alice_secret = }\KeywordTok{alice}\NormalTok{.}\FunctionTok{computeSecret}\NormalTok{(}\KeywordTok{bob}\NormalTok{.}\FunctionTok{getPublicKey}\NormalTok{(), }\CharTok{'binary'}\NormalTok{, }\CharTok{'hex'}\NormalTok{);}
\KeywordTok{var} \NormalTok{bob_secret = }\KeywordTok{bob}\NormalTok{.}\FunctionTok{computeSecret}\NormalTok{(}\KeywordTok{alice}\NormalTok{.}\FunctionTok{getPublicKey}\NormalTok{(), }\CharTok{'binary'}\NormalTok{, }\CharTok{'hex'}\NormalTok{);}

\CommentTok{/* alice_secret and bob_secret should be the same */}
\KeywordTok{console}\NormalTok{.}\FunctionTok{log}\NormalTok{(alice_secret == bob_secret);}
\end{Highlighting}
\end{Shaded}

\subsection{crypto.pbkdf2(password, salt, iterations, keylen, callback)}

Asynchronous PBKDF2 applies pseudorandom function HMAC-SHA1 to derive a
key of given length from the given password, salt and iterations. The
callback gets two arguments \texttt{(err, derivedKey)}.

\subsection{crypto.randomBytes(size, {[}callback{]})}

Generates cryptographically strong pseudo-random data. Usage:

\begin{Shaded}
\begin{Highlighting}[]
\CommentTok{// async}
\KeywordTok{crypto}\NormalTok{.}\FunctionTok{randomBytes}\NormalTok{(}\DecValTok{256}\NormalTok{, }\KeywordTok{function}\NormalTok{(ex, buf) \{}
  \KeywordTok{if} \NormalTok{(ex) }\KeywordTok{throw} \NormalTok{ex;}
  \KeywordTok{console}\NormalTok{.}\FunctionTok{log}\NormalTok{(}\CharTok{'Have %d bytes of random data: %s'}\NormalTok{, }\KeywordTok{buf}\NormalTok{.}\FunctionTok{length}\NormalTok{, buf);}
\NormalTok{\});}

\CommentTok{// sync}
\KeywordTok{try} \NormalTok{\{}
  \KeywordTok{var} \NormalTok{buf = }\KeywordTok{crypto}\NormalTok{.}\FunctionTok{randomBytes}\NormalTok{(}\DecValTok{256}\NormalTok{);}
  \KeywordTok{console}\NormalTok{.}\FunctionTok{log}\NormalTok{(}\CharTok{'Have %d bytes of random data: %s'}\NormalTok{, }\KeywordTok{buf}\NormalTok{.}\FunctionTok{length}\NormalTok{, buf);}
\NormalTok{\} }\KeywordTok{catch} \NormalTok{(ex) \{}
  \CommentTok{// handle error}
\NormalTok{\}}
\end{Highlighting}
\end{Shaded}

