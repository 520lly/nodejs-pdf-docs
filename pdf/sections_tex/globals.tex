\section{Global Objects}

These objects are available in all modules. Some of these objects aren't
actually in the global scope but in the module scope - this will be
noted.

\subsection{global}

\begin{itemize}
\item
  \{Object\} The global namespace object.
\end{itemize}

In browsers, the top-level scope is the global scope. That means that in
browsers if you're in the global scope \texttt{var something} will
define a global variable. In Node this is different. The top-level scope
is not the global scope; \texttt{var something} inside a Node module
will be local to that module.

\subsection{process}

\begin{itemize}
\item
  \{Object\}
\end{itemize}

The process object. See the
\href{process.html\#process\_process}{process object} section.

\subsection{console}

\begin{itemize}
\item
  \{Object\}
\end{itemize}

Used to print to stdout and stderr. See the \href{stdio.html}{stdio}
section.

\subsection{Class: Buffer}

\begin{itemize}
\item
  \{Function\}
\end{itemize}

Used to handle binary data. See the \href{buffer.html}{buffer section}

\subsection{require()}

\begin{itemize}
\item
  \{Function\}
\end{itemize}

To require modules. See the
\href{modules.html\#modules\_modules}{Modules} section. \texttt{require}
isn't actually a global but rather local to each module.

\subsubsection{require.resolve()}

Use the internal \texttt{require()} machinery to look up the location of
a module, but rather than loading the module, just return the resolved
filename.

\subsubsection{require.cache}

\begin{itemize}
\item
  \{Object\}
\end{itemize}

Modules are cached in this object when they are required. By deleting a
key value from this object, the next \texttt{require} will reload the
module.

\subsubsection{require.extensions}

\begin{Shaded}
\begin{Highlighting}[]
\DataTypeTok{Stability}\NormalTok{: }\DecValTok{0} \NormalTok{- Deprecated}
\end{Highlighting}
\end{Shaded}

\begin{itemize}
\item
  \{Object\}
\end{itemize}

Instruct \texttt{require} on how to handle certain file extensions.

Process files with the extension \texttt{.sjs} as \texttt{.js}:

\begin{Shaded}
\begin{Highlighting}[]
\KeywordTok{require}\NormalTok{.}\FunctionTok{extensions}\NormalTok{[}\CharTok{'.sjs'}\NormalTok{] = }\KeywordTok{require}\NormalTok{.}\FunctionTok{extensions}\NormalTok{[}\CharTok{'.js'}\NormalTok{];}
\end{Highlighting}
\end{Shaded}

\textbf{Deprecated} In the past, this list has been used to load
non-JavaScript modules into Node by compiling them on-demand. However,
in practice, there are much better ways to do this, such as loading
modules via some other Node program, or compiling them to JavaScript
ahead of time.

Since the Module system is locked, this feature will probably never go
away. However, it may have subtle bugs and complexities that are best
left untouched.

\subsection{\_\_filename}

\begin{itemize}
\item
  \{String\}
\end{itemize}

The filename of the code being executed. This is the resolved absolute
path of this code file. For a main program this is not necessarily the
same filename used in the command line. The value inside a module is the
path to that module file.

Example: running \texttt{node example.js} from \texttt{/Users/mjr}

\begin{Shaded}
\begin{Highlighting}[]
\KeywordTok{console}\NormalTok{.}\FunctionTok{log}\NormalTok{(__filename);}
\CommentTok{// /Users/mjr/example.js}
\end{Highlighting}
\end{Shaded}

\texttt{\_\_filename} isn't actually a global but rather local to each
module.

\subsection{\_\_dirname}

\begin{itemize}
\item
  \{String\}
\end{itemize}

The name of the directory that the currently executing script resides
in.

Example: running \texttt{node example.js} from \texttt{/Users/mjr}

\begin{Shaded}
\begin{Highlighting}[]
\KeywordTok{console}\NormalTok{.}\FunctionTok{log}\NormalTok{(__dirname);}
\CommentTok{// /Users/mjr}
\end{Highlighting}
\end{Shaded}

\texttt{\_\_dirname} isn't actually a global but rather local to each
module.

\subsection{module}

\begin{itemize}
\item
  \{Object\}
\end{itemize}

A reference to the current module. In particular \texttt{module.exports}
is the same as the \texttt{exports} object. \texttt{module} isn't
actually a global but rather local to each module.

See the \href{modules.html}{module system documentation} for more
information.

\subsection{exports}

A reference to the \texttt{module.exports} object which is shared
between all instances of the current module and made accessible through
\texttt{require()}. See \href{modules.html}{module system documentation}
for details on when to use \texttt{exports} and when to use
\texttt{module.exports}. \texttt{exports} isn't actually a global but
rather local to each module.

See the \href{modules.html}{module system documentation} for more
information.

See the \href{modules.html}{module section} for more information.

\subsection{setTimeout(cb, ms)}

Run callback \texttt{cb} after \emph{at least} \texttt{ms} milliseconds.
The actual delay depends on external factors like OS timer granularity
and system load.

The timeout must be in the range of 1-2,147,483,647 inclusive. If the
value is outside that range, it's changed to 1 millisecond. Broadly
speaking, a timer cannot span more than 24.8 days.

Returns an opaque value that represents the timer.

\subsection{clearTimeout(t)}

Stop a timer that was previously created with \texttt{setTimeout()}. The
callback will not execute.

\subsection{setInterval(cb, ms)}

Run callback \texttt{cb} repeatedly every \texttt{ms} milliseconds. Note
that the actual interval may vary, depending on external factors like OS
timer granularity and system load. It's never less than \texttt{ms} but
it may be longer.

The interval must be in the range of 1-2,147,483,647 inclusive. If the
value is outside that range, it's changed to 1 millisecond. Broadly
speaking, a timer cannot span more than 24.8 days.

Returns an opaque value that represents the timer.

\subsection{clearInterval(t)}

Stop a timer that was previously created with \texttt{setInterval()}.
The callback will not execute.

The timer functions are global variables. See the
\href{timers.html}{timers} section.
