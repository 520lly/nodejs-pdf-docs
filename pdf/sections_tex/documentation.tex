\section{About this Documentation}

The goal of this documentation is to comprehensively explain the Node.js
API, both from a reference as well as a conceptual point of view. Each
section describes a built-in module or high-level concept.

Where appropriate, property types, method arguments, and the arguments
provided to event handlers are detailed in a list underneath the topic
heading.

Every \texttt{.html} document has a corresponding \texttt{.json}
document presenting the same information in a structured manner. This
feature is experimental, and added for the benefit of IDEs and other
utilities that wish to do programmatic things with the documentation.

Every \texttt{.html} and \texttt{.json} file is generated based on the
corresponding \texttt{.markdown} file in the \texttt{doc/api/} folder in
node's source tree. The documentation is generated using the
\texttt{tools/doc/generate.js} program. The HTML template is located at
\texttt{doc/template.html}.

\subsection{Stability Index}

Throughout the documentation, you will see indications of a section's
stability. The Node.js API is still somewhat changing, and as it
matures, certain parts are more reliable than others. Some are so
proven, and so relied upon, that they are unlikely to ever change at
all. Others are brand new and experimental, or known to be hazardous and
in the process of being redesigned.

The notices look like this:

\begin{Shaded}
\begin{Highlighting}[]
\DataTypeTok{Stability}\NormalTok{: }\DecValTok{1} \NormalTok{Experimental}
\end{Highlighting}
\end{Shaded}

The stability indices are as follows:

\begin{itemize}
\item
  \textbf{0 - Deprecated} This feature is known to be problematic, and
  changes are planned. Do not rely on it. Use of the feature may cause
  warnings. Backwards compatibility should not be expected.
\item
  \textbf{1 - Experimental} This feature was introduced recently, and
  may change or be removed in future versions. Please try it out and
  provide feedback. If it addresses a use-case that is important to you,
  tell the node core team.
\item
  \textbf{2 - Unstable} The API is in the process of settling, but has
  not yet had sufficient real-world testing to be considered stable.
  Backwards-compatibility will be maintained if reasonable.
\item
  \textbf{3 - Stable} The API has proven satisfactory, but cleanup in
  the underlying code may cause minor changes. Backwards-compatibility
  is guaranteed.
\item
  \textbf{4 - API Frozen} This API has been tested extensively in
  production and is unlikely to ever have to change.
\item
  \textbf{5 - Locked} Unless serious bugs are found, this code will not
  ever change. Please do not suggest changes in this area; they will be
  refused.
\end{itemize}

\subsection{JSON Output}

\begin{Shaded}
\begin{Highlighting}[]
\DataTypeTok{Stability}\NormalTok{: }\DecValTok{1} \NormalTok{- Experimental}
\end{Highlighting}
\end{Shaded}

Every HTML file in the markdown has a corresponding JSON file with the
same data.

This feature is new as of node v0.6.12. It is experimental.
