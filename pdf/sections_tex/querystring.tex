\section{Query String}\label{query-string}

\begin{Shaded}
\begin{Highlighting}[]
\NormalTok{Stability: }\DecValTok{3} \NormalTok{- Stable}
\end{Highlighting}
\end{Shaded}

This module provides utilities for dealing with query strings. It
provides the following methods:

\subsection{querystring.stringify(obj{[}, sep{]}{[}, eq{]}{[},
options{]})}\label{querystring.stringifyobj-sep-eq-options}

Serialize an object to a query string. Optionally override the default
separator (\texttt{\textquotesingle{}\&\textquotesingle{}}) and
assignment (\texttt{\textquotesingle{}=\textquotesingle{}}) characters.

Options object may contain \texttt{encodeURIComponent} property
(\texttt{querystring.escape} by default), it can be used to encode
string with \texttt{non-utf8} encoding if necessary.

Example:

\begin{Shaded}
\begin{Highlighting}[]
\OtherTok{querystring}\NormalTok{.}\FunctionTok{stringify}\NormalTok{(\{ }\DataTypeTok{foo}\NormalTok{: }\StringTok{'bar'}\NormalTok{, }\DataTypeTok{baz}\NormalTok{: [}\StringTok{'qux'}\NormalTok{, }\StringTok{'quux'}\NormalTok{], }\DataTypeTok{corge}\NormalTok{: }\StringTok{''} \NormalTok{\})}
\CommentTok{// returns}
\StringTok{'foo=bar&baz=qux&baz=quux&corge='}

\OtherTok{querystring}\NormalTok{.}\FunctionTok{stringify}\NormalTok{(\{}\DataTypeTok{foo}\NormalTok{: }\StringTok{'bar'}\NormalTok{, }\DataTypeTok{baz}\NormalTok{: }\StringTok{'qux'}\NormalTok{\}, }\StringTok{';'}\NormalTok{, }\StringTok{':'}\NormalTok{)}
\CommentTok{// returns}
\StringTok{'foo:bar;baz:qux'}

\CommentTok{// Suppose gbkEncodeURIComponent function already exists,}
\CommentTok{// it can encode string with `gbk` encoding}
\OtherTok{querystring}\NormalTok{.}\FunctionTok{stringify}\NormalTok{(\{ }\DataTypeTok{w}\NormalTok{: }\StringTok{'中文'}\NormalTok{, }\DataTypeTok{foo}\NormalTok{: }\StringTok{'bar'} \NormalTok{\}, }\KeywordTok{null}\NormalTok{, }\KeywordTok{null}\NormalTok{,}
  \NormalTok{\{ }\DataTypeTok{encodeURIComponent}\NormalTok{: gbkEncodeURIComponent \})}
\CommentTok{// returns}
\StringTok{'w=%D6%D0%CE%C4&foo=bar'}
\end{Highlighting}
\end{Shaded}

\subsection{querystring.parse(str{[}, sep{]}{[}, eq{]}{[},
options{]})}\label{querystring.parsestr-sep-eq-options}

Deserialize a query string to an object. Optionally override the default
separator (\texttt{\textquotesingle{}\&\textquotesingle{}}) and
assignment (\texttt{\textquotesingle{}=\textquotesingle{}}) characters.

Options object may contain \texttt{maxKeys} property (equal to 1000 by
default), it'll be used to limit processed keys. Set it to 0 to remove
key count limitation.

Options object may contain \texttt{decodeURIComponent} property
(\texttt{querystring.unescape} by default), it can be used to decode a
\texttt{non-utf8} encoding string if necessary.

Example:

\begin{Shaded}
\begin{Highlighting}[]
\OtherTok{querystring}\NormalTok{.}\FunctionTok{parse}\NormalTok{(}\StringTok{'foo=bar&baz=qux&baz=quux&corge'}\NormalTok{)}
\CommentTok{// returns}
\NormalTok{\{ }\DataTypeTok{foo}\NormalTok{: }\StringTok{'bar'}\NormalTok{, }\DataTypeTok{baz}\NormalTok{: [}\StringTok{'qux'}\NormalTok{, }\StringTok{'quux'}\NormalTok{], }\DataTypeTok{corge}\NormalTok{: }\StringTok{''} \NormalTok{\}}

\CommentTok{// Suppose gbkDecodeURIComponent function already exists,}
\CommentTok{// it can decode `gbk` encoding string}
\OtherTok{querystring}\NormalTok{.}\FunctionTok{parse}\NormalTok{(}\StringTok{'w=%D6%D0%CE%C4&foo=bar'}\NormalTok{, }\KeywordTok{null}\NormalTok{, }\KeywordTok{null}\NormalTok{,}
  \NormalTok{\{ }\DataTypeTok{decodeURIComponent}\NormalTok{: gbkDecodeURIComponent \})}
\CommentTok{// returns}
\NormalTok{\{ }\DataTypeTok{w}\NormalTok{: }\StringTok{'中文'}\NormalTok{, }\DataTypeTok{foo}\NormalTok{: }\StringTok{'bar'} \NormalTok{\}}
\end{Highlighting}
\end{Shaded}

\subsection{querystring.escape}\label{querystring.escape}

The escape function used by \texttt{querystring.stringify}, provided so
that it could be overridden if necessary.

\subsection{querystring.unescape}\label{querystring.unescape}

The unescape function used by \texttt{querystring.parse}, provided so
that it could be overridden if necessary.

It will try to use \texttt{decodeURIComponent} in the first place, but
if that fails it falls back to a safer equivalent that doesn't throw on
malformed URLs.
