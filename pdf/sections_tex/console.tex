\section{console}\label{console}

\begin{Shaded}
\begin{Highlighting}[]
\NormalTok{Stability: }\DecValTok{4} \NormalTok{- API Frozen}
\end{Highlighting}
\end{Shaded}

\begin{itemize}
\itemsep1pt\parskip0pt\parsep0pt
\item
  \{Object\}
\end{itemize}

For printing to stdout and stderr. Similar to the console object
functions provided by most web browsers, here the output is sent to
stdout or stderr.

The console functions are synchronous when the destination is a terminal
or a file (to avoid lost messages in case of premature exit) and
asynchronous when it's a pipe (to avoid blocking for long periods of
time).

That is, in the following example, stdout is non-blocking while stderr
is blocking:

\begin{Shaded}
\begin{Highlighting}[]
\NormalTok{$ node }\OtherTok{script}\NormalTok{.}\FunctionTok{js} \DecValTok{2}\NormalTok{> }\OtherTok{error}\NormalTok{.}\FunctionTok{log} \NormalTok{| tee }\OtherTok{info}\NormalTok{.}\FunctionTok{log}
\end{Highlighting}
\end{Shaded}

In daily use, the blocking/non-blocking dichotomy is not something you
should worry about unless you log huge amounts of data.

\subsection{console.log({[}data{]}{[},
\ldots{}{]})}\label{console.logdata}

Prints to stdout with newline. This function can take multiple arguments
in a \texttt{printf()}-like way. Example:

\begin{Shaded}
\begin{Highlighting}[]
\KeywordTok{var} \NormalTok{count = }\DecValTok{5}\NormalTok{;}
\OtherTok{console}\NormalTok{.}\FunctionTok{log}\NormalTok{(}\StringTok{'count: %d'}\NormalTok{, count);}
\CommentTok{// prints 'count: 5'}
\end{Highlighting}
\end{Shaded}

If formatting elements are not found in the first string then
\texttt{util.inspect} is used on each argument. See
\href{util.html\#util_util_format_format}{util.format()} for more
information.

\subsection{console.info({[}data{]}{[},
\ldots{}{]})}\label{console.infodata}

Same as \texttt{console.log}.

\subsection{console.error({[}data{]}{[},
\ldots{}{]})}\label{console.errordata}

Same as \texttt{console.log} but prints to stderr.

\subsection{console.warn({[}data{]}{[},
\ldots{}{]})}\label{console.warndata}

Same as \texttt{console.error}.

\subsection{console.dir(obj{[},
options{]})}\label{console.dirobj-options}

Uses \texttt{util.inspect} on \texttt{obj} and prints resulting string
to stdout. This function bypasses any custom \texttt{inspect()} function
on \texttt{obj}. An optional \emph{options} object may be passed that
alters certain aspects of the formatted string:

\begin{itemize}
\item
  \texttt{showHidden} - if \texttt{true} then the object's
  non-enumerable properties will be shown too. Defaults to
  \texttt{false}.
\item
  \texttt{depth} - tells \texttt{inspect} how many times to recurse
  while formatting the object. This is useful for inspecting large
  complicated objects. Defaults to \texttt{2}. To make it recurse
  indefinitely pass \texttt{null}.
\item
  \texttt{colors} - if \texttt{true}, then the output will be styled
  with ANSI color codes. Defaults to \texttt{false}. Colors are
  customizable, see below.
\end{itemize}

\subsection{console.time(label)}\label{console.timelabel}

Mark a time.

\subsection{console.timeEnd(label)}\label{console.timeendlabel}

Finish timer, record output. Example:

\begin{Shaded}
\begin{Highlighting}[]
\OtherTok{console}\NormalTok{.}\FunctionTok{time}\NormalTok{(}\StringTok{'100-elements'}\NormalTok{);}
\KeywordTok{for} \NormalTok{(}\KeywordTok{var} \NormalTok{i = }\DecValTok{0}\NormalTok{; i < }\DecValTok{100}\NormalTok{; i++) \{}
  \NormalTok{;}
\NormalTok{\}}
\OtherTok{console}\NormalTok{.}\FunctionTok{timeEnd}\NormalTok{(}\StringTok{'100-elements'}\NormalTok{);}
\CommentTok{// prints 100-elements: 262ms}
\end{Highlighting}
\end{Shaded}

\subsection{console.trace(message{[},
\ldots{}{]})}\label{console.tracemessage}

Print to stderr \texttt{\textquotesingle{}Trace\ :\textquotesingle{}},
followed by the formatted message and stack trace to the current
position.

\subsection{console.assert(value{[}, message{]}{[},
\ldots{}{]})}\label{console.assertvalue-message}

Similar to
\href{assert.html\#assert_assert_value_message_assert_ok_value_message}{assert.ok()},
but the error message is formatted as \texttt{util.format(message...)}.
