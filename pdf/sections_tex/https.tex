\section{HTTPS}

\begin{Shaded}
\begin{Highlighting}[]
\DataTypeTok{Stability}\NormalTok{: }\DecValTok{3} \NormalTok{- Stable}
\end{Highlighting}
\end{Shaded}

HTTPS is the HTTP protocol over TLS/SSL. In Node this is implemented as
a separate module.

\subsection{Class: https.Server}

This class is a subclass of \texttt{tls.Server} and emits events same as
\texttt{http.Server}. See \texttt{http.Server} for more information.

\subsection{https.createServer(options, {[}requestListener{]})}

Returns a new HTTPS web server object. The \texttt{options} is similar
to
\href{tls.html\#tls\_tls\_createserver\_options\_secureconnectionlistener}{tls.createServer()}.
The \texttt{requestListener} is a function which is automatically added
to the \texttt{'request'} event.

Example:

\begin{Shaded}
\begin{Highlighting}[]
\CommentTok{// curl -k https://localhost:8000/}
\KeywordTok{var} \NormalTok{https = require(}\CharTok{'https'}\NormalTok{);}
\KeywordTok{var} \NormalTok{fs = require(}\CharTok{'fs'}\NormalTok{);}

\KeywordTok{var} \NormalTok{options = \{}
  \DataTypeTok{key}\NormalTok{: }\KeywordTok{fs}\NormalTok{.}\FunctionTok{readFileSync}\NormalTok{(}\CharTok{'test/fixtures/keys/agent2-key.pem'}\NormalTok{),}
  \DataTypeTok{cert}\NormalTok{: }\KeywordTok{fs}\NormalTok{.}\FunctionTok{readFileSync}\NormalTok{(}\CharTok{'test/fixtures/keys/agent2-cert.pem'}\NormalTok{)}
\NormalTok{\};}

\KeywordTok{https}\NormalTok{.}\FunctionTok{createServer}\NormalTok{(options, }\KeywordTok{function} \NormalTok{(req, res) \{}
  \KeywordTok{res}\NormalTok{.}\FunctionTok{writeHead}\NormalTok{(}\DecValTok{200}\NormalTok{);}
  \KeywordTok{res}\NormalTok{.}\FunctionTok{end}\NormalTok{(}\StringTok{"hello world}\CharTok{\textbackslash{}n}\StringTok{"}\NormalTok{);}
\NormalTok{\}).}\FunctionTok{listen}\NormalTok{(}\DecValTok{8000}\NormalTok{);}
\end{Highlighting}
\end{Shaded}

Or

\begin{Shaded}
\begin{Highlighting}[]
\KeywordTok{var} \NormalTok{https = require(}\CharTok{'https'}\NormalTok{);}
\KeywordTok{var} \NormalTok{fs = require(}\CharTok{'fs'}\NormalTok{);}

\KeywordTok{var} \NormalTok{options = \{}
  \DataTypeTok{pfx}\NormalTok{: }\KeywordTok{fs}\NormalTok{.}\FunctionTok{readFileSync}\NormalTok{(}\CharTok{'server.pfx'}\NormalTok{)}
\NormalTok{\};}

\KeywordTok{https}\NormalTok{.}\FunctionTok{createServer}\NormalTok{(options, }\KeywordTok{function} \NormalTok{(req, res) \{}
  \KeywordTok{res}\NormalTok{.}\FunctionTok{writeHead}\NormalTok{(}\DecValTok{200}\NormalTok{);}
  \KeywordTok{res}\NormalTok{.}\FunctionTok{end}\NormalTok{(}\StringTok{"hello world}\CharTok{\textbackslash{}n}\StringTok{"}\NormalTok{);}
\NormalTok{\}).}\FunctionTok{listen}\NormalTok{(}\DecValTok{8000}\NormalTok{);}
\end{Highlighting}
\end{Shaded}

\subsection{https.request(options, callback)}

Makes a request to a secure web server. All options from
\href{http.html\#http\_http\_request\_options\_callback}{http.request()}
are valid.

Example:

\begin{Shaded}
\begin{Highlighting}[]
\KeywordTok{var} \NormalTok{https = require(}\CharTok{'https'}\NormalTok{);}

\KeywordTok{var} \NormalTok{options = \{}
  \DataTypeTok{host}\NormalTok{: }\CharTok{'encrypted.google.com'}\NormalTok{,}
  \DataTypeTok{port}\NormalTok{: }\DecValTok{443}\NormalTok{,}
  \DataTypeTok{path}\NormalTok{: }\CharTok{'/'}\NormalTok{,}
  \DataTypeTok{method}\NormalTok{: }\CharTok{'GET'}
\NormalTok{\};}

\KeywordTok{var} \NormalTok{req = }\KeywordTok{https}\NormalTok{.}\FunctionTok{request}\NormalTok{(options, }\KeywordTok{function}\NormalTok{(res) \{}
  \KeywordTok{console}\NormalTok{.}\FunctionTok{log}\NormalTok{(}\StringTok{"statusCode: "}\NormalTok{, }\KeywordTok{res}\NormalTok{.}\FunctionTok{statusCode}\NormalTok{);}
  \KeywordTok{console}\NormalTok{.}\FunctionTok{log}\NormalTok{(}\StringTok{"headers: "}\NormalTok{, }\KeywordTok{res}\NormalTok{.}\FunctionTok{headers}\NormalTok{);}

  \KeywordTok{res}\NormalTok{.}\FunctionTok{on}\NormalTok{(}\CharTok{'data'}\NormalTok{, }\KeywordTok{function}\NormalTok{(d) \{}
    \KeywordTok{process.stdout}\NormalTok{.}\FunctionTok{write}\NormalTok{(d);}
  \NormalTok{\});}
\NormalTok{\});}
\KeywordTok{req}\NormalTok{.}\FunctionTok{end}\NormalTok{();}

\KeywordTok{req}\NormalTok{.}\FunctionTok{on}\NormalTok{(}\CharTok{'error'}\NormalTok{, }\KeywordTok{function}\NormalTok{(e) \{}
  \KeywordTok{console}\NormalTok{.}\FunctionTok{error}\NormalTok{(e);}
\NormalTok{\});}
\end{Highlighting}
\end{Shaded}

The options argument has the following options

\begin{itemize}
\item
  host: IP or domain of host to make request to. Defaults to
  \texttt{'localhost'}.
\item
  port: port of host to request to. Defaults to 443.
\item
  path: Path to request. Default \texttt{'/'}.
\item
  method: HTTP request method. Default \texttt{'GET'}.
\item
  \texttt{host}: A domain name or IP address of the server to issue the
  request to. Defaults to \texttt{'localhost'}.
\item
  \texttt{hostname}: To support \texttt{url.parse()} \texttt{hostname}
  is preferred over \texttt{host}
\item
  \texttt{port}: Port of remote server. Defaults to 443.
\item
  \texttt{method}: A string specifying the HTTP request method. Defaults
  to \texttt{'GET'}.
\item
  \texttt{path}: Request path. Defaults to \texttt{'/'}. Should include
  query string if any. E.G. \texttt{'/index.html?page=12'}
\item
  \texttt{headers}: An object containing request headers.
\item
  \texttt{auth}: Basic authentication i.e. \texttt{'user:password'} to
  compute an Authorization header.
\item
  \texttt{agent}: Controls \hyperref[https\_class\_https\_agent]{Agent}
  behavior. When an Agent is used request will default to
  \texttt{Connection: keep-alive}. Possible values:
\item
  \texttt{undefined} (default): use
  \hyperref[https\_https\_globalagent]{globalAgent} for this host and
  port.
\item
  \texttt{Agent} object: explicitly use the passed in \texttt{Agent}.
\item
  \texttt{false}: opts out of connection pooling with an Agent, defaults
  request to \texttt{Connection: close}.
\end{itemize}

The following options from
\href{tls.html\#tls\_tls\_connect\_options\_secureconnectlistener}{tls.connect()}
can also be specified. However, a
\hyperref[https\_https\_globalagent]{globalAgent} silently ignores
these.

\begin{itemize}
\item
  \texttt{pfx}: Certificate, Private key and CA certificates to use for
  SSL. Default \texttt{null}.
\item
  \texttt{key}: Private key to use for SSL. Default \texttt{null}.
\item
  \texttt{passphrase}: A string of passphrase for the private key or
  pfx. Default \texttt{null}.
\item
  \texttt{cert}: Public x509 certificate to use. Default \texttt{null}.
\item
  \texttt{ca}: An authority certificate or array of authority
  certificates to check the remote host against.
\item
  \texttt{ciphers}: A string describing the ciphers to use or exclude.
  Consult
  \url{http://www.openssl.org/docs/apps/ciphers.html#CIPHER_LIST_FORMAT}
  for details on the format.
\item
  \texttt{rejectUnauthorized}: If \texttt{true}, the server certificate
  is verified against the list of supplied CAs. An \texttt{'error'}
  event is emitted if verification fails. Verification happens at the
  connection level, \emph{before} the HTTP request is sent. Default
  \texttt{false}.
\end{itemize}

In order to specify these options, use a custom \texttt{Agent}.

Example:

\begin{Shaded}
\begin{Highlighting}[]
\KeywordTok{var} \NormalTok{options = \{}
  \DataTypeTok{host}\NormalTok{: }\CharTok{'encrypted.google.com'}\NormalTok{,}
  \DataTypeTok{port}\NormalTok{: }\DecValTok{443}\NormalTok{,}
  \DataTypeTok{path}\NormalTok{: }\CharTok{'/'}\NormalTok{,}
  \DataTypeTok{method}\NormalTok{: }\CharTok{'GET'}\NormalTok{,}
  \DataTypeTok{key}\NormalTok{: }\KeywordTok{fs}\NormalTok{.}\FunctionTok{readFileSync}\NormalTok{(}\CharTok{'test/fixtures/keys/agent2-key.pem'}\NormalTok{),}
  \DataTypeTok{cert}\NormalTok{: }\KeywordTok{fs}\NormalTok{.}\FunctionTok{readFileSync}\NormalTok{(}\CharTok{'test/fixtures/keys/agent2-cert.pem'}\NormalTok{)}
\NormalTok{\};}
\KeywordTok{options}\NormalTok{.}\FunctionTok{agent} \NormalTok{= }\KeywordTok{new} \KeywordTok{https}\NormalTok{.}\FunctionTok{Agent}\NormalTok{(options);}

\KeywordTok{var} \NormalTok{req = }\KeywordTok{https}\NormalTok{.}\FunctionTok{request}\NormalTok{(options, }\KeywordTok{function}\NormalTok{(res) \{}
  \NormalTok{...}
\NormalTok{\}}
\end{Highlighting}
\end{Shaded}

Or does not use an \texttt{Agent}.

Example:

\begin{Shaded}
\begin{Highlighting}[]
\KeywordTok{var} \NormalTok{options = \{}
  \DataTypeTok{host}\NormalTok{: }\CharTok{'encrypted.google.com'}\NormalTok{,}
  \DataTypeTok{port}\NormalTok{: }\DecValTok{443}\NormalTok{,}
  \DataTypeTok{path}\NormalTok{: }\CharTok{'/'}\NormalTok{,}
  \DataTypeTok{method}\NormalTok{: }\CharTok{'GET'}\NormalTok{,}
  \DataTypeTok{key}\NormalTok{: }\KeywordTok{fs}\NormalTok{.}\FunctionTok{readFileSync}\NormalTok{(}\CharTok{'test/fixtures/keys/agent2-key.pem'}\NormalTok{),}
  \DataTypeTok{cert}\NormalTok{: }\KeywordTok{fs}\NormalTok{.}\FunctionTok{readFileSync}\NormalTok{(}\CharTok{'test/fixtures/keys/agent2-cert.pem'}\NormalTok{),}
  \DataTypeTok{agent}\NormalTok{: }\KeywordTok{false}
\NormalTok{\};}

\KeywordTok{var} \NormalTok{req = }\KeywordTok{https}\NormalTok{.}\FunctionTok{request}\NormalTok{(options, }\KeywordTok{function}\NormalTok{(res) \{}
  \NormalTok{...}
\NormalTok{\}}
\end{Highlighting}
\end{Shaded}

\subsection{https.get(options, callback)}

Like \texttt{http.get()} but for HTTPS.

Example:

\begin{Shaded}
\begin{Highlighting}[]
\KeywordTok{var} \NormalTok{https = require(}\CharTok{'https'}\NormalTok{);}

\KeywordTok{https}\NormalTok{.}\FunctionTok{get}\NormalTok{(\{ }\DataTypeTok{host}\NormalTok{: }\CharTok{'encrypted.google.com'}\NormalTok{, }\DataTypeTok{path}\NormalTok{: }\CharTok{'/'} \NormalTok{\}, }\KeywordTok{function}\NormalTok{(res) \{}
  \KeywordTok{console}\NormalTok{.}\FunctionTok{log}\NormalTok{(}\StringTok{"statusCode: "}\NormalTok{, }\KeywordTok{res}\NormalTok{.}\FunctionTok{statusCode}\NormalTok{);}
  \KeywordTok{console}\NormalTok{.}\FunctionTok{log}\NormalTok{(}\StringTok{"headers: "}\NormalTok{, }\KeywordTok{res}\NormalTok{.}\FunctionTok{headers}\NormalTok{);}

  \KeywordTok{res}\NormalTok{.}\FunctionTok{on}\NormalTok{(}\CharTok{'data'}\NormalTok{, }\KeywordTok{function}\NormalTok{(d) \{}
    \KeywordTok{process.stdout}\NormalTok{.}\FunctionTok{write}\NormalTok{(d);}
  \NormalTok{\});}

\NormalTok{\}).}\FunctionTok{on}\NormalTok{(}\CharTok{'error'}\NormalTok{, }\KeywordTok{function}\NormalTok{(e) \{}
  \KeywordTok{console}\NormalTok{.}\FunctionTok{error}\NormalTok{(e);}
\NormalTok{\});}
\end{Highlighting}
\end{Shaded}

\subsection{Class: https.Agent}

An Agent object for HTTPS similar to
\href{http.html\#http\_class\_http\_agent}{http.Agent}. See
{[}https.request(){]}{[}{]} for more information.

\subsection{https.globalAgent}

Global instance of \hyperref[https\_class\_https\_agent]{https.Agent}
for all HTTPS client requests.
