\section{TLS (SSL)}\label{tls-ssl}

\begin{Shaded}
\begin{Highlighting}[]
\NormalTok{Stability: }\DecValTok{3} \NormalTok{- Stable}
\end{Highlighting}
\end{Shaded}

Use \texttt{require(\textquotesingle{}tls\textquotesingle{})} to access
this module.

The \texttt{tls} module uses OpenSSL to provide Transport Layer Security
and/or Secure Socket Layer: encrypted stream communication.

TLS/SSL is a public/private key infrastructure. Each client and each
server must have a private key. A private key is created like this:

\begin{Shaded}
\begin{Highlighting}[]
\NormalTok{openssl genrsa -out ryans-}\OtherTok{key}\NormalTok{.}\FunctionTok{pem} \DecValTok{2048}
\end{Highlighting}
\end{Shaded}

All servers and some clients need to have a certificate. Certificates
are public keys signed by a Certificate Authority or self-signed. The
first step to getting a certificate is to create a ``Certificate Signing
Request'' (CSR) file. This is done with:

\begin{Shaded}
\begin{Highlighting}[]
\NormalTok{openssl req -}\KeywordTok{new} \NormalTok{-sha256 -key ryans-}\OtherTok{key}\NormalTok{.}\FunctionTok{pem} \NormalTok{-out ryans-}\OtherTok{csr}\NormalTok{.}\FunctionTok{pem}
\end{Highlighting}
\end{Shaded}

To create a self-signed certificate with the CSR, do this:

\begin{Shaded}
\begin{Highlighting}[]
\NormalTok{openssl x509 -req -}\KeywordTok{in} \NormalTok{ryans-}\OtherTok{csr}\NormalTok{.}\FunctionTok{pem} \NormalTok{-signkey ryans-}\OtherTok{key}\NormalTok{.}\FunctionTok{pem} \NormalTok{-out ryans-}\OtherTok{cert}\NormalTok{.}\FunctionTok{pem}
\end{Highlighting}
\end{Shaded}

Alternatively you can send the CSR to a Certificate Authority for
signing.

For Perfect Forward Secrecy, it is required to generate Diffie-Hellman
parameters:

\begin{Shaded}
\begin{Highlighting}[]
\NormalTok{openssl dhparam -outform PEM -out }\OtherTok{dhparam}\NormalTok{.}\FunctionTok{pem} \DecValTok{2048}
\end{Highlighting}
\end{Shaded}

To create .pfx or .p12, do this:

\begin{Shaded}
\begin{Highlighting}[]
\NormalTok{openssl pkcs12 -}\KeywordTok{export} \NormalTok{-}\KeywordTok{in} \NormalTok{agent5-}\OtherTok{cert}\NormalTok{.}\FunctionTok{pem} \NormalTok{-inkey agent5-}\OtherTok{key}\NormalTok{.}\FunctionTok{pem} \NormalTok{\textbackslash{}}
    \NormalTok{-certfile ca-}\OtherTok{cert}\NormalTok{.}\FunctionTok{pem} \NormalTok{-out }\OtherTok{agent5}\NormalTok{.}\FunctionTok{pfx}
\end{Highlighting}
\end{Shaded}

\begin{itemize}
\itemsep1pt\parskip0pt\parsep0pt
\item
  \texttt{in}: certificate
\item
  \texttt{inkey}: private key
\item
  \texttt{certfile}: all CA certs concatenated in one file like
  \texttt{cat\ ca1-cert.pem\ ca2-cert.pem\ \textgreater{}\ ca-cert.pem}
\end{itemize}

\subsection{Protocol support}\label{protocol-support}

Node.js is compiled with SSLv2 and SSLv3 protocol support by default,
but these protocols are \textbf{disabled}. They are considered insecure
and could be easily compromised as was shown by
\href{https://access.redhat.com/articles/1232123}{CVE-2014-3566}.
However, in some situations, it may cause problems with legacy
clients/servers (such as Internet Explorer 6). If you wish to enable
SSLv2 or SSLv3, run node with the \texttt{-\/-enable-ssl2} or
\texttt{-\/-enable-ssl3} flag respectively. In future versions of
Node.js SSLv2 and SSLv3 will not be compiled in by default.

There is a way to force node into using SSLv3 or SSLv2 only mode by
explicitly specifying \texttt{secureProtocol} to
\texttt{\textquotesingle{}SSLv3\_method\textquotesingle{}} or
\texttt{\textquotesingle{}SSLv2\_method\textquotesingle{}}.

The default protocol method Node.js uses is \texttt{SSLv23\_method}
which would be more accurately named \texttt{AutoNegotiate\_method}.
This method will try and negotiate from the highest level down to
whatever the client supports. To provide a secure default, Node.js
(since v0.10.33) explicitly disables the use of SSLv3 and SSLv2 by
setting the \texttt{secureOptions} to be
\texttt{SSL\_OP\_NO\_SSLv3\textbar{}SSL\_OP\_NO\_SSLv2} (again, unless
you have passed \texttt{-\/-enable-ssl3}, or \texttt{-\/-enable-ssl2},
or \texttt{SSLv3\_method} as \texttt{secureProtocol}).

If you have set \texttt{secureOptions} to anything, we will not override
your options.

The ramifications of this behavior change:

\begin{itemize}
\itemsep1pt\parskip0pt\parsep0pt
\item
  If your application is behaving as a secure server, clients who are
  \texttt{SSLv3} only will now not be able to appropriately negotiate a
  connection and will be refused. In this case your server will emit a
  \texttt{clientError} event. The error message will include
  \texttt{\textquotesingle{}wrong\ version\ number\textquotesingle{}}.
\item
  If your application is behaving as a secure client and communicating
  with a server that doesn't support methods more secure than SSLv3 then
  your connection won't be able to negotiate and will fail. In this case
  your client will emit a an \texttt{error} event. The error message
  will include
  \texttt{\textquotesingle{}wrong\ version\ number\textquotesingle{}}.
\end{itemize}

\subsection{Client-initiated renegotiation attack
mitigation}\label{client-initiated-renegotiation-attack-mitigation}

The TLS protocol lets the client renegotiate certain aspects of the TLS
session. Unfortunately, session renegotiation requires a disproportional
amount of server-side resources, which makes it a potential vector for
denial-of-service attacks.

To mitigate this, renegotiations are limited to three times every 10
minutes. An error is emitted on the
\hyperref[tlsux5fclassux5ftlsux5ftlssocket]{tls.TLSSocket} instance when
the threshold is exceeded. The limits are configurable:

\begin{itemize}
\item
  \texttt{tls.CLIENT\_RENEG\_LIMIT}: renegotiation limit, default is 3.
\item
  \texttt{tls.CLIENT\_RENEG\_WINDOW}: renegotiation window in seconds,
  default is 10 minutes.
\end{itemize}

Don't change the defaults unless you know what you are doing.

To test your server, connect to it with
\texttt{openssl\ s\_client\ -connect\ address:port} and tap
\texttt{R\textless{}CR\textgreater{}} (that's the letter \texttt{R}
followed by a carriage return) a few times.

\subsection{NPN and SNI}\label{npn-and-sni}

NPN (Next Protocol Negotiation) and SNI (Server Name Indication) are TLS
handshake extensions allowing you:

\begin{itemize}
\itemsep1pt\parskip0pt\parsep0pt
\item
  NPN - to use one TLS server for multiple protocols (HTTP, SPDY)
\item
  SNI - to use one TLS server for multiple hostnames with different SSL
  certificates.
\end{itemize}

\subsection{Perfect Forward Secrecy}\label{perfect-forward-secrecy}

The term
``\href{http://en.wikipedia.org/wiki/Perfect_forward_secrecy}{Forward
Secrecy}'' or ``Perfect Forward Secrecy'' describes a feature of
key-agreement (i.e.~key-exchange) methods. Practically it means that
even if the private key of a (your) server is compromised, communication
can only be decrypted by eavesdroppers if they manage to obtain the
key-pair specifically generated for each session.

This is achieved by randomly generating a key pair for key-agreement on
every handshake (in contrary to the same key for all sessions). Methods
implementing this technique, thus offering Perfect Forward Secrecy, are
called ``ephemeral''.

Currently two methods are commonly used to achieve Perfect Forward
Secrecy (note the character ``E'' appended to the traditional
abbreviations):

\begin{itemize}
\itemsep1pt\parskip0pt\parsep0pt
\item
  \href{https://en.wikipedia.org/wiki/Diffie\%E2\%80\%93Hellman_key_exchange}{DHE}
  - An ephemeral version of the Diffie Hellman key-agreement protocol.
\item
  \href{https://en.wikipedia.org/wiki/Elliptic_curve_Diffie\%E2\%80\%93Hellman}{ECDHE}
  - An ephemeral version of the Elliptic Curve Diffie Hellman
  key-agreement protocol.
\end{itemize}

Ephemeral methods may have some performance drawbacks, because key
generation is expensive.

\subsection{Modifying the Default Cipher
Suite}\label{modifying-the-default-cipher-suite}

Node.js is built with a default suite of enabled and disabled ciphers.
Currently, the default cipher suite is:

\begin{Shaded}
\begin{Highlighting}[]
\NormalTok{ECDHE-RSA-AES256-SHA384:DHE-RSA-AES256-SHA384:ECDHE-RSA-AES256-SHA256:}
\NormalTok{DHE-RSA-AES256-SHA256:ECDHE-RSA-AES128-SHA256:DHE-RSA-AES128-SHA256:}
\NormalTok{HIGH:!aNULL:!eNULL:!EXPORT:!DES:!RC4:!MD5:!PSK:!SRP:!CAMELLIA}
\end{Highlighting}
\end{Shaded}

This default can be overridden entirely using the
\texttt{-\/-cipher-list} command line switch or
\texttt{NODE\_CIPHER\_LIST} environment variable. For instance:

\begin{Shaded}
\begin{Highlighting}[]
\NormalTok{node --cipher-list=ECDHE-RSA-AES256-SHA384:DHE-RSA-AES256-SHA384}
\end{Highlighting}
\end{Shaded}

Setting the environment variable would have the same effect:

\begin{Shaded}
\begin{Highlighting}[]
\NormalTok{NODE_CIPHER_LIST=ECDHE-RSA-AES256-SHA384:DHE-RSA-AES256-SHA384}
\end{Highlighting}
\end{Shaded}

CAUTION: The default cipher suite has been carefully selected to reflect
current security best practices and risk mitigation. Changing the
default cipher suite can have a significant impact on the security of an
application. The \texttt{-\/-cipher-list} and
\texttt{NODE\_CIPHER\_LIST} options should only be used if absolutely
necessary.

\subsubsection{Using Legacy Default Cipher
Suite}\label{using-legacy-default-cipher-suite}

It is possible for the built-in default cipher suite to change from one
release of Node.js to another. For instance, v0.10.38 uses a different
default than v0.12.2. Such changes can cause issues with applications
written to assume certain specific defaults. To help buffer applications
against such changes, the \texttt{-\/-enable-legacy-cipher-list} command
line switch or \texttt{NODE\_LEGACY\_CIPHER\_LIST} environment variable
can be set to specify a specific preset default:

\begin{Shaded}
\begin{Highlighting}[]
\NormalTok{# Use the }\OtherTok{v0}\NormalTok{.}\FloatTok{10.38} \NormalTok{defaults}
\NormalTok{node --enable-legacy-cipher-list=}\OtherTok{v0}\NormalTok{.}\FloatTok{10.38}
\CommentTok{// or}
\NormalTok{NODE_LEGACY_CIPHER_LIST=}\OtherTok{v0}\NormalTok{.}\FloatTok{10.38}

\NormalTok{# Use the }\OtherTok{v0}\NormalTok{.}\FloatTok{12.2} \NormalTok{defaults}
\NormalTok{node --enable-legacy-cipher-list=}\OtherTok{v0}\NormalTok{.}\FloatTok{12.2}
\CommentTok{// or}
\NormalTok{NODE_LEGACY_CIPHER_LIST=}\OtherTok{v0}\NormalTok{.}\FloatTok{12.2}
\end{Highlighting}
\end{Shaded}

Currently, the values supported for the
\texttt{enable-legacy-cipher-list} switch and
\texttt{NODE\_LEGACY\_CIPHER\_LIST} environment variable include:

\begin{Shaded}
\begin{Highlighting}[]
\OtherTok{v0}\NormalTok{.}\FloatTok{10.38} \NormalTok{- To enable the }\KeywordTok{default} \NormalTok{cipher suite used }\KeywordTok{in} \OtherTok{v0}\NormalTok{.}\FloatTok{10.38}

  \NormalTok{ECDHE-RSA-AES128-SHA256:AES128-GCM-SHA256:RC4:HIGH:!MD5:!aNULL:!EDH}

\OtherTok{v0}\NormalTok{.}\FloatTok{10.39} \NormalTok{- To enable the }\KeywordTok{default} \NormalTok{cipher suite used }\KeywordTok{in} \OtherTok{v0}\NormalTok{.}\FloatTok{10.39}

  \NormalTok{ECDHE-RSA-AES128-SHA256:AES128-GCM-SHA256:HIGH:!RC4:!MD5:!aNULL:!EDH}

\OtherTok{v0}\NormalTok{.}\FloatTok{12.2} \NormalTok{- To enable the }\KeywordTok{default} \NormalTok{cipher suite used }\KeywordTok{in} \OtherTok{v0}\NormalTok{.}\FloatTok{12.2}

  \NormalTok{ECDHE-RSA-AES128-SHA256:DHE-RSA-AES128-SHA256:AES128-GCM-SHA256:RC4:}
  \NormalTok{HIGH:!MD5:!aNULL}

\OtherTok{v}\NormalTok{.}\FloatTok{0.12.3} \NormalTok{- To enable the }\KeywordTok{default} \NormalTok{cipher suite used }\KeywordTok{in} \OtherTok{v0}\NormalTok{.}\FloatTok{12.3}

  \NormalTok{ECDHE-RSA-AES128-SHA256:DHE-RSA-AES128-SHA256:AES128-GCM-SHA256:HIGH:}
  \NormalTok{!RC4:!MD5:!aNULL}
\end{Highlighting}
\end{Shaded}

These legacy cipher suites are also made available for use via the
\texttt{getLegacyCiphers()} method:

\begin{Shaded}
\begin{Highlighting}[]
\KeywordTok{var} \NormalTok{tls = }\FunctionTok{require}\NormalTok{(}\StringTok{'tls'}\NormalTok{);}
\OtherTok{console}\NormalTok{.}\FunctionTok{log}\NormalTok{(}\OtherTok{tls}\NormalTok{.}\FunctionTok{getLegacyCiphers}\NormalTok{(}\StringTok{'v0.10.38'}\NormalTok{));}
\end{Highlighting}
\end{Shaded}

CAUTION: Changes to the default cipher suite are typically made in order
to strengthen the default security for applications running within
Node.js. Reverting back to the defaults used by older releases can
weaken the security of your applications. The legacy cipher suites
should only be used if absolutely necessary.

\subsection{tls.getCiphers()}\label{tls.getciphers}

Returns an array with the names of the supported SSL ciphers.

Example:

\begin{Shaded}
\begin{Highlighting}[]
\KeywordTok{var} \NormalTok{ciphers = }\OtherTok{tls}\NormalTok{.}\FunctionTok{getCiphers}\NormalTok{();}
\OtherTok{console}\NormalTok{.}\FunctionTok{log}\NormalTok{(ciphers); }\CommentTok{// ['AES128-SHA', 'AES256-SHA', ...]}
\end{Highlighting}
\end{Shaded}

\subsection{tls.getLegacyCiphers(version)}\label{tls.getlegacyciphersversion}

Returns the legacy default cipher suite for the specified Node.js
release.

Example: var cipher\_suite = tls.getLegacyCiphers(`v0.10.38');

\subsection{tls.createServer(options{[},
secureConnectionListener{]})}\label{tls.createserveroptions-secureconnectionlistener}

Creates a new \hyperref[tlsux5fclassux5ftlsux5fserver]{tls.Server}. The
\texttt{connectionListener} argument is automatically set as a listener
for the \hyperref[tlsux5feventux5fsecureconnection]{secureConnection}
event. The \texttt{options} object has these possibilities:

\begin{itemize}
\item
  \texttt{pfx}: A string or \texttt{Buffer} containing the private key,
  certificate and CA certs of the server in PFX or PKCS12 format.
  (Mutually exclusive with the \texttt{key}, \texttt{cert} and
  \texttt{ca} options.)
\item
  \texttt{key}: A string or \texttt{Buffer} containing the private key
  of the server in PEM format. (Could be an array of keys). (Required)
\item
  \texttt{passphrase}: A string of passphrase for the private key or
  pfx.
\item
  \texttt{cert}: A string or \texttt{Buffer} containing the certificate
  key of the server in PEM format. (Could be an array of certs).
  (Required)
\item
  \texttt{ca}: An array of strings or \texttt{Buffer}s of trusted
  certificates in PEM format. If this is omitted several well known
  ``root'' CAs will be used, like VeriSign. These are used to authorize
  connections.
\item
  \texttt{crl} : Either a string or list of strings of PEM encoded CRLs
  (Certificate Revocation List)
\item
  \texttt{ciphers}: A string describing the ciphers to use or exclude,
  separated by \texttt{:}. The default cipher suite is:

\begin{Shaded}
\begin{Highlighting}[]
\NormalTok{ECDHE-RSA-AES256-SHA384:DHE-RSA-AES256-SHA384:ECDHE-RSA-AES256-SHA256:}
\NormalTok{DHE-RSA-AES256-SHA256:ECDHE-RSA-AES128-SHA256:DHE-RSA-AES128-SHA256:}
\NormalTok{HIGH:!aNULL:!eNULL:!EXPORT:!DES:!RC4:!MD5:!PSK:!SRP:!CAMELLIA}
\end{Highlighting}
\end{Shaded}

  The default cipher suite prefers ECDHE and DHE ciphers for Perfect
  Forward secrecy, while offering \emph{some} backward compatibility.
  Old clients which rely on insecure and deprecated RC4 or DES-based
  ciphers (like Internet Explorer 6) aren't able to complete the
  handshake with the default configuration. If you absolutely must
  support these clients, the
  \href{https://wiki.mozilla.org/Security/Server_Side_TLS}{TLS
  recommendations} may offer a compatible cipher suite. For more details
  on the format, see the
  \href{http://www.openssl.org/docs/apps/ciphers.html\#CIPHER_LIST_FORMAT}{OpenSSL
  cipher list format documentation}.
\item
  \texttt{ecdhCurve}: A string describing a named curve to use for ECDH
  key agreement or false to disable ECDH.

  Defaults to \texttt{prime256v1}. Consult
  \href{http://www.rfc-editor.org/rfc/rfc4492.txt}{RFC 4492} for more
  details.
\item
  \texttt{dhparam}: DH parameter file to use for DHE key agreement. Use
  \texttt{openssl\ dhparam} command to create it. If the file is invalid
  to load, it is silently discarded.
\item
  \texttt{handshakeTimeout}: Abort the connection if the SSL/TLS
  handshake does not finish in this many milliseconds. The default is
  120 seconds.

  A \texttt{\textquotesingle{}clientError\textquotesingle{}} is emitted
  on the \texttt{tls.Server} object whenever a handshake times out.
\item
  \texttt{honorCipherOrder} : When choosing a cipher, use the server's
  preferences instead of the client preferences. Default: \texttt{true}.

  Although, this option is disabled by default, it is \emph{recommended}
  that you use this option in conjunction with the \texttt{ciphers}
  option to mitigate BEAST attacks.

  Note: If SSLv2 is used, the server will send its list of preferences
  to the client, and the client chooses the cipher. Support for SSLv2 is
  disabled unless node.js was configured with
  \texttt{./configure\ -\/-with-sslv2}.
\item
  \texttt{requestCert}: If \texttt{true} the server will request a
  certificate from clients that connect and attempt to verify that
  certificate. Default: \texttt{false}.
\item
  \texttt{rejectUnauthorized}: If \texttt{true} the server will reject
  any connection which is not authorized with the list of supplied CAs.
  This option only has an effect if \texttt{requestCert} is
  \texttt{true}. Default: \texttt{false}.
\item
  \texttt{checkServerIdentity(servername,\ cert)}: Provide an override
  for checking server's hostname against the certificate. Should return
  an error if verification fails. Return \texttt{undefined} if passing.
\item
  \texttt{NPNProtocols}: An array or \texttt{Buffer} of possible NPN
  protocols. (Protocols should be ordered by their priority).
\item
  \texttt{SNICallback(servername,\ cb)}: A function that will be called
  if client supports SNI TLS extension. Two argument will be passed to
  it: \texttt{servername}, and \texttt{cb}. \texttt{SNICallback} should
  invoke \texttt{cb(null,\ ctx)}, where \texttt{ctx} is a SecureContext
  instance. (You can use \texttt{tls.createSecureContext(...)} to get
  proper SecureContext). If \texttt{SNICallback} wasn't provided -
  default callback with high-level API will be used (see below).
\item
  \texttt{sessionTimeout}: An integer specifying the seconds after which
  TLS session identifiers and TLS session tickets created by the server
  are timed out. See
  \href{http://www.openssl.org/docs/ssl/SSL_CTX_set_timeout.html}{SSL\_CTX\_set\_timeout}
  for more details.
\item
  \texttt{ticketKeys}: A 48-byte \texttt{Buffer} instance consisting of
  16-byte prefix, 16-byte hmac key, 16-byte AES key. You could use it to
  accept tls session tickets on multiple instances of tls server.

  NOTE: Automatically shared between \texttt{cluster} module workers.
\item
  \texttt{sessionIdContext}: A string containing an opaque identifier
  for session resumption. If \texttt{requestCert} is \texttt{true}, the
  default is MD5 hash value generated from command-line. Otherwise, the
  default is not provided.
\item
  \texttt{secureProtocol}: The SSL method to use, e.g.
  \texttt{SSLv3\_method} to force SSL version 3. The possible values
  depend on your installation of OpenSSL and are defined in the constant
  \href{http://www.openssl.org/docs/ssl/ssl.html\#DEALING_WITH_PROTOCOL_METHODS}{SSL\_METHODS}.
\item
  \texttt{secureOptions}: Set server options. For example, to disable
  the SSLv3 protocol set the \texttt{SSL\_OP\_NO\_SSLv3} flag. See
  \href{https://www.openssl.org/docs/ssl/SSL_CTX_set_options.html}{SSL\_CTX\_set\_options}
  for all available options.
\end{itemize}

Here is a simple example echo server:

\begin{Shaded}
\begin{Highlighting}[]
\KeywordTok{var} \NormalTok{tls = }\FunctionTok{require}\NormalTok{(}\StringTok{'tls'}\NormalTok{);}
\KeywordTok{var} \NormalTok{fs = }\FunctionTok{require}\NormalTok{(}\StringTok{'fs'}\NormalTok{);}

\KeywordTok{var} \NormalTok{options = \{}
  \DataTypeTok{key}\NormalTok{: }\OtherTok{fs}\NormalTok{.}\FunctionTok{readFileSync}\NormalTok{(}\StringTok{'server-key.pem'}\NormalTok{),}
  \DataTypeTok{cert}\NormalTok{: }\OtherTok{fs}\NormalTok{.}\FunctionTok{readFileSync}\NormalTok{(}\StringTok{'server-cert.pem'}\NormalTok{),}

  \CommentTok{// This is necessary only if using the client certificate authentication.}
  \DataTypeTok{requestCert}\NormalTok{: }\KeywordTok{true}\NormalTok{,}

  \CommentTok{// This is necessary only if the client uses the self-signed certificate.}
  \DataTypeTok{ca}\NormalTok{: [ }\OtherTok{fs}\NormalTok{.}\FunctionTok{readFileSync}\NormalTok{(}\StringTok{'client-cert.pem'}\NormalTok{) ]}
\NormalTok{\};}

\KeywordTok{var} \NormalTok{server = }\OtherTok{tls}\NormalTok{.}\FunctionTok{createServer}\NormalTok{(options, }\KeywordTok{function}\NormalTok{(socket) \{}
  \OtherTok{console}\NormalTok{.}\FunctionTok{log}\NormalTok{(}\StringTok{'server connected'}\NormalTok{,}
              \OtherTok{socket}\NormalTok{.}\FunctionTok{authorized} \NormalTok{? }\StringTok{'authorized'} \NormalTok{: }\StringTok{'unauthorized'}\NormalTok{);}
  \OtherTok{socket}\NormalTok{.}\FunctionTok{write}\NormalTok{(}\StringTok{"welcome!}\CharTok{\textbackslash{}n}\StringTok{"}\NormalTok{);}
  \OtherTok{socket}\NormalTok{.}\FunctionTok{setEncoding}\NormalTok{(}\StringTok{'utf8'}\NormalTok{);}
  \OtherTok{socket}\NormalTok{.}\FunctionTok{pipe}\NormalTok{(socket);}
\NormalTok{\});}
\OtherTok{server}\NormalTok{.}\FunctionTok{listen}\NormalTok{(}\DecValTok{8000}\NormalTok{, }\KeywordTok{function}\NormalTok{() \{}
  \OtherTok{console}\NormalTok{.}\FunctionTok{log}\NormalTok{(}\StringTok{'server bound'}\NormalTok{);}
\NormalTok{\});}
\end{Highlighting}
\end{Shaded}

Or

\begin{Shaded}
\begin{Highlighting}[]
\KeywordTok{var} \NormalTok{tls = }\FunctionTok{require}\NormalTok{(}\StringTok{'tls'}\NormalTok{);}
\KeywordTok{var} \NormalTok{fs = }\FunctionTok{require}\NormalTok{(}\StringTok{'fs'}\NormalTok{);}

\KeywordTok{var} \NormalTok{options = \{}
  \DataTypeTok{pfx}\NormalTok{: }\OtherTok{fs}\NormalTok{.}\FunctionTok{readFileSync}\NormalTok{(}\StringTok{'server.pfx'}\NormalTok{),}

  \CommentTok{// This is necessary only if using the client certificate authentication.}
  \DataTypeTok{requestCert}\NormalTok{: }\KeywordTok{true}\NormalTok{,}

\NormalTok{\};}

\KeywordTok{var} \NormalTok{server = }\OtherTok{tls}\NormalTok{.}\FunctionTok{createServer}\NormalTok{(options, }\KeywordTok{function}\NormalTok{(socket) \{}
  \OtherTok{console}\NormalTok{.}\FunctionTok{log}\NormalTok{(}\StringTok{'server connected'}\NormalTok{,}
              \OtherTok{socket}\NormalTok{.}\FunctionTok{authorized} \NormalTok{? }\StringTok{'authorized'} \NormalTok{: }\StringTok{'unauthorized'}\NormalTok{);}
  \OtherTok{socket}\NormalTok{.}\FunctionTok{write}\NormalTok{(}\StringTok{"welcome!}\CharTok{\textbackslash{}n}\StringTok{"}\NormalTok{);}
  \OtherTok{socket}\NormalTok{.}\FunctionTok{setEncoding}\NormalTok{(}\StringTok{'utf8'}\NormalTok{);}
  \OtherTok{socket}\NormalTok{.}\FunctionTok{pipe}\NormalTok{(socket);}
\NormalTok{\});}
\OtherTok{server}\NormalTok{.}\FunctionTok{listen}\NormalTok{(}\DecValTok{8000}\NormalTok{, }\KeywordTok{function}\NormalTok{() \{}
  \OtherTok{console}\NormalTok{.}\FunctionTok{log}\NormalTok{(}\StringTok{'server bound'}\NormalTok{);}
\NormalTok{\});}
\end{Highlighting}
\end{Shaded}

You can test this server by connecting to it with
\texttt{openssl\ s\_client}:

\begin{Shaded}
\begin{Highlighting}[]
\NormalTok{openssl s_client -connect }\FloatTok{127.0.0.1}\NormalTok{:}\DecValTok{8000}
\end{Highlighting}
\end{Shaded}

\subsection{tls.connect(options{[},
callback{]})}\label{tls.connectoptions-callback}

\subsection{tls.connect(port{[}, host{]}{[}, options{]}{[},
callback{]})}\label{tls.connectport-host-options-callback}

Creates a new client connection to the given \texttt{port} and
\texttt{host} (old API) or \texttt{options.port} and
\texttt{options.host}. (If \texttt{host} is omitted, it defaults to
\texttt{localhost}.) \texttt{options} should be an object which
specifies:

\begin{itemize}
\item
  \texttt{host}: Host the client should connect to
\item
  \texttt{port}: Port the client should connect to
\item
  \texttt{socket}: Establish secure connection on a given socket rather
  than creating a new socket. If this option is specified, \texttt{host}
  and \texttt{port} are ignored.
\item
  \texttt{path}: Creates unix socket connection to path. If this option
  is specified, \texttt{host} and \texttt{port} are ignored.
\item
  \texttt{ciphers}: A string describing the ciphers to use or exclude.

  Defaults to
  \texttt{ECDHE-RSA-AES128-SHA256:DHE-RSA-AES128-SHA256:AES128-GCM-SHA256:HIGH:!RC4:!MD5:!aNULL}.
  Consult the
  \href{http://www.openssl.org/docs/apps/ciphers.html\#CIPHER_LIST_FORMAT}{OpenSSL
  cipher list format documentation} for details on the format.

  The full list of available ciphers can be obtained via
  \hyperref[tlsux5ftlsux5fgetciphers]{tls.getCiphers}.

  \texttt{ECDHE-RSA-AES128-SHA256}, \texttt{DHE-RSA-AES128-SHA256} and
  \texttt{AES128-GCM-SHA256} are TLS v1.2 ciphers and used when Node.js
  is linked against OpenSSL 1.0.1 or newer, such as the bundled version
  of OpenSSL.
\item
  \texttt{pfx}: A string or \texttt{Buffer} containing the private key,
  certificate and CA certs of the client in PFX or PKCS12 format.
\item
  \texttt{key}: A string or \texttt{Buffer} containing the private key
  of the client in PEM format. (Could be an array of keys).
\item
  \texttt{passphrase}: A string of passphrase for the private key or
  pfx.
\item
  \texttt{cert}: A string or \texttt{Buffer} containing the certificate
  key of the client in PEM format. (Could be an array of certs).
\item
  \texttt{ca}: An array of strings or \texttt{Buffer}s of trusted
  certificates in PEM format. If this is omitted several well known
  ``root'' CAs will be used, like VeriSign. These are used to authorize
  connections.
\item
  \texttt{rejectUnauthorized}: If \texttt{true}, the server certificate
  is verified against the list of supplied CAs. An
  \texttt{\textquotesingle{}error\textquotesingle{}} event is emitted if
  verification fails; \texttt{err.code} contains the OpenSSL error code.
  Default: \texttt{true}.
\item
  \texttt{NPNProtocols}: An array of strings or \texttt{Buffer}s
  containing supported NPN protocols. \texttt{Buffer}s should have
  following format: \texttt{0x05hello0x05world}, where first byte is
  next protocol name's length. (Passing array should usually be much
  simpler:
  \texttt{{[}\textquotesingle{}hello\textquotesingle{},\ \textquotesingle{}world\textquotesingle{}{]}}.)
\item
  \texttt{servername}: Servername for SNI (Server Name Indication) TLS
  extension.
\item
  \texttt{secureProtocol}: The SSL method to use, e.g.
  \texttt{SSLv3\_method} to force SSL version 3. The possible values
  depend on your installation of OpenSSL and are defined in the constant
  \href{http://www.openssl.org/docs/ssl/ssl.html\#DEALING_WITH_PROTOCOL_METHODS}{SSL\_METHODS}.
\item
  \texttt{session}: A \texttt{Buffer} instance, containing TLS session.
\end{itemize}

The \texttt{callback} parameter will be added as a listener for the
\hyperref[tlsux5feventux5fsecureconnect]{`secureConnect'} event.

\texttt{tls.connect()} returns a
\hyperref[tlsux5fclassux5ftlsux5ftlssocket]{tls.TLSSocket} object.

Here is an example of a client of echo server as described previously:

\begin{Shaded}
\begin{Highlighting}[]
\KeywordTok{var} \NormalTok{tls = }\FunctionTok{require}\NormalTok{(}\StringTok{'tls'}\NormalTok{);}
\KeywordTok{var} \NormalTok{fs = }\FunctionTok{require}\NormalTok{(}\StringTok{'fs'}\NormalTok{);}

\KeywordTok{var} \NormalTok{options = \{}
  \CommentTok{// These are necessary only if using the client certificate authentication}
  \DataTypeTok{key}\NormalTok{: }\OtherTok{fs}\NormalTok{.}\FunctionTok{readFileSync}\NormalTok{(}\StringTok{'client-key.pem'}\NormalTok{),}
  \DataTypeTok{cert}\NormalTok{: }\OtherTok{fs}\NormalTok{.}\FunctionTok{readFileSync}\NormalTok{(}\StringTok{'client-cert.pem'}\NormalTok{),}

  \CommentTok{// This is necessary only if the server uses the self-signed certificate}
  \DataTypeTok{ca}\NormalTok{: [ }\OtherTok{fs}\NormalTok{.}\FunctionTok{readFileSync}\NormalTok{(}\StringTok{'server-cert.pem'}\NormalTok{) ]}
\NormalTok{\};}

\KeywordTok{var} \NormalTok{socket = }\OtherTok{tls}\NormalTok{.}\FunctionTok{connect}\NormalTok{(}\DecValTok{8000}\NormalTok{, options, }\KeywordTok{function}\NormalTok{() \{}
  \OtherTok{console}\NormalTok{.}\FunctionTok{log}\NormalTok{(}\StringTok{'client connected'}\NormalTok{,}
              \OtherTok{socket}\NormalTok{.}\FunctionTok{authorized} \NormalTok{? }\StringTok{'authorized'} \NormalTok{: }\StringTok{'unauthorized'}\NormalTok{);}
  \OtherTok{process}\NormalTok{.}\OtherTok{stdin}\NormalTok{.}\FunctionTok{pipe}\NormalTok{(socket);}
  \OtherTok{process}\NormalTok{.}\OtherTok{stdin}\NormalTok{.}\FunctionTok{resume}\NormalTok{();}
\NormalTok{\});}
\OtherTok{socket}\NormalTok{.}\FunctionTok{setEncoding}\NormalTok{(}\StringTok{'utf8'}\NormalTok{);}
\OtherTok{socket}\NormalTok{.}\FunctionTok{on}\NormalTok{(}\StringTok{'data'}\NormalTok{, }\KeywordTok{function}\NormalTok{(data) \{}
  \OtherTok{console}\NormalTok{.}\FunctionTok{log}\NormalTok{(data);}
\NormalTok{\});}
\OtherTok{socket}\NormalTok{.}\FunctionTok{on}\NormalTok{(}\StringTok{'end'}\NormalTok{, }\KeywordTok{function}\NormalTok{() \{}
  \OtherTok{server}\NormalTok{.}\FunctionTok{close}\NormalTok{();}
\NormalTok{\});}
\end{Highlighting}
\end{Shaded}

Or

\begin{Shaded}
\begin{Highlighting}[]
\KeywordTok{var} \NormalTok{tls = }\FunctionTok{require}\NormalTok{(}\StringTok{'tls'}\NormalTok{);}
\KeywordTok{var} \NormalTok{fs = }\FunctionTok{require}\NormalTok{(}\StringTok{'fs'}\NormalTok{);}

\KeywordTok{var} \NormalTok{options = \{}
  \DataTypeTok{pfx}\NormalTok{: }\OtherTok{fs}\NormalTok{.}\FunctionTok{readFileSync}\NormalTok{(}\StringTok{'client.pfx'}\NormalTok{)}
\NormalTok{\};}

\KeywordTok{var} \NormalTok{socket = }\OtherTok{tls}\NormalTok{.}\FunctionTok{connect}\NormalTok{(}\DecValTok{8000}\NormalTok{, options, }\KeywordTok{function}\NormalTok{() \{}
  \OtherTok{console}\NormalTok{.}\FunctionTok{log}\NormalTok{(}\StringTok{'client connected'}\NormalTok{,}
              \OtherTok{socket}\NormalTok{.}\FunctionTok{authorized} \NormalTok{? }\StringTok{'authorized'} \NormalTok{: }\StringTok{'unauthorized'}\NormalTok{);}
  \OtherTok{process}\NormalTok{.}\OtherTok{stdin}\NormalTok{.}\FunctionTok{pipe}\NormalTok{(socket);}
  \OtherTok{process}\NormalTok{.}\OtherTok{stdin}\NormalTok{.}\FunctionTok{resume}\NormalTok{();}
\NormalTok{\});}
\OtherTok{socket}\NormalTok{.}\FunctionTok{setEncoding}\NormalTok{(}\StringTok{'utf8'}\NormalTok{);}
\OtherTok{socket}\NormalTok{.}\FunctionTok{on}\NormalTok{(}\StringTok{'data'}\NormalTok{, }\KeywordTok{function}\NormalTok{(data) \{}
  \OtherTok{console}\NormalTok{.}\FunctionTok{log}\NormalTok{(data);}
\NormalTok{\});}
\OtherTok{socket}\NormalTok{.}\FunctionTok{on}\NormalTok{(}\StringTok{'end'}\NormalTok{, }\KeywordTok{function}\NormalTok{() \{}
  \OtherTok{server}\NormalTok{.}\FunctionTok{close}\NormalTok{();}
\NormalTok{\});}
\end{Highlighting}
\end{Shaded}

\subsection{Class: tls.TLSSocket}\label{class-tls.tlssocket}

Wrapper for instance of
\href{net.html\#net_class_net_socket}{net.Socket}, replaces internal
socket read/write routines to perform transparent encryption/decryption
of incoming/outgoing data.

\subsection{new tls.TLSSocket(socket,
options)}\label{new-tls.tlssocketsocket-options}

Construct a new TLSSocket object from existing TCP socket.

\texttt{socket} is an instance of
\href{net.html\#net_class_net_socket}{net.Socket}

\texttt{options} is an object that might contain following properties:

\begin{itemize}
\item
  \texttt{secureContext}: An optional TLS context object from
  \texttt{tls.createSecureContext(\ ...\ )}
\item
  \texttt{isServer}: If true - TLS socket will be instantiated in
  server-mode
\item
  \texttt{server}: An optional
  \href{net.html\#net_class_net_server}{net.Server} instance
\item
  \texttt{requestCert}: Optional, see
  \hyperref[tlsux5ftlsux5fcreatesecurepairux5fcredentialsux5fisserverux5frequestcertux5frejectunauthorized]{tls.createSecurePair}
\item
  \texttt{rejectUnauthorized}: Optional, see
  \hyperref[tlsux5ftlsux5fcreatesecurepairux5fcredentialsux5fisserverux5frequestcertux5frejectunauthorized]{tls.createSecurePair}
\item
  \texttt{NPNProtocols}: Optional, see
  \hyperref[tlsux5ftlsux5fcreateserverux5foptionsux5fsecureconnectionlistener]{tls.createServer}
\item
  \texttt{SNICallback}: Optional, see
  \hyperref[tlsux5ftlsux5fcreateserverux5foptionsux5fsecureconnectionlistener]{tls.createServer}
\item
  \texttt{session}: Optional, a \texttt{Buffer} instance, containing TLS
  session
\item
  \texttt{requestOCSP}: Optional, if \texttt{true} - OCSP status request
  extension would be added to client hello, and \texttt{OCSPResponse}
  event will be emitted on socket before establishing secure
  communication
\end{itemize}

\subsection{tls.createSecureContext(details)}\label{tls.createsecurecontextdetails}

Creates a credentials object, with the optional details being a
dictionary with keys:

\begin{itemize}
\itemsep1pt\parskip0pt\parsep0pt
\item
  \texttt{pfx} : A string or buffer holding the PFX or PKCS12 encoded
  private key, certificate and CA certificates
\item
  \texttt{key} : A string holding the PEM encoded private key
\item
  \texttt{passphrase} : A string of passphrase for the private key or
  pfx
\item
  \texttt{cert} : A string holding the PEM encoded certificate
\item
  \texttt{ca} : Either a string or list of strings of PEM encoded CA
  certificates to trust.
\item
  \texttt{crl} : Either a string or list of strings of PEM encoded CRLs
  (Certificate Revocation List)
\item
  \texttt{ciphers}: A string describing the ciphers to use or exclude.
  Consult
  \url{http://www.openssl.org/docs/apps/ciphers.html\#CIPHER_LIST_FORMAT}
  for details on the format.
\item
  \texttt{honorCipherOrder} : When choosing a cipher, use the server's
  preferences instead of the client preferences. For further details see
  \texttt{tls} module documentation.
\end{itemize}

If no `ca' details are given, then node.js will use the default publicly
trusted list of CAs as given in
\url{http://mxr.mozilla.org/mozilla/source/security/nss/lib/ckfw/builtins/certdata.txt}.

\subsection{tls.createSecurePair({[}context{]}{[}, isServer{]}{[},
requestCert{]}{[},
rejectUnauthorized{]})}\label{tls.createsecurepaircontext-isserver-requestcert-rejectunauthorized}

Creates a new secure pair object with two streams, one of which
reads/writes encrypted data, and one reads/writes cleartext data.
Generally the encrypted one is piped to/from an incoming encrypted data
stream, and the cleartext one is used as a replacement for the initial
encrypted stream.

\begin{itemize}
\item
  \texttt{context}: A secure context object from
  tls.createSecureContext( \ldots{} )
\item
  \texttt{isServer}: A boolean indicating whether this tls connection
  should be opened as a server or a client.
\item
  \texttt{requestCert}: A boolean indicating whether a server should
  request a certificate from a connecting client. Only applies to server
  connections.
\item
  \texttt{rejectUnauthorized}: A boolean indicating whether a server
  should automatically reject clients with invalid certificates. Only
  applies to servers with \texttt{requestCert} enabled.
\end{itemize}

\texttt{tls.createSecurePair()} returns a SecurePair object with
\texttt{cleartext} and \texttt{encrypted} stream properties.

NOTE: \texttt{cleartext} has the same APIs as
\hyperref[tlsux5fclassux5ftlsux5ftlssocket]{tls.TLSSocket}

\subsection{Class: SecurePair}\label{class-securepair}

Returned by tls.createSecurePair.

\subsubsection{\texorpdfstring{Event:
`secure'}{Event: secure}}\label{event-secure}

The event is emitted from the SecurePair once the pair has successfully
established a secure connection.

Similarly to the checking for the server `secureConnection' event,
pair.cleartext.authorized should be checked to confirm whether the
certificate used properly authorized.

\subsection{Class: tls.Server}\label{class-tls.server}

This class is a subclass of \texttt{net.Server} and has the same methods
on it. Instead of accepting just raw TCP connections, this accepts
encrypted connections using TLS or SSL.

\subsubsection{\texorpdfstring{Event:
`secureConnection'}{Event: secureConnection}}\label{event-secureconnection}

\texttt{function\ (tlsSocket)\ \{\}}

This event is emitted after a new connection has been successfully
handshaked. The argument is an instance of
\hyperref[tlsux5fclassux5ftlsux5ftlssocket]{tls.TLSSocket}. It has all
the common stream methods and events.

\texttt{socket.authorized} is a boolean value which indicates if the
client has verified by one of the supplied certificate authorities for
the server. If \texttt{socket.authorized} is false, then
\texttt{socket.authorizationError} is set to describe how authorization
failed. Implied but worth mentioning: depending on the settings of the
TLS server, you unauthorized connections may be accepted.
\texttt{socket.npnProtocol} is a string containing selected NPN
protocol. \texttt{socket.servername} is a string containing servername
requested with SNI.

\subsubsection{\texorpdfstring{Event:
`clientError'}{Event: clientError}}\label{event-clienterror}

\texttt{function\ (exception,\ tlsSocket)\ \{\ \}}

When a client connection emits an `error' event before secure connection
is established - it will be forwarded here.

\texttt{tlsSocket} is the
\hyperref[tlsux5fclassux5ftlsux5ftlssocket]{tls.TLSSocket} that the
error originated from.

\subsubsection{\texorpdfstring{Event:
`newSession'}{Event: newSession}}\label{event-newsession}

\texttt{function\ (sessionId,\ sessionData,\ callback)\ \{\ \}}

Emitted on creation of TLS session. May be used to store sessions in
external storage. \texttt{callback} must be invoked eventually,
otherwise no data will be sent or received from secure connection.

NOTE: adding this event listener will have an effect only on connections
established after addition of event listener.

\subsubsection{\texorpdfstring{Event:
`resumeSession'}{Event: resumeSession}}\label{event-resumesession}

\texttt{function\ (sessionId,\ callback)\ \{\ \}}

Emitted when client wants to resume previous TLS session. Event listener
may perform lookup in external storage using given \texttt{sessionId},
and invoke \texttt{callback(null,\ sessionData)} once finished. If
session can't be resumed (i.e.~doesn't exist in storage) one may call
\texttt{callback(null,\ null)}. Calling \texttt{callback(err)} will
terminate incoming connection and destroy socket.

NOTE: adding this event listener will have an effect only on connections
established after addition of event listener.

\subsubsection{\texorpdfstring{Event:
`OCSPRequest'}{Event: OCSPRequest}}\label{event-ocsprequest}

\texttt{function\ (certificate,\ issuer,\ callback)\ \{\ \}}

Emitted when the client sends a certificate status request. You could
parse server's current certificate to obtain OCSP url and certificate
id, and after obtaining OCSP response invoke
\texttt{callback(null,\ resp)}, where \texttt{resp} is a \texttt{Buffer}
instance. Both \texttt{certificate} and \texttt{issuer} are a
\texttt{Buffer} DER-representations of the primary and issuer's
certificates. They could be used to obtain OCSP certificate id and OCSP
endpoint url.

Alternatively, \texttt{callback(null,\ null)} could be called, meaning
that there is no OCSP response.

Calling \texttt{callback(err)} will result in a
\texttt{socket.destroy(err)} call.

Typical flow:

\begin{enumerate}
\def\labelenumi{\arabic{enumi}.}
\itemsep1pt\parskip0pt\parsep0pt
\item
  Client connects to server and sends \texttt{OCSPRequest} to it (via
  status info extension in ClientHello.)
\item
  Server receives request and invokes \texttt{OCSPRequest} event
  listener if present
\item
  Server grabs OCSP url from either \texttt{certificate} or
  \texttt{issuer} and performs an
  \href{http://en.wikipedia.org/wiki/OCSP_stapling}{OCSP request} to the
  CA
\item
  Server receives \texttt{OCSPResponse} from CA and sends it back to
  client via \texttt{callback} argument
\item
  Client validates the response and either destroys socket or performs a
  handshake.
\end{enumerate}

NOTE: \texttt{issuer} could be null, if the certificate is self-signed
or if the issuer is not in the root certificates list. (You could
provide an issuer via \texttt{ca} option.)

NOTE: adding this event listener will have an effect only on connections
established after addition of event listener.

NOTE: you may want to use some npm module like
\href{http://npmjs.org/package/asn1.js}{asn1.js} to parse the
certificates.

\subsubsection{server.listen(port{[}, host{]}{[},
callback{]})}\label{server.listenport-host-callback}

Begin accepting connections on the specified \texttt{port} and
\texttt{host}. If the \texttt{host} is omitted, the server will accept
connections directed to any IPv4 address (\texttt{INADDR\_ANY}).

This function is asynchronous. The last parameter \texttt{callback} will
be called when the server has been bound.

See \texttt{net.Server} for more information.

\subsubsection{server.close()}\label{server.close}

Stops the server from accepting new connections. This function is
asynchronous, the server is finally closed when the server emits a
\texttt{\textquotesingle{}close\textquotesingle{}} event.

\subsubsection{server.address()}\label{server.address}

Returns the bound address, the address family name and port of the
server as reported by the operating system. See
\href{net.html\#net_server_address}{net.Server.address()} for more
information.

\subsubsection{server.addContext(hostname,
context)}\label{server.addcontexthostname-context}

Add secure context that will be used if client request's SNI hostname is
matching passed \texttt{hostname} (wildcards can be used).
\texttt{context} can contain \texttt{key}, \texttt{cert}, \texttt{ca}
and/or any other properties from \texttt{tls.createSecureContext}
\texttt{options} argument.

\subsubsection{server.maxConnections}\label{server.maxconnections}

Set this property to reject connections when the server's connection
count gets high.

\subsubsection{server.connections}\label{server.connections}

The number of concurrent connections on the server.

\subsection{Class: CryptoStream}\label{class-cryptostream}

\begin{Shaded}
\begin{Highlighting}[]
\NormalTok{Stability: }\DecValTok{0} \NormalTok{- }\OtherTok{Deprecated}\NormalTok{. }\FunctionTok{Use} \OtherTok{tls}\NormalTok{.}\FunctionTok{TLSSocket} \OtherTok{instead}\NormalTok{.}
\end{Highlighting}
\end{Shaded}

This is an encrypted stream.

\subsubsection{cryptoStream.bytesWritten}\label{cryptostream.byteswritten}

A proxy to the underlying socket's bytesWritten accessor, this will
return the total bytes written to the socket, \emph{including the TLS
overhead}.

\subsection{Class: CleartextStream}\label{class-cleartextstream}

The CleartextStream class in Node.js version v0.10.39 and prior has been
deprecated and removed.

\subsection{Class: tls.TLSSocket}\label{class-tls.tlssocket-1}

This is a wrapped version of
\href{net.html\#net_class_net_socket}{net.Socket} that does transparent
encryption of written data and all required TLS negotiation.

This instance implements a duplex
\href{stream.html\#stream_stream}{Stream} interfaces. It has all the
common stream methods and events.

\subsubsection{\texorpdfstring{Event:
`secureConnect'}{Event: secureConnect}}\label{event-secureconnect}

This event is emitted after a new connection has been successfully
handshaked. The listener will be called no matter if the server's
certificate was authorized or not. It is up to the user to test
\texttt{tlsSocket.authorized} to see if the server certificate was
signed by one of the specified CAs. If
\texttt{tlsSocket.authorized\ ===\ false} then the error can be found in
\texttt{tlsSocket.authorizationError}. Also if NPN was used - you can
check \texttt{tlsSocket.npnProtocol} for negotiated protocol.

\subsubsection{\texorpdfstring{Event:
`OCSPResponse'}{Event: OCSPResponse}}\label{event-ocspresponse}

\texttt{function\ (response)\ \{\ \}}

This event will be emitted if \texttt{requestOCSP} option was set.
\texttt{response} is a buffer object, containing server's OCSP response.

Traditionally, the \texttt{response} is a signed object from the
server's CA that contains information about server's certificate
revocation status.

\subsubsection{tlsSocket.encrypted}\label{tlssocket.encrypted}

Static boolean value, always \texttt{true}. May be used to distinguish
TLS sockets from regular ones.

\subsubsection{tlsSocket.authorized}\label{tlssocket.authorized}

A boolean that is \texttt{true} if the peer certificate was signed by
one of the specified CAs, otherwise \texttt{false}

\subsubsection{tlsSocket.authorizationError}\label{tlssocket.authorizationerror}

The reason why the peer's certificate has not been verified. This
property becomes available only when
\texttt{tlsSocket.authorized\ ===\ false}.

\subsubsection{tlsSocket.getPeerCertificate({[} detailed
{]})}\label{tlssocket.getpeercertificate-detailed}

Returns an object representing the peer's certificate. The returned
object has some properties corresponding to the field of the
certificate. If \texttt{detailed} argument is \texttt{true} - the full
chain with \texttt{issuer} property will be returned, if \texttt{false}
- only the top certificate without \texttt{issuer} property.

Example:

\begin{Shaded}
\begin{Highlighting}[]
\NormalTok{\{ }\DataTypeTok{subject}\NormalTok{:}
   \NormalTok{\{ }\DataTypeTok{C}\NormalTok{: }\StringTok{'UK'}\NormalTok{,}
     \DataTypeTok{ST}\NormalTok{: }\StringTok{'Acknack Ltd'}\NormalTok{,}
     \DataTypeTok{L}\NormalTok{: }\StringTok{'Rhys Jones'}\NormalTok{,}
     \DataTypeTok{O}\NormalTok{: }\StringTok{'node.js'}\NormalTok{,}
     \DataTypeTok{OU}\NormalTok{: }\StringTok{'Test TLS Certificate'}\NormalTok{,}
     \DataTypeTok{CN}\NormalTok{: }\StringTok{'localhost'} \NormalTok{\},}
  \DataTypeTok{issuerInfo}\NormalTok{:}
   \NormalTok{\{ }\DataTypeTok{C}\NormalTok{: }\StringTok{'UK'}\NormalTok{,}
     \DataTypeTok{ST}\NormalTok{: }\StringTok{'Acknack Ltd'}\NormalTok{,}
     \DataTypeTok{L}\NormalTok{: }\StringTok{'Rhys Jones'}\NormalTok{,}
     \DataTypeTok{O}\NormalTok{: }\StringTok{'node.js'}\NormalTok{,}
     \DataTypeTok{OU}\NormalTok{: }\StringTok{'Test TLS Certificate'}\NormalTok{,}
     \DataTypeTok{CN}\NormalTok{: }\StringTok{'localhost'} \NormalTok{\},}
  \DataTypeTok{issuer}\NormalTok{:}
   \NormalTok{\{ ... }\FunctionTok{another} \OtherTok{certificate} \NormalTok{... \},}
  \DataTypeTok{raw}\NormalTok{: < RAW DER buffer >,}
  \DataTypeTok{valid_from}\NormalTok{: }\StringTok{'Nov 11 09:52:22 2009 GMT'}\NormalTok{,}
  \DataTypeTok{valid_to}\NormalTok{: }\StringTok{'Nov  6 09:52:22 2029 GMT'}\NormalTok{,}
  \DataTypeTok{fingerprint}\NormalTok{: }\StringTok{'2A:7A:C2:DD:E5:F9:CC:53:72:35:99:7A:02:5A:71:38:52:EC:8A:DF'}\NormalTok{,}
  \DataTypeTok{serialNumber}\NormalTok{: }\StringTok{'B9B0D332A1AA5635'} \NormalTok{\}}
\end{Highlighting}
\end{Shaded}

If the peer does not provide a certificate, it returns \texttt{null} or
an empty object.

\subsubsection{tlsSocket.getCipher()}\label{tlssocket.getcipher}

Returns an object representing the cipher name and the SSL/TLS protocol
version of the current connection.

Example: \{ name: `AES256-SHA', version: `TLSv1/SSLv3' \}

See SSL\_CIPHER\_get\_name() and SSL\_CIPHER\_get\_version() in
http://www.openssl.org/docs/ssl/ssl.html\#DEALING\_WITH\_CIPHERS for
more information.

\subsubsection{tlsSocket.renegotiate(options,
callback)}\label{tlssocket.renegotiateoptions-callback}

Initiate TLS renegotiation process. The \texttt{options} may contain the
following fields: \texttt{rejectUnauthorized}, \texttt{requestCert} (See
\hyperref[tlsux5ftlsux5fcreateserverux5foptionsux5fsecureconnectionlistener]{tls.createServer}
for details). \texttt{callback(err)} will be executed with \texttt{null}
as \texttt{err}, once the renegotiation is successfully completed.

NOTE: Can be used to request peer's certificate after the secure
connection has been established.

ANOTHER NOTE: When running as the server, socket will be destroyed with
an error after \texttt{handshakeTimeout} timeout.

\subsubsection{tlsSocket.setMaxSendFragment(size)}\label{tlssocket.setmaxsendfragmentsize}

Set maximum TLS fragment size (default and maximum value is:
\texttt{16384}, minimum is: \texttt{512}). Returns \texttt{true} on
success, \texttt{false} otherwise.

Smaller fragment size decreases buffering latency on the client: large
fragments are buffered by the TLS layer until the entire fragment is
received and its integrity is verified; large fragments can span
multiple roundtrips, and their processing can be delayed due to packet
loss or reordering. However, smaller fragments add extra TLS framing
bytes and CPU overhead, which may decrease overall server throughput.

\subsubsection{tlsSocket.getSession()}\label{tlssocket.getsession}

Return ASN.1 encoded TLS session or \texttt{undefined} if none was
negotiated. Could be used to speed up handshake establishment when
reconnecting to the server.

\subsubsection{tlsSocket.getTLSTicket()}\label{tlssocket.gettlsticket}

NOTE: Works only with client TLS sockets. Useful only for debugging, for
session reuse provide \texttt{session} option to \texttt{tls.connect}.

Return TLS session ticket or \texttt{undefined} if none was negotiated.

\subsubsection{tlsSocket.address()}\label{tlssocket.address}

Returns the bound address, the address family name and port of the
underlying socket as reported by the operating system. Returns an object
with three properties, e.g.
\texttt{\{\ port:\ 12346,\ family:\ \textquotesingle{}IPv4\textquotesingle{},\ address:\ \textquotesingle{}127.0.0.1\textquotesingle{}\ \}}

\subsubsection{tlsSocket.remoteAddress}\label{tlssocket.remoteaddress}

The string representation of the remote IP address. For example,
\texttt{\textquotesingle{}74.125.127.100\textquotesingle{}} or
\texttt{\textquotesingle{}2001:4860:a005::68\textquotesingle{}}.

\subsubsection{tlsSocket.remoteFamily}\label{tlssocket.remotefamily}

The string representation of the remote IP family.
\texttt{\textquotesingle{}IPv4\textquotesingle{}} or
\texttt{\textquotesingle{}IPv6\textquotesingle{}}.

\subsubsection{tlsSocket.remotePort}\label{tlssocket.remoteport}

The numeric representation of the remote port. For example,
\texttt{443}.

\subsubsection{tlsSocket.localAddress}\label{tlssocket.localaddress}

The string representation of the local IP address.

\subsubsection{tlsSocket.localPort}\label{tlssocket.localport}

The numeric representation of the local port.
