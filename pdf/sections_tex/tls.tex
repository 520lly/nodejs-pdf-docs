\section{TLS (SSL)}

\begin{Shaded}
\begin{Highlighting}[]
\DataTypeTok{Stability}\NormalTok{: }\DecValTok{3} \NormalTok{- Stable}
\end{Highlighting}
\end{Shaded}

Use \texttt{require('tls')} to access this module.

The \texttt{tls} module uses OpenSSL to provide Transport Layer Security
and/or Secure Socket Layer: encrypted stream communication.

TLS/SSL is a public/private key infrastructure. Each client and each
server must have a private key. A private key is created like this

\begin{Shaded}
\begin{Highlighting}[]
\NormalTok{openssl genrsa -out ryans-}\KeywordTok{key}\NormalTok{.}\FunctionTok{pem} \DecValTok{1024}
\end{Highlighting}
\end{Shaded}

All severs and some clients need to have a certificate. Certificates are
public keys signed by a Certificate Authority or self-signed. The first
step to getting a certificate is to create a ``Certificate Signing
Request'' (CSR) file. This is done with:

\begin{Shaded}
\begin{Highlighting}[]
\NormalTok{openssl req -}\KeywordTok{new} \NormalTok{-key ryans-}\KeywordTok{key}\NormalTok{.}\FunctionTok{pem} \NormalTok{-out ryans-}\KeywordTok{csr}\NormalTok{.}\FunctionTok{pem}
\end{Highlighting}
\end{Shaded}

To create a self-signed certificate with the CSR, do this:

\begin{Shaded}
\begin{Highlighting}[]
\NormalTok{openssl x509 -req -}\KeywordTok{in} \NormalTok{ryans-}\KeywordTok{csr}\NormalTok{.}\FunctionTok{pem} \NormalTok{-signkey ryans-}\KeywordTok{key}\NormalTok{.}\FunctionTok{pem} \NormalTok{-out ryans-}\KeywordTok{cert}\NormalTok{.}\FunctionTok{pem}
\end{Highlighting}
\end{Shaded}

Alternatively you can send the CSR to a Certificate Authority for
signing.

(TODO: docs on creating a CA, for now interested users should just look
at \texttt{test/fixtures/keys/Makefile} in the Node source code)

To create .pfx or .p12, do this:

\begin{Shaded}
\begin{Highlighting}[]
\NormalTok{openssl pkcs12 -export -}\KeywordTok{in} \NormalTok{agent5-}\KeywordTok{cert}\NormalTok{.}\FunctionTok{pem} \NormalTok{-inkey agent5-}\KeywordTok{key}\NormalTok{.}\FunctionTok{pem} \NormalTok{\textbackslash{}}
    \NormalTok{-certfile ca-}\KeywordTok{cert}\NormalTok{.}\FunctionTok{pem} \NormalTok{-out }\KeywordTok{agent5}\NormalTok{.}\FunctionTok{pfx}
\end{Highlighting}
\end{Shaded}

\begin{itemize}
\item
  \texttt{in}: certificate
\item
  \texttt{inkey}: private key
\item
  \texttt{certfile}: all CA certs concatenated in one file like
  \texttt{cat ca1-cert.pem ca2-cert.pem \textgreater{} ca-cert.pem}
\end{itemize}

\subsection{Client-initiated renegotiation attack mitigation}

The TLS protocol lets the client renegotiate certain aspects of the TLS
session. Unfortunately, session renegotiation requires a disproportional
amount of server-side resources, which makes it a potential vector for
denial-of-service attacks.

To mitigate this, renegotiations are limited to three times every 10
minutes. An error is emitted on the
\hyperref[tls_class_tls_cleartextstream]{CleartextStream} instance
when the threshold is exceeded. The limits are configurable:

\begin{itemize}
\item
  \texttt{tls.CLIENT\_RENEG\_LIMIT}: renegotiation limit, default is 3.
\item
  \texttt{tls.CLIENT\_RENEG\_WINDOW}: renegotiation window in seconds,
  default is 10 minutes.
\end{itemize}

Don't change the defaults unless you know what you are doing.

To test your server, connect to it with
\texttt{openssl s\_client -connect address:port} and tap
\texttt{R\textless{}CR\textgreater{}} (that's the letter \texttt{R}
followed by a carriage return) a few times.

\subsection{NPN and SNI}

NPN (Next Protocol Negotiation) and SNI (Server Name Indication) are TLS
handshake extensions allowing you:

\begin{itemize}
\item
  NPN - to use one TLS server for multiple protocols (HTTP, SPDY)
\item
  SNI - to use one TLS server for multiple hostnames with different SSL
  certificates.
\end{itemize}

\subsection{tls.getCiphers()}

Returns an array with the names of the supported SSL ciphers.

Example:

\begin{Shaded}
\begin{Highlighting}[]
\KeywordTok{var} \NormalTok{ciphers = }\KeywordTok{tls}\NormalTok{.}\FunctionTok{getCiphers}\NormalTok{();}
\KeywordTok{console}\NormalTok{.}\FunctionTok{log}\NormalTok{(ciphers); }\CommentTok{// ['AES128-SHA', 'AES256-SHA', ...]}
\end{Highlighting}
\end{Shaded}

\subsection{tls.createServer(options, {[}secureConnectionListener{]})}

Creates a new \hyperref[tls_class_tls_server]{tls.Server}. The
\texttt{connectionListener} argument is automatically set as a listener
for the \hyperref[tls_event_secureconnection]{secureConnection} event.
The \texttt{options} object has these possibilities:

\begin{itemize}
\item
  \texttt{pfx}: A string or \texttt{Buffer} containing the private key,
  certificate and CA certs of the server in PFX or PKCS12 format.
  (Mutually exclusive with the \texttt{key}, \texttt{cert} and
  \texttt{ca} options.)
\item
  \texttt{key}: A string or \texttt{Buffer} containing the private key
  of the server in PEM format. (Required)
\item
  \texttt{passphrase}: A string of passphrase for the private key or
  pfx.
\item
  \texttt{cert}: A string or \texttt{Buffer} containing the certificate
  key of the server in PEM format. (Required)
\item
  \texttt{ca}: An array of strings or \texttt{Buffer}s of trusted
  certificates. If this is omitted several well known ``root'' CAs will
  be used, like VeriSign. These are used to authorize connections.
\item
  \texttt{crl} : Either a string or list of strings of PEM encoded CRLs
  (Certificate Revocation List)
\item
  \texttt{ciphers}: A string describing the ciphers to use or exclude.

  To mitigate
  \href{http://blog.ivanristic.com/2011/10/mitigating-the-beast-attack-on-tls.html}{BEAST
  attacks} it is recommended that you use this option in conjunction
  with the \texttt{honorCipherOrder} option described below to
  prioritize the non-CBC cipher.

  Defaults to
  \texttt{ECDHE-RSA-AES128-SHA256:AES128-GCM-SHA256:RC4:HIGH:!MD5:!aNULL:!EDH}.
  Consult the
  \href{http://www.openssl.org/docs/apps/ciphers.html\#CIPHER\_LIST\_FORMAT}{OpenSSL
  cipher list format documentation} for details on the format.

  \texttt{ECDHE-RSA-AES128-SHA256} and \texttt{AES128-GCM-SHA256} are
  used when node.js is linked against OpenSSL 1.0.1 or newer and the
  client speaks TLS 1.2, RC4 is used as a secure fallback.

  \textbf{NOTE}: Previous revisions of this section suggested
  \texttt{AES256-SHA} as an acceptable cipher. Unfortunately,
  \texttt{AES256-SHA} is a CBC cipher and therefore susceptible to BEAST
  attacks. Do \emph{not} use it.
\item
  \texttt{handshakeTimeout}: Abort the connection if the SSL/TLS
  handshake does not finish in this many milliseconds. The default is
  120 seconds.

  A \texttt{'clientError'} is emitted on the \texttt{tls.Server} object
  whenever a handshake times out.
\item
  \texttt{honorCipherOrder} : When choosing a cipher, use the server's
  preferences instead of the client preferences.

  Note that if SSLv2 is used, the server will send its list of
  preferences to the client, and the client chooses the cipher.

  Although, this option is disabled by default, it is \emph{recommended}
  that you use this option in conjunction with the \texttt{ciphers}
  option to mitigate BEAST attacks.
\item
  \texttt{requestCert}: If \texttt{true} the server will request a
  certificate from clients that connect and attempt to verify that
  certificate. Default: \texttt{false}.
\item
  \texttt{rejectUnauthorized}: If \texttt{true} the server will reject
  any connection which is not authorized with the list of supplied CAs.
  This option only has an effect if \texttt{requestCert} is
  \texttt{true}. Default: \texttt{false}.
\item
  \texttt{NPNProtocols}: An array or \texttt{Buffer} of possible NPN
  protocols. (Protocols should be ordered by their priority).
\item
  \texttt{SNICallback}: A function that will be called if client
  supports SNI TLS extension. Only one argument will be passed to it:
  \texttt{servername}. And \texttt{SNICallback} should return
  SecureContext instance. (You can use
  \texttt{crypto.createCredentials(...).context} to get proper
  SecureContext). If \texttt{SNICallback} wasn't provided - default
  callback with high-level API will be used (see below).
\item
  \texttt{sessionTimeout}: An integer specifying the seconds after which
  TLS session identifiers and TLS session tickets created by the server
  are timed out. See
  \href{http://www.openssl.org/docs/ssl/SSL\_CTX\_set\_timeout.html}{SSL\_CTX\_set\_timeout}
  for more details.
\item
  \texttt{sessionIdContext}: A string containing a opaque identifier for
  session resumption. If \texttt{requestCert} is \texttt{true}, the
  default is MD5 hash value generated from command-line. Otherwise, the
  default is not provided.
\end{itemize}

Here is a simple example echo server:

\begin{Shaded}
\begin{Highlighting}[]
\KeywordTok{var} \NormalTok{tls = require(}\CharTok{'tls'}\NormalTok{);}
\KeywordTok{var} \NormalTok{fs = require(}\CharTok{'fs'}\NormalTok{);}

\KeywordTok{var} \NormalTok{options = \{}
  \DataTypeTok{key}\NormalTok{: }\KeywordTok{fs}\NormalTok{.}\FunctionTok{readFileSync}\NormalTok{(}\CharTok{'server-key.pem'}\NormalTok{),}
  \DataTypeTok{cert}\NormalTok{: }\KeywordTok{fs}\NormalTok{.}\FunctionTok{readFileSync}\NormalTok{(}\CharTok{'server-cert.pem'}\NormalTok{),}

  \CommentTok{// This is necessary only if using the client certificate authentication.}
  \DataTypeTok{requestCert}\NormalTok{: }\KeywordTok{true}\NormalTok{,}

  \CommentTok{// This is necessary only if the client uses the self-signed certificate.}
  \DataTypeTok{ca}\NormalTok{: [ }\KeywordTok{fs}\NormalTok{.}\FunctionTok{readFileSync}\NormalTok{(}\CharTok{'client-cert.pem'}\NormalTok{) ]}
\NormalTok{\};}

\KeywordTok{var} \NormalTok{server = }\KeywordTok{tls}\NormalTok{.}\FunctionTok{createServer}\NormalTok{(options, }\KeywordTok{function}\NormalTok{(cleartextStream) \{}
  \KeywordTok{console}\NormalTok{.}\FunctionTok{log}\NormalTok{(}\CharTok{'server connected'}\NormalTok{,}
              \KeywordTok{cleartextStream}\NormalTok{.}\FunctionTok{authorized} \NormalTok{? }\CharTok{'authorized'} \NormalTok{: }\CharTok{'unauthorized'}\NormalTok{);}
  \KeywordTok{cleartextStream}\NormalTok{.}\FunctionTok{write}\NormalTok{(}\StringTok{"welcome!}\CharTok{\textbackslash{}n}\StringTok{"}\NormalTok{);}
  \KeywordTok{cleartextStream}\NormalTok{.}\FunctionTok{setEncoding}\NormalTok{(}\CharTok{'utf8'}\NormalTok{);}
  \KeywordTok{cleartextStream}\NormalTok{.}\FunctionTok{pipe}\NormalTok{(cleartextStream);}
\NormalTok{\});}
\KeywordTok{server}\NormalTok{.}\FunctionTok{listen}\NormalTok{(}\DecValTok{8000}\NormalTok{, }\KeywordTok{function}\NormalTok{() \{}
  \KeywordTok{console}\NormalTok{.}\FunctionTok{log}\NormalTok{(}\CharTok{'server bound'}\NormalTok{);}
\NormalTok{\});}
\end{Highlighting}
\end{Shaded}

Or

\begin{Shaded}
\begin{Highlighting}[]
\KeywordTok{var} \NormalTok{tls = require(}\CharTok{'tls'}\NormalTok{);}
\KeywordTok{var} \NormalTok{fs = require(}\CharTok{'fs'}\NormalTok{);}

\KeywordTok{var} \NormalTok{options = \{}
  \DataTypeTok{pfx}\NormalTok{: }\KeywordTok{fs}\NormalTok{.}\FunctionTok{readFileSync}\NormalTok{(}\CharTok{'server.pfx'}\NormalTok{),}

  \CommentTok{// This is necessary only if using the client certificate authentication.}
  \DataTypeTok{requestCert}\NormalTok{: }\KeywordTok{true}\NormalTok{,}

\NormalTok{\};}

\KeywordTok{var} \NormalTok{server = }\KeywordTok{tls}\NormalTok{.}\FunctionTok{createServer}\NormalTok{(options, }\KeywordTok{function}\NormalTok{(cleartextStream) \{}
  \KeywordTok{console}\NormalTok{.}\FunctionTok{log}\NormalTok{(}\CharTok{'server connected'}\NormalTok{,}
              \KeywordTok{cleartextStream}\NormalTok{.}\FunctionTok{authorized} \NormalTok{? }\CharTok{'authorized'} \NormalTok{: }\CharTok{'unauthorized'}\NormalTok{);}
  \KeywordTok{cleartextStream}\NormalTok{.}\FunctionTok{write}\NormalTok{(}\StringTok{"welcome!}\CharTok{\textbackslash{}n}\StringTok{"}\NormalTok{);}
  \KeywordTok{cleartextStream}\NormalTok{.}\FunctionTok{setEncoding}\NormalTok{(}\CharTok{'utf8'}\NormalTok{);}
  \KeywordTok{cleartextStream}\NormalTok{.}\FunctionTok{pipe}\NormalTok{(cleartextStream);}
\NormalTok{\});}
\KeywordTok{server}\NormalTok{.}\FunctionTok{listen}\NormalTok{(}\DecValTok{8000}\NormalTok{, }\KeywordTok{function}\NormalTok{() \{}
  \KeywordTok{console}\NormalTok{.}\FunctionTok{log}\NormalTok{(}\CharTok{'server bound'}\NormalTok{);}
\NormalTok{\});}
\end{Highlighting}
\end{Shaded}

You can test this server by connecting to it with
\texttt{openssl s\_client}:

\begin{Shaded}
\begin{Highlighting}[]
\NormalTok{openssl s_client -connect }\FloatTok{127.0.0.1}\NormalTok{:}\DecValTok{8000}
\end{Highlighting}
\end{Shaded}

\subsection{tls.SLAB\_BUFFER\_SIZE}

Size of slab buffer used by all tls servers and clients. Default:
\texttt{10 * 1024 * 1024}.

Don't change the defaults unless you know what you are doing.

\subsection{tls.connect(options, {[}callback{]})}

\subsection{tls.connect(port, {[}host{]}, {[}options{]},
{[}callback{]})}

Creates a new client connection to the given \texttt{port} and
\texttt{host} (old API) or \texttt{options.port} and
\texttt{options.host}. (If \texttt{host} is omitted, it defaults to
\texttt{localhost}.) \texttt{options} should be an object which
specifies:

\begin{itemize}
\item
  \texttt{host}: Host the client should connect to
\item
  \texttt{port}: Port the client should connect to
\item
  \texttt{socket}: Establish secure connection on a given socket rather
  than creating a new socket. If this option is specified, \texttt{host}
  and \texttt{port} are ignored.
\item
  \texttt{pfx}: A string or \texttt{Buffer} containing the private key,
  certificate and CA certs of the server in PFX or PKCS12 format.
\item
  \texttt{key}: A string or \texttt{Buffer} containing the private key
  of the client in PEM format.
\item
  \texttt{passphrase}: A string of passphrase for the private key or
  pfx.
\item
  \texttt{cert}: A string or \texttt{Buffer} containing the certificate
  key of the client in PEM format.
\item
  \texttt{ca}: An array of strings or \texttt{Buffer}s of trusted
  certificates. If this is omitted several well known ``root'' CAs will
  be used, like VeriSign. These are used to authorize connections.
\item
  \texttt{rejectUnauthorized}: If \texttt{true}, the server certificate
  is verified against the list of supplied CAs. An \texttt{'error'}
  event is emitted if verification fails. Default: \texttt{true}.
\item
  \texttt{NPNProtocols}: An array of string or \texttt{Buffer}
  containing supported NPN protocols. \texttt{Buffer} should have
  following format: \texttt{0x05hello0x05world}, where first byte is
  next protocol name's length. (Passing array should usually be much
  simpler: \texttt{{[}'hello', 'world'{]}}.)
\item
  \texttt{servername}: Servername for SNI (Server Name Indication) TLS
  extension.
\end{itemize}

The \texttt{callback} parameter will be added as a listener for the
\hyperref[tls_event_secureconnect]{`secureConnect'} event.

\texttt{tls.connect()} returns a
\hyperref[tls_class_tls_cleartextstream]{CleartextStream} object.

Here is an example of a client of echo server as described previously:

\begin{Shaded}
\begin{Highlighting}[]
\KeywordTok{var} \NormalTok{tls = require(}\CharTok{'tls'}\NormalTok{);}
\KeywordTok{var} \NormalTok{fs = require(}\CharTok{'fs'}\NormalTok{);}

\KeywordTok{var} \NormalTok{options = \{}
  \CommentTok{// These are necessary only if using the client certificate authentication}
  \DataTypeTok{key}\NormalTok{: }\KeywordTok{fs}\NormalTok{.}\FunctionTok{readFileSync}\NormalTok{(}\CharTok{'client-key.pem'}\NormalTok{),}
  \DataTypeTok{cert}\NormalTok{: }\KeywordTok{fs}\NormalTok{.}\FunctionTok{readFileSync}\NormalTok{(}\CharTok{'client-cert.pem'}\NormalTok{),}

  \CommentTok{// This is necessary only if the server uses the self-signed certificate}
  \DataTypeTok{ca}\NormalTok{: [ }\KeywordTok{fs}\NormalTok{.}\FunctionTok{readFileSync}\NormalTok{(}\CharTok{'server-cert.pem'}\NormalTok{) ]}
\NormalTok{\};}

\KeywordTok{var} \NormalTok{cleartextStream = }\KeywordTok{tls}\NormalTok{.}\FunctionTok{connect}\NormalTok{(}\DecValTok{8000}\NormalTok{, options, }\KeywordTok{function}\NormalTok{() \{}
  \KeywordTok{console}\NormalTok{.}\FunctionTok{log}\NormalTok{(}\CharTok{'client connected'}\NormalTok{,}
              \KeywordTok{cleartextStream}\NormalTok{.}\FunctionTok{authorized} \NormalTok{? }\CharTok{'authorized'} \NormalTok{: }\CharTok{'unauthorized'}\NormalTok{);}
  \KeywordTok{process.stdin}\NormalTok{.}\FunctionTok{pipe}\NormalTok{(cleartextStream);}
  \KeywordTok{process.stdin}\NormalTok{.}\FunctionTok{resume}\NormalTok{();}
\NormalTok{\});}
\KeywordTok{cleartextStream}\NormalTok{.}\FunctionTok{setEncoding}\NormalTok{(}\CharTok{'utf8'}\NormalTok{);}
\KeywordTok{cleartextStream}\NormalTok{.}\FunctionTok{on}\NormalTok{(}\CharTok{'data'}\NormalTok{, }\KeywordTok{function}\NormalTok{(data) \{}
  \KeywordTok{console}\NormalTok{.}\FunctionTok{log}\NormalTok{(data);}
\NormalTok{\});}
\KeywordTok{cleartextStream}\NormalTok{.}\FunctionTok{on}\NormalTok{(}\CharTok{'end'}\NormalTok{, }\KeywordTok{function}\NormalTok{() \{}
  \KeywordTok{server}\NormalTok{.}\FunctionTok{close}\NormalTok{();}
\NormalTok{\});}
\end{Highlighting}
\end{Shaded}

Or

\begin{Shaded}
\begin{Highlighting}[]
\KeywordTok{var} \NormalTok{tls = require(}\CharTok{'tls'}\NormalTok{);}
\KeywordTok{var} \NormalTok{fs = require(}\CharTok{'fs'}\NormalTok{);}

\KeywordTok{var} \NormalTok{options = \{}
  \DataTypeTok{pfx}\NormalTok{: }\KeywordTok{fs}\NormalTok{.}\FunctionTok{readFileSync}\NormalTok{(}\CharTok{'client.pfx'}\NormalTok{)}
\NormalTok{\};}

\KeywordTok{var} \NormalTok{cleartextStream = }\KeywordTok{tls}\NormalTok{.}\FunctionTok{connect}\NormalTok{(}\DecValTok{8000}\NormalTok{, options, }\KeywordTok{function}\NormalTok{() \{}
  \KeywordTok{console}\NormalTok{.}\FunctionTok{log}\NormalTok{(}\CharTok{'client connected'}\NormalTok{,}
              \KeywordTok{cleartextStream}\NormalTok{.}\FunctionTok{authorized} \NormalTok{? }\CharTok{'authorized'} \NormalTok{: }\CharTok{'unauthorized'}\NormalTok{);}
  \KeywordTok{process.stdin}\NormalTok{.}\FunctionTok{pipe}\NormalTok{(cleartextStream);}
  \KeywordTok{process.stdin}\NormalTok{.}\FunctionTok{resume}\NormalTok{();}
\NormalTok{\});}
\KeywordTok{cleartextStream}\NormalTok{.}\FunctionTok{setEncoding}\NormalTok{(}\CharTok{'utf8'}\NormalTok{);}
\KeywordTok{cleartextStream}\NormalTok{.}\FunctionTok{on}\NormalTok{(}\CharTok{'data'}\NormalTok{, }\KeywordTok{function}\NormalTok{(data) \{}
  \KeywordTok{console}\NormalTok{.}\FunctionTok{log}\NormalTok{(data);}
\NormalTok{\});}
\KeywordTok{cleartextStream}\NormalTok{.}\FunctionTok{on}\NormalTok{(}\CharTok{'end'}\NormalTok{, }\KeywordTok{function}\NormalTok{() \{}
  \KeywordTok{server}\NormalTok{.}\FunctionTok{close}\NormalTok{();}
\NormalTok{\});}
\end{Highlighting}
\end{Shaded}

\subsection{tls.createSecurePair({[}credentials{]}, {[}isServer{]},
{[}requestCert{]}, {[}rejectUnauthorized{]})}

Creates a new secure pair object with two streams, one of which
reads/writes encrypted data, and one reads/writes cleartext data.
Generally the encrypted one is piped to/from an incoming encrypted data
stream, and the cleartext one is used as a replacement for the initial
encrypted stream.

\begin{itemize}
\item
  \texttt{credentials}: A credentials object from
  crypto.createCredentials( \ldots{} )
\item
  \texttt{isServer}: A boolean indicating whether this tls connection
  should be opened as a server or a client.
\item
  \texttt{requestCert}: A boolean indicating whether a server should
  request a certificate from a connecting client. Only applies to server
  connections.
\item
  \texttt{rejectUnauthorized}: A boolean indicating whether a server
  should automatically reject clients with invalid certificates. Only
  applies to servers with \texttt{requestCert} enabled.
\end{itemize}

\texttt{tls.createSecurePair()} returns a SecurePair object with
{[}cleartext{]}{[}{]} and \texttt{encrypted} stream properties.

\subsection{Class: SecurePair}

Returned by tls.createSecurePair.

\subsubsection{Event: `secure'}

The event is emitted from the SecurePair once the pair has successfully
established a secure connection.

Similarly to the checking for the server `secureConnection' event,
pair.cleartext.authorized should be checked to confirm whether the
certificate used properly authorized.

\subsection{Class: tls.Server}

This class is a subclass of \texttt{net.Server} and has the same methods
on it. Instead of accepting just raw TCP connections, this accepts
encrypted connections using TLS or SSL.

\subsubsection{Event: `secureConnection'}

\texttt{function (cleartextStream) \{\}}

This event is emitted after a new connection has been successfully
handshaked. The argument is a instance of
\hyperref[tls_class_tls_cleartextstream]{CleartextStream}. It has all
the common stream methods and events.

\texttt{cleartextStream.authorized} is a boolean value which indicates
if the client has verified by one of the supplied certificate
authorities for the server. If \texttt{cleartextStream.authorized} is
false, then \texttt{cleartextStream.authorizationError} is set to
describe how authorization failed. Implied but worth mentioning:
depending on the settings of the TLS server, you unauthorized
connections may be accepted. \texttt{cleartextStream.npnProtocol} is a
string containing selected NPN protocol.
\texttt{cleartextStream.servername} is a string containing servername
requested with SNI.

\subsubsection{Event: `clientError'}

\texttt{function (exception, securePair) \{ \}}

When a client connection emits an `error' event before secure connection
is established - it will be forwarded here.

\texttt{securePair} is the \texttt{tls.SecurePair} that the error
originated from.

\subsubsection{Event: `newSession'}

\texttt{function (sessionId, sessionData) \{ \}}

Emitted on creation of TLS session. May be used to store sessions in
external storage.

\subsubsection{Event: `resumeSession'}

\texttt{function (sessionId, callback) \{ \}}

Emitted when client wants to resume previous TLS session. Event listener
may perform lookup in external storage using given \texttt{sessionId},
and invoke \texttt{callback(null, sessionData)} once finished. If
session can't be resumed (i.e.~doesn't exist in storage) one may call
\texttt{callback(null, null)}. Calling \texttt{callback(err)} will
terminate incoming connection and destroy socket.

\subsubsection{server.listen(port, {[}host{]}, {[}callback{]})}

Begin accepting connections on the specified \texttt{port} and
\texttt{host}. If the \texttt{host} is omitted, the server will accept
connections directed to any IPv4 address (\texttt{INADDR\_ANY}).

This function is asynchronous. The last parameter \texttt{callback} will
be called when the server has been bound.

See \texttt{net.Server} for more information.

\subsubsection{server.close()}

Stops the server from accepting new connections. This function is
asynchronous, the server is finally closed when the server emits a
\texttt{'close'} event.

\subsubsection{server.address()}

Returns the bound address, the address family name and port of the
server as reported by the operating system. See
\href{net.html\#net\_server\_address}{net.Server.address()} for more
information.

\subsubsection{server.addContext(hostname, credentials)}

Add secure context that will be used if client request's SNI hostname is
matching passed \texttt{hostname} (wildcards can be used).
\texttt{credentials} can contain \texttt{key}, \texttt{cert} and
\texttt{ca}.

\subsubsection{server.maxConnections}

Set this property to reject connections when the server's connection
count gets high.

\subsubsection{server.connections}

The number of concurrent connections on the server.

\subsection{Class: CryptoStream}

This is an encrypted stream.

\subsubsection{cryptoStream.bytesWritten}

A proxy to the underlying socket's bytesWritten accessor, this will
return the total bytes written to the socket, \emph{including the TLS
overhead}.

\subsection{Class: tls.CleartextStream}

This is a stream on top of the \emph{Encrypted} stream that makes it
possible to read/write an encrypted data as a cleartext data.

This instance implements a duplex
\href{stream.html\#stream\_stream}{Stream} interfaces. It has all the
common stream methods and events.

A ClearTextStream is the \texttt{clear} member of a SecurePair object.

\subsubsection{Event: `secureConnect'}

This event is emitted after a new connection has been successfully
handshaked. The listener will be called no matter if the server's
certificate was authorized or not. It is up to the user to test
\texttt{cleartextStream.authorized} to see if the server certificate was
signed by one of the specified CAs. If
\texttt{cleartextStream.authorized === false} then the error can be
found in \texttt{cleartextStream.authorizationError}. Also if NPN was
used - you can check \texttt{cleartextStream.npnProtocol} for negotiated
protocol.

\subsubsection{cleartextStream.authorized}

A boolean that is \texttt{true} if the peer certificate was signed by
one of the specified CAs, otherwise \texttt{false}

\subsubsection{cleartextStream.authorizationError}

The reason why the peer's certificate has not been verified. This
property becomes available only when
\texttt{cleartextStream.authorized === false}.

\subsubsection{cleartextStream.getPeerCertificate()}

Returns an object representing the peer's certificate. The returned
object has some properties corresponding to the field of the
certificate.

Example:

\begin{Shaded}
\begin{Highlighting}[]
\NormalTok{\{ }\DataTypeTok{subject}\NormalTok{: }
   \NormalTok{\{ }\DataTypeTok{C}\NormalTok{: }\CharTok{'UK'}\NormalTok{,}
     \DataTypeTok{ST}\NormalTok{: }\CharTok{'Acknack Ltd'}\NormalTok{,}
     \DataTypeTok{L}\NormalTok{: }\CharTok{'Rhys Jones'}\NormalTok{,}
     \DataTypeTok{O}\NormalTok{: }\CharTok{'node.js'}\NormalTok{,}
     \DataTypeTok{OU}\NormalTok{: }\CharTok{'Test TLS Certificate'}\NormalTok{,}
     \DataTypeTok{CN}\NormalTok{: }\CharTok{'localhost'} \NormalTok{\},}
  \DataTypeTok{issuer}\NormalTok{: }
   \NormalTok{\{ }\DataTypeTok{C}\NormalTok{: }\CharTok{'UK'}\NormalTok{,}
     \DataTypeTok{ST}\NormalTok{: }\CharTok{'Acknack Ltd'}\NormalTok{,}
     \DataTypeTok{L}\NormalTok{: }\CharTok{'Rhys Jones'}\NormalTok{,}
     \DataTypeTok{O}\NormalTok{: }\CharTok{'node.js'}\NormalTok{,}
     \DataTypeTok{OU}\NormalTok{: }\CharTok{'Test TLS Certificate'}\NormalTok{,}
     \DataTypeTok{CN}\NormalTok{: }\CharTok{'localhost'} \NormalTok{\},}
  \DataTypeTok{valid_from}\NormalTok{: }\CharTok{'Nov 11 09:52:22 2009 GMT'}\NormalTok{,}
  \DataTypeTok{valid_to}\NormalTok{: }\CharTok{'Nov  6 09:52:22 2029 GMT'}\NormalTok{,}
  \DataTypeTok{fingerprint}\NormalTok{: }\CharTok{'2A:7A:C2:DD:E5:F9:CC:53:72:35:99:7A:02:5A:71:38:52:EC:8A:DF'} \NormalTok{\}}
\end{Highlighting}
\end{Shaded}

If the peer does not provide a certificate, it returns \texttt{null} or
an empty object.

\subsubsection{cleartextStream.getCipher()}

Returns an object representing the cipher name and the SSL/TLS protocol
version of the current connection.

Example: \{ name: `AES256-SHA', version: `TLSv1/SSLv3' \}

See SSL\_CIPHER\_get\_name() and SSL\_CIPHER\_get\_version() in
http://www.openssl.org/docs/ssl/ssl.html\#DEALING\_WITH\_CIPHERS for
more information.

\subsubsection{cleartextStream.address()}

Returns the bound address, the address family name and port of the
underlying socket as reported by the operating system. Returns an object
with three properties, e.g.
\texttt{\{ port: 12346, family: 'IPv4', address: '127.0.0.1' \}}

\subsubsection{cleartextStream.remoteAddress}

The string representation of the remote IP address. For example,
\texttt{'74.125.127.100'} or \texttt{'2001:4860:a005::68'}.

\subsubsection{cleartextStream.remotePort}

The numeric representation of the remote port. For example,
\texttt{443}.
