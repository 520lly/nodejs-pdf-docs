\section{Events}

\begin{Shaded}
\begin{Highlighting}[]
\DataTypeTok{Stability}\NormalTok{: }\DecValTok{4} \NormalTok{- API Frozen}
\end{Highlighting}
\end{Shaded}

Many objects in Node emit events: a \texttt{net.Server} emits an event
each time a peer connects to it, a \texttt{fs.readStream} emits an event
when the file is opened. All objects which emit events are instances of
\texttt{events.EventEmitter}. You can access this module by doing:
\texttt{require("events");}

Typically, event names are represented by a camel-cased string, however,
there aren't any strict restrictions on that, as any string will be
accepted.

Functions can then be attached to objects, to be executed when an event
is emitted. These functions are called \emph{listeners}.

\subsection{Class: events.EventEmitter}

To access the EventEmitter class,
\texttt{require('events').EventEmitter}.

When an \texttt{EventEmitter} instance experiences an error, the typical
action is to emit an \texttt{'error'} event. Error events are treated as
a special case in node. If there is no listener for it, then the default
action is to print a stack trace and exit the program.

All EventEmitters emit the event \texttt{'newListener'} when new
listeners are added and \texttt{'removeListener'} when a listener is
removed.

\subsubsection{emitter.addListener(event, listener)}

\subsubsection{emitter.on(event, listener)}

Adds a listener to the end of the listeners array for the specified
event.

\begin{Shaded}
\begin{Highlighting}[]
\KeywordTok{server}\NormalTok{.}\FunctionTok{on}\NormalTok{(}\CharTok{'connection'}\NormalTok{, }\KeywordTok{function} \NormalTok{(stream) \{}
  \KeywordTok{console}\NormalTok{.}\FunctionTok{log}\NormalTok{(}\CharTok{'someone connected!'}\NormalTok{);}
\NormalTok{\});}
\end{Highlighting}
\end{Shaded}

\subsubsection{emitter.once(event, listener)}

Adds a \textbf{one time} listener for the event. This listener is
invoked only the next time the event is fired, after which it is
removed.

\begin{Shaded}
\begin{Highlighting}[]
\KeywordTok{server}\NormalTok{.}\FunctionTok{once}\NormalTok{(}\CharTok{'connection'}\NormalTok{, }\KeywordTok{function} \NormalTok{(stream) \{}
  \KeywordTok{console}\NormalTok{.}\FunctionTok{log}\NormalTok{(}\CharTok{'Ah, we have our first user!'}\NormalTok{);}
\NormalTok{\});}
\end{Highlighting}
\end{Shaded}

\subsubsection{emitter.removeListener(event, listener)}

Remove a listener from the listener array for the specified event.
\textbf{Caution}: changes array indices in the listener array behind the
listener.

\begin{Shaded}
\begin{Highlighting}[]
\KeywordTok{var} \NormalTok{callback = }\KeywordTok{function}\NormalTok{(stream) \{}
  \KeywordTok{console}\NormalTok{.}\FunctionTok{log}\NormalTok{(}\CharTok{'someone connected!'}\NormalTok{);}
\NormalTok{\};}
\KeywordTok{server}\NormalTok{.}\FunctionTok{on}\NormalTok{(}\CharTok{'connection'}\NormalTok{, callback);}
\CommentTok{// ...}
\KeywordTok{server}\NormalTok{.}\FunctionTok{removeListener}\NormalTok{(}\CharTok{'connection'}\NormalTok{, callback);}
\end{Highlighting}
\end{Shaded}

\subsubsection{emitter.removeAllListeners({[}event{]})}

Removes all listeners, or those of the specified event.

\subsubsection{emitter.setMaxListeners(n)}

By default EventEmitters will print a warning if more than 10 listeners
are added for a particular event. This is a useful default which helps
finding memory leaks. Obviously not all Emitters should be limited to
10. This function allows that to be increased. Set to zero for
unlimited.

\subsubsection{emitter.listeners(event)}

Returns an array of listeners for the specified event.

\begin{Shaded}
\begin{Highlighting}[]
\KeywordTok{server}\NormalTok{.}\FunctionTok{on}\NormalTok{(}\CharTok{'connection'}\NormalTok{, }\KeywordTok{function} \NormalTok{(stream) \{}
  \KeywordTok{console}\NormalTok{.}\FunctionTok{log}\NormalTok{(}\CharTok{'someone connected!'}\NormalTok{);}
\NormalTok{\});}
\KeywordTok{console}\NormalTok{.}\FunctionTok{log}\NormalTok{(}\KeywordTok{util}\NormalTok{.}\FunctionTok{inspect}\NormalTok{(}\KeywordTok{server}\NormalTok{.}\FunctionTok{listeners}\NormalTok{(}\CharTok{'connection'}\NormalTok{))); }\CommentTok{// [ [Function] ]}
\end{Highlighting}
\end{Shaded}

\subsubsection{emitter.emit(event, {[}arg1{]}, {[}arg2{]},
{[}\ldots{}{]})}

Execute each of the listeners in order with the supplied arguments.

\subsubsection{Class Method: EventEmitter.listenerCount(emitter, event)}

Return the number of listeners for a given event.

\subsubsection{Event: `newListener'}

\begin{itemize}
\item
  \texttt{event} \{String\} The event name
\item
  \texttt{listener} \{Function\} The event handler function
\end{itemize}

This event is emitted any time someone adds a new listener.

\subsubsection{Event: `removeListener'}

\begin{itemize}
\item
  \texttt{event} \{String\} The event name
\item
  \texttt{listener} \{Function\} The event handler function
\end{itemize}

This event is emitted any time someone removes a listener.
