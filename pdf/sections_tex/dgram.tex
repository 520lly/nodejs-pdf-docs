\section{UDP / Datagram Sockets}

\begin{Shaded}
\begin{Highlighting}[]
\DataTypeTok{Stability}\NormalTok{: }\DecValTok{3} \NormalTok{- Stable}
\end{Highlighting}
\end{Shaded}

Datagram sockets are available through \texttt{require('dgram')}.

Important note: the behavior of \texttt{dgram.Socket\#bind()} has
changed in v0.10 and is always asynchronous now. If you have code that
looks like this:

\begin{Shaded}
\begin{Highlighting}[]
\KeywordTok{var} \NormalTok{s = }\KeywordTok{dgram}\NormalTok{.}\FunctionTok{createSocket}\NormalTok{(}\CharTok{'udp4'}\NormalTok{);}
\KeywordTok{s}\NormalTok{.}\FunctionTok{bind}\NormalTok{(}\DecValTok{1234}\NormalTok{);}
\KeywordTok{s}\NormalTok{.}\FunctionTok{addMembership}\NormalTok{(}\CharTok{'224.0.0.114'}\NormalTok{);}
\end{Highlighting}
\end{Shaded}

You have to change it to this:

\begin{Shaded}
\begin{Highlighting}[]
\KeywordTok{var} \NormalTok{s = }\KeywordTok{dgram}\NormalTok{.}\FunctionTok{createSocket}\NormalTok{(}\CharTok{'udp4'}\NormalTok{);}
\KeywordTok{s}\NormalTok{.}\FunctionTok{bind}\NormalTok{(}\DecValTok{1234}\NormalTok{, }\KeywordTok{function}\NormalTok{() \{}
  \KeywordTok{s}\NormalTok{.}\FunctionTok{addMembership}\NormalTok{(}\CharTok{'224.0.0.114'}\NormalTok{);}
\NormalTok{\});}
\end{Highlighting}
\end{Shaded}

\subsection{dgram.createSocket(type, {[}callback{]})}

\begin{itemize}
\item
  \texttt{type} String. Either `udp4' or `udp6'
\item
  \texttt{callback} Function. Attached as a listener to \texttt{message}
  events. Optional
\item
  Returns: Socket object
\end{itemize}

Creates a datagram Socket of the specified types. Valid types are
\texttt{udp4} and \texttt{udp6}.

Takes an optional callback which is added as a listener for
\texttt{message} events.

Call \texttt{socket.bind} if you want to receive datagrams.
\texttt{socket.bind()} will bind to the ``all interfaces'' address on a
random port (it does the right thing for both \texttt{udp4} and
\texttt{udp6} sockets). You can then retrieve the address and port with
\texttt{socket.address().address} and \texttt{socket.address().port}.

\subsection{Class: Socket}

The dgram Socket class encapsulates the datagram functionality. It
should be created via \texttt{dgram.createSocket(type, {[}callback{]})}.

\subsubsection{Event: `message'}

\begin{itemize}
\item
  \texttt{msg} Buffer object. The message
\item
  \texttt{rinfo} Object. Remote address information
\end{itemize}

Emitted when a new datagram is available on a socket. \texttt{msg} is a
\texttt{Buffer} and \texttt{rinfo} is an object with the sender's
address information and the number of bytes in the datagram.

\subsubsection{Event: `listening'}

Emitted when a socket starts listening for datagrams. This happens as
soon as UDP sockets are created.

\subsubsection{Event: `close'}

Emitted when a socket is closed with \texttt{close()}. No new
\texttt{message} events will be emitted on this socket.

\subsubsection{Event: `error'}

\begin{itemize}
\item
  \texttt{exception} Error object
\end{itemize}

Emitted when an error occurs.

\subsubsection{dgram.send(buf, offset, length, port, address,
{[}callback{]})}

\begin{itemize}
\item
  \texttt{buf} Buffer object. Message to be sent
\item
  \texttt{offset} Integer. Offset in the buffer where the message
  starts.
\item
  \texttt{length} Integer. Number of bytes in the message.
\item
  \texttt{port} Integer. destination port
\item
  \texttt{address} String. destination IP
\item
  \texttt{callback} Function. Callback when message is done being
  delivered. Optional.
\end{itemize}

For UDP sockets, the destination port and IP address must be specified.
A string may be supplied for the \texttt{address} parameter, and it will
be resolved with DNS. An optional callback may be specified to detect
any DNS errors and when \texttt{buf} may be re-used. Note that DNS
lookups will delay the time that a send takes place, at least until the
next tick. The only way to know for sure that a send has taken place is
to use the callback.

If the socket has not been previously bound with a call to
\texttt{bind}, it's assigned a random port number and bound to the ``all
interfaces'' address (0.0.0.0 for \texttt{udp4} sockets, ::0 for
\texttt{udp6} sockets).

Example of sending a UDP packet to a random port on \texttt{localhost};

\begin{Shaded}
\begin{Highlighting}[]
\KeywordTok{var} \NormalTok{dgram = require(}\CharTok{'dgram'}\NormalTok{);}
\KeywordTok{var} \NormalTok{message = }\KeywordTok{new} \NormalTok{Buffer(}\StringTok{"Some bytes"}\NormalTok{);}
\KeywordTok{var} \NormalTok{client = }\KeywordTok{dgram}\NormalTok{.}\FunctionTok{createSocket}\NormalTok{(}\StringTok{"udp4"}\NormalTok{);}
\KeywordTok{client}\NormalTok{.}\FunctionTok{send}\NormalTok{(message, }\DecValTok{0}\NormalTok{, }\KeywordTok{message}\NormalTok{.}\FunctionTok{length}\NormalTok{, }\DecValTok{41234}\NormalTok{, }\StringTok{"localhost"}\NormalTok{, }\KeywordTok{function}\NormalTok{(err, bytes) \{}
  \KeywordTok{client}\NormalTok{.}\FunctionTok{close}\NormalTok{();}
\NormalTok{\});}
\end{Highlighting}
\end{Shaded}

\textbf{A Note about UDP datagram size}

The maximum size of an \texttt{IPv4/v6} datagram depends on the
\texttt{MTU} (\emph{Maximum Transmission Unit}) and on the
\texttt{Payload Length} field size.

\begin{itemize}
\item
  The \texttt{Payload Length} field is \texttt{16 bits} wide, which
  means that a normal payload cannot be larger than 64K octets including
  internet header and data (65,507 bytes = 65,535 − 8 bytes UDP header −
  20 bytes IP header); this is generally true for loopback interfaces,
  but such long datagrams are impractical for most hosts and networks.
\item
  The \texttt{MTU} is the largest size a given link layer technology can
  support for datagrams. For any link, \texttt{IPv4} mandates a minimum
  \texttt{MTU} of \texttt{68} octets, while the recommended \texttt{MTU}
  for IPv4 is \texttt{576} (typically recommended as the \texttt{MTU}
  for dial-up type applications), whether they arrive whole or in
  fragments.
\end{itemize}

For \texttt{IPv6}, the minimum \texttt{MTU} is \texttt{1280} octets,
however, the mandatory minimum fragment reassembly buffer size is
\texttt{1500} octets. The value of \texttt{68} octets is very small,
since most current link layer technologies have a minimum \texttt{MTU}
of \texttt{1500} (like Ethernet).

Note that it's impossible to know in advance the MTU of each link
through which a packet might travel, and that generally sending a
datagram greater than the (receiver) \texttt{MTU} won't work (the packet
gets silently dropped, without informing the source that the data did
not reach its intended recipient).

\subsubsection{dgram.bind(port, {[}address{]}, {[}callback{]})}

\begin{itemize}
\item
  \texttt{port} Integer
\item
  \texttt{address} String, Optional
\item
  \texttt{callback} Function, Optional
\end{itemize}

For UDP sockets, listen for datagrams on a named \texttt{port} and
optional \texttt{address}. If \texttt{address} is not specified, the OS
will try to listen on all addresses.

The \texttt{callback} argument, if provided, is added as a one-shot
\texttt{'listening'} event listener.

Example of a UDP server listening on port 41234:

\begin{Shaded}
\begin{Highlighting}[]
\KeywordTok{var} \NormalTok{dgram = require(}\StringTok{"dgram"}\NormalTok{);}

\KeywordTok{var} \NormalTok{server = }\KeywordTok{dgram}\NormalTok{.}\FunctionTok{createSocket}\NormalTok{(}\StringTok{"udp4"}\NormalTok{);}

\KeywordTok{server}\NormalTok{.}\FunctionTok{on}\NormalTok{(}\StringTok{"message"}\NormalTok{, }\KeywordTok{function} \NormalTok{(msg, rinfo) \{}
  \KeywordTok{console}\NormalTok{.}\FunctionTok{log}\NormalTok{(}\StringTok{"server got: "} \NormalTok{+ msg + }\StringTok{" from "} \NormalTok{+}
    \KeywordTok{rinfo}\NormalTok{.}\FunctionTok{address} \NormalTok{+ }\StringTok{":"} \NormalTok{+ }\KeywordTok{rinfo}\NormalTok{.}\FunctionTok{port}\NormalTok{);}
\NormalTok{\});}

\KeywordTok{server}\NormalTok{.}\FunctionTok{on}\NormalTok{(}\StringTok{"listening"}\NormalTok{, }\KeywordTok{function} \NormalTok{() \{}
  \KeywordTok{var} \NormalTok{address = }\KeywordTok{server}\NormalTok{.}\FunctionTok{address}\NormalTok{();}
  \KeywordTok{console}\NormalTok{.}\FunctionTok{log}\NormalTok{(}\StringTok{"server listening "} \NormalTok{+}
      \KeywordTok{address}\NormalTok{.}\FunctionTok{address} \NormalTok{+ }\StringTok{":"} \NormalTok{+ }\KeywordTok{address}\NormalTok{.}\FunctionTok{port}\NormalTok{);}
\NormalTok{\});}

\KeywordTok{server}\NormalTok{.}\FunctionTok{bind}\NormalTok{(}\DecValTok{41234}\NormalTok{);}
\CommentTok{// server listening 0.0.0.0:41234}
\end{Highlighting}
\end{Shaded}

\subsubsection{dgram.close()}

Close the underlying socket and stop listening for data on it.

\subsubsection{dgram.address()}

Returns an object containing the address information for a socket. For
UDP sockets, this object will contain \texttt{address} , \texttt{family}
and \texttt{port}.

\subsubsection{dgram.setBroadcast(flag)}

\begin{itemize}
\item
  \texttt{flag} Boolean
\end{itemize}

Sets or clears the \texttt{SO\_BROADCAST} socket option. When this
option is set, UDP packets may be sent to a local interface's broadcast
address.

\subsubsection{dgram.setTTL(ttl)}

\begin{itemize}
\item
  \texttt{ttl} Integer
\end{itemize}

Sets the \texttt{IP\_TTL} socket option. TTL stands for ``Time to
Live,'' but in this context it specifies the number of IP hops that a
packet is allowed to go through. Each router or gateway that forwards a
packet decrements the TTL. If the TTL is decremented to 0 by a router,
it will not be forwarded. Changing TTL values is typically done for
network probes or when multicasting.

The argument to \texttt{setTTL()} is a number of hops between 1 and 255.
The default on most systems is 64.

\subsubsection{dgram.setMulticastTTL(ttl)}

\begin{itemize}
\item
  \texttt{ttl} Integer
\end{itemize}

Sets the \texttt{IP\_MULTICAST\_TTL} socket option. TTL stands for
``Time to Live,'' but in this context it specifies the number of IP hops
that a packet is allowed to go through, specifically for multicast
traffic. Each router or gateway that forwards a packet decrements the
TTL. If the TTL is decremented to 0 by a router, it will not be
forwarded.

The argument to \texttt{setMulticastTTL()} is a number of hops between 0
and 255. The default on most systems is 1.

\subsubsection{dgram.setMulticastLoopback(flag)}

\begin{itemize}
\item
  \texttt{flag} Boolean
\end{itemize}

Sets or clears the \texttt{IP\_MULTICAST\_LOOP} socket option. When this
option is set, multicast packets will also be received on the local
interface.

\subsubsection{dgram.addMembership(multicastAddress,
{[}multicastInterface{]})}

\begin{itemize}
\item
  \texttt{multicastAddress} String
\item
  \texttt{multicastInterface} String, Optional
\end{itemize}

Tells the kernel to join a multicast group with
\texttt{IP\_ADD\_MEMBERSHIP} socket option.

If \texttt{multicastInterface} is not specified, the OS will try to add
membership to all valid interfaces.

\subsubsection{dgram.dropMembership(multicastAddress,
{[}multicastInterface{]})}

\begin{itemize}
\item
  \texttt{multicastAddress} String
\item
  \texttt{multicastInterface} String, Optional
\end{itemize}

Opposite of \texttt{addMembership} - tells the kernel to leave a
multicast group with \texttt{IP\_DROP\_MEMBERSHIP} socket option. This
is automatically called by the kernel when the socket is closed or
process terminates, so most apps will never need to call this.

If \texttt{multicastInterface} is not specified, the OS will try to drop
membership to all valid interfaces.

\subsubsection{dgram.unref()}

Calling \texttt{unref} on a socket will allow the program to exit if
this is the only active socket in the event system. If the socket is
already \texttt{unref}d calling \texttt{unref} again will have no
effect.

\subsubsection{dgram.ref()}

Opposite of \texttt{unref}, calling \texttt{ref} on a previously
\texttt{unref}d socket will \emph{not} let the program exit if it's the
only socket left (the default behavior). If the socket is \texttt{ref}d
calling \texttt{ref} again will have no effect.
