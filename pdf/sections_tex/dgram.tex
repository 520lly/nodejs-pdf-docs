\section{UDP / Datagram Sockets}\label{udp-datagram-sockets}

\begin{Shaded}
\begin{Highlighting}[]
\NormalTok{Stability: }\DecValTok{3} \NormalTok{- Stable}
\end{Highlighting}
\end{Shaded}

Datagram sockets are available through \texttt{require('dgram')}.

Important note: the behavior of \texttt{dgram.Socket\#bind()} has
changed in v0.10 and is always asynchronous now. If you have code that
looks like this:

\begin{Shaded}
\begin{Highlighting}[]
\KeywordTok{var} \NormalTok{s = }\OtherTok{dgram}\NormalTok{.}\FunctionTok{createSocket}\NormalTok{(}\StringTok{'udp4'}\NormalTok{);}
\OtherTok{s}\NormalTok{.}\FunctionTok{bind}\NormalTok{(}\DecValTok{1234}\NormalTok{);}
\OtherTok{s}\NormalTok{.}\FunctionTok{addMembership}\NormalTok{(}\StringTok{'224.0.0.114'}\NormalTok{);}
\end{Highlighting}
\end{Shaded}

You have to change it to this:

\begin{Shaded}
\begin{Highlighting}[]
\KeywordTok{var} \NormalTok{s = }\OtherTok{dgram}\NormalTok{.}\FunctionTok{createSocket}\NormalTok{(}\StringTok{'udp4'}\NormalTok{);}
\OtherTok{s}\NormalTok{.}\FunctionTok{bind}\NormalTok{(}\DecValTok{1234}\NormalTok{, }\KeywordTok{function}\NormalTok{() \{}
  \OtherTok{s}\NormalTok{.}\FunctionTok{addMembership}\NormalTok{(}\StringTok{'224.0.0.114'}\NormalTok{);}
\NormalTok{\});}
\end{Highlighting}
\end{Shaded}

\subsection{dgram.createSocket(type,
{[}callback{]})}\label{dgram.createsockettype-callback}

\subsection{dgram.createSocket(options,
{[}callback{]})}\label{dgram.createsocketoptions-callback}

\begin{itemize}
\itemsep1pt\parskip0pt\parsep0pt
\item
  \texttt{type} String. Either `udp4' or `udp6'
\item
  \texttt{options} Object. Should contain a \texttt{type} property and
  could contain \texttt{reuseAddr} property. \texttt{false} by default.
  When \texttt{reuseAddr} is \texttt{true} - \texttt{socket.bind()} will
  reuse address, even if the other process has already bound a socket on
  it.
\item
  \texttt{callback} Function. Attached as a listener to \texttt{message}
  events. Optional
\item
  Returns: Socket object
\end{itemize}

Creates a datagram Socket of the specified types. Valid types are
\texttt{udp4} and \texttt{udp6}.

Takes an optional callback which is added as a listener for
\texttt{message} events.

Call \texttt{socket.bind} if you want to receive datagrams.
\texttt{socket.bind()} will bind to the ``all interfaces'' address on a
random port (it does the right thing for both \texttt{udp4} and
\texttt{udp6} sockets). You can then retrieve the address and port with
\texttt{socket.address().address} and \texttt{socket.address().port}.

\subsection{Class: dgram.Socket}\label{class-dgram.socket}

The dgram Socket class encapsulates the datagram functionality. It
should be created via \texttt{dgram.createSocket(...)}

\subsubsection{Event: `message'}\label{event-message}

\begin{itemize}
\itemsep1pt\parskip0pt\parsep0pt
\item
  \texttt{msg} Buffer object. The message
\item
  \texttt{rinfo} Object. Remote address information
\end{itemize}

Emitted when a new datagram is available on a socket. \texttt{msg} is a
\texttt{Buffer} and \texttt{rinfo} is an object with the sender's
address information:

\begin{Shaded}
\begin{Highlighting}[]
\OtherTok{socket}\NormalTok{.}\FunctionTok{on}\NormalTok{(}\StringTok{'message'}\NormalTok{, }\KeywordTok{function}\NormalTok{(msg, rinfo) \{}
  \OtherTok{console}\NormalTok{.}\FunctionTok{log}\NormalTok{(}\StringTok{'Received %d bytes from %s:%d}\CharTok{\textbackslash{}n}\StringTok{'}\NormalTok{,}
              \OtherTok{msg}\NormalTok{.}\FunctionTok{length}\NormalTok{, }\OtherTok{rinfo}\NormalTok{.}\FunctionTok{address}\NormalTok{, }\OtherTok{rinfo}\NormalTok{.}\FunctionTok{port}\NormalTok{);}
\NormalTok{\});}
\end{Highlighting}
\end{Shaded}

\subsubsection{Event: `listening'}\label{event-listening}

Emitted when a socket starts listening for datagrams. This happens as
soon as UDP sockets are created.

\subsubsection{Event: `close'}\label{event-close}

Emitted when a socket is closed with \texttt{close()}. No new
\texttt{message} events will be emitted on this socket.

\subsubsection{Event: `error'}\label{event-error}

\begin{itemize}
\itemsep1pt\parskip0pt\parsep0pt
\item
  \texttt{exception} Error object
\end{itemize}

Emitted when an error occurs.

\subsubsection{socket.send(buf, offset, length, port, address,
{[}callback{]})}\label{socket.sendbuf-offset-length-port-address-callback}

\begin{itemize}
\itemsep1pt\parskip0pt\parsep0pt
\item
  \texttt{buf} Buffer object or string. Message to be sent
\item
  \texttt{offset} Integer. Offset in the buffer where the message
  starts.
\item
  \texttt{length} Integer. Number of bytes in the message.
\item
  \texttt{port} Integer. Destination port.
\item
  \texttt{address} String. Destination hostname or IP address.
\item
  \texttt{callback} Function. Called when the message has been sent.
  Optional.
\end{itemize}

For UDP sockets, the destination port and address must be specified. A
string may be supplied for the \texttt{address} parameter, and it will
be resolved with DNS.

If the address is omitted or is an empty string, \texttt{'0.0.0.0'} or
\texttt{'::0'} is used instead. Depending on the network configuration,
those defaults may or may not work; it's best to be explicit about the
destination address.

If the socket has not been previously bound with a call to
\texttt{bind}, it gets assigned a random port number and is bound to the
``all interfaces'' address (\texttt{'0.0.0.0'} for \texttt{udp4}
sockets, \texttt{'::0'} for \texttt{udp6} sockets.)

An optional callback may be specified to detect DNS errors or for
determining when it's safe to reuse the \texttt{buf} object. Note that
DNS lookups delay the time to send for at least one tick. The only way
to know for sure that the datagram has been sent is by using a callback.

With consideration for multi-byte characters, \texttt{offset} and
\texttt{length} will be calculated with respect to
\href{buffer.html\#buffer_class_method_buffer_bytelength_string_encoding}{byte
length} and not the character position.

Example of sending a UDP packet to a random port on \texttt{localhost};

\begin{Shaded}
\begin{Highlighting}[]
\KeywordTok{var} \NormalTok{dgram = }\FunctionTok{require}\NormalTok{(}\StringTok{'dgram'}\NormalTok{);}
\KeywordTok{var} \NormalTok{message = }\KeywordTok{new} \FunctionTok{Buffer}\NormalTok{(}\StringTok{"Some bytes"}\NormalTok{);}
\KeywordTok{var} \NormalTok{client = }\OtherTok{dgram}\NormalTok{.}\FunctionTok{createSocket}\NormalTok{(}\StringTok{"udp4"}\NormalTok{);}
\OtherTok{client}\NormalTok{.}\FunctionTok{send}\NormalTok{(message, }\DecValTok{0}\NormalTok{, }\OtherTok{message}\NormalTok{.}\FunctionTok{length}\NormalTok{, }\DecValTok{41234}\NormalTok{, }\StringTok{"localhost"}\NormalTok{, }\KeywordTok{function}\NormalTok{(err) \{}
  \OtherTok{client}\NormalTok{.}\FunctionTok{close}\NormalTok{();}
\NormalTok{\});}
\end{Highlighting}
\end{Shaded}

\textbf{A Note about UDP datagram size}

The maximum size of an \texttt{IPv4/v6} datagram depends on the
\texttt{MTU} (\emph{Maximum Transmission Unit}) and on the
\texttt{Payload Length} field size.

\begin{itemize}
\item
  The \texttt{Payload Length} field is \texttt{16 bits} wide, which
  means that a normal payload cannot be larger than 64K octets including
  internet header and data (65,507 bytes = 65,535 − 8 bytes UDP header −
  20 bytes IP header); this is generally true for loopback interfaces,
  but such long datagrams are impractical for most hosts and networks.
\item
  The \texttt{MTU} is the largest size a given link layer technology can
  support for datagrams. For any link, \texttt{IPv4} mandates a minimum
  \texttt{MTU} of \texttt{68} octets, while the recommended \texttt{MTU}
  for IPv4 is \texttt{576} (typically recommended as the \texttt{MTU}
  for dial-up type applications), whether they arrive whole or in
  fragments.
\end{itemize}

For \texttt{IPv6}, the minimum \texttt{MTU} is \texttt{1280} octets,
however, the mandatory minimum fragment reassembly buffer size is
\texttt{1500} octets. The value of \texttt{68} octets is very small,
since most current link layer technologies have a minimum \texttt{MTU}
of \texttt{1500} (like Ethernet).

Note that it's impossible to know in advance the MTU of each link
through which a packet might travel, and that generally sending a
datagram greater than the (receiver) \texttt{MTU} won't work (the packet
gets silently dropped, without informing the source that the data did
not reach its intended recipient).

\subsubsection{socket.bind(port, {[}address{]},
{[}callback{]})}\label{socket.bindport-address-callback}

\begin{itemize}
\itemsep1pt\parskip0pt\parsep0pt
\item
  \texttt{port} Integer
\item
  \texttt{address} String, Optional
\item
  \texttt{callback} Function with no parameters, Optional. Callback when
  binding is done.
\end{itemize}

For UDP sockets, listen for datagrams on a named \texttt{port} and
optional \texttt{address}. If \texttt{address} is not specified, the OS
will try to listen on all addresses. After binding is done, a
``listening'' event is emitted and the \texttt{callback}(if specified)
is called. Specifying both a ``listening'' event listener and
\texttt{callback} is not harmful but not very useful.

A bound datagram socket keeps the node process running to receive
datagrams.

If binding fails, an ``error'' event is generated. In rare case
(e.g.~binding a closed socket), an \texttt{Error} may be thrown by this
method.

Example of a UDP server listening on port 41234:

\begin{Shaded}
\begin{Highlighting}[]
\KeywordTok{var} \NormalTok{dgram = }\FunctionTok{require}\NormalTok{(}\StringTok{"dgram"}\NormalTok{);}

\KeywordTok{var} \NormalTok{server = }\OtherTok{dgram}\NormalTok{.}\FunctionTok{createSocket}\NormalTok{(}\StringTok{"udp4"}\NormalTok{);}

\OtherTok{server}\NormalTok{.}\FunctionTok{on}\NormalTok{(}\StringTok{"error"}\NormalTok{, }\KeywordTok{function} \NormalTok{(err) \{}
  \OtherTok{console}\NormalTok{.}\FunctionTok{log}\NormalTok{(}\StringTok{"server error:}\CharTok{\textbackslash{}n}\StringTok{"} \NormalTok{+ }\OtherTok{err}\NormalTok{.}\FunctionTok{stack}\NormalTok{);}
  \OtherTok{server}\NormalTok{.}\FunctionTok{close}\NormalTok{();}
\NormalTok{\});}

\OtherTok{server}\NormalTok{.}\FunctionTok{on}\NormalTok{(}\StringTok{"message"}\NormalTok{, }\KeywordTok{function} \NormalTok{(msg, rinfo) \{}
  \OtherTok{console}\NormalTok{.}\FunctionTok{log}\NormalTok{(}\StringTok{"server got: "} \NormalTok{+ msg + }\StringTok{" from "} \NormalTok{+}
    \OtherTok{rinfo}\NormalTok{.}\FunctionTok{address} \NormalTok{+ }\StringTok{":"} \NormalTok{+ }\OtherTok{rinfo}\NormalTok{.}\FunctionTok{port}\NormalTok{);}
\NormalTok{\});}

\OtherTok{server}\NormalTok{.}\FunctionTok{on}\NormalTok{(}\StringTok{"listening"}\NormalTok{, }\KeywordTok{function} \NormalTok{() \{}
  \KeywordTok{var} \NormalTok{address = }\OtherTok{server}\NormalTok{.}\FunctionTok{address}\NormalTok{();}
  \OtherTok{console}\NormalTok{.}\FunctionTok{log}\NormalTok{(}\StringTok{"server listening "} \NormalTok{+}
      \OtherTok{address}\NormalTok{.}\FunctionTok{address} \NormalTok{+ }\StringTok{":"} \NormalTok{+ }\OtherTok{address}\NormalTok{.}\FunctionTok{port}\NormalTok{);}
\NormalTok{\});}

\OtherTok{server}\NormalTok{.}\FunctionTok{bind}\NormalTok{(}\DecValTok{41234}\NormalTok{);}
\CommentTok{// server listening 0.0.0.0:41234}
\end{Highlighting}
\end{Shaded}

\subsubsection{socket.bind(options,
{[}callback{]})}\label{socket.bindoptions-callback}

\begin{itemize}
\itemsep1pt\parskip0pt\parsep0pt
\item
  \texttt{options} \{Object\} - Required. Supports the following
  properties:
\item
  \texttt{port} \{Number\} - Required.
\item
  \texttt{address} \{String\} - Optional.
\item
  \texttt{exclusive} \{Boolean\} - Optional.
\item
  \texttt{callback} \{Function\} - Optional.
\end{itemize}

The \texttt{port} and \texttt{address} properties of \texttt{options},
as well as the optional callback function, behave as they do on a call
to
\hyperref[dgramux5fsocketux5fbindux5fportux5faddressux5fcallback]{socket.bind(port,
{[}address{]}, {[}callback{]})}.

If \texttt{exclusive} is \texttt{false} (default), then cluster workers
will use the same underlying handle, allowing connection handling duties
to be shared. When \texttt{exclusive} is \texttt{true}, the handle is
not shared, and attempted port sharing results in an error. An example
which listens on an exclusive port is shown below.

\begin{Shaded}
\begin{Highlighting}[]
\OtherTok{socket}\NormalTok{.}\FunctionTok{bind}\NormalTok{(\{}
  \DataTypeTok{address}\NormalTok{: }\StringTok{'localhost'}\NormalTok{,}
  \DataTypeTok{port}\NormalTok{: }\DecValTok{8000}\NormalTok{,}
  \DataTypeTok{exclusive}\NormalTok{: }\KeywordTok{true}
\NormalTok{\});}
\end{Highlighting}
\end{Shaded}

\subsubsection{socket.close()}\label{socket.close}

Close the underlying socket and stop listening for data on it.

\subsubsection{socket.address()}\label{socket.address}

Returns an object containing the address information for a socket. For
UDP sockets, this object will contain \texttt{address} , \texttt{family}
and \texttt{port}.

\subsubsection{socket.setBroadcast(flag)}\label{socket.setbroadcastflag}

\begin{itemize}
\itemsep1pt\parskip0pt\parsep0pt
\item
  \texttt{flag} Boolean
\end{itemize}

Sets or clears the \texttt{SO\_BROADCAST} socket option. When this
option is set, UDP packets may be sent to a local interface's broadcast
address.

\subsubsection{socket.setTTL(ttl)}\label{socket.setttlttl}

\begin{itemize}
\itemsep1pt\parskip0pt\parsep0pt
\item
  \texttt{ttl} Integer
\end{itemize}

Sets the \texttt{IP\_TTL} socket option. TTL stands for ``Time to
Live,'' but in this context it specifies the number of IP hops that a
packet is allowed to go through. Each router or gateway that forwards a
packet decrements the TTL. If the TTL is decremented to 0 by a router,
it will not be forwarded. Changing TTL values is typically done for
network probes or when multicasting.

The argument to \texttt{setTTL()} is a number of hops between 1 and 255.
The default on most systems is 64.

\subsubsection{socket.setMulticastTTL(ttl)}\label{socket.setmulticastttlttl}

\begin{itemize}
\itemsep1pt\parskip0pt\parsep0pt
\item
  \texttt{ttl} Integer
\end{itemize}

Sets the \texttt{IP\_MULTICAST\_TTL} socket option. TTL stands for
``Time to Live,'' but in this context it specifies the number of IP hops
that a packet is allowed to go through, specifically for multicast
traffic. Each router or gateway that forwards a packet decrements the
TTL. If the TTL is decremented to 0 by a router, it will not be
forwarded.

The argument to \texttt{setMulticastTTL()} is a number of hops between 0
and 255. The default on most systems is 1.

\subsubsection{socket.setMulticastLoopback(flag)}\label{socket.setmulticastloopbackflag}

\begin{itemize}
\itemsep1pt\parskip0pt\parsep0pt
\item
  \texttt{flag} Boolean
\end{itemize}

Sets or clears the \texttt{IP\_MULTICAST\_LOOP} socket option. When this
option is set, multicast packets will also be received on the local
interface.

\subsubsection{socket.addMembership(multicastAddress,
{[}multicastInterface{]})}\label{socket.addmembershipmulticastaddress-multicastinterface}

\begin{itemize}
\itemsep1pt\parskip0pt\parsep0pt
\item
  \texttt{multicastAddress} String
\item
  \texttt{multicastInterface} String, Optional
\end{itemize}

Tells the kernel to join a multicast group with
\texttt{IP\_ADD\_MEMBERSHIP} socket option.

If \texttt{multicastInterface} is not specified, the OS will try to add
membership to all valid interfaces.

\subsubsection{socket.dropMembership(multicastAddress,
{[}multicastInterface{]})}\label{socket.dropmembershipmulticastaddress-multicastinterface}

\begin{itemize}
\itemsep1pt\parskip0pt\parsep0pt
\item
  \texttt{multicastAddress} String
\item
  \texttt{multicastInterface} String, Optional
\end{itemize}

Opposite of \texttt{addMembership} - tells the kernel to leave a
multicast group with \texttt{IP\_DROP\_MEMBERSHIP} socket option. This
is automatically called by the kernel when the socket is closed or
process terminates, so most apps will never need to call this.

If \texttt{multicastInterface} is not specified, the OS will try to drop
membership to all valid interfaces.

\subsubsection{socket.unref()}\label{socket.unref}

Calling \texttt{unref} on a socket will allow the program to exit if
this is the only active socket in the event system. If the socket is
already \texttt{unref}d calling \texttt{unref} again will have no
effect.

\subsubsection{socket.ref()}\label{socket.ref}

Opposite of \texttt{unref}, calling \texttt{ref} on a previously
\texttt{unref}d socket will \emph{not} let the program exit if it's the
only socket left (the default behavior). If the socket is \texttt{ref}d
calling \texttt{ref} again will have no effect.
