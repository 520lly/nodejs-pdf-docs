\section{Readline}

\begin{Shaded}
\begin{Highlighting}[]
\DataTypeTok{Stability}\NormalTok{: }\DecValTok{2} \NormalTok{- Unstable}
\end{Highlighting}
\end{Shaded}

To use this module, do \texttt{require('readline')}. Readline allows
reading of a stream (such as \texttt{process.stdin}) on a line-by-line
basis.

Note that once you've invoked this module, your node program will not
terminate until you've closed the interface. Here's how to allow your
program to gracefully exit:

\begin{Shaded}
\begin{Highlighting}[]
\KeywordTok{var} \NormalTok{readline = require(}\CharTok{'readline'}\NormalTok{);}

\KeywordTok{var} \NormalTok{rl = }\KeywordTok{readline}\NormalTok{.}\FunctionTok{createInterface}\NormalTok{(\{}
  \DataTypeTok{input}\NormalTok{: }\KeywordTok{process}\NormalTok{.}\FunctionTok{stdin}\NormalTok{,}
  \DataTypeTok{output}\NormalTok{: }\KeywordTok{process}\NormalTok{.}\FunctionTok{stdout}
\NormalTok{\});}

\KeywordTok{rl}\NormalTok{.}\FunctionTok{question}\NormalTok{(}\StringTok{"What do you think of node.js? "}\NormalTok{, }\KeywordTok{function}\NormalTok{(answer) \{}
  \CommentTok{// TODO: Log the answer in a database}
  \KeywordTok{console}\NormalTok{.}\FunctionTok{log}\NormalTok{(}\StringTok{"Thank you for your valuable feedback:"}\NormalTok{, answer);}

  \KeywordTok{rl}\NormalTok{.}\FunctionTok{close}\NormalTok{();}
\NormalTok{\});}
\end{Highlighting}
\end{Shaded}

\subsection{readline.createInterface(options)}

Creates a readline \texttt{Interface} instance. Accepts an ``options''
Object that takes the following values:

\begin{itemize}
\item
  \texttt{input} - the readable stream to listen to (Required).
\item
  \texttt{output} - the writable stream to write readline data to
  (Required).
\item
  \texttt{completer} - an optional function that is used for Tab
  autocompletion. See below for an example of using this.
\item
  \texttt{terminal} - pass \texttt{true} if the \texttt{input} and
  \texttt{output} streams should be treated like a TTY, and have
  ANSI/VT100 escape codes written to it. Defaults to checking
  \texttt{isTTY} on the \texttt{output} stream upon instantiation.
\end{itemize}

The \texttt{completer} function is given a the current line entered by
the user, and is supposed to return an Array with 2 entries:

\begin{enumerate}
\item
  An Array with matching entries for the completion.
\item
  The substring that was used for the matching.
\end{enumerate}

Which ends up looking something like:
\texttt{{[}{[}substr1, substr2, ...{]}, originalsubstring{]}}.

Example:

\begin{Shaded}
\begin{Highlighting}[]
\KeywordTok{function} \NormalTok{completer(line) \{}
  \KeywordTok{var} \NormalTok{completions = }\CharTok{'.help .error .exit .quit .q'}\NormalTok{.}\FunctionTok{split}\NormalTok{(}\CharTok{' '}\NormalTok{)}
  \KeywordTok{var} \NormalTok{hits = }\KeywordTok{completions}\NormalTok{.}\FunctionTok{filter}\NormalTok{(}\KeywordTok{function}\NormalTok{(c) \{ }\KeywordTok{return} \KeywordTok{c}\NormalTok{.}\FunctionTok{indexOf}\NormalTok{(line) == }\DecValTok{0} \NormalTok{\})}
  \CommentTok{// show all completions if none found}
  \KeywordTok{return} \NormalTok{[}\KeywordTok{hits}\NormalTok{.}\FunctionTok{length} \NormalTok{? }\DataTypeTok{hits }\NormalTok{: completions, line]}
\NormalTok{\}}
\end{Highlighting}
\end{Shaded}

Also \texttt{completer} can be run in async mode if it accepts two
arguments:

\begin{Shaded}
\begin{Highlighting}[]
\KeywordTok{function} \NormalTok{completer(linePartial, callback) \{}
  \NormalTok{callback(null, [[}\CharTok{'123'}\NormalTok{], linePartial]);}
\NormalTok{\}}
\end{Highlighting}
\end{Shaded}

\texttt{createInterface} is commonly used with \texttt{process.stdin}
and \texttt{process.stdout} in order to accept user input:

\begin{Shaded}
\begin{Highlighting}[]
\KeywordTok{var} \NormalTok{readline = require(}\CharTok{'readline'}\NormalTok{);}
\KeywordTok{var} \NormalTok{rl = }\KeywordTok{readline}\NormalTok{.}\FunctionTok{createInterface}\NormalTok{(\{}
  \DataTypeTok{input}\NormalTok{: }\KeywordTok{process}\NormalTok{.}\FunctionTok{stdin}\NormalTok{,}
  \DataTypeTok{output}\NormalTok{: }\KeywordTok{process}\NormalTok{.}\FunctionTok{stdout}
\NormalTok{\});}
\end{Highlighting}
\end{Shaded}

Once you have a readline instance, you most commonly listen for the
\texttt{"line"} event.

If \texttt{terminal} is \texttt{true} for this instance then the
\texttt{output} stream will get the best compatibility if it defines an
\texttt{output.columns} property, and fires a \texttt{"resize"} event on
the \texttt{output} if/when the columns ever change
(\texttt{process.stdout} does this automatically when it is a TTY).

\subsection{Class: Interface}

The class that represents a readline interface with an input and output
stream.

\subsubsection{rl.setPrompt(prompt)}

Sets the prompt, for example when you run \texttt{node} on the command
line, you see \texttt{\textgreater{}}, which is node's prompt.

\subsubsection{rl.prompt({[}preserveCursor{]})}

Readies readline for input from the user, putting the current
\texttt{setPrompt} options on a new line, giving the user a new spot to
write. Set \texttt{preserveCursor} to \texttt{true} to prevent the
cursor placement being reset to \texttt{0}.

This will also resume the \texttt{input} stream used with
\texttt{createInterface} if it has been paused.

\subsubsection{rl.question(query, callback)}

Prepends the prompt with \texttt{query} and invokes \texttt{callback}
with the user's response. Displays the query to the user, and then
invokes \texttt{callback} with the user's response after it has been
typed.

This will also resume the \texttt{input} stream used with
\texttt{createInterface} if it has been paused.

Example usage:

\begin{Shaded}
\begin{Highlighting}[]
\KeywordTok{interface}\NormalTok{.}\FunctionTok{question}\NormalTok{(}\CharTok{'What is your favorite food?'}\NormalTok{, }\KeywordTok{function}\NormalTok{(answer) \{}
  \KeywordTok{console}\NormalTok{.}\FunctionTok{log}\NormalTok{(}\CharTok{'Oh, so your favorite food is '} \NormalTok{+ answer);}
\NormalTok{\});}
\end{Highlighting}
\end{Shaded}

\subsubsection{rl.pause()}

Pauses the readline \texttt{input} stream, allowing it to be resumed
later if needed.

\subsubsection{rl.resume()}

Resumes the readline \texttt{input} stream.

\subsubsection{rl.close()}

Closes the \texttt{Interface} instance, relinquishing control on the
\texttt{input} and \texttt{output} streams. The ``close'' event will
also be emitted.

\subsubsection{rl.write(data, {[}key{]})}

Writes \texttt{data} to \texttt{output} stream. \texttt{key} is an
object literal to represent a key sequence; available if the terminal is
a TTY.

This will also resume the \texttt{input} stream if it has been paused.

Example:

\begin{Shaded}
\begin{Highlighting}[]
\KeywordTok{rl}\NormalTok{.}\FunctionTok{write}\NormalTok{(}\CharTok{'Delete me!'}\NormalTok{);}
\CommentTok{// Simulate ctrl+u to delete the line written previously}
\KeywordTok{rl}\NormalTok{.}\FunctionTok{write}\NormalTok{(null, \{}\DataTypeTok{ctrl}\NormalTok{: }\KeywordTok{true}\NormalTok{, }\DataTypeTok{name}\NormalTok{: }\CharTok{'u'}\NormalTok{\});}
\end{Highlighting}
\end{Shaded}

\subsection{Events}

\subsubsection{Event: `line'}

\texttt{function (line) \{\}}

Emitted whenever the \texttt{input} stream receives a
\texttt{\textbackslash{}n}, usually received when the user hits enter,
or return. This is a good hook to listen for user input.

Example of listening for \texttt{line}:

\begin{Shaded}
\begin{Highlighting}[]
\KeywordTok{rl}\NormalTok{.}\FunctionTok{on}\NormalTok{(}\CharTok{'line'}\NormalTok{, }\KeywordTok{function} \NormalTok{(cmd) \{}
  \KeywordTok{console}\NormalTok{.}\FunctionTok{log}\NormalTok{(}\CharTok{'You just typed: '}\NormalTok{+cmd);}
\NormalTok{\});}
\end{Highlighting}
\end{Shaded}

\subsubsection{Event: `pause'}

\texttt{function () \{\}}

Emitted whenever the \texttt{input} stream is paused.

Also emitted whenever the \texttt{input} stream is not paused and
receives the \texttt{SIGCONT} event. (See events \texttt{SIGTSTP} and
\texttt{SIGCONT})

Example of listening for \texttt{pause}:

\begin{Shaded}
\begin{Highlighting}[]
\KeywordTok{rl}\NormalTok{.}\FunctionTok{on}\NormalTok{(}\CharTok{'pause'}\NormalTok{, }\KeywordTok{function}\NormalTok{() \{}
  \KeywordTok{console}\NormalTok{.}\FunctionTok{log}\NormalTok{(}\CharTok{'Readline paused.'}\NormalTok{);}
\NormalTok{\});}
\end{Highlighting}
\end{Shaded}

\subsubsection{Event: `resume'}

\texttt{function () \{\}}

Emitted whenever the \texttt{input} stream is resumed.

Example of listening for \texttt{resume}:

\begin{Shaded}
\begin{Highlighting}[]
\KeywordTok{rl}\NormalTok{.}\FunctionTok{on}\NormalTok{(}\CharTok{'resume'}\NormalTok{, }\KeywordTok{function}\NormalTok{() \{}
  \KeywordTok{console}\NormalTok{.}\FunctionTok{log}\NormalTok{(}\CharTok{'Readline resumed.'}\NormalTok{);}
\NormalTok{\});}
\end{Highlighting}
\end{Shaded}

\subsubsection{Event: `close'}

\texttt{function () \{\}}

Emitted when \texttt{close()} is called.

Also emitted when the \texttt{input} stream receives its ``end'' event.
The \texttt{Interface} instance should be considered ``finished'' once
this is emitted. For example, when the \texttt{input} stream receives
\texttt{\^{}D}, respectively known as \texttt{EOT}.

This event is also called if there is no \texttt{SIGINT} event listener
present when the \texttt{input} stream receives a \texttt{\^{}C},
respectively known as \texttt{SIGINT}.

\subsubsection{Event: `SIGINT'}

\texttt{function () \{\}}

Emitted whenever the \texttt{input} stream receives a \texttt{\^{}C},
respectively known as \texttt{SIGINT}. If there is no \texttt{SIGINT}
event listener present when the \texttt{input} stream receives a
\texttt{SIGINT}, \texttt{pause} will be triggered.

Example of listening for \texttt{SIGINT}:

\begin{Shaded}
\begin{Highlighting}[]
\KeywordTok{rl}\NormalTok{.}\FunctionTok{on}\NormalTok{(}\CharTok{'SIGINT'}\NormalTok{, }\KeywordTok{function}\NormalTok{() \{}
  \KeywordTok{rl}\NormalTok{.}\FunctionTok{question}\NormalTok{(}\CharTok{'Are you sure you want to exit?'}\NormalTok{, }\KeywordTok{function}\NormalTok{(answer) \{}
    \KeywordTok{if} \NormalTok{(}\KeywordTok{answer}\NormalTok{.}\FunctionTok{match}\NormalTok{(}\OtherTok{/}\FloatTok{^}\OtherTok{y}\FloatTok{(}\OtherTok{es}\FloatTok{)?$}\OtherTok{/i}\NormalTok{)) }\KeywordTok{rl}\NormalTok{.}\FunctionTok{pause}\NormalTok{();}
  \NormalTok{\});}
\NormalTok{\});}
\end{Highlighting}
\end{Shaded}

\subsubsection{Event: `SIGTSTP'}

\texttt{function () \{\}}

\textbf{This does not work on Windows.}

Emitted whenever the \texttt{input} stream receives a \texttt{\^{}Z},
respectively known as \texttt{SIGTSTP}. If there is no \texttt{SIGTSTP}
event listener present when the \texttt{input} stream receives a
\texttt{SIGTSTP}, the program will be sent to the background.

When the program is resumed with \texttt{fg}, the \texttt{pause} and
\texttt{SIGCONT} events will be emitted. You can use either to resume
the stream.

The \texttt{pause} and \texttt{SIGCONT} events will not be triggered if
the stream was paused before the program was sent to the background.

Example of listening for \texttt{SIGTSTP}:

\begin{Shaded}
\begin{Highlighting}[]
\KeywordTok{rl}\NormalTok{.}\FunctionTok{on}\NormalTok{(}\CharTok{'SIGTSTP'}\NormalTok{, }\KeywordTok{function}\NormalTok{() \{}
  \CommentTok{// This will override SIGTSTP and prevent the program from going to the}
  \CommentTok{// background.}
  \KeywordTok{console}\NormalTok{.}\FunctionTok{log}\NormalTok{(}\CharTok{'Caught SIGTSTP.'}\NormalTok{);}
\NormalTok{\});}
\end{Highlighting}
\end{Shaded}

\subsubsection{Event: `SIGCONT'}

\texttt{function () \{\}}

\textbf{This does not work on Windows.}

Emitted whenever the \texttt{input} stream is sent to the background
with \texttt{\^{}Z}, respectively known as \texttt{SIGTSTP}, and then
continued with \texttt{fg(1)}. This event only emits if the stream was
not paused before sending the program to the background.

Example of listening for \texttt{SIGCONT}:

\begin{Shaded}
\begin{Highlighting}[]
\KeywordTok{rl}\NormalTok{.}\FunctionTok{on}\NormalTok{(}\CharTok{'SIGCONT'}\NormalTok{, }\KeywordTok{function}\NormalTok{() \{}
  \CommentTok{// `prompt` will automatically resume the stream}
  \KeywordTok{rl}\NormalTok{.}\FunctionTok{prompt}\NormalTok{();}
\NormalTok{\});}
\end{Highlighting}
\end{Shaded}

\subsection{Example: Tiny CLI}

Here's an example of how to use all these together to craft a tiny
command line interface:

\begin{Shaded}
\begin{Highlighting}[]
\KeywordTok{var} \NormalTok{readline = require(}\CharTok{'readline'}\NormalTok{),}
    \NormalTok{rl = }\KeywordTok{readline}\NormalTok{.}\FunctionTok{createInterface}\NormalTok{(}\KeywordTok{process}\NormalTok{.}\FunctionTok{stdin}\NormalTok{, }\KeywordTok{process}\NormalTok{.}\FunctionTok{stdout}\NormalTok{);}

\KeywordTok{rl}\NormalTok{.}\FunctionTok{setPrompt}\NormalTok{(}\CharTok{'OHAI> '}\NormalTok{);}
\KeywordTok{rl}\NormalTok{.}\FunctionTok{prompt}\NormalTok{();}

\KeywordTok{rl}\NormalTok{.}\FunctionTok{on}\NormalTok{(}\CharTok{'line'}\NormalTok{, }\KeywordTok{function}\NormalTok{(line) \{}
  \KeywordTok{switch}\NormalTok{(}\KeywordTok{line}\NormalTok{.}\FunctionTok{trim}\NormalTok{()) \{}
    \KeywordTok{case} \CharTok{'hello'}\NormalTok{:}
      \KeywordTok{console}\NormalTok{.}\FunctionTok{log}\NormalTok{(}\CharTok{'world!'}\NormalTok{);}
      \KeywordTok{break}\NormalTok{;}
    \DataTypeTok{default}\NormalTok{:}
      \KeywordTok{console}\NormalTok{.}\FunctionTok{log}\NormalTok{(}\CharTok{'Say what? I might have heard `'} \NormalTok{+ }\KeywordTok{line}\NormalTok{.}\FunctionTok{trim}\NormalTok{() + }\CharTok{'`'}\NormalTok{);}
      \KeywordTok{break}\NormalTok{;}
  \NormalTok{\}}
  \KeywordTok{rl}\NormalTok{.}\FunctionTok{prompt}\NormalTok{();}
\NormalTok{\}).}\FunctionTok{on}\NormalTok{(}\CharTok{'close'}\NormalTok{, }\KeywordTok{function}\NormalTok{() \{}
  \KeywordTok{console}\NormalTok{.}\FunctionTok{log}\NormalTok{(}\CharTok{'Have a great day!'}\NormalTok{);}
  \KeywordTok{process}\NormalTok{.}\FunctionTok{exit}\NormalTok{(}\DecValTok{0}\NormalTok{);}
\NormalTok{\});}
\end{Highlighting}
\end{Shaded}

