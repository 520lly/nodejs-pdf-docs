\section{DNS}\label{dns}

\begin{Shaded}
\begin{Highlighting}[]
\NormalTok{Stability: }\DecValTok{3} \NormalTok{- Stable}
\end{Highlighting}
\end{Shaded}

Use \texttt{require(\textquotesingle{}dns\textquotesingle{})} to access
this module.

This module contains functions that belong to two different categories:

\begin{enumerate}
\def\labelenumi{\arabic{enumi})}
\itemsep1pt\parskip0pt\parsep0pt
\item
  Functions that use the underlying operating system facilities to
  perform name resolution, and that do not necessarily do any network
  communication. This category contains only one function:
  \texttt{dns.lookup}. \textbf{Developers looking to perform name
  resolution in the same way that other applications on the same
  operating system behave should use \texttt{dns.lookup}.}
\end{enumerate}

Here is an example that does a lookup of \texttt{www.google.com}.

\begin{Shaded}
\begin{Highlighting}[]
\KeywordTok{var} \NormalTok{dns = }\FunctionTok{require}\NormalTok{(}\StringTok{'dns'}\NormalTok{);}

\OtherTok{dns}\NormalTok{.}\FunctionTok{lookup}\NormalTok{(}\StringTok{'www.google.com'}\NormalTok{, }\KeywordTok{function} \FunctionTok{onLookup}\NormalTok{(err, addresses, family) \{}
  \OtherTok{console}\NormalTok{.}\FunctionTok{log}\NormalTok{(}\StringTok{'addresses:'}\NormalTok{, addresses);}
\NormalTok{\});}
\end{Highlighting}
\end{Shaded}

\begin{enumerate}
\def\labelenumi{\arabic{enumi})}
\setcounter{enumi}{1}
\itemsep1pt\parskip0pt\parsep0pt
\item
  Functions that connect to an actual DNS server to perform name
  resolution, and that \emph{always} use the network to perform DNS
  queries. This category contains all functions in the \texttt{dns}
  module but \texttt{dns.lookup}. These functions do not use the same
  set of configuration files than what \texttt{dns.lookup} uses. For
  instance, \emph{they do not use the configuration from
  \texttt{/etc/hosts}}. These functions should be used by developers who
  do not want to use the underlying operating system's facilities for
  name resolution, and instead want to \emph{always} perform DNS
  queries.
\end{enumerate}

Here is an example which resolves
\texttt{\textquotesingle{}www.google.com\textquotesingle{}} then reverse
resolves the IP addresses which are returned.

\begin{Shaded}
\begin{Highlighting}[]
\KeywordTok{var} \NormalTok{dns = }\FunctionTok{require}\NormalTok{(}\StringTok{'dns'}\NormalTok{);}

\OtherTok{dns}\NormalTok{.}\FunctionTok{resolve4}\NormalTok{(}\StringTok{'www.google.com'}\NormalTok{, }\KeywordTok{function} \NormalTok{(err, addresses) \{}
  \KeywordTok{if} \NormalTok{(err) }\KeywordTok{throw} \NormalTok{err;}

  \OtherTok{console}\NormalTok{.}\FunctionTok{log}\NormalTok{(}\StringTok{'addresses: '} \NormalTok{+ }\OtherTok{JSON}\NormalTok{.}\FunctionTok{stringify}\NormalTok{(addresses));}

  \OtherTok{addresses}\NormalTok{.}\FunctionTok{forEach}\NormalTok{(}\KeywordTok{function} \NormalTok{(a) \{}
    \OtherTok{dns}\NormalTok{.}\FunctionTok{reverse}\NormalTok{(a, }\KeywordTok{function} \NormalTok{(err, hostnames) \{}
      \KeywordTok{if} \NormalTok{(err) \{}
        \KeywordTok{throw} \NormalTok{err;}
      \NormalTok{\}}

      \OtherTok{console}\NormalTok{.}\FunctionTok{log}\NormalTok{(}\StringTok{'reverse for '} \NormalTok{+ a + }\StringTok{': '} \NormalTok{+ }\OtherTok{JSON}\NormalTok{.}\FunctionTok{stringify}\NormalTok{(hostnames));}
    \NormalTok{\});}
  \NormalTok{\});}
\NormalTok{\});}
\end{Highlighting}
\end{Shaded}

There are subtle consequences in choosing one or another, please consult
the \hyperref[dnsux5fimplementationux5fconsiderations]{Implementation
considerations section} for more information.

\subsection{dns.lookup(hostname{[}, options{]},
callback)}\label{dns.lookuphostname-options-callback}

Resolves a hostname (e.g.
\texttt{\textquotesingle{}google.com\textquotesingle{}}) into the first
found A (IPv4) or AAAA (IPv6) record. \texttt{options} can be an object
or integer. If \texttt{options} is not provided, then IP v4 and v6
addresses are both valid. If \texttt{options} is an integer, then it
must be \texttt{4} or \texttt{6}.

Alternatively, \texttt{options} can be an object containing two
properties, \texttt{family} and \texttt{hints}. Both properties are
optional. If \texttt{family} is provided, it must be the integer
\texttt{4} or \texttt{6}. If \texttt{family} is not provided then IP v4
and v6 addresses are accepted. The \texttt{hints} field, if present,
should be one or more of the supported \texttt{getaddrinfo} flags. If
\texttt{hints} is not provided, then no flags are passed to
\texttt{getaddrinfo}. Multiple flags can be passed through
\texttt{hints} by logically \texttt{OR}ing their values. An example
usage of \texttt{options} is shown below.

\begin{verbatim}
{
  family: 4,
  hints: dns.ADDRCONFIG | dns.V4MAPPED
}
\end{verbatim}

See \hyperref[dnsux5fsupportedux5fgetaddrinfoux5fflags]{supported
\texttt{getaddrinfo} flags} below for more information on supported
flags.

The callback has arguments \texttt{(err,\ address,\ family)}. The
\texttt{address} argument is a string representation of a IP v4 or v6
address. The \texttt{family} argument is either the integer 4 or 6 and
denotes the family of \texttt{address} (not necessarily the value
initially passed to \texttt{lookup}).

On error, \texttt{err} is an \texttt{Error} object, where
\texttt{err.code} is the error code. Keep in mind that \texttt{err.code}
will be set to \texttt{\textquotesingle{}ENOENT\textquotesingle{}} not
only when the hostname does not exist but also when the lookup fails in
other ways such as no available file descriptors.

\texttt{dns.lookup} doesn't necessarily have anything to do with the DNS
protocol. It's only an operating system facility that can associate name
with addresses, and vice versa.

Its implementation can have subtle but important consequences on the
behavior of any Node.js program. Please take some time to consult the
\hyperref[dnsux5fimplementationux5fconsiderations]{Implementation
considerations section} before using it.

\section{dns.lookupService(address, port,
callback)}\label{dns.lookupserviceaddress-port-callback}

Resolves the given address and port into a hostname and service using
\texttt{getnameinfo}.

The callback has arguments \texttt{(err,\ hostname,\ service)}. The
\texttt{hostname} and \texttt{service} arguments are strings (e.g.
\texttt{\textquotesingle{}localhost\textquotesingle{}} and
\texttt{\textquotesingle{}http\textquotesingle{}} respectively).

On error, \texttt{err} is an \texttt{Error} object, where
\texttt{err.code} is the error code.

\subsection{dns.resolve(hostname{[}, rrtype{]},
callback)}\label{dns.resolvehostname-rrtype-callback}

Resolves a hostname (e.g.
\texttt{\textquotesingle{}google.com\textquotesingle{}}) into an array
of the record types specified by rrtype.

Valid rrtypes are:

\begin{itemize}
\itemsep1pt\parskip0pt\parsep0pt
\item
  \texttt{\textquotesingle{}A\textquotesingle{}} (IPV4 addresses,
  default)
\item
  \texttt{\textquotesingle{}AAAA\textquotesingle{}} (IPV6 addresses)
\item
  \texttt{\textquotesingle{}MX\textquotesingle{}} (mail exchange
  records)
\item
  \texttt{\textquotesingle{}TXT\textquotesingle{}} (text records)
\item
  \texttt{\textquotesingle{}SRV\textquotesingle{}} (SRV records)
\item
  \texttt{\textquotesingle{}PTR\textquotesingle{}} (used for reverse IP
  lookups)
\item
  \texttt{\textquotesingle{}NS\textquotesingle{}} (name server records)
\item
  \texttt{\textquotesingle{}CNAME\textquotesingle{}} (canonical name
  records)
\item
  \texttt{\textquotesingle{}SOA\textquotesingle{}} (start of authority
  record)
\end{itemize}

The callback has arguments \texttt{(err,\ addresses)}. The type of each
item in \texttt{addresses} is determined by the record type, and
described in the documentation for the corresponding lookup methods
below.

On error, \texttt{err} is an \texttt{Error} object, where
\texttt{err.code} is one of the error codes listed below.

\subsection{dns.resolve4(hostname,
callback)}\label{dns.resolve4hostname-callback}

The same as \texttt{dns.resolve()}, but only for IPv4 queries
(\texttt{A} records). \texttt{addresses} is an array of IPv4 addresses
(e.g.
\texttt{{[}\textquotesingle{}74.125.79.104\textquotesingle{},\ \textquotesingle{}74.125.79.105\textquotesingle{},\ \textquotesingle{}74.125.79.106\textquotesingle{}{]}}).

\subsection{dns.resolve6(hostname,
callback)}\label{dns.resolve6hostname-callback}

The same as \texttt{dns.resolve4()} except for IPv6 queries (an
\texttt{AAAA} query).

\subsection{dns.resolveMx(hostname,
callback)}\label{dns.resolvemxhostname-callback}

The same as \texttt{dns.resolve()}, but only for mail exchange queries
(\texttt{MX} records).

\texttt{addresses} is an array of MX records, each with a priority and
an exchange attribute (e.g.
\texttt{{[}\{\textquotesingle{}priority\textquotesingle{}:\ 10,\ \textquotesingle{}exchange\textquotesingle{}:\ \textquotesingle{}mx.example.com\textquotesingle{}\},...{]}}).

\subsection{dns.resolveTxt(hostname,
callback)}\label{dns.resolvetxthostname-callback}

The same as \texttt{dns.resolve()}, but only for text queries
(\texttt{TXT} records). \texttt{addresses} is an 2-d array of the text
records available for \texttt{hostname} (e.g.,
\texttt{{[}\ {[}\textquotesingle{}v=spf1\ ip4:0.0.0.0\ \textquotesingle{},\ \textquotesingle{}\textasciitilde{}all\textquotesingle{}\ {]}\ {]}}).
Each sub-array contains TXT chunks of one record. Depending on the use
case, the could be either joined together or treated separately.

\subsection{dns.resolveSrv(hostname,
callback)}\label{dns.resolvesrvhostname-callback}

The same as \texttt{dns.resolve()}, but only for service records
(\texttt{SRV} records). \texttt{addresses} is an array of the SRV
records available for \texttt{hostname}. Properties of SRV records are
priority, weight, port, and name (e.g.,
\texttt{{[}\{\textquotesingle{}priority\textquotesingle{}:\ 10,\ \textquotesingle{}weight\textquotesingle{}:\ 5,\ \textquotesingle{}port\textquotesingle{}:\ 21223,\ \textquotesingle{}name\textquotesingle{}:\ \textquotesingle{}service.example.com\textquotesingle{}\},\ ...{]}}).

\subsection{dns.resolveSoa(hostname,
callback)}\label{dns.resolvesoahostname-callback}

The same as \texttt{dns.resolve()}, but only for start of authority
record queries (\texttt{SOA} record).

\texttt{addresses} is an object with the following structure:

\begin{verbatim}
{
  nsname: 'ns.example.com',
  hostmaster: 'root.example.com',
  serial: 2013101809,
  refresh: 10000,
  retry: 2400,
  expire: 604800,
  minttl: 3600
}
\end{verbatim}

\subsection{dns.resolveNs(hostname,
callback)}\label{dns.resolvenshostname-callback}

The same as \texttt{dns.resolve()}, but only for name server records
(\texttt{NS} records). \texttt{addresses} is an array of the name server
records available for \texttt{hostname} (e.g.,
\texttt{{[}\textquotesingle{}ns1.example.com\textquotesingle{},\ \textquotesingle{}ns2.example.com\textquotesingle{}{]}}).

\subsection{dns.resolveCname(hostname,
callback)}\label{dns.resolvecnamehostname-callback}

The same as \texttt{dns.resolve()}, but only for canonical name records
(\texttt{CNAME} records). \texttt{addresses} is an array of the
canonical name records available for \texttt{hostname} (e.g.,
\texttt{{[}\textquotesingle{}bar.example.com\textquotesingle{}{]}}).

\subsection{dns.reverse(ip, callback)}\label{dns.reverseip-callback}

Reverse resolves an ip address to an array of hostnames.

The callback has arguments \texttt{(err,\ hostnames)}.

On error, \texttt{err} is an \texttt{Error} object, where
\texttt{err.code} is one of the error codes listed below.

\subsection{dns.getServers()}\label{dns.getservers}

Returns an array of IP addresses as strings that are currently being
used for resolution

\subsection{dns.setServers(servers)}\label{dns.setserversservers}

Given an array of IP addresses as strings, set them as the servers to
use for resolving

If you specify a port with the address it will be stripped, as the
underlying library doesn't support that.

This will throw if you pass invalid input.

\subsection{Error codes}\label{error-codes}

Each DNS query can return one of the following error codes:

\begin{itemize}
\itemsep1pt\parskip0pt\parsep0pt
\item
  \texttt{dns.NODATA}: DNS server returned answer with no data.
\item
  \texttt{dns.FORMERR}: DNS server claims query was misformatted.
\item
  \texttt{dns.SERVFAIL}: DNS server returned general failure.
\item
  \texttt{dns.NOTFOUND}: Domain name not found.
\item
  \texttt{dns.NOTIMP}: DNS server does not implement requested
  operation.
\item
  \texttt{dns.REFUSED}: DNS server refused query.
\item
  \texttt{dns.BADQUERY}: Misformatted DNS query.
\item
  \texttt{dns.BADNAME}: Misformatted hostname.
\item
  \texttt{dns.BADFAMILY}: Unsupported address family.
\item
  \texttt{dns.BADRESP}: Misformatted DNS reply.
\item
  \texttt{dns.CONNREFUSED}: Could not contact DNS servers.
\item
  \texttt{dns.TIMEOUT}: Timeout while contacting DNS servers.
\item
  \texttt{dns.EOF}: End of file.
\item
  \texttt{dns.FILE}: Error reading file.
\item
  \texttt{dns.NOMEM}: Out of memory.
\item
  \texttt{dns.DESTRUCTION}: Channel is being destroyed.
\item
  \texttt{dns.BADSTR}: Misformatted string.
\item
  \texttt{dns.BADFLAGS}: Illegal flags specified.
\item
  \texttt{dns.NONAME}: Given hostname is not numeric.
\item
  \texttt{dns.BADHINTS}: Illegal hints flags specified.
\item
  \texttt{dns.NOTINITIALIZED}: c-ares library initialization not yet
  performed.
\item
  \texttt{dns.LOADIPHLPAPI}: Error loading iphlpapi.dll.
\item
  \texttt{dns.ADDRGETNETWORKPARAMS}: Could not find GetNetworkParams
  function.
\item
  \texttt{dns.CANCELLED}: DNS query cancelled.
\end{itemize}

\subsection{Supported getaddrinfo
flags}\label{supported-getaddrinfo-flags}

The following flags can be passed as hints to \texttt{dns.lookup}.

\begin{itemize}
\itemsep1pt\parskip0pt\parsep0pt
\item
  \texttt{dns.ADDRCONFIG}: Returned address types are determined by the
  types of addresses supported by the current system. For example, IPv4
  addresses are only returned if the current system has at least one
  IPv4 address configured. Loopback addresses are not considered.
\item
  \texttt{dns.V4MAPPED}: If the IPv6 family was specified, but no IPv6
  addresses were found, then return IPv4 mapped IPv6 addresses. Note
  that it is not supported on some operating systems (e.g FreeBSD 10.1).
\end{itemize}

\subsection{Implementation
considerations}\label{implementation-considerations}

Although \texttt{dns.lookup} and \texttt{dns.resolve*/dns.reverse}
functions have the same goal of associating a network name with a
network address (or vice versa), their behavior is quite different.
These differences can have subtle but significant consequences on the
behavior of Node.js programs.

\subsubsection{dns.lookup}\label{dns.lookup}

Under the hood, \texttt{dns.lookup} uses the same operating system
facilities as most other programs. For instance, \texttt{dns.lookup}
will almost always resolve a given name the same way as the
\texttt{ping} command. On most POSIX-like operating systems, the
behavior of the \texttt{dns.lookup} function can be tweaked by changing
settings in \texttt{nsswitch.conf(5)} and/or \texttt{resolv.conf(5)},
but be careful that changing these files will change the behavior of all
other programs running on the same operating system.

Though the call will be asynchronous from JavaScript's perspective, it
is implemented as a synchronous call to \texttt{getaddrinfo(3)} that
runs on libuv's threadpool. Because libuv's threadpool has a fixed size,
it means that if for whatever reason the call to \texttt{getaddrinfo(3)}
takes a long time, other operations that could run on libuv's threadpool
(such as filesystem operations) will experience degraded performance. In
order to mitigate this issue, one potential solution is to increase the
size of libuv's threadpool by setting the `UV\_THREADPOOL\_SIZE'
environment variable to a value greater than 4 (its current default
value). For more information on libuv's threadpool, see
\href{http://docs.libuv.org/en/latest/threadpool.html}{the official
libuv documentation}.

\subsubsection{dns.resolve, functions starting with dns.resolve and
dns.reverse}\label{dns.resolve-functions-starting-with-dns.resolve-and-dns.reverse}

These functions are implemented quite differently than
\texttt{dns.lookup}. They do not use \texttt{getaddrinfo(3)} and they
\emph{always} perform a DNS query on the network. This network
communication is always done asynchronously, and does not use libuv's
threadpool.

As a result, these functions cannot have the same negative impact on
other processing that happens on libuv's threadpool that
\texttt{dns.lookup} can have.

They do not use the same set of configuration files than what
\texttt{dns.lookup} uses. For instance, \emph{they do not use the
configuration from \texttt{/etc/hosts}}.
