\section{Cluster}

\begin{Shaded}
\begin{Highlighting}[]
\DataTypeTok{Stability}\NormalTok{: }\DecValTok{1} \NormalTok{- Experimental}
\end{Highlighting}
\end{Shaded}

A single instance of Node runs in a single thread. To take advantage of
multi-core systems the user will sometimes want to launch a cluster of
Node processes to handle the load.

The cluster module allows you to easily create a network of processes
that all share server ports.

\begin{Shaded}
\begin{Highlighting}[]
\KeywordTok{var} \NormalTok{cluster = require(}\CharTok{'cluster'}\NormalTok{);}
\KeywordTok{var} \NormalTok{http = require(}\CharTok{'http'}\NormalTok{);}
\KeywordTok{var} \NormalTok{numCPUs = require(}\CharTok{'os'}\NormalTok{).}\FunctionTok{cpus}\NormalTok{().}\FunctionTok{length}\NormalTok{;}

\KeywordTok{if} \NormalTok{(}\KeywordTok{cluster}\NormalTok{.}\FunctionTok{isMaster}\NormalTok{) \{}
  \CommentTok{// Fork workers.}
  \KeywordTok{for} \NormalTok{(}\KeywordTok{var} \NormalTok{i = }\DecValTok{0}\NormalTok{; i < numCPUs; i++) \{}
    \KeywordTok{cluster}\NormalTok{.}\FunctionTok{fork}\NormalTok{();}
  \NormalTok{\}}

  \KeywordTok{cluster}\NormalTok{.}\FunctionTok{on}\NormalTok{(}\CharTok{'exit'}\NormalTok{, }\KeywordTok{function}\NormalTok{(worker, code, signal) \{}
    \KeywordTok{console}\NormalTok{.}\FunctionTok{log}\NormalTok{(}\CharTok{'worker '} \NormalTok{+ }\KeywordTok{worker.process}\NormalTok{.}\FunctionTok{pid} \NormalTok{+ }\CharTok{' died'}\NormalTok{);}
  \NormalTok{\});}
\NormalTok{\} }\KeywordTok{else} \NormalTok{\{}
  \CommentTok{// Workers can share any TCP connection}
  \CommentTok{// In this case its a HTTP server}
  \KeywordTok{http}\NormalTok{.}\FunctionTok{createServer}\NormalTok{(}\KeywordTok{function}\NormalTok{(req, res) \{}
    \KeywordTok{res}\NormalTok{.}\FunctionTok{writeHead}\NormalTok{(}\DecValTok{200}\NormalTok{);}
    \KeywordTok{res}\NormalTok{.}\FunctionTok{end}\NormalTok{(}\StringTok{"hello world}\CharTok{\textbackslash{}n}\StringTok{"}\NormalTok{);}
  \NormalTok{\}).}\FunctionTok{listen}\NormalTok{(}\DecValTok{8000}\NormalTok{);}
\NormalTok{\}}
\end{Highlighting}
\end{Shaded}

Running node will now share port 8000 between the workers:

\begin{Shaded}
\begin{Highlighting}[]
\NormalTok{% node }\KeywordTok{server}\NormalTok{.}\FunctionTok{js}
\FunctionTok{Worker} \DecValTok{2438} \NormalTok{online}
\NormalTok{Worker }\DecValTok{2437} \NormalTok{online}
\end{Highlighting}
\end{Shaded}

This feature was introduced recently, and may change in future versions.
Please try it out and provide feedback.

Also note that, on Windows, it is not yet possible to set up a named
pipe server in a worker.

\subsection{How It Works}

The worker processes are spawned using the \texttt{child\_process.fork}
method, so that they can communicate with the parent via IPC and pass
server handles back and forth.

When you call \texttt{server.listen(...)} in a worker, it serializes the
arguments and passes the request to the master process. If the master
process already has a listening server matching the worker's
requirements, then it passes the handle to the worker. If it does not
already have a listening server matching that requirement, then it will
create one, and pass the handle to the child.

This causes potentially surprising behavior in three edge cases:

\begin{enumerate}[1.]
\item
  \texttt{server.listen(\{fd: 7\})} Because the message is passed to the
  master, file descriptor 7 \textbf{in the parent} will be listened on,
  and the handle passed to the worker, rather than listening to the
  worker's idea of what the number 7 file descriptor references.
\item
  \texttt{server.listen(handle)} Listening on handles explicitly will
  cause the worker to use the supplied handle, rather than talk to the
  master process. If the worker already has the handle, then it's
  presumed that you know what you are doing.
\item
  \texttt{server.listen(0)} Normally, this will cause servers to listen
  on a random port. However, in a cluster, each worker will receive the
  same ``random'' port each time they do \texttt{listen(0)}. In essence,
  the port is random the first time, but predictable thereafter. If you
  want to listen on a unique port, generate a port number based on the
  cluster worker ID.
\end{enumerate}

When multiple processes are all \texttt{accept()}ing on the same
underlying resource, the operating system load-balances across them very
efficiently. There is no routing logic in Node.js, or in your program,
and no shared state between the workers. Therefore, it is important to
design your program such that it does not rely too heavily on in-memory
data objects for things like sessions and login.

Because workers are all separate processes, they can be killed or
re-spawned depending on your program's needs, without affecting other
workers. As long as there are some workers still alive, the server will
continue to accept connections. Node does not automatically manage the
number of workers for you, however. It is your responsibility to manage
the worker pool for your application's needs.

\subsection{cluster.settings}

\begin{itemize}
\item
  \{Object\}
\item
  \texttt{exec} \{String\} file path to worker file.
  (Default=\texttt{\_\_filename})
\item
  \texttt{args} \{Array\} string arguments passed to worker.
  (Default=\texttt{process.argv.slice(2)})
\item
  \texttt{silent} \{Boolean\} whether or not to send output to parent's
  stdio. (Default=\texttt{false})
\end{itemize}

All settings set by the \texttt{.setupMaster} is stored in this settings
object. This object is not supposed to be change or set manually, by
you.

\subsection{cluster.isMaster}

\begin{itemize}
\item
  \{Boolean\}
\end{itemize}

True if the process is a master. This is determined by the
\texttt{process.env.NODE\_UNIQUE\_ID}. If
\texttt{process.env.NODE\_UNIQUE\_ID} is undefined, then
\texttt{isMaster} is \texttt{true}.

\subsection{cluster.isWorker}

\begin{itemize}
\item
  \{Boolean\}
\end{itemize}

This boolean flag is true if the process is a worker forked from a
master. If the \texttt{process.env.NODE\_UNIQUE\_ID} is set to a value,
then \texttt{isWorker} is \texttt{true}.

\subsection{Event: `fork'}

\begin{itemize}
\item
  \texttt{worker} \{Worker object\}
\end{itemize}

When a new worker is forked the cluster module will emit a `fork' event.
This can be used to log worker activity, and create you own timeout.

\begin{Shaded}
\begin{Highlighting}[]
\KeywordTok{var} \NormalTok{timeouts = [];}
\KeywordTok{function} \NormalTok{errorMsg() \{}
  \KeywordTok{console}\NormalTok{.}\FunctionTok{error}\NormalTok{(}\StringTok{"Something must be wrong with the connection ..."}\NormalTok{);}
\NormalTok{\}}

\KeywordTok{cluster}\NormalTok{.}\FunctionTok{on}\NormalTok{(}\CharTok{'fork'}\NormalTok{, }\KeywordTok{function}\NormalTok{(worker) \{}
  \NormalTok{timeouts[}\KeywordTok{worker}\NormalTok{.}\FunctionTok{id}\NormalTok{] = setTimeout(errorMsg, }\DecValTok{2000}\NormalTok{);}
\NormalTok{\});}
\KeywordTok{cluster}\NormalTok{.}\FunctionTok{on}\NormalTok{(}\CharTok{'listening'}\NormalTok{, }\KeywordTok{function}\NormalTok{(worker, address) \{}
  \NormalTok{clearTimeout(timeouts[}\KeywordTok{worker}\NormalTok{.}\FunctionTok{id}\NormalTok{]);}
\NormalTok{\});}
\KeywordTok{cluster}\NormalTok{.}\FunctionTok{on}\NormalTok{(}\CharTok{'exit'}\NormalTok{, }\KeywordTok{function}\NormalTok{(worker, code, signal) \{}
  \NormalTok{clearTimeout(timeouts[}\KeywordTok{worker}\NormalTok{.}\FunctionTok{id}\NormalTok{]);}
  \NormalTok{errorMsg();}
\NormalTok{\});}
\end{Highlighting}
\end{Shaded}

\subsection{Event: `online'}

\begin{itemize}
\item
  \texttt{worker} \{Worker object\}
\end{itemize}

After forking a new worker, the worker should respond with a online
message. When the master receives a online message it will emit such
event. The difference between `fork' and `online' is that fork is
emitted when the master tries to fork a worker, and `online' is emitted
when the worker is being executed.

\begin{Shaded}
\begin{Highlighting}[]
\KeywordTok{cluster}\NormalTok{.}\FunctionTok{on}\NormalTok{(}\CharTok{'online'}\NormalTok{, }\KeywordTok{function}\NormalTok{(worker) \{}
  \KeywordTok{console}\NormalTok{.}\FunctionTok{log}\NormalTok{(}\StringTok{"Yay, the worker responded after it was forked"}\NormalTok{);}
\NormalTok{\});}
\end{Highlighting}
\end{Shaded}

\subsection{Event: `listening'}

\begin{itemize}
\item
  \texttt{worker} \{Worker object\}
\item
  \texttt{address} \{Object\}
\end{itemize}

When calling \texttt{listen()} from a worker, a `listening' event is
automatically assigned to the server instance. When the server is
listening a message is send to the master where the `listening' event is
emitted.

The event handler is executed with two arguments, the \texttt{worker}
contains the worker object and the \texttt{address} object contains the
following connection properties: \texttt{address}, \texttt{port} and
\texttt{addressType}. This is very useful if the worker is listening on
more than one address.

\begin{Shaded}
\begin{Highlighting}[]
\KeywordTok{cluster}\NormalTok{.}\FunctionTok{on}\NormalTok{(}\CharTok{'listening'}\NormalTok{, }\KeywordTok{function}\NormalTok{(worker, address) \{}
  \KeywordTok{console}\NormalTok{.}\FunctionTok{log}\NormalTok{(}\StringTok{"A worker is now connected to "} \NormalTok{+ }\KeywordTok{address}\NormalTok{.}\FunctionTok{address} \NormalTok{+ }\StringTok{":"} \NormalTok{+ }\KeywordTok{address}\NormalTok{.}\FunctionTok{port}\NormalTok{);}
\NormalTok{\});}
\end{Highlighting}
\end{Shaded}

\subsection{Event: `disconnect'}

\begin{itemize}
\item
  \texttt{worker} \{Worker object\}
\end{itemize}

When a workers IPC channel has disconnected this event is emitted. This
will happen when the worker dies, usually after calling
\texttt{.destroy()}.

When calling \texttt{.disconnect()}, there may be a delay between the
\texttt{disconnect} and \texttt{exit} events. This event can be used to
detect if the process is stuck in a cleanup or if there are long-living
connections.

\begin{Shaded}
\begin{Highlighting}[]
\KeywordTok{cluster}\NormalTok{.}\FunctionTok{on}\NormalTok{(}\CharTok{'disconnect'}\NormalTok{, }\KeywordTok{function}\NormalTok{(worker) \{}
  \KeywordTok{console}\NormalTok{.}\FunctionTok{log}\NormalTok{(}\CharTok{'The worker #'} \NormalTok{+ }\KeywordTok{worker}\NormalTok{.}\FunctionTok{id} \NormalTok{+ }\CharTok{' has disconnected'}\NormalTok{);}
\NormalTok{\});}
\end{Highlighting}
\end{Shaded}

\subsection{Event: `exit'}

\begin{itemize}
\item
  \texttt{worker} \{Worker object\}
\item
  \texttt{code} \{Number\} the exit code, if it exited normally.
\item
  \texttt{signal} \{String\} the name of the signal (eg.
  \texttt{'SIGHUP'}) that caused the process to be killed.
\end{itemize}

When any of the workers die the cluster module will emit the `exit'
event. This can be used to restart the worker by calling \texttt{fork()}
again.

\begin{Shaded}
\begin{Highlighting}[]
\KeywordTok{cluster}\NormalTok{.}\FunctionTok{on}\NormalTok{(}\CharTok{'exit'}\NormalTok{, }\KeywordTok{function}\NormalTok{(worker, code, signal) \{}
  \KeywordTok{var} \NormalTok{exitCode = }\KeywordTok{worker.process}\NormalTok{.}\FunctionTok{exitCode}\NormalTok{;}
  \KeywordTok{console}\NormalTok{.}\FunctionTok{log}\NormalTok{(}\CharTok{'worker '} \NormalTok{+ }\KeywordTok{worker.process}\NormalTok{.}\FunctionTok{pid} \NormalTok{+ }\CharTok{' died ('}\NormalTok{+exitCode+}\CharTok{'). restarting...'}\NormalTok{);}
  \KeywordTok{cluster}\NormalTok{.}\FunctionTok{fork}\NormalTok{();}
\NormalTok{\});}
\end{Highlighting}
\end{Shaded}

\subsection{Event: `setup'}

\begin{itemize}
\item
  \texttt{worker} \{Worker object\}
\end{itemize}

When the \texttt{.setupMaster()} function has been executed this event
emits. If \texttt{.setupMaster()} was not executed before
\texttt{fork()} this function will call \texttt{.setupMaster()} with no
arguments.

\subsection{cluster.setupMaster({[}settings{]})}

\begin{itemize}
\item
  \texttt{settings} \{Object\}
\item
  \texttt{exec} \{String\} file path to worker file.
  (Default=\texttt{\_\_filename})
\item
  \texttt{args} \{Array\} string arguments passed to worker.
  (Default=\texttt{process.argv.slice(2)})
\item
  \texttt{silent} \{Boolean\} whether or not to send output to parent's
  stdio. (Default=\texttt{false})
\end{itemize}

\texttt{setupMaster} is used to change the default `fork' behavior. The
new settings are effective immediately and permanently, they cannot be
changed later on.

Example:

\begin{Shaded}
\begin{Highlighting}[]
\KeywordTok{var} \NormalTok{cluster = require(}\StringTok{"cluster"}\NormalTok{);}
\KeywordTok{cluster}\NormalTok{.}\FunctionTok{setupMaster}\NormalTok{(\{}
  \DataTypeTok{exec }\NormalTok{: }\StringTok{"worker.js"}\NormalTok{,}
  \DataTypeTok{args }\NormalTok{: [}\StringTok{"--use"}\NormalTok{, }\StringTok{"https"}\NormalTok{],}
  \DataTypeTok{silent }\NormalTok{: }\KeywordTok{true}
\NormalTok{\});}
\KeywordTok{cluster}\NormalTok{.}\FunctionTok{fork}\NormalTok{();}
\end{Highlighting}
\end{Shaded}

\subsection{cluster.fork({[}env{]})}

\begin{itemize}
\item
  \texttt{env} \{Object\} Key/value pairs to add to child process
  environment.
\item
  return \{Worker object\}
\end{itemize}

Spawn a new worker process. This can only be called from the master
process.

\subsection{cluster.disconnect({[}callback{]})}

\begin{itemize}
\item
  \texttt{callback} \{Function\} called when all workers are
  disconnected and handlers are closed
\end{itemize}

When calling this method, all workers will commit a graceful suicide.
When they are disconnected all internal handlers will be closed,
allowing the master process to die graceful if no other event is
waiting.

The method takes an optional callback argument which will be called when
finished.

\subsection{cluster.worker}

\begin{itemize}
\item
  \{Object\}
\end{itemize}

A reference to the current worker object. Not available in the master
process.

\begin{Shaded}
\begin{Highlighting}[]
\KeywordTok{var} \NormalTok{cluster = require(}\CharTok{'cluster'}\NormalTok{);}

\KeywordTok{if} \NormalTok{(}\KeywordTok{cluster}\NormalTok{.}\FunctionTok{isMaster}\NormalTok{) \{}
  \KeywordTok{console}\NormalTok{.}\FunctionTok{log}\NormalTok{(}\CharTok{'I am master'}\NormalTok{);}
  \KeywordTok{cluster}\NormalTok{.}\FunctionTok{fork}\NormalTok{();}
  \KeywordTok{cluster}\NormalTok{.}\FunctionTok{fork}\NormalTok{();}
\NormalTok{\} }\KeywordTok{else} \KeywordTok{if} \NormalTok{(}\KeywordTok{cluster}\NormalTok{.}\FunctionTok{isWorker}\NormalTok{) \{}
  \KeywordTok{console}\NormalTok{.}\FunctionTok{log}\NormalTok{(}\CharTok{'I am worker #'} \NormalTok{+ }\KeywordTok{cluster.worker}\NormalTok{.}\FunctionTok{id}\NormalTok{);}
\NormalTok{\}}
\end{Highlighting}
\end{Shaded}

\subsection{cluster.workers}

\begin{itemize}
\item
  \{Object\}
\end{itemize}

A hash that stores the active worker objects, keyed by \texttt{id}
field. Makes it easy to loop through all the workers. It is only
available in the master process.

\begin{Shaded}
\begin{Highlighting}[]
\CommentTok{// Go through all workers}
\KeywordTok{function} \NormalTok{eachWorker(callback) \{}
  \KeywordTok{for} \NormalTok{(}\KeywordTok{var} \NormalTok{id }\KeywordTok{in} \KeywordTok{cluster}\NormalTok{.}\FunctionTok{workers}\NormalTok{) \{}
    \NormalTok{callback(}\KeywordTok{cluster}\NormalTok{.}\FunctionTok{workers}\NormalTok{[id]);}
  \NormalTok{\}}
\NormalTok{\}}
\NormalTok{eachWorker(}\KeywordTok{function}\NormalTok{(worker) \{}
  \KeywordTok{worker}\NormalTok{.}\FunctionTok{send}\NormalTok{(}\CharTok{'big announcement to all workers'}\NormalTok{);}
\NormalTok{\});}
\end{Highlighting}
\end{Shaded}

Should you wish to reference a worker over a communication channel,
using the worker's unique id is the easiest way to find the worker.

\begin{Shaded}
\begin{Highlighting}[]
\KeywordTok{socket}\NormalTok{.}\FunctionTok{on}\NormalTok{(}\CharTok{'data'}\NormalTok{, }\KeywordTok{function}\NormalTok{(id) \{}
  \KeywordTok{var} \NormalTok{worker = }\KeywordTok{cluster}\NormalTok{.}\FunctionTok{workers}\NormalTok{[id];}
\NormalTok{\});}
\end{Highlighting}
\end{Shaded}

\subsection{Class: Worker}

A Worker object contains all public information and method about a
worker. In the master it can be obtained using \texttt{cluster.workers}.
In a worker it can be obtained using \texttt{cluster.worker}.

\subsubsection{worker.id}

\begin{itemize}
\item
  \{String\}
\end{itemize}

Each new worker is given its own unique id, this id is stored in the
\texttt{id}.

While a worker is alive, this is the key that indexes it in
cluster.workers

\subsubsection{worker.process}

\begin{itemize}
\item
  \{ChildProcess object\}
\end{itemize}

All workers are created using \texttt{child\_process.fork()}, the
returned object from this function is stored in process.

See: \href{child\_process.html}{Child Process module}

\subsubsection{worker.suicide}

\begin{itemize}
\item
  \{Boolean\}
\end{itemize}

This property is a boolean. It is set when a worker dies after calling
\texttt{.destroy()} or immediately after calling the
\texttt{.disconnect()} method. Until then it is \texttt{undefined}.

\subsubsection{worker.send(message, {[}sendHandle{]})}

\begin{itemize}
\item
  \texttt{message} \{Object\}
\item
  \texttt{sendHandle} \{Handle object\}
\end{itemize}

This function is equal to the send methods provided by
\texttt{child\_process.fork()}. In the master you should use this
function to send a message to a specific worker. However in a worker you
can also use \texttt{process.send(message)}, since this is the same
function.

This example will echo back all messages from the master:

\begin{Shaded}
\begin{Highlighting}[]
\KeywordTok{if} \NormalTok{(}\KeywordTok{cluster}\NormalTok{.}\FunctionTok{isMaster}\NormalTok{) \{}
  \KeywordTok{var} \NormalTok{worker = }\KeywordTok{cluster}\NormalTok{.}\FunctionTok{fork}\NormalTok{();}
  \KeywordTok{worker}\NormalTok{.}\FunctionTok{send}\NormalTok{(}\CharTok{'hi there'}\NormalTok{);}

\NormalTok{\} }\KeywordTok{else} \KeywordTok{if} \NormalTok{(}\KeywordTok{cluster}\NormalTok{.}\FunctionTok{isWorker}\NormalTok{) \{}
  \KeywordTok{process}\NormalTok{.}\FunctionTok{on}\NormalTok{(}\CharTok{'message'}\NormalTok{, }\KeywordTok{function}\NormalTok{(msg) \{}
    \KeywordTok{process}\NormalTok{.}\FunctionTok{send}\NormalTok{(msg);}
  \NormalTok{\});}
\NormalTok{\}}
\end{Highlighting}
\end{Shaded}

\subsubsection{worker.destroy()}

This function will kill the worker, and inform the master to not spawn a
new worker. The boolean \texttt{suicide} lets you distinguish between
voluntary and accidental exit.

\begin{Shaded}
\begin{Highlighting}[]
\KeywordTok{cluster}\NormalTok{.}\FunctionTok{on}\NormalTok{(}\CharTok{'exit'}\NormalTok{, }\KeywordTok{function}\NormalTok{(worker, code, signal) \{}
  \KeywordTok{if} \NormalTok{(}\KeywordTok{worker}\NormalTok{.}\FunctionTok{suicide} \NormalTok{=== }\KeywordTok{true}\NormalTok{) \{}
    \KeywordTok{console}\NormalTok{.}\FunctionTok{log}\NormalTok{(}\CharTok{'Oh, it was just suicide\textbackslash{}' – no need to worry'}\NormalTok{).}
  \NormalTok{\}}
\NormalTok{\});}

\CommentTok{// destroy worker}
\KeywordTok{worker}\NormalTok{.}\FunctionTok{destroy}\NormalTok{();}
\end{Highlighting}
\end{Shaded}

\subsubsection{worker.disconnect()}

When calling this function the worker will no longer accept new
connections, but they will be handled by any other listening worker.
Existing connection will be allowed to exit as usual. When no more
connections exist, the IPC channel to the worker will close allowing it
to die graceful. When the IPC channel is closed the \texttt{disconnect}
event will emit, this is then followed by the \texttt{exit} event, there
is emitted when the worker finally die.

Because there might be long living connections, it is useful to
implement a timeout. This example ask the worker to disconnect and after
2 seconds it will destroy the server. An alternative would be to execute
\texttt{worker.destroy()} after 2 seconds, but that would normally not
allow the worker to do any cleanup if needed.

\begin{Shaded}
\begin{Highlighting}[]
\KeywordTok{if} \NormalTok{(}\KeywordTok{cluster}\NormalTok{.}\FunctionTok{isMaster}\NormalTok{) \{}
  \KeywordTok{var} \NormalTok{worker = }\KeywordTok{cluster}\NormalTok{.}\FunctionTok{fork}\NormalTok{();}
  \KeywordTok{var} \NormalTok{timeout;}

  \KeywordTok{worker}\NormalTok{.}\FunctionTok{on}\NormalTok{(}\CharTok{'listening'}\NormalTok{, }\KeywordTok{function}\NormalTok{(address) \{}
    \KeywordTok{worker}\NormalTok{.}\FunctionTok{disconnect}\NormalTok{();}
    \NormalTok{timeout = setTimeout(}\KeywordTok{function}\NormalTok{() \{}
      \KeywordTok{worker}\NormalTok{.}\FunctionTok{send}\NormalTok{(}\CharTok{'force kill'}\NormalTok{);}
    \NormalTok{\}, }\DecValTok{2000}\NormalTok{);}
  \NormalTok{\});}

  \KeywordTok{worker}\NormalTok{.}\FunctionTok{on}\NormalTok{(}\CharTok{'disconnect'}\NormalTok{, }\KeywordTok{function}\NormalTok{() \{}
    \NormalTok{clearTimeout(timeout);}
  \NormalTok{\});}

\NormalTok{\} }\KeywordTok{else} \KeywordTok{if} \NormalTok{(}\KeywordTok{cluster}\NormalTok{.}\FunctionTok{isWorker}\NormalTok{) \{}
  \KeywordTok{var} \NormalTok{net = require(}\CharTok{'net'}\NormalTok{);}
  \KeywordTok{var} \NormalTok{server = }\KeywordTok{net}\NormalTok{.}\FunctionTok{createServer}\NormalTok{(}\KeywordTok{function}\NormalTok{(socket) \{}
    \CommentTok{// connection never end}
  \NormalTok{\});}

  \KeywordTok{server}\NormalTok{.}\FunctionTok{listen}\NormalTok{(}\DecValTok{8000}\NormalTok{);}

  \KeywordTok{server}\NormalTok{.}\FunctionTok{on}\NormalTok{(}\CharTok{'close'}\NormalTok{, }\KeywordTok{function}\NormalTok{() \{}
    \CommentTok{// cleanup}
  \NormalTok{\});}

  \KeywordTok{process}\NormalTok{.}\FunctionTok{on}\NormalTok{(}\CharTok{'message'}\NormalTok{, }\KeywordTok{function}\NormalTok{(msg) \{}
    \KeywordTok{if} \NormalTok{(msg === }\CharTok{'force kill'}\NormalTok{) \{}
      \KeywordTok{server}\NormalTok{.}\FunctionTok{destroy}\NormalTok{();}
    \NormalTok{\}}
  \NormalTok{\});}
\NormalTok{\}}
\end{Highlighting}
\end{Shaded}

\subsubsection{Event: `message'}

\begin{itemize}
\item
  \texttt{message} \{Object\}
\end{itemize}

This event is the same as the one provided by
\texttt{child\_process.fork()}. In the master you should use this event,
however in a worker you can also use \texttt{process.on('message')}

As an example, here is a cluster that keeps count of the number of
requests in the master process using the message system:

\begin{Shaded}
\begin{Highlighting}[]
\KeywordTok{var} \NormalTok{cluster = require(}\CharTok{'cluster'}\NormalTok{);}
\KeywordTok{var} \NormalTok{http = require(}\CharTok{'http'}\NormalTok{);}

\KeywordTok{if} \NormalTok{(}\KeywordTok{cluster}\NormalTok{.}\FunctionTok{isMaster}\NormalTok{) \{}

  \CommentTok{// Keep track of http requests}
  \KeywordTok{var} \NormalTok{numReqs = }\DecValTok{0}\NormalTok{;}
  \NormalTok{setInterval(}\KeywordTok{function}\NormalTok{() \{}
    \KeywordTok{console}\NormalTok{.}\FunctionTok{log}\NormalTok{(}\StringTok{"numReqs ="}\NormalTok{, numReqs);}
  \NormalTok{\}, }\DecValTok{1000}\NormalTok{);}

  \CommentTok{// Count requestes}
  \KeywordTok{function} \NormalTok{messageHandler(msg) \{}
    \KeywordTok{if} \NormalTok{(}\KeywordTok{msg}\NormalTok{.}\FunctionTok{cmd} \NormalTok{&& }\KeywordTok{msg}\NormalTok{.}\FunctionTok{cmd} \NormalTok{== }\CharTok{'notifyRequest'}\NormalTok{) \{}
      \NormalTok{numReqs += }\DecValTok{1}\NormalTok{;}
    \NormalTok{\}}
  \NormalTok{\}}

  \CommentTok{// Start workers and listen for messages containing notifyRequest}
  \KeywordTok{var} \NormalTok{numCPUs = require(}\CharTok{'os'}\NormalTok{).}\FunctionTok{cpus}\NormalTok{().}\FunctionTok{length}\NormalTok{;}
  \KeywordTok{for} \NormalTok{(}\KeywordTok{var} \NormalTok{i = }\DecValTok{0}\NormalTok{; i < numCPUs; i++) \{}
    \KeywordTok{cluster}\NormalTok{.}\FunctionTok{fork}\NormalTok{();}
  \NormalTok{\}}

  \KeywordTok{Object}\NormalTok{.}\FunctionTok{keys}\NormalTok{(}\KeywordTok{cluster}\NormalTok{.}\FunctionTok{workers}\NormalTok{).}\FunctionTok{forEach}\NormalTok{(}\KeywordTok{function}\NormalTok{(id) \{}
    \KeywordTok{cluster}\NormalTok{.}\FunctionTok{workers}\NormalTok{[id].}\FunctionTok{on}\NormalTok{(}\CharTok{'message'}\NormalTok{, messageHandler);}
  \NormalTok{\});}

\NormalTok{\} }\KeywordTok{else} \NormalTok{\{}

  \CommentTok{// Worker processes have a http server.}
  \KeywordTok{http}\NormalTok{.}\FunctionTok{Server}\NormalTok{(}\KeywordTok{function}\NormalTok{(req, res) \{}
    \KeywordTok{res}\NormalTok{.}\FunctionTok{writeHead}\NormalTok{(}\DecValTok{200}\NormalTok{);}
    \KeywordTok{res}\NormalTok{.}\FunctionTok{end}\NormalTok{(}\StringTok{"hello world}\CharTok{\textbackslash{}n}\StringTok{"}\NormalTok{);}

    \CommentTok{// notify master about the request}
    \KeywordTok{process}\NormalTok{.}\FunctionTok{send}\NormalTok{(\{ }\DataTypeTok{cmd}\NormalTok{: }\CharTok{'notifyRequest'} \NormalTok{\});}
  \NormalTok{\}).}\FunctionTok{listen}\NormalTok{(}\DecValTok{8000}\NormalTok{);}
\NormalTok{\}}
\end{Highlighting}
\end{Shaded}

\subsubsection{Event: `online'}

Same as the \texttt{cluster.on('online')} event, but emits only when the
state change on the specified worker.

\begin{Shaded}
\begin{Highlighting}[]
\KeywordTok{cluster}\NormalTok{.}\FunctionTok{fork}\NormalTok{().}\FunctionTok{on}\NormalTok{(}\CharTok{'online'}\NormalTok{, }\KeywordTok{function}\NormalTok{() \{}
  \CommentTok{// Worker is online}
\NormalTok{\};}
\end{Highlighting}
\end{Shaded}

\subsubsection{Event: `listening'}

\begin{itemize}
\item
  \texttt{address} \{Object\}
\end{itemize}

Same as the \texttt{cluster.on('listening')} event, but emits only when
the state change on the specified worker.

\begin{Shaded}
\begin{Highlighting}[]
\KeywordTok{cluster}\NormalTok{.}\FunctionTok{fork}\NormalTok{().}\FunctionTok{on}\NormalTok{(}\CharTok{'listening'}\NormalTok{, }\KeywordTok{function}\NormalTok{(address) \{}
  \CommentTok{// Worker is listening}
\NormalTok{\};}
\end{Highlighting}
\end{Shaded}

\subsubsection{Event: `disconnect'}

Same as the \texttt{cluster.on('disconnect')} event, but emits only when
the state change on the specified worker.

\begin{Shaded}
\begin{Highlighting}[]
\KeywordTok{cluster}\NormalTok{.}\FunctionTok{fork}\NormalTok{().}\FunctionTok{on}\NormalTok{(}\CharTok{'disconnect'}\NormalTok{, }\KeywordTok{function}\NormalTok{() \{}
  \CommentTok{// Worker has disconnected}
\NormalTok{\};}
\end{Highlighting}
\end{Shaded}

\subsubsection{Event: `exit'}

\begin{itemize}
\item
  \texttt{code} \{Number\} the exit code, if it exited normally.
\item
  \texttt{signal} \{String\} the name of the signal (eg.
  \texttt{'SIGHUP'}) that caused the process to be killed.
\end{itemize}

Emitted by the individual worker instance, when the underlying child
process is terminated. See
\href{child\_process.html\#child\_process\_event\_exit}{child\_process
event: `exit'}.

\begin{Shaded}
\begin{Highlighting}[]
\KeywordTok{var} \NormalTok{worker = }\KeywordTok{cluster}\NormalTok{.}\FunctionTok{fork}\NormalTok{();}
\KeywordTok{worker}\NormalTok{.}\FunctionTok{on}\NormalTok{(}\CharTok{'exit'}\NormalTok{, }\KeywordTok{function}\NormalTok{(code, signal) \{}
  \KeywordTok{if}\NormalTok{( signal ) \{}
    \KeywordTok{console}\NormalTok{.}\FunctionTok{log}\NormalTok{(}\StringTok{"worker was killed by signal: "}\NormalTok{+signal);}
  \NormalTok{\} }\KeywordTok{else} \KeywordTok{if}\NormalTok{( code !== }\DecValTok{0} \NormalTok{) \{}
    \KeywordTok{console}\NormalTok{.}\FunctionTok{log}\NormalTok{(}\StringTok{"worker exited with error code: "}\NormalTok{+code);}
  \NormalTok{\} }\KeywordTok{else} \NormalTok{\{}
    \KeywordTok{console}\NormalTok{.}\FunctionTok{log}\NormalTok{(}\StringTok{"worker success!"}\NormalTok{);}
  \NormalTok{\}}
\NormalTok{\};}
\end{Highlighting}
\end{Shaded}

