\section{Cluster}\label{cluster}

\begin{Shaded}
\begin{Highlighting}[]
\NormalTok{Stability: }\DecValTok{2} \NormalTok{- Unstable}
\end{Highlighting}
\end{Shaded}

A single instance of Node runs in a single thread. To take advantage of
multi-core systems the user will sometimes want to launch a cluster of
Node processes to handle the load.

The cluster module allows you to easily create child processes that all
share server ports.

\begin{Shaded}
\begin{Highlighting}[]
\KeywordTok{var} \NormalTok{cluster = }\FunctionTok{require}\NormalTok{(}\StringTok{'cluster'}\NormalTok{);}
\KeywordTok{var} \NormalTok{http = }\FunctionTok{require}\NormalTok{(}\StringTok{'http'}\NormalTok{);}
\KeywordTok{var} \NormalTok{numCPUs = }\FunctionTok{require}\NormalTok{(}\StringTok{'os'}\NormalTok{).}\FunctionTok{cpus}\NormalTok{().}\FunctionTok{length}\NormalTok{;}

\KeywordTok{if} \NormalTok{(}\OtherTok{cluster}\NormalTok{.}\FunctionTok{isMaster}\NormalTok{) \{}
  \CommentTok{// Fork workers.}
  \KeywordTok{for} \NormalTok{(}\KeywordTok{var} \NormalTok{i = }\DecValTok{0}\NormalTok{; i < numCPUs; i++) \{}
    \OtherTok{cluster}\NormalTok{.}\FunctionTok{fork}\NormalTok{();}
  \NormalTok{\}}

  \OtherTok{cluster}\NormalTok{.}\FunctionTok{on}\NormalTok{(}\StringTok{'exit'}\NormalTok{, }\KeywordTok{function}\NormalTok{(worker, code, signal) \{}
    \OtherTok{console}\NormalTok{.}\FunctionTok{log}\NormalTok{(}\StringTok{'worker '} \NormalTok{+ }\OtherTok{worker}\NormalTok{.}\OtherTok{process}\NormalTok{.}\FunctionTok{pid} \NormalTok{+ }\StringTok{' died'}\NormalTok{);}
  \NormalTok{\});}
\NormalTok{\} }\KeywordTok{else} \NormalTok{\{}
  \CommentTok{// Workers can share any TCP connection}
  \CommentTok{// In this case its a HTTP server}
  \OtherTok{http}\NormalTok{.}\FunctionTok{createServer}\NormalTok{(}\KeywordTok{function}\NormalTok{(req, res) \{}
    \OtherTok{res}\NormalTok{.}\FunctionTok{writeHead}\NormalTok{(}\DecValTok{200}\NormalTok{);}
    \OtherTok{res}\NormalTok{.}\FunctionTok{end}\NormalTok{(}\StringTok{"hello world}\CharTok{\textbackslash{}n}\StringTok{"}\NormalTok{);}
  \NormalTok{\}).}\FunctionTok{listen}\NormalTok{(}\DecValTok{8000}\NormalTok{);}
\NormalTok{\}}
\end{Highlighting}
\end{Shaded}

Running node will now share port 8000 between the workers:

\begin{Shaded}
\begin{Highlighting}[]
\NormalTok{% NODE_DEBUG=cluster node }\OtherTok{server}\NormalTok{.}\FunctionTok{js}
\DecValTok{23521}\NormalTok{,Master Worker }\DecValTok{23524} \NormalTok{online}
\DecValTok{23521}\NormalTok{,Master Worker }\DecValTok{23526} \NormalTok{online}
\DecValTok{23521}\NormalTok{,Master Worker }\DecValTok{23523} \NormalTok{online}
\DecValTok{23521}\NormalTok{,Master Worker }\DecValTok{23528} \NormalTok{online}
\end{Highlighting}
\end{Shaded}

This feature was introduced recently, and may change in future versions.
Please try it out and provide feedback.

Also note that, on Windows, it is not yet possible to set up a named
pipe server in a worker.

\subsection{How It Works}\label{how-it-works}

The worker processes are spawned using the \texttt{child\_process.fork}
method, so that they can communicate with the parent via IPC and pass
server handles back and forth.

The cluster module supports two methods of distributing incoming
connections.

The first one (and the default one on all platforms except Windows), is
the round-robin approach, where the master process listens on a port,
accepts new connections and distributes them across the workers in a
round-robin fashion, with some built-in smarts to avoid overloading a
worker process.

The second approach is where the master process creates the listen
socket and sends it to interested workers. The workers then accept
incoming connections directly.

The second approach should, in theory, give the best performance. In
practice however, distribution tends to be very unbalanced due to
operating system scheduler vagaries. Loads have been observed where over
70\% of all connections ended up in just two processes, out of a total
of eight.

Because \texttt{server.listen()} hands off most of the work to the
master process, there are three cases where the behavior between a
normal node.js process and a cluster worker differs:

\begin{enumerate}
\def\labelenumi{\arabic{enumi}.}
\itemsep1pt\parskip0pt\parsep0pt
\item
  \texttt{server.listen(\{fd: 7\})} Because the message is passed to the
  master, file descriptor 7 \textbf{in the parent} will be listened on,
  and the handle passed to the worker, rather than listening to the
  worker's idea of what the number 7 file descriptor references.
\item
  \texttt{server.listen(handle)} Listening on handles explicitly will
  cause the worker to use the supplied handle, rather than talk to the
  master process. If the worker already has the handle, then it's
  presumed that you know what you are doing.
\item
  \texttt{server.listen(0)} Normally, this will cause servers to listen
  on a random port. However, in a cluster, each worker will receive the
  same ``random'' port each time they do \texttt{listen(0)}. In essence,
  the port is random the first time, but predictable thereafter. If you
  want to listen on a unique port, generate a port number based on the
  cluster worker ID.
\end{enumerate}

There is no routing logic in Node.js, or in your program, and no shared
state between the workers. Therefore, it is important to design your
program such that it does not rely too heavily on in-memory data objects
for things like sessions and login.

Because workers are all separate processes, they can be killed or
re-spawned depending on your program's needs, without affecting other
workers. As long as there are some workers still alive, the server will
continue to accept connections. Node does not automatically manage the
number of workers for you, however. It is your responsibility to manage
the worker pool for your application's needs.

\subsection{cluster.schedulingPolicy}\label{cluster.schedulingpolicy}

The scheduling policy, either \texttt{cluster.SCHED\_RR} for round-robin
or \texttt{cluster.SCHED\_NONE} to leave it to the operating system.
This is a global setting and effectively frozen once you spawn the first
worker or call \texttt{cluster.setupMaster()}, whatever comes first.

\texttt{SCHED\_RR} is the default on all operating systems except
Windows. Windows will change to \texttt{SCHED\_RR} once libuv is able to
effectively distribute IOCP handles without incurring a large
performance hit.

\texttt{cluster.schedulingPolicy} can also be set through the
\texttt{NODE\_CLUSTER\_SCHED\_POLICY} environment variable. Valid values
are \texttt{"rr"} and \texttt{"none"}.

\subsection{cluster.settings}\label{cluster.settings}

\begin{itemize}
\itemsep1pt\parskip0pt\parsep0pt
\item
  \{Object\}
\item
  \texttt{execArgv} \{Array\} list of string arguments passed to the
  node executable. (Default=\texttt{process.execArgv})
\item
  \texttt{exec} \{String\} file path to worker file.
  (Default=\texttt{process.argv{[}1{]}})
\item
  \texttt{args} \{Array\} string arguments passed to worker.
  (Default=\texttt{process.argv.slice(2)})
\item
  \texttt{silent} \{Boolean\} whether or not to send output to parent's
  stdio. (Default=\texttt{false})
\item
  \texttt{uid} \{Number\} Sets the user identity of the process. (See
  setuid(2).)
\item
  \texttt{gid} \{Number\} Sets the group identity of the process. (See
  setgid(2).)
\end{itemize}

After calling \texttt{.setupMaster()} (or \texttt{.fork()}) this
settings object will contain the settings, including the default values.

It is effectively frozen after being set, because
\texttt{.setupMaster()} can only be called once.

This object is not supposed to be changed or set manually, by you.

\subsection{cluster.isMaster}\label{cluster.ismaster}

\begin{itemize}
\itemsep1pt\parskip0pt\parsep0pt
\item
  \{Boolean\}
\end{itemize}

True if the process is a master. This is determined by the
\texttt{process.env.NODE\_UNIQUE\_ID}. If
\texttt{process.env.NODE\_UNIQUE\_ID} is undefined, then
\texttt{isMaster} is \texttt{true}.

\subsection{cluster.isWorker}\label{cluster.isworker}

\begin{itemize}
\itemsep1pt\parskip0pt\parsep0pt
\item
  \{Boolean\}
\end{itemize}

True if the process is not a master (it is the negation of
\texttt{cluster.isMaster}).

\subsection{Event: `fork'}\label{event-fork}

\begin{itemize}
\itemsep1pt\parskip0pt\parsep0pt
\item
  \texttt{worker} \{Worker object\}
\end{itemize}

When a new worker is forked the cluster module will emit a `fork' event.
This can be used to log worker activity, and create your own timeout.

\begin{Shaded}
\begin{Highlighting}[]
\KeywordTok{var} \NormalTok{timeouts = [];}
\KeywordTok{function} \FunctionTok{errorMsg}\NormalTok{() \{}
  \OtherTok{console}\NormalTok{.}\FunctionTok{error}\NormalTok{(}\StringTok{"Something must be wrong with the connection ..."}\NormalTok{);}
\NormalTok{\}}

\OtherTok{cluster}\NormalTok{.}\FunctionTok{on}\NormalTok{(}\StringTok{'fork'}\NormalTok{, }\KeywordTok{function}\NormalTok{(worker) \{}
  \NormalTok{timeouts[}\OtherTok{worker}\NormalTok{.}\FunctionTok{id}\NormalTok{] = }\FunctionTok{setTimeout}\NormalTok{(errorMsg, }\DecValTok{2000}\NormalTok{);}
\NormalTok{\});}
\OtherTok{cluster}\NormalTok{.}\FunctionTok{on}\NormalTok{(}\StringTok{'listening'}\NormalTok{, }\KeywordTok{function}\NormalTok{(worker, address) \{}
  \FunctionTok{clearTimeout}\NormalTok{(timeouts[}\OtherTok{worker}\NormalTok{.}\FunctionTok{id}\NormalTok{]);}
\NormalTok{\});}
\OtherTok{cluster}\NormalTok{.}\FunctionTok{on}\NormalTok{(}\StringTok{'exit'}\NormalTok{, }\KeywordTok{function}\NormalTok{(worker, code, signal) \{}
  \FunctionTok{clearTimeout}\NormalTok{(timeouts[}\OtherTok{worker}\NormalTok{.}\FunctionTok{id}\NormalTok{]);}
  \FunctionTok{errorMsg}\NormalTok{();}
\NormalTok{\});}
\end{Highlighting}
\end{Shaded}

\subsection{Event: `online'}\label{event-online}

\begin{itemize}
\itemsep1pt\parskip0pt\parsep0pt
\item
  \texttt{worker} \{Worker object\}
\end{itemize}

After forking a new worker, the worker should respond with an online
message. When the master receives an online message it will emit this
event. The difference between `fork' and `online' is that fork is
emitted when the master forks a worker, and `online' is emitted when the
worker is running.

\begin{Shaded}
\begin{Highlighting}[]
\OtherTok{cluster}\NormalTok{.}\FunctionTok{on}\NormalTok{(}\StringTok{'online'}\NormalTok{, }\KeywordTok{function}\NormalTok{(worker) \{}
  \OtherTok{console}\NormalTok{.}\FunctionTok{log}\NormalTok{(}\StringTok{"Yay, the worker responded after it was forked"}\NormalTok{);}
\NormalTok{\});}
\end{Highlighting}
\end{Shaded}

\subsection{Event: `listening'}\label{event-listening}

\begin{itemize}
\itemsep1pt\parskip0pt\parsep0pt
\item
  \texttt{worker} \{Worker object\}
\item
  \texttt{address} \{Object\}
\end{itemize}

After calling \texttt{listen()} from a worker, when the `listening'
event is emitted on the server, a listening event will also be emitted
on \texttt{cluster} in the master.

The event handler is executed with two arguments, the \texttt{worker}
contains the worker object and the \texttt{address} object contains the
following connection properties: \texttt{address}, \texttt{port} and
\texttt{addressType}. This is very useful if the worker is listening on
more than one address.

\begin{Shaded}
\begin{Highlighting}[]
\OtherTok{cluster}\NormalTok{.}\FunctionTok{on}\NormalTok{(}\StringTok{'listening'}\NormalTok{, }\KeywordTok{function}\NormalTok{(worker, address) \{}
  \OtherTok{console}\NormalTok{.}\FunctionTok{log}\NormalTok{(}\StringTok{"A worker is now connected to "} \NormalTok{+ }\OtherTok{address}\NormalTok{.}\FunctionTok{address} \NormalTok{+ }\StringTok{":"} \NormalTok{+ }\OtherTok{address}\NormalTok{.}\FunctionTok{port}\NormalTok{);}
\NormalTok{\});}
\end{Highlighting}
\end{Shaded}

The \texttt{addressType} is one of:

\begin{itemize}
\itemsep1pt\parskip0pt\parsep0pt
\item
  \texttt{4} (TCPv4)
\item
  \texttt{6} (TCPv6)
\item
  \texttt{-1} (unix domain socket)
\item
  \texttt{"udp4"} or \texttt{"udp6"} (UDP v4 or v6)
\end{itemize}

\subsection{Event: `disconnect'}\label{event-disconnect}

\begin{itemize}
\itemsep1pt\parskip0pt\parsep0pt
\item
  \texttt{worker} \{Worker object\}
\end{itemize}

Emitted after the worker IPC channel has disconnected. This can occur
when a worker exits gracefully, is killed, or is disconnected manually
(such as with worker.disconnect()).

There may be a delay between the \texttt{disconnect} and \texttt{exit}
events. These events can be used to detect if the process is stuck in a
cleanup or if there are long-living connections.

\begin{Shaded}
\begin{Highlighting}[]
\OtherTok{cluster}\NormalTok{.}\FunctionTok{on}\NormalTok{(}\StringTok{'disconnect'}\NormalTok{, }\KeywordTok{function}\NormalTok{(worker) \{}
  \OtherTok{console}\NormalTok{.}\FunctionTok{log}\NormalTok{(}\StringTok{'The worker #'} \NormalTok{+ }\OtherTok{worker}\NormalTok{.}\FunctionTok{id} \NormalTok{+ }\StringTok{' has disconnected'}\NormalTok{);}
\NormalTok{\});}
\end{Highlighting}
\end{Shaded}

\subsection{Event: `exit'}\label{event-exit}

\begin{itemize}
\itemsep1pt\parskip0pt\parsep0pt
\item
  \texttt{worker} \{Worker object\}
\item
  \texttt{code} \{Number\} the exit code, if it exited normally.
\item
  \texttt{signal} \{String\} the name of the signal (eg.
  \texttt{'SIGHUP'}) that caused the process to be killed.
\end{itemize}

When any of the workers die the cluster module will emit the `exit'
event.

This can be used to restart the worker by calling \texttt{.fork()}
again.

\begin{Shaded}
\begin{Highlighting}[]
\OtherTok{cluster}\NormalTok{.}\FunctionTok{on}\NormalTok{(}\StringTok{'exit'}\NormalTok{, }\KeywordTok{function}\NormalTok{(worker, code, signal) \{}
  \OtherTok{console}\NormalTok{.}\FunctionTok{log}\NormalTok{(}\StringTok{'worker %d died (%s). restarting...'}\NormalTok{,}
    \OtherTok{worker}\NormalTok{.}\OtherTok{process}\NormalTok{.}\FunctionTok{pid}\NormalTok{, signal || code);}
  \OtherTok{cluster}\NormalTok{.}\FunctionTok{fork}\NormalTok{();}
\NormalTok{\});}
\end{Highlighting}
\end{Shaded}

See \href{child_process.html\#child_process_event_exit}{child\_process
event: `exit'}.

\subsection{Event: `setup'}\label{event-setup}

\begin{itemize}
\itemsep1pt\parskip0pt\parsep0pt
\item
  \texttt{settings} \{Object\}
\end{itemize}

Emitted every time \texttt{.setupMaster()} is called.

The \texttt{settings} object is the \texttt{cluster.settings} object at
the time \texttt{.setupMaster()} was called and is advisory only, since
multiple calls to \texttt{.setupMaster()} can be made in a single tick.

If accuracy is important, use \texttt{cluster.settings}.

\subsection{cluster.setupMaster({[}settings{]})}\label{cluster.setupmastersettings}

\begin{itemize}
\itemsep1pt\parskip0pt\parsep0pt
\item
  \texttt{settings} \{Object\}
\item
  \texttt{exec} \{String\} file path to worker file.
  (Default=\texttt{process.argv{[}1{]}})
\item
  \texttt{args} \{Array\} string arguments passed to worker.
  (Default=\texttt{process.argv.slice(2)})
\item
  \texttt{silent} \{Boolean\} whether or not to send output to parent's
  stdio. (Default=\texttt{false})
\end{itemize}

\texttt{setupMaster} is used to change the default `fork' behavior. Once
called, the settings will be present in \texttt{cluster.settings}.

Note that:

\begin{itemize}
\itemsep1pt\parskip0pt\parsep0pt
\item
  any settings changes only affect future calls to \texttt{.fork()} and
  have no effect on workers that are already running
\item
  The \emph{only} attribute of a worker that cannot be set via
  \texttt{.setupMaster()} is the \texttt{env} passed to \texttt{.fork()}
\item
  the defaults above apply to the first call only, the defaults for
  later calls is the current value at the time of
  \texttt{cluster.setupMaster()} is called
\end{itemize}

Example:

\begin{Shaded}
\begin{Highlighting}[]
\KeywordTok{var} \NormalTok{cluster = }\FunctionTok{require}\NormalTok{(}\StringTok{'cluster'}\NormalTok{);}
\OtherTok{cluster}\NormalTok{.}\FunctionTok{setupMaster}\NormalTok{(\{}
  \DataTypeTok{exec}\NormalTok{: }\StringTok{'worker.js'}\NormalTok{,}
  \DataTypeTok{args}\NormalTok{: [}\StringTok{'--use'}\NormalTok{, }\StringTok{'https'}\NormalTok{],}
  \DataTypeTok{silent}\NormalTok{: }\KeywordTok{true}
\NormalTok{\});}
\OtherTok{cluster}\NormalTok{.}\FunctionTok{fork}\NormalTok{(); }\CommentTok{// https worker}
\OtherTok{cluster}\NormalTok{.}\FunctionTok{setupMaster}\NormalTok{(\{}
  \DataTypeTok{args}\NormalTok{: [}\StringTok{'--use'}\NormalTok{, }\StringTok{'http'}\NormalTok{]}
\NormalTok{\});}
\OtherTok{cluster}\NormalTok{.}\FunctionTok{fork}\NormalTok{(); }\CommentTok{// http worker}
\end{Highlighting}
\end{Shaded}

This can only be called from the master process.

\subsection{cluster.fork({[}env{]})}\label{cluster.forkenv}

\begin{itemize}
\itemsep1pt\parskip0pt\parsep0pt
\item
  \texttt{env} \{Object\} Key/value pairs to add to worker process
  environment.
\item
  return \{Worker object\}
\end{itemize}

Spawn a new worker process.

This can only be called from the master process.

\subsection{cluster.disconnect({[}callback{]})}\label{cluster.disconnectcallback}

\begin{itemize}
\itemsep1pt\parskip0pt\parsep0pt
\item
  \texttt{callback} \{Function\} called when all workers are
  disconnected and handles are closed
\end{itemize}

Calls \texttt{.disconnect()} on each worker in \texttt{cluster.workers}.

When they are disconnected all internal handles will be closed, allowing
the master process to die gracefully if no other event is waiting.

The method takes an optional callback argument which will be called when
finished.

This can only be called from the master process.

\subsection{cluster.worker}\label{cluster.worker}

\begin{itemize}
\itemsep1pt\parskip0pt\parsep0pt
\item
  \{Object\}
\end{itemize}

A reference to the current worker object. Not available in the master
process.

\begin{Shaded}
\begin{Highlighting}[]
\KeywordTok{var} \NormalTok{cluster = }\FunctionTok{require}\NormalTok{(}\StringTok{'cluster'}\NormalTok{);}

\KeywordTok{if} \NormalTok{(}\OtherTok{cluster}\NormalTok{.}\FunctionTok{isMaster}\NormalTok{) \{}
  \OtherTok{console}\NormalTok{.}\FunctionTok{log}\NormalTok{(}\StringTok{'I am master'}\NormalTok{);}
  \OtherTok{cluster}\NormalTok{.}\FunctionTok{fork}\NormalTok{();}
  \OtherTok{cluster}\NormalTok{.}\FunctionTok{fork}\NormalTok{();}
\NormalTok{\} }\KeywordTok{else} \KeywordTok{if} \NormalTok{(}\OtherTok{cluster}\NormalTok{.}\FunctionTok{isWorker}\NormalTok{) \{}
  \OtherTok{console}\NormalTok{.}\FunctionTok{log}\NormalTok{(}\StringTok{'I am worker #'} \NormalTok{+ }\OtherTok{cluster}\NormalTok{.}\OtherTok{worker}\NormalTok{.}\FunctionTok{id}\NormalTok{);}
\NormalTok{\}}
\end{Highlighting}
\end{Shaded}

\subsection{cluster.workers}\label{cluster.workers}

\begin{itemize}
\itemsep1pt\parskip0pt\parsep0pt
\item
  \{Object\}
\end{itemize}

A hash that stores the active worker objects, keyed by \texttt{id}
field. Makes it easy to loop through all the workers. It is only
available in the master process.

A worker is removed from cluster.workers after the worker has
disconnected \emph{and} exited. The order between these two events
cannot be determined in advance. However, it is guaranteed that the
removal from the cluster.workers list happens before last
\texttt{'disconnect'} or \texttt{'exit'} event is emitted.

\begin{Shaded}
\begin{Highlighting}[]
\CommentTok{// Go through all workers}
\KeywordTok{function} \FunctionTok{eachWorker}\NormalTok{(callback) \{}
  \KeywordTok{for} \NormalTok{(}\KeywordTok{var} \NormalTok{id }\KeywordTok{in} \OtherTok{cluster}\NormalTok{.}\FunctionTok{workers}\NormalTok{) \{}
    \FunctionTok{callback}\NormalTok{(}\OtherTok{cluster}\NormalTok{.}\FunctionTok{workers}\NormalTok{[id]);}
  \NormalTok{\}}
\NormalTok{\}}
\FunctionTok{eachWorker}\NormalTok{(}\KeywordTok{function}\NormalTok{(worker) \{}
  \OtherTok{worker}\NormalTok{.}\FunctionTok{send}\NormalTok{(}\StringTok{'big announcement to all workers'}\NormalTok{);}
\NormalTok{\});}
\end{Highlighting}
\end{Shaded}

Should you wish to reference a worker over a communication channel,
using the worker's unique id is the easiest way to find the worker.

\begin{Shaded}
\begin{Highlighting}[]
\OtherTok{socket}\NormalTok{.}\FunctionTok{on}\NormalTok{(}\StringTok{'data'}\NormalTok{, }\KeywordTok{function}\NormalTok{(id) \{}
  \KeywordTok{var} \NormalTok{worker = }\OtherTok{cluster}\NormalTok{.}\FunctionTok{workers}\NormalTok{[id];}
\NormalTok{\});}
\end{Highlighting}
\end{Shaded}

\subsection{Class: Worker}\label{class-worker}

A Worker object contains all public information and method about a
worker. In the master it can be obtained using \texttt{cluster.workers}.
In a worker it can be obtained using \texttt{cluster.worker}.

\subsubsection{worker.id}\label{worker.id}

\begin{itemize}
\itemsep1pt\parskip0pt\parsep0pt
\item
  \{String\}
\end{itemize}

Each new worker is given its own unique id, this id is stored in the
\texttt{id}.

While a worker is alive, this is the key that indexes it in
cluster.workers

\subsubsection{worker.process}\label{worker.process}

\begin{itemize}
\itemsep1pt\parskip0pt\parsep0pt
\item
  \{ChildProcess object\}
\end{itemize}

All workers are created using \texttt{child\_process.fork()}, the
returned object from this function is stored as \texttt{.process}. In a
worker, the global \texttt{process} is stored.

See:
\href{child_process.html\#child_process_child_process_fork_modulepath_args_options}{Child
Process module}

Note that workers will call \texttt{process.exit(0)} if the
\texttt{'disconnect'} event occurs on \texttt{process} and
\texttt{.suicide} is not \texttt{true}. This protects against accidental
disconnection.

\subsubsection{worker.suicide}\label{worker.suicide}

\begin{itemize}
\itemsep1pt\parskip0pt\parsep0pt
\item
  \{Boolean\}
\end{itemize}

Set by calling \texttt{.kill()} or \texttt{.disconnect()}, until then it
is \texttt{undefined}.

The boolean \texttt{worker.suicide} lets you distinguish between
voluntary and accidental exit, the master may choose not to respawn a
worker based on this value.

\begin{Shaded}
\begin{Highlighting}[]
\OtherTok{cluster}\NormalTok{.}\FunctionTok{on}\NormalTok{(}\StringTok{'exit'}\NormalTok{, }\KeywordTok{function}\NormalTok{(worker, code, signal) \{}
  \KeywordTok{if} \NormalTok{(}\OtherTok{worker}\NormalTok{.}\FunctionTok{suicide} \NormalTok{=== }\KeywordTok{true}\NormalTok{) \{}
    \OtherTok{console}\NormalTok{.}\FunctionTok{log}\NormalTok{(}\StringTok{'Oh, it was just suicide}\CharTok{\textbackslash{}'}\StringTok{ – no need to worry'}\NormalTok{).}
  \NormalTok{\}}
\NormalTok{\});}

\CommentTok{// kill worker}
\OtherTok{worker}\NormalTok{.}\FunctionTok{kill}\NormalTok{();}
\end{Highlighting}
\end{Shaded}

\subsubsection{worker.send(message,
{[}sendHandle{]})}\label{worker.sendmessage-sendhandle}

\begin{itemize}
\itemsep1pt\parskip0pt\parsep0pt
\item
  \texttt{message} \{Object\}
\item
  \texttt{sendHandle} \{Handle object\}
\end{itemize}

This function is equal to the send methods provided by
\texttt{child\_process.fork()}. In the master you should use this
function to send a message to a specific worker.

In a worker you can also use \texttt{process.send(message)}, it is the
same function.

This example will echo back all messages from the master:

\begin{Shaded}
\begin{Highlighting}[]
\KeywordTok{if} \NormalTok{(}\OtherTok{cluster}\NormalTok{.}\FunctionTok{isMaster}\NormalTok{) \{}
  \KeywordTok{var} \NormalTok{worker = }\OtherTok{cluster}\NormalTok{.}\FunctionTok{fork}\NormalTok{();}
  \OtherTok{worker}\NormalTok{.}\FunctionTok{send}\NormalTok{(}\StringTok{'hi there'}\NormalTok{);}

\NormalTok{\} }\KeywordTok{else} \KeywordTok{if} \NormalTok{(}\OtherTok{cluster}\NormalTok{.}\FunctionTok{isWorker}\NormalTok{) \{}
  \OtherTok{process}\NormalTok{.}\FunctionTok{on}\NormalTok{(}\StringTok{'message'}\NormalTok{, }\KeywordTok{function}\NormalTok{(msg) \{}
    \OtherTok{process}\NormalTok{.}\FunctionTok{send}\NormalTok{(msg);}
  \NormalTok{\});}
\NormalTok{\}}
\end{Highlighting}
\end{Shaded}

\subsubsection{worker.kill({[}signal=`SIGTERM'{]})}\label{worker.killsignalsigterm}

\begin{itemize}
\itemsep1pt\parskip0pt\parsep0pt
\item
  \texttt{signal} \{String\} Name of the kill signal to send to the
  worker process.
\end{itemize}

This function will kill the worker. In the master, it does this by
disconnecting the \texttt{worker.process}, and once disconnected,
killing with \texttt{signal}. In the worker, it does it by disconnecting
the channel, and then exiting with code \texttt{0}.

Causes \texttt{.suicide} to be set.

This method is aliased as \texttt{worker.destroy()} for backwards
compatibility.

Note that in a worker, \texttt{process.kill()} exists, but it is not
this function, it is
\href{process.html\#process_process_kill_pid_signal}{kill}.

\subsubsection{worker.disconnect()}\label{worker.disconnect}

In a worker, this function will close all servers, wait for the `close'
event on those servers, and then disconnect the IPC channel.

In the master, an internal message is sent to the worker causing it to
call \texttt{.disconnect()} on itself.

Causes \texttt{.suicide} to be set.

Note that after a server is closed, it will no longer accept new
connections, but connections may be accepted by any other listening
worker. Existing connections will be allowed to close as usual. When no
more connections exist, see
\href{net.html\#net_event_close}{server.close()}, the IPC channel to the
worker will close allowing it to die gracefully.

The above applies \emph{only} to server connections, client connections
are not automatically closed by workers, and disconnect does not wait
for them to close before exiting.

Note that in a worker, \texttt{process.disconnect} exists, but it is not
this function, it is
\href{child_process.html\#child_process_child_disconnect}{disconnect}.

Because long living server connections may block workers from
disconnecting, it may be useful to send a message, so application
specific actions may be taken to close them. It also may be useful to
implement a timeout, killing a worker if the \texttt{disconnect} event
has not been emitted after some time.

\begin{Shaded}
\begin{Highlighting}[]
\KeywordTok{if} \NormalTok{(}\OtherTok{cluster}\NormalTok{.}\FunctionTok{isMaster}\NormalTok{) \{}
  \KeywordTok{var} \NormalTok{worker = }\OtherTok{cluster}\NormalTok{.}\FunctionTok{fork}\NormalTok{();}
  \KeywordTok{var} \NormalTok{timeout;}

  \OtherTok{worker}\NormalTok{.}\FunctionTok{on}\NormalTok{(}\StringTok{'listening'}\NormalTok{, }\KeywordTok{function}\NormalTok{(address) \{}
    \OtherTok{worker}\NormalTok{.}\FunctionTok{send}\NormalTok{(}\StringTok{'shutdown'}\NormalTok{);}
    \OtherTok{worker}\NormalTok{.}\FunctionTok{disconnect}\NormalTok{();}
    \NormalTok{timeout = }\FunctionTok{setTimeout}\NormalTok{(}\KeywordTok{function}\NormalTok{() \{}
      \OtherTok{worker}\NormalTok{.}\FunctionTok{kill}\NormalTok{();}
    \NormalTok{\}, }\DecValTok{2000}\NormalTok{);}
  \NormalTok{\});}

  \OtherTok{worker}\NormalTok{.}\FunctionTok{on}\NormalTok{(}\StringTok{'disconnect'}\NormalTok{, }\KeywordTok{function}\NormalTok{() \{}
    \FunctionTok{clearTimeout}\NormalTok{(timeout);}
  \NormalTok{\});}

\NormalTok{\} }\KeywordTok{else} \KeywordTok{if} \NormalTok{(}\OtherTok{cluster}\NormalTok{.}\FunctionTok{isWorker}\NormalTok{) \{}
  \KeywordTok{var} \NormalTok{net = }\FunctionTok{require}\NormalTok{(}\StringTok{'net'}\NormalTok{);}
  \KeywordTok{var} \NormalTok{server = }\OtherTok{net}\NormalTok{.}\FunctionTok{createServer}\NormalTok{(}\KeywordTok{function}\NormalTok{(socket) \{}
    \CommentTok{// connections never end}
  \NormalTok{\});}

  \OtherTok{server}\NormalTok{.}\FunctionTok{listen}\NormalTok{(}\DecValTok{8000}\NormalTok{);}

  \OtherTok{process}\NormalTok{.}\FunctionTok{on}\NormalTok{(}\StringTok{'message'}\NormalTok{, }\KeywordTok{function}\NormalTok{(msg) \{}
    \KeywordTok{if}\NormalTok{(msg === }\StringTok{'shutdown'}\NormalTok{) \{}
      \CommentTok{// initiate graceful close of any connections to server}
    \NormalTok{\}}
  \NormalTok{\});}
\NormalTok{\}}
\end{Highlighting}
\end{Shaded}

\subsubsection{worker.isDead()}\label{worker.isdead}

This function returns \texttt{true} if the worker's process has
terminated (either because of exiting or being signaled). Otherwise, it
returns \texttt{false}.

\subsubsection{worker.isConnected()}\label{worker.isconnected}

This function returns \texttt{true} if the worker is connected to its
master via its IPC channel, \texttt{false} otherwise. A worker is
connected to its master after it's been created. It is disconnected
after the \texttt{disconnect} event is emitted.

\subsubsection{Event: `message'}\label{event-message}

\begin{itemize}
\itemsep1pt\parskip0pt\parsep0pt
\item
  \texttt{message} \{Object\}
\end{itemize}

This event is the same as the one provided by
\texttt{child\_process.fork()}.

In a worker you can also use \texttt{process.on('message')}.

As an example, here is a cluster that keeps count of the number of
requests in the master process using the message system:

\begin{Shaded}
\begin{Highlighting}[]
\KeywordTok{var} \NormalTok{cluster = }\FunctionTok{require}\NormalTok{(}\StringTok{'cluster'}\NormalTok{);}
\KeywordTok{var} \NormalTok{http = }\FunctionTok{require}\NormalTok{(}\StringTok{'http'}\NormalTok{);}

\KeywordTok{if} \NormalTok{(}\OtherTok{cluster}\NormalTok{.}\FunctionTok{isMaster}\NormalTok{) \{}

  \CommentTok{// Keep track of http requests}
  \KeywordTok{var} \NormalTok{numReqs = }\DecValTok{0}\NormalTok{;}
  \FunctionTok{setInterval}\NormalTok{(}\KeywordTok{function}\NormalTok{() \{}
    \OtherTok{console}\NormalTok{.}\FunctionTok{log}\NormalTok{(}\StringTok{"numReqs ="}\NormalTok{, numReqs);}
  \NormalTok{\}, }\DecValTok{1000}\NormalTok{);}

  \CommentTok{// Count requestes}
  \KeywordTok{function} \FunctionTok{messageHandler}\NormalTok{(msg) \{}
    \KeywordTok{if} \NormalTok{(}\OtherTok{msg}\NormalTok{.}\FunctionTok{cmd} \NormalTok{&& }\OtherTok{msg}\NormalTok{.}\FunctionTok{cmd} \NormalTok{== }\StringTok{'notifyRequest'}\NormalTok{) \{}
      \NormalTok{numReqs += }\DecValTok{1}\NormalTok{;}
    \NormalTok{\}}
  \NormalTok{\}}

  \CommentTok{// Start workers and listen for messages containing notifyRequest}
  \KeywordTok{var} \NormalTok{numCPUs = }\FunctionTok{require}\NormalTok{(}\StringTok{'os'}\NormalTok{).}\FunctionTok{cpus}\NormalTok{().}\FunctionTok{length}\NormalTok{;}
  \KeywordTok{for} \NormalTok{(}\KeywordTok{var} \NormalTok{i = }\DecValTok{0}\NormalTok{; i < numCPUs; i++) \{}
    \OtherTok{cluster}\NormalTok{.}\FunctionTok{fork}\NormalTok{();}
  \NormalTok{\}}

  \OtherTok{Object}\NormalTok{.}\FunctionTok{keys}\NormalTok{(}\OtherTok{cluster}\NormalTok{.}\FunctionTok{workers}\NormalTok{).}\FunctionTok{forEach}\NormalTok{(}\KeywordTok{function}\NormalTok{(id) \{}
    \OtherTok{cluster}\NormalTok{.}\FunctionTok{workers}\NormalTok{[id].}\FunctionTok{on}\NormalTok{(}\StringTok{'message'}\NormalTok{, messageHandler);}
  \NormalTok{\});}

\NormalTok{\} }\KeywordTok{else} \NormalTok{\{}

  \CommentTok{// Worker processes have a http server.}
  \OtherTok{http}\NormalTok{.}\FunctionTok{Server}\NormalTok{(}\KeywordTok{function}\NormalTok{(req, res) \{}
    \OtherTok{res}\NormalTok{.}\FunctionTok{writeHead}\NormalTok{(}\DecValTok{200}\NormalTok{);}
    \OtherTok{res}\NormalTok{.}\FunctionTok{end}\NormalTok{(}\StringTok{"hello world}\CharTok{\textbackslash{}n}\StringTok{"}\NormalTok{);}

    \CommentTok{// notify master about the request}
    \OtherTok{process}\NormalTok{.}\FunctionTok{send}\NormalTok{(\{ }\DataTypeTok{cmd}\NormalTok{: }\StringTok{'notifyRequest'} \NormalTok{\});}
  \NormalTok{\}).}\FunctionTok{listen}\NormalTok{(}\DecValTok{8000}\NormalTok{);}
\NormalTok{\}}
\end{Highlighting}
\end{Shaded}

\subsubsection{Event: `online'}\label{event-online-1}

Similar to the \texttt{cluster.on('online')} event, but specific to this
worker.

\begin{Shaded}
\begin{Highlighting}[]
\OtherTok{cluster}\NormalTok{.}\FunctionTok{fork}\NormalTok{().}\FunctionTok{on}\NormalTok{(}\StringTok{'online'}\NormalTok{, }\KeywordTok{function}\NormalTok{() \{}
  \CommentTok{// Worker is online}
\NormalTok{\});}
\end{Highlighting}
\end{Shaded}

It is not emitted in the worker.

\subsubsection{Event: `listening'}\label{event-listening-1}

\begin{itemize}
\itemsep1pt\parskip0pt\parsep0pt
\item
  \texttt{address} \{Object\}
\end{itemize}

Similar to the \texttt{cluster.on('listening')} event, but specific to
this worker.

\begin{Shaded}
\begin{Highlighting}[]
\OtherTok{cluster}\NormalTok{.}\FunctionTok{fork}\NormalTok{().}\FunctionTok{on}\NormalTok{(}\StringTok{'listening'}\NormalTok{, }\KeywordTok{function}\NormalTok{(address) \{}
  \CommentTok{// Worker is listening}
\NormalTok{\});}
\end{Highlighting}
\end{Shaded}

It is not emitted in the worker.

\subsubsection{Event: `disconnect'}\label{event-disconnect-1}

Similar to the \texttt{cluster.on('disconnect')} event, but specfic to
this worker.

\begin{Shaded}
\begin{Highlighting}[]
\OtherTok{cluster}\NormalTok{.}\FunctionTok{fork}\NormalTok{().}\FunctionTok{on}\NormalTok{(}\StringTok{'disconnect'}\NormalTok{, }\KeywordTok{function}\NormalTok{() \{}
  \CommentTok{// Worker has disconnected}
\NormalTok{\});}
\end{Highlighting}
\end{Shaded}

\subsubsection{Event: `exit'}\label{event-exit-1}

\begin{itemize}
\itemsep1pt\parskip0pt\parsep0pt
\item
  \texttt{code} \{Number\} the exit code, if it exited normally.
\item
  \texttt{signal} \{String\} the name of the signal (eg.
  \texttt{'SIGHUP'}) that caused the process to be killed.
\end{itemize}

Similar to the \texttt{cluster.on('exit')} event, but specific to this
worker.

\begin{Shaded}
\begin{Highlighting}[]
\KeywordTok{var} \NormalTok{worker = }\OtherTok{cluster}\NormalTok{.}\FunctionTok{fork}\NormalTok{();}
\OtherTok{worker}\NormalTok{.}\FunctionTok{on}\NormalTok{(}\StringTok{'exit'}\NormalTok{, }\KeywordTok{function}\NormalTok{(code, signal) \{}
  \KeywordTok{if}\NormalTok{( signal ) \{}
    \OtherTok{console}\NormalTok{.}\FunctionTok{log}\NormalTok{(}\StringTok{"worker was killed by signal: "}\NormalTok{+signal);}
  \NormalTok{\} }\KeywordTok{else} \KeywordTok{if}\NormalTok{( code !== }\DecValTok{0} \NormalTok{) \{}
    \OtherTok{console}\NormalTok{.}\FunctionTok{log}\NormalTok{(}\StringTok{"worker exited with error code: "}\NormalTok{+code);}
  \NormalTok{\} }\KeywordTok{else} \NormalTok{\{}
    \OtherTok{console}\NormalTok{.}\FunctionTok{log}\NormalTok{(}\StringTok{"worker success!"}\NormalTok{);}
  \NormalTok{\}}
\NormalTok{\});}
\end{Highlighting}
\end{Shaded}

\subsubsection{Event: `error'}\label{event-error}

This event is the same as the one provided by
\texttt{child\_process.fork()}.

In a worker you can also use \texttt{process.on('error')}.
