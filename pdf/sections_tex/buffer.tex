\section{Buffer}

\begin{Shaded}
\begin{Highlighting}[]
\DataTypeTok{Stability}\NormalTok{: }\DecValTok{3} \NormalTok{- Stable}
\end{Highlighting}
\end{Shaded}

Pure JavaScript is Unicode friendly but not nice to binary data. When
dealing with TCP streams or the file system, it's necessary to handle
octet streams. Node has several strategies for manipulating, creating,
and consuming octet streams.

Raw data is stored in instances of the \texttt{Buffer} class. A
\texttt{Buffer} is similar to an array of integers but corresponds to a
raw memory allocation outside the V8 heap. A \texttt{Buffer} cannot be
resized.

The \texttt{Buffer} class is a global, making it very rare that one
would need to ever \texttt{require('buffer')}.

Converting between Buffers and JavaScript string objects requires an
explicit encoding method. Here are the different string encodings.

\begin{itemize}
\item
  \texttt{'ascii'} - for 7 bit ASCII data only. This encoding method is
  very fast, and will strip the high bit if set. Note that this encoding
  converts a null character (\texttt{'\textbackslash{}0'} or
  \texttt{'\textbackslash{}u0000'}) into \texttt{0x20} (character code
  of a space). If you want to convert a null character into
  \texttt{0x00}, you should use \texttt{'utf8'}.
\item
  \texttt{'utf8'} - Multibyte encoded Unicode characters. Many web pages
  and other document formats use UTF-8.
\item
  \texttt{'utf16le'} - 2 or 4 bytes, little endian encoded Unicode
  characters. Surrogate pairs (U+10000 to U+10FFFF) are supported.
\item
  \texttt{'ucs2'} - Alias of \texttt{'utf16le'}.
\item
  \texttt{'base64'} - Base64 string encoding.
\item
  \texttt{'binary'} - A way of encoding raw binary data into strings by
  using only the first 8 bits of each character. This encoding method is
  deprecated and should be avoided in favor of \texttt{Buffer} objects
  where possible. This encoding will be removed in future versions of
  Node.
\item
  \texttt{'hex'} - Encode each byte as two hexadecimal characters.
\end{itemize}

A \texttt{Buffer} object can also be used with typed arrays. The buffer
object is cloned to an \texttt{ArrayBuffer} that is used as the backing
store for the typed array. The memory of the buffer and the
\texttt{ArrayBuffer} is not shared.

NOTE: Node.js v0.8 simply retained a reference to the buffer in
\texttt{array.buffer} instead of cloning it.

While more efficient, it introduces subtle incompatibilities with the
typed arrays specification. \texttt{ArrayBuffer\#slice()} makes a copy
of the slice while \texttt{Buffer\#slice()} creates a view.

\subsection{Class: Buffer}

The Buffer class is a global type for dealing with binary data directly.
It can be constructed in a variety of ways.

\subsubsection{new Buffer(size)}

\begin{itemize}
\item
  \texttt{size} Number
\end{itemize}

Allocates a new buffer of \texttt{size} octets.

\subsubsection{new Buffer(array)}

\begin{itemize}
\item
  \texttt{array} Array
\end{itemize}

Allocates a new buffer using an \texttt{array} of octets.

\subsubsection{new Buffer(str, {[}encoding{]})}

\begin{itemize}
\item
  \texttt{str} String - string to encode.
\item
  \texttt{encoding} String - encoding to use, Optional.
\end{itemize}

Allocates a new buffer containing the given \texttt{str}.
\texttt{encoding} defaults to \texttt{'utf8'}.

\subsubsection{Class Method: Buffer.isEncoding(encoding)}

\begin{itemize}
\item
  \texttt{encoding} \{String\} The encoding string to test
\end{itemize}

Returns true if the \texttt{encoding} is a valid encoding argument, or
false otherwise.

\subsubsection{buf.write(string, {[}offset{]}, {[}length{]},
{[}encoding{]})}

\begin{itemize}
\item
  \texttt{string} String - data to be written to buffer
\item
  \texttt{offset} Number, Optional, Default: 0
\item
  \texttt{length} Number, Optional, Default:
  \texttt{buffer.length - offset}
\item
  \texttt{encoding} String, Optional, Default: `utf8'
\end{itemize}

Writes \texttt{string} to the buffer at \texttt{offset} using the given
encoding. \texttt{offset} defaults to \texttt{0}, \texttt{encoding}
defaults to \texttt{'utf8'}. \texttt{length} is the number of bytes to
write. Returns number of octets written. If \texttt{buffer} did not
contain enough space to fit the entire string, it will write a partial
amount of the string. \texttt{length} defaults to
\texttt{buffer.length - offset}. The method will not write partial
characters.

\begin{Shaded}
\begin{Highlighting}[]
\NormalTok{buf = }\KeywordTok{new} \NormalTok{Buffer(}\DecValTok{256}\NormalTok{);}
\NormalTok{len = }\KeywordTok{buf}\NormalTok{.}\FunctionTok{write}\NormalTok{(}\CharTok{'\textbackslash{}u00bd + \textbackslash{}u00bc = \textbackslash{}u00be'}\NormalTok{, }\DecValTok{0}\NormalTok{);}
\KeywordTok{console}\NormalTok{.}\FunctionTok{log}\NormalTok{(len + }\StringTok{" bytes: "} \NormalTok{+ }\KeywordTok{buf}\NormalTok{.}\FunctionTok{toString}\NormalTok{(}\CharTok{'utf8'}\NormalTok{, }\DecValTok{0}\NormalTok{, len));}
\end{Highlighting}
\end{Shaded}

The number of characters written (which may be different than the number
of bytes written) is set in \texttt{Buffer.\_charsWritten} and will be
overwritten the next time \texttt{buf.write()} is called.

\subsubsection{buf.toString({[}encoding{]}, {[}start{]}, {[}end{]})}

\begin{itemize}
\item
  \texttt{encoding} String, Optional, Default: `utf8'
\item
  \texttt{start} Number, Optional, Default: 0
\item
  \texttt{end} Number, Optional, Default: \texttt{buffer.length}
\end{itemize}

Decodes and returns a string from buffer data encoded with
\texttt{encoding} (defaults to \texttt{'utf8'}) beginning at
\texttt{start} (defaults to \texttt{0}) and ending at \texttt{end}
(defaults to \texttt{buffer.length}).

See \texttt{buffer.write()} example, above.

\subsubsection{buf.toJSON()}

Returns a JSON-representation of the Buffer instance, which is identical
to the output for JSON Arrays. \texttt{JSON.stringify} implicitly calls
this function when stringifying a Buffer instance.

Example:

\begin{Shaded}
\begin{Highlighting}[]
\KeywordTok{var} \NormalTok{buf = }\KeywordTok{new} \NormalTok{Buffer(}\CharTok{'test'}\NormalTok{);}
\KeywordTok{var} \NormalTok{json = }\KeywordTok{JSON}\NormalTok{.}\FunctionTok{stringify}\NormalTok{(buf);}

\KeywordTok{console}\NormalTok{.}\FunctionTok{log}\NormalTok{(json);}
\CommentTok{// '[116,101,115,116]'}

\KeywordTok{var} \NormalTok{copy = }\KeywordTok{new} \NormalTok{Buffer(}\KeywordTok{JSON}\NormalTok{.}\FunctionTok{parse}\NormalTok{(json));}

\KeywordTok{console}\NormalTok{.}\FunctionTok{log}\NormalTok{(copy);}
\CommentTok{// <Buffer 74 65 73 74>}
\end{Highlighting}
\end{Shaded}

\subsubsection{buf{[}index{]}}

Get and set the octet at \texttt{index}. The values refer to individual
bytes, so the legal range is between \texttt{0x00} and \texttt{0xFF} hex
or \texttt{0} and \texttt{255}.

Example: copy an ASCII string into a buffer, one byte at a time:

\begin{Shaded}
\begin{Highlighting}[]
\NormalTok{str = }\StringTok{"node.js"}\NormalTok{;}
\NormalTok{buf = }\KeywordTok{new} \NormalTok{Buffer(}\KeywordTok{str}\NormalTok{.}\FunctionTok{length}\NormalTok{);}

\KeywordTok{for} \NormalTok{(}\KeywordTok{var} \NormalTok{i = }\DecValTok{0}\NormalTok{; i < }\KeywordTok{str}\NormalTok{.}\FunctionTok{length} \NormalTok{; i++) \{}
  \NormalTok{buf[i] = }\KeywordTok{str}\NormalTok{.}\FunctionTok{charCodeAt}\NormalTok{(i);}
\NormalTok{\}}

\KeywordTok{console}\NormalTok{.}\FunctionTok{log}\NormalTok{(buf);}

\CommentTok{// node.js}
\end{Highlighting}
\end{Shaded}

\subsubsection{Class Method: Buffer.isBuffer(obj)}

\begin{itemize}
\item
  \texttt{obj} Object
\item
  Return: Boolean
\end{itemize}

Tests if \texttt{obj} is a \texttt{Buffer}.

\subsubsection{Class Method: Buffer.byteLength(string, {[}encoding{]})}

\begin{itemize}
\item
  \texttt{string} String
\item
  \texttt{encoding} String, Optional, Default: `utf8'
\item
  Return: Number
\end{itemize}

Gives the actual byte length of a string. \texttt{encoding} defaults to
\texttt{'utf8'}. This is not the same as
\texttt{String.prototype.length} since that returns the number of
\emph{characters} in a string.

Example:

\begin{Shaded}
\begin{Highlighting}[]
\NormalTok{str = }\CharTok{'\textbackslash{}u00bd + \textbackslash{}u00bc = \textbackslash{}u00be'}\NormalTok{;}

\KeywordTok{console}\NormalTok{.}\FunctionTok{log}\NormalTok{(str + }\StringTok{": "} \NormalTok{+ }\KeywordTok{str}\NormalTok{.}\FunctionTok{length} \NormalTok{+ }\StringTok{" characters, "} \NormalTok{+}
  \KeywordTok{Buffer}\NormalTok{.}\FunctionTok{byteLength}\NormalTok{(str, }\CharTok{'utf8'}\NormalTok{) + }\StringTok{" bytes"}\NormalTok{);}

\CommentTok{// ½ + ¼ = ¾: 9 characters, 12 bytes}
\end{Highlighting}
\end{Shaded}

\subsubsection{Class Method: Buffer.concat(list, {[}totalLength{]})}

\begin{itemize}
\item
  \texttt{list} \{Array\} List of Buffer objects to concat
\item
  \texttt{totalLength} \{Number\} Total length of the buffers when
  concatenated
\end{itemize}

Returns a buffer which is the result of concatenating all the buffers in
the list together.

If the list has no items, or if the totalLength is 0, then it returns a
zero-length buffer.

If the list has exactly one item, then the first item of the list is
returned.

If the list has more than one item, then a new Buffer is created.

If totalLength is not provided, it is read from the buffers in the list.
However, this adds an additional loop to the function, so it is faster
to provide the length explicitly.

\subsubsection{buf.length}

\begin{itemize}
\item
  Number
\end{itemize}

The size of the buffer in bytes. Note that this is not necessarily the
size of the contents. \texttt{length} refers to the amount of memory
allocated for the buffer object. It does not change when the contents of
the buffer are changed.

\begin{Shaded}
\begin{Highlighting}[]
\NormalTok{buf = }\KeywordTok{new} \NormalTok{Buffer(}\DecValTok{1234}\NormalTok{);}

\KeywordTok{console}\NormalTok{.}\FunctionTok{log}\NormalTok{(}\KeywordTok{buf}\NormalTok{.}\FunctionTok{length}\NormalTok{);}
\KeywordTok{buf}\NormalTok{.}\FunctionTok{write}\NormalTok{(}\StringTok{"some string"}\NormalTok{, }\DecValTok{0}\NormalTok{, }\StringTok{"ascii"}\NormalTok{);}
\KeywordTok{console}\NormalTok{.}\FunctionTok{log}\NormalTok{(}\KeywordTok{buf}\NormalTok{.}\FunctionTok{length}\NormalTok{);}

\CommentTok{// 1234}
\CommentTok{// 1234}
\end{Highlighting}
\end{Shaded}

\subsubsection{buf.copy(targetBuffer, {[}targetStart{]},
{[}sourceStart{]}, {[}sourceEnd{]})}

\begin{itemize}
\item
  \texttt{targetBuffer} Buffer object - Buffer to copy into
\item
  \texttt{targetStart} Number, Optional, Default: 0
\item
  \texttt{sourceStart} Number, Optional, Default: 0
\item
  \texttt{sourceEnd} Number, Optional, Default: \texttt{buffer.length}
\end{itemize}

Does copy between buffers. The source and target regions can be
overlapped. \texttt{targetStart} and \texttt{sourceStart} default to
\texttt{0}. \texttt{sourceEnd} defaults to \texttt{buffer.length}.

All values passed that are \texttt{undefined}/\texttt{NaN} or are out of
bounds are set equal to their respective defaults.

Example: build two Buffers, then copy \texttt{buf1} from byte 16 through
byte 19 into \texttt{buf2}, starting at the 8th byte in \texttt{buf2}.

\begin{Shaded}
\begin{Highlighting}[]
\NormalTok{buf1 = }\KeywordTok{new} \NormalTok{Buffer(}\DecValTok{26}\NormalTok{);}
\NormalTok{buf2 = }\KeywordTok{new} \NormalTok{Buffer(}\DecValTok{26}\NormalTok{);}

\KeywordTok{for} \NormalTok{(}\KeywordTok{var} \NormalTok{i = }\DecValTok{0} \NormalTok{; i < }\DecValTok{26} \NormalTok{; i++) \{}
  \NormalTok{buf1[i] = i + }\DecValTok{97}\NormalTok{; }\CommentTok{// 97 is ASCII a}
  \NormalTok{buf2[i] = }\DecValTok{33}\NormalTok{; }\CommentTok{// ASCII !}
\NormalTok{\}}

\KeywordTok{buf1}\NormalTok{.}\FunctionTok{copy}\NormalTok{(buf2, }\DecValTok{8}\NormalTok{, }\DecValTok{16}\NormalTok{, }\DecValTok{20}\NormalTok{);}
\KeywordTok{console}\NormalTok{.}\FunctionTok{log}\NormalTok{(}\KeywordTok{buf2}\NormalTok{.}\FunctionTok{toString}\NormalTok{(}\CharTok{'ascii'}\NormalTok{, }\DecValTok{0}\NormalTok{, }\DecValTok{25}\NormalTok{));}

\CommentTok{// !!!!!!!!qrst!!!!!!!!!!!!!}
\end{Highlighting}
\end{Shaded}

\subsubsection{buf.slice({[}start{]}, {[}end{]})}

\begin{itemize}
\item
  \texttt{start} Number, Optional, Default: 0
\item
  \texttt{end} Number, Optional, Default: \texttt{buffer.length}
\end{itemize}

Returns a new buffer which references the same memory as the old, but
offset and cropped by the \texttt{start} (defaults to \texttt{0}) and
\texttt{end} (defaults to \texttt{buffer.length}) indexes. Negative
indexes start from the end of the buffer.

\textbf{Modifying the new buffer slice will modify memory in the
original buffer!}

Example: build a Buffer with the ASCII alphabet, take a slice, then
modify one byte from the original Buffer.

\begin{Shaded}
\begin{Highlighting}[]
\KeywordTok{var} \NormalTok{buf1 = }\KeywordTok{new} \NormalTok{Buffer(}\DecValTok{26}\NormalTok{);}

\KeywordTok{for} \NormalTok{(}\KeywordTok{var} \NormalTok{i = }\DecValTok{0} \NormalTok{; i < }\DecValTok{26} \NormalTok{; i++) \{}
  \NormalTok{buf1[i] = i + }\DecValTok{97}\NormalTok{; }\CommentTok{// 97 is ASCII a}
\NormalTok{\}}

\KeywordTok{var} \NormalTok{buf2 = }\KeywordTok{buf1}\NormalTok{.}\FunctionTok{slice}\NormalTok{(}\DecValTok{0}\NormalTok{, }\DecValTok{3}\NormalTok{);}
\KeywordTok{console}\NormalTok{.}\FunctionTok{log}\NormalTok{(}\KeywordTok{buf2}\NormalTok{.}\FunctionTok{toString}\NormalTok{(}\CharTok{'ascii'}\NormalTok{, }\DecValTok{0}\NormalTok{, }\KeywordTok{buf2}\NormalTok{.}\FunctionTok{length}\NormalTok{));}
\NormalTok{buf1[}\DecValTok{0}\NormalTok{] = }\DecValTok{33}\NormalTok{;}
\KeywordTok{console}\NormalTok{.}\FunctionTok{log}\NormalTok{(}\KeywordTok{buf2}\NormalTok{.}\FunctionTok{toString}\NormalTok{(}\CharTok{'ascii'}\NormalTok{, }\DecValTok{0}\NormalTok{, }\KeywordTok{buf2}\NormalTok{.}\FunctionTok{length}\NormalTok{));}

\CommentTok{// abc}
\CommentTok{// !bc}
\end{Highlighting}
\end{Shaded}

\subsubsection{buf.readUInt8(offset, {[}noAssert{]})}

\begin{itemize}
\item
  \texttt{offset} Number
\item
  \texttt{noAssert} Boolean, Optional, Default: false
\item
  Return: Number
\end{itemize}

Reads an unsigned 8 bit integer from the buffer at the specified offset.

Set \texttt{noAssert} to true to skip validation of \texttt{offset}.
This means that \texttt{offset} may be beyond the end of the buffer.
Defaults to \texttt{false}.

Example:

\begin{Shaded}
\begin{Highlighting}[]
\KeywordTok{var} \NormalTok{buf = }\KeywordTok{new} \NormalTok{Buffer(}\DecValTok{4}\NormalTok{);}

\NormalTok{buf[}\DecValTok{0}\NormalTok{] = }\DecValTok{0x3}\NormalTok{;}
\NormalTok{buf[}\DecValTok{1}\NormalTok{] = }\DecValTok{0x4}\NormalTok{;}
\NormalTok{buf[}\DecValTok{2}\NormalTok{] = }\DecValTok{0x23}\NormalTok{;}
\NormalTok{buf[}\DecValTok{3}\NormalTok{] = }\DecValTok{0x42}\NormalTok{;}

\KeywordTok{for} \NormalTok{(ii = }\DecValTok{0}\NormalTok{; ii < }\KeywordTok{buf}\NormalTok{.}\FunctionTok{length}\NormalTok{; ii++) \{}
  \KeywordTok{console}\NormalTok{.}\FunctionTok{log}\NormalTok{(}\KeywordTok{buf}\NormalTok{.}\FunctionTok{readUInt8}\NormalTok{(ii));}
\NormalTok{\}}

\CommentTok{// 0x3}
\CommentTok{// 0x4}
\CommentTok{// 0x23}
\CommentTok{// 0x42}
\end{Highlighting}
\end{Shaded}

\subsubsection{buf.readUInt16LE(offset, {[}noAssert{]})}

\subsubsection{buf.readUInt16BE(offset, {[}noAssert{]})}

\begin{itemize}
\item
  \texttt{offset} Number
\item
  \texttt{noAssert} Boolean, Optional, Default: false
\item
  Return: Number
\end{itemize}

Reads an unsigned 16 bit integer from the buffer at the specified offset
with specified endian format.

Set \texttt{noAssert} to true to skip validation of \texttt{offset}.
This means that \texttt{offset} may be beyond the end of the buffer.
Defaults to \texttt{false}.

Example:

\begin{Shaded}
\begin{Highlighting}[]
\KeywordTok{var} \NormalTok{buf = }\KeywordTok{new} \NormalTok{Buffer(}\DecValTok{4}\NormalTok{);}

\NormalTok{buf[}\DecValTok{0}\NormalTok{] = }\DecValTok{0x3}\NormalTok{;}
\NormalTok{buf[}\DecValTok{1}\NormalTok{] = }\DecValTok{0x4}\NormalTok{;}
\NormalTok{buf[}\DecValTok{2}\NormalTok{] = }\DecValTok{0x23}\NormalTok{;}
\NormalTok{buf[}\DecValTok{3}\NormalTok{] = }\DecValTok{0x42}\NormalTok{;}

\KeywordTok{console}\NormalTok{.}\FunctionTok{log}\NormalTok{(}\KeywordTok{buf}\NormalTok{.}\FunctionTok{readUInt16BE}\NormalTok{(}\DecValTok{0}\NormalTok{));}
\KeywordTok{console}\NormalTok{.}\FunctionTok{log}\NormalTok{(}\KeywordTok{buf}\NormalTok{.}\FunctionTok{readUInt16LE}\NormalTok{(}\DecValTok{0}\NormalTok{));}
\KeywordTok{console}\NormalTok{.}\FunctionTok{log}\NormalTok{(}\KeywordTok{buf}\NormalTok{.}\FunctionTok{readUInt16BE}\NormalTok{(}\DecValTok{1}\NormalTok{));}
\KeywordTok{console}\NormalTok{.}\FunctionTok{log}\NormalTok{(}\KeywordTok{buf}\NormalTok{.}\FunctionTok{readUInt16LE}\NormalTok{(}\DecValTok{1}\NormalTok{));}
\KeywordTok{console}\NormalTok{.}\FunctionTok{log}\NormalTok{(}\KeywordTok{buf}\NormalTok{.}\FunctionTok{readUInt16BE}\NormalTok{(}\DecValTok{2}\NormalTok{));}
\KeywordTok{console}\NormalTok{.}\FunctionTok{log}\NormalTok{(}\KeywordTok{buf}\NormalTok{.}\FunctionTok{readUInt16LE}\NormalTok{(}\DecValTok{2}\NormalTok{));}

\CommentTok{// 0x0304}
\CommentTok{// 0x0403}
\CommentTok{// 0x0423}
\CommentTok{// 0x2304}
\CommentTok{// 0x2342}
\CommentTok{// 0x4223}
\end{Highlighting}
\end{Shaded}

\subsubsection{buf.readUInt32LE(offset, {[}noAssert{]})}

\subsubsection{buf.readUInt32BE(offset, {[}noAssert{]})}

\begin{itemize}
\item
  \texttt{offset} Number
\item
  \texttt{noAssert} Boolean, Optional, Default: false
\item
  Return: Number
\end{itemize}

Reads an unsigned 32 bit integer from the buffer at the specified offset
with specified endian format.

Set \texttt{noAssert} to true to skip validation of \texttt{offset}.
This means that \texttt{offset} may be beyond the end of the buffer.
Defaults to \texttt{false}.

Example:

\begin{Shaded}
\begin{Highlighting}[]
\KeywordTok{var} \NormalTok{buf = }\KeywordTok{new} \NormalTok{Buffer(}\DecValTok{4}\NormalTok{);}

\NormalTok{buf[}\DecValTok{0}\NormalTok{] = }\DecValTok{0x3}\NormalTok{;}
\NormalTok{buf[}\DecValTok{1}\NormalTok{] = }\DecValTok{0x4}\NormalTok{;}
\NormalTok{buf[}\DecValTok{2}\NormalTok{] = }\DecValTok{0x23}\NormalTok{;}
\NormalTok{buf[}\DecValTok{3}\NormalTok{] = }\DecValTok{0x42}\NormalTok{;}

\KeywordTok{console}\NormalTok{.}\FunctionTok{log}\NormalTok{(}\KeywordTok{buf}\NormalTok{.}\FunctionTok{readUInt32BE}\NormalTok{(}\DecValTok{0}\NormalTok{));}
\KeywordTok{console}\NormalTok{.}\FunctionTok{log}\NormalTok{(}\KeywordTok{buf}\NormalTok{.}\FunctionTok{readUInt32LE}\NormalTok{(}\DecValTok{0}\NormalTok{));}

\CommentTok{// 0x03042342}
\CommentTok{// 0x42230403}
\end{Highlighting}
\end{Shaded}

\subsubsection{buf.readInt8(offset, {[}noAssert{]})}

\begin{itemize}
\item
  \texttt{offset} Number
\item
  \texttt{noAssert} Boolean, Optional, Default: false
\item
  Return: Number
\end{itemize}

Reads a signed 8 bit integer from the buffer at the specified offset.

Set \texttt{noAssert} to true to skip validation of \texttt{offset}.
This means that \texttt{offset} may be beyond the end of the buffer.
Defaults to \texttt{false}.

Works as \texttt{buffer.readUInt8}, except buffer contents are treated
as two's complement signed values.

\subsubsection{buf.readInt16LE(offset, {[}noAssert{]})}

\subsubsection{buf.readInt16BE(offset, {[}noAssert{]})}

\begin{itemize}
\item
  \texttt{offset} Number
\item
  \texttt{noAssert} Boolean, Optional, Default: false
\item
  Return: Number
\end{itemize}

Reads a signed 16 bit integer from the buffer at the specified offset
with specified endian format.

Set \texttt{noAssert} to true to skip validation of \texttt{offset}.
This means that \texttt{offset} may be beyond the end of the buffer.
Defaults to \texttt{false}.

Works as \texttt{buffer.readUInt16*}, except buffer contents are treated
as two's complement signed values.

\subsubsection{buf.readInt32LE(offset, {[}noAssert{]})}

\subsubsection{buf.readInt32BE(offset, {[}noAssert{]})}

\begin{itemize}
\item
  \texttt{offset} Number
\item
  \texttt{noAssert} Boolean, Optional, Default: false
\item
  Return: Number
\end{itemize}

Reads a signed 32 bit integer from the buffer at the specified offset
with specified endian format.

Set \texttt{noAssert} to true to skip validation of \texttt{offset}.
This means that \texttt{offset} may be beyond the end of the buffer.
Defaults to \texttt{false}.

Works as \texttt{buffer.readUInt32*}, except buffer contents are treated
as two's complement signed values.

\subsubsection{buf.readFloatLE(offset, {[}noAssert{]})}

\subsubsection{buf.readFloatBE(offset, {[}noAssert{]})}

\begin{itemize}
\item
  \texttt{offset} Number
\item
  \texttt{noAssert} Boolean, Optional, Default: false
\item
  Return: Number
\end{itemize}

Reads a 32 bit float from the buffer at the specified offset with
specified endian format.

Set \texttt{noAssert} to true to skip validation of \texttt{offset}.
This means that \texttt{offset} may be beyond the end of the buffer.
Defaults to \texttt{false}.

Example:

\begin{Shaded}
\begin{Highlighting}[]
\KeywordTok{var} \NormalTok{buf = }\KeywordTok{new} \NormalTok{Buffer(}\DecValTok{4}\NormalTok{);}

\NormalTok{buf[}\DecValTok{0}\NormalTok{] = }\DecValTok{0x00}\NormalTok{;}
\NormalTok{buf[}\DecValTok{1}\NormalTok{] = }\DecValTok{0x00}\NormalTok{;}
\NormalTok{buf[}\DecValTok{2}\NormalTok{] = }\DecValTok{0x80}\NormalTok{;}
\NormalTok{buf[}\DecValTok{3}\NormalTok{] = }\DecValTok{0x3f}\NormalTok{;}

\KeywordTok{console}\NormalTok{.}\FunctionTok{log}\NormalTok{(}\KeywordTok{buf}\NormalTok{.}\FunctionTok{readFloatLE}\NormalTok{(}\DecValTok{0}\NormalTok{));}

\CommentTok{// 0x01}
\end{Highlighting}
\end{Shaded}

\subsubsection{buf.readDoubleLE(offset, {[}noAssert{]})}

\subsubsection{buf.readDoubleBE(offset, {[}noAssert{]})}

\begin{itemize}
\item
  \texttt{offset} Number
\item
  \texttt{noAssert} Boolean, Optional, Default: false
\item
  Return: Number
\end{itemize}

Reads a 64 bit double from the buffer at the specified offset with
specified endian format.

Set \texttt{noAssert} to true to skip validation of \texttt{offset}.
This means that \texttt{offset} may be beyond the end of the buffer.
Defaults to \texttt{false}.

Example:

\begin{Shaded}
\begin{Highlighting}[]
\KeywordTok{var} \NormalTok{buf = }\KeywordTok{new} \NormalTok{Buffer(}\DecValTok{8}\NormalTok{);}

\NormalTok{buf[}\DecValTok{0}\NormalTok{] = }\DecValTok{0x55}\NormalTok{;}
\NormalTok{buf[}\DecValTok{1}\NormalTok{] = }\DecValTok{0x55}\NormalTok{;}
\NormalTok{buf[}\DecValTok{2}\NormalTok{] = }\DecValTok{0x55}\NormalTok{;}
\NormalTok{buf[}\DecValTok{3}\NormalTok{] = }\DecValTok{0x55}\NormalTok{;}
\NormalTok{buf[}\DecValTok{4}\NormalTok{] = }\DecValTok{0x55}\NormalTok{;}
\NormalTok{buf[}\DecValTok{5}\NormalTok{] = }\DecValTok{0x55}\NormalTok{;}
\NormalTok{buf[}\DecValTok{6}\NormalTok{] = }\DecValTok{0xd5}\NormalTok{;}
\NormalTok{buf[}\DecValTok{7}\NormalTok{] = }\DecValTok{0x3f}\NormalTok{;}

\KeywordTok{console}\NormalTok{.}\FunctionTok{log}\NormalTok{(}\KeywordTok{buf}\NormalTok{.}\FunctionTok{readDoubleLE}\NormalTok{(}\DecValTok{0}\NormalTok{));}

\CommentTok{// 0.3333333333333333}
\end{Highlighting}
\end{Shaded}

\subsubsection{buf.writeUInt8(value, offset, {[}noAssert{]})}

\begin{itemize}
\item
  \texttt{value} Number
\item
  \texttt{offset} Number
\item
  \texttt{noAssert} Boolean, Optional, Default: false
\end{itemize}

Writes \texttt{value} to the buffer at the specified offset. Note,
\texttt{value} must be a valid unsigned 8 bit integer.

Set \texttt{noAssert} to true to skip validation of \texttt{value} and
\texttt{offset}. This means that \texttt{value} may be too large for the
specific function and \texttt{offset} may be beyond the end of the
buffer leading to the values being silently dropped. This should not be
used unless you are certain of correctness. Defaults to \texttt{false}.

Example:

\begin{Shaded}
\begin{Highlighting}[]
\KeywordTok{var} \NormalTok{buf = }\KeywordTok{new} \NormalTok{Buffer(}\DecValTok{4}\NormalTok{);}
\KeywordTok{buf}\NormalTok{.}\FunctionTok{writeUInt8}\NormalTok{(}\DecValTok{0x3}\NormalTok{, }\DecValTok{0}\NormalTok{);}
\KeywordTok{buf}\NormalTok{.}\FunctionTok{writeUInt8}\NormalTok{(}\DecValTok{0x4}\NormalTok{, }\DecValTok{1}\NormalTok{);}
\KeywordTok{buf}\NormalTok{.}\FunctionTok{writeUInt8}\NormalTok{(}\DecValTok{0x23}\NormalTok{, }\DecValTok{2}\NormalTok{);}
\KeywordTok{buf}\NormalTok{.}\FunctionTok{writeUInt8}\NormalTok{(}\DecValTok{0x42}\NormalTok{, }\DecValTok{3}\NormalTok{);}

\KeywordTok{console}\NormalTok{.}\FunctionTok{log}\NormalTok{(buf);}

\CommentTok{// <Buffer 03 04 23 42>}
\end{Highlighting}
\end{Shaded}

\subsubsection{buf.writeUInt16LE(value, offset, {[}noAssert{]})}

\subsubsection{buf.writeUInt16BE(value, offset, {[}noAssert{]})}

\begin{itemize}
\item
  \texttt{value} Number
\item
  \texttt{offset} Number
\item
  \texttt{noAssert} Boolean, Optional, Default: false
\end{itemize}

Writes \texttt{value} to the buffer at the specified offset with
specified endian format. Note, \texttt{value} must be a valid unsigned
16 bit integer.

Set \texttt{noAssert} to true to skip validation of \texttt{value} and
\texttt{offset}. This means that \texttt{value} may be too large for the
specific function and \texttt{offset} may be beyond the end of the
buffer leading to the values being silently dropped. This should not be
used unless you are certain of correctness. Defaults to \texttt{false}.

Example:

\begin{Shaded}
\begin{Highlighting}[]
\KeywordTok{var} \NormalTok{buf = }\KeywordTok{new} \NormalTok{Buffer(}\DecValTok{4}\NormalTok{);}
\KeywordTok{buf}\NormalTok{.}\FunctionTok{writeUInt16BE}\NormalTok{(}\DecValTok{0xdead}\NormalTok{, }\DecValTok{0}\NormalTok{);}
\KeywordTok{buf}\NormalTok{.}\FunctionTok{writeUInt16BE}\NormalTok{(}\DecValTok{0xbeef}\NormalTok{, }\DecValTok{2}\NormalTok{);}

\KeywordTok{console}\NormalTok{.}\FunctionTok{log}\NormalTok{(buf);}

\KeywordTok{buf}\NormalTok{.}\FunctionTok{writeUInt16LE}\NormalTok{(}\DecValTok{0xdead}\NormalTok{, }\DecValTok{0}\NormalTok{);}
\KeywordTok{buf}\NormalTok{.}\FunctionTok{writeUInt16LE}\NormalTok{(}\DecValTok{0xbeef}\NormalTok{, }\DecValTok{2}\NormalTok{);}

\KeywordTok{console}\NormalTok{.}\FunctionTok{log}\NormalTok{(buf);}

\CommentTok{// <Buffer de ad be ef>}
\CommentTok{// <Buffer ad de ef be>}
\end{Highlighting}
\end{Shaded}

\subsubsection{buf.writeUInt32LE(value, offset, {[}noAssert{]})}

\subsubsection{buf.writeUInt32BE(value, offset, {[}noAssert{]})}

\begin{itemize}
\item
  \texttt{value} Number
\item
  \texttt{offset} Number
\item
  \texttt{noAssert} Boolean, Optional, Default: false
\end{itemize}

Writes \texttt{value} to the buffer at the specified offset with
specified endian format. Note, \texttt{value} must be a valid unsigned
32 bit integer.

Set \texttt{noAssert} to true to skip validation of \texttt{value} and
\texttt{offset}. This means that \texttt{value} may be too large for the
specific function and \texttt{offset} may be beyond the end of the
buffer leading to the values being silently dropped. This should not be
used unless you are certain of correctness. Defaults to \texttt{false}.

Example:

\begin{Shaded}
\begin{Highlighting}[]
\KeywordTok{var} \NormalTok{buf = }\KeywordTok{new} \NormalTok{Buffer(}\DecValTok{4}\NormalTok{);}
\KeywordTok{buf}\NormalTok{.}\FunctionTok{writeUInt32BE}\NormalTok{(}\DecValTok{0xfeedface}\NormalTok{, }\DecValTok{0}\NormalTok{);}

\KeywordTok{console}\NormalTok{.}\FunctionTok{log}\NormalTok{(buf);}

\KeywordTok{buf}\NormalTok{.}\FunctionTok{writeUInt32LE}\NormalTok{(}\DecValTok{0xfeedface}\NormalTok{, }\DecValTok{0}\NormalTok{);}

\KeywordTok{console}\NormalTok{.}\FunctionTok{log}\NormalTok{(buf);}

\CommentTok{// <Buffer fe ed fa ce>}
\CommentTok{// <Buffer ce fa ed fe>}
\end{Highlighting}
\end{Shaded}

\subsubsection{buf.writeInt8(value, offset, {[}noAssert{]})}

\begin{itemize}
\item
  \texttt{value} Number
\item
  \texttt{offset} Number
\item
  \texttt{noAssert} Boolean, Optional, Default: false
\end{itemize}

Writes \texttt{value} to the buffer at the specified offset. Note,
\texttt{value} must be a valid signed 8 bit integer.

Set \texttt{noAssert} to true to skip validation of \texttt{value} and
\texttt{offset}. This means that \texttt{value} may be too large for the
specific function and \texttt{offset} may be beyond the end of the
buffer leading to the values being silently dropped. This should not be
used unless you are certain of correctness. Defaults to \texttt{false}.

Works as \texttt{buffer.writeUInt8}, except value is written out as a
two's complement signed integer into \texttt{buffer}.

\subsubsection{buf.writeInt16LE(value, offset, {[}noAssert{]})}

\subsubsection{buf.writeInt16BE(value, offset, {[}noAssert{]})}

\begin{itemize}
\item
  \texttt{value} Number
\item
  \texttt{offset} Number
\item
  \texttt{noAssert} Boolean, Optional, Default: false
\end{itemize}

Writes \texttt{value} to the buffer at the specified offset with
specified endian format. Note, \texttt{value} must be a valid signed 16
bit integer.

Set \texttt{noAssert} to true to skip validation of \texttt{value} and
\texttt{offset}. This means that \texttt{value} may be too large for the
specific function and \texttt{offset} may be beyond the end of the
buffer leading to the values being silently dropped. This should not be
used unless you are certain of correctness. Defaults to \texttt{false}.

Works as \texttt{buffer.writeUInt16*}, except value is written out as a
two's complement signed integer into \texttt{buffer}.

\subsubsection{buf.writeInt32LE(value, offset, {[}noAssert{]})}

\subsubsection{buf.writeInt32BE(value, offset, {[}noAssert{]})}

\begin{itemize}
\item
  \texttt{value} Number
\item
  \texttt{offset} Number
\item
  \texttt{noAssert} Boolean, Optional, Default: false
\end{itemize}

Writes \texttt{value} to the buffer at the specified offset with
specified endian format. Note, \texttt{value} must be a valid signed 32
bit integer.

Set \texttt{noAssert} to true to skip validation of \texttt{value} and
\texttt{offset}. This means that \texttt{value} may be too large for the
specific function and \texttt{offset} may be beyond the end of the
buffer leading to the values being silently dropped. This should not be
used unless you are certain of correctness. Defaults to \texttt{false}.

Works as \texttt{buffer.writeUInt32*}, except value is written out as a
two's complement signed integer into \texttt{buffer}.

\subsubsection{buf.writeFloatLE(value, offset, {[}noAssert{]})}

\subsubsection{buf.writeFloatBE(value, offset, {[}noAssert{]})}

\begin{itemize}
\item
  \texttt{value} Number
\item
  \texttt{offset} Number
\item
  \texttt{noAssert} Boolean, Optional, Default: false
\end{itemize}

Writes \texttt{value} to the buffer at the specified offset with
specified endian format. Note, behavior is unspecified if \texttt{value}
is not a 32 bit float.

Set \texttt{noAssert} to true to skip validation of \texttt{value} and
\texttt{offset}. This means that \texttt{value} may be too large for the
specific function and \texttt{offset} may be beyond the end of the
buffer leading to the values being silently dropped. This should not be
used unless you are certain of correctness. Defaults to \texttt{false}.

Example:

\begin{Shaded}
\begin{Highlighting}[]
\KeywordTok{var} \NormalTok{buf = }\KeywordTok{new} \NormalTok{Buffer(}\DecValTok{4}\NormalTok{);}
\KeywordTok{buf}\NormalTok{.}\FunctionTok{writeFloatBE}\NormalTok{(}\DecValTok{0xcafebabe}\NormalTok{, }\DecValTok{0}\NormalTok{);}

\KeywordTok{console}\NormalTok{.}\FunctionTok{log}\NormalTok{(buf);}

\KeywordTok{buf}\NormalTok{.}\FunctionTok{writeFloatLE}\NormalTok{(}\DecValTok{0xcafebabe}\NormalTok{, }\DecValTok{0}\NormalTok{);}

\KeywordTok{console}\NormalTok{.}\FunctionTok{log}\NormalTok{(buf);}

\CommentTok{// <Buffer 4f 4a fe bb>}
\CommentTok{// <Buffer bb fe 4a 4f>}
\end{Highlighting}
\end{Shaded}

\subsubsection{buf.writeDoubleLE(value, offset, {[}noAssert{]})}

\subsubsection{buf.writeDoubleBE(value, offset, {[}noAssert{]})}

\begin{itemize}
\item
  \texttt{value} Number
\item
  \texttt{offset} Number
\item
  \texttt{noAssert} Boolean, Optional, Default: false
\end{itemize}

Writes \texttt{value} to the buffer at the specified offset with
specified endian format. Note, \texttt{value} must be a valid 64 bit
double.

Set \texttt{noAssert} to true to skip validation of \texttt{value} and
\texttt{offset}. This means that \texttt{value} may be too large for the
specific function and \texttt{offset} may be beyond the end of the
buffer leading to the values being silently dropped. This should not be
used unless you are certain of correctness. Defaults to \texttt{false}.

Example:

\begin{Shaded}
\begin{Highlighting}[]
\KeywordTok{var} \NormalTok{buf = }\KeywordTok{new} \NormalTok{Buffer(}\DecValTok{8}\NormalTok{);}
\KeywordTok{buf}\NormalTok{.}\FunctionTok{writeDoubleBE}\NormalTok{(}\DecValTok{0xdeadbeefcafebabe}\NormalTok{, }\DecValTok{0}\NormalTok{);}

\KeywordTok{console}\NormalTok{.}\FunctionTok{log}\NormalTok{(buf);}

\KeywordTok{buf}\NormalTok{.}\FunctionTok{writeDoubleLE}\NormalTok{(}\DecValTok{0xdeadbeefcafebabe}\NormalTok{, }\DecValTok{0}\NormalTok{);}

\KeywordTok{console}\NormalTok{.}\FunctionTok{log}\NormalTok{(buf);}

\CommentTok{// <Buffer 43 eb d5 b7 dd f9 5f d7>}
\CommentTok{// <Buffer d7 5f f9 dd b7 d5 eb 43>}
\end{Highlighting}
\end{Shaded}

\subsubsection{buf.fill(value, {[}offset{]}, {[}end{]})}

\begin{itemize}
\item
  \texttt{value}
\item
  \texttt{offset} Number, Optional
\item
  \texttt{end} Number, Optional
\end{itemize}

Fills the buffer with the specified value. If the \texttt{offset}
(defaults to \texttt{0}) and \texttt{end} (defaults to
\texttt{buffer.length}) are not given it will fill the entire buffer.

\begin{Shaded}
\begin{Highlighting}[]
\KeywordTok{var} \NormalTok{b = }\KeywordTok{new} \NormalTok{Buffer(}\DecValTok{50}\NormalTok{);}
\KeywordTok{b}\NormalTok{.}\FunctionTok{fill}\NormalTok{(}\StringTok{"h"}\NormalTok{);}
\end{Highlighting}
\end{Shaded}

\subsection{buffer.INSPECT\_MAX\_BYTES}

\begin{itemize}
\item
  Number, Default: 50
\end{itemize}

How many bytes will be returned when \texttt{buffer.inspect()} is
called. This can be overridden by user modules.

Note that this is a property on the buffer module returned by
\texttt{require('buffer')}, not on the Buffer global, or a buffer
instance.

\subsection{Class: SlowBuffer}

This class is primarily for internal use. JavaScript programs should use
Buffer instead of using SlowBuffer.

In order to avoid the overhead of allocating many C++ Buffer objects for
small blocks of memory in the lifetime of a server, Node allocates
memory in 8Kb (8192 byte) chunks. If a buffer is smaller than this size,
then it will be backed by a parent SlowBuffer object. If it is larger
than this, then Node will allocate a SlowBuffer slab for it directly.
