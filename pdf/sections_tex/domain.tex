\section{Domain}

\begin{Shaded}
\begin{Highlighting}[]
\DataTypeTok{Stability}\NormalTok{: }\DecValTok{2} \NormalTok{- Unstable}
\end{Highlighting}
\end{Shaded}

Domains provide a way to handle multiple different IO operations as a
single group. If any of the event emitters or callbacks registered to a
domain emit an \texttt{error} event, or throw an error, then the domain
object will be notified, rather than losing the context of the error in
the \texttt{process.on('uncaughtException')} handler, or causing the
program to exit immediately with an error code.

\subsection{Warning: Don't Ignore Errors!}

Domain error handlers are not a substitute for closing down your process
when an error occurs.

By the very nature of how \texttt{throw} works in JavaScript, there is
almost never any way to safely ``pick up where you left off'', without
leaking references, or creating some other sort of undefined brittle
state.

The safest way to respond to a thrown error is to shut down the process.
Of course, in a normal web server, you might have many connections open,
and it is not reasonable to abruptly shut those down because an error
was triggered by someone else.

The better approach is send an error response to the request that
triggered the error, while letting the others finish in their normal
time, and stop listening for new requests in that worker.

In this way, \texttt{domain} usage goes hand-in-hand with the cluster
module, since the master process can fork a new worker when a worker
encounters an error. For node programs that scale to multiple machines,
the terminating proxy or service registry can take note of the failure,
and react accordingly.

For example, this is not a good idea:

\begin{Shaded}
\begin{Highlighting}[]
\CommentTok{// XXX WARNING!  BAD IDEA!}

\KeywordTok{var} \NormalTok{d = require(}\CharTok{'domain'}\NormalTok{).}\FunctionTok{create}\NormalTok{();}
\KeywordTok{d}\NormalTok{.}\FunctionTok{on}\NormalTok{(}\CharTok{'error'}\NormalTok{, }\KeywordTok{function}\NormalTok{(er) \{}
  \CommentTok{// The error won't crash the process, but what it does is worse!}
  \CommentTok{// Though we've prevented abrupt process restarting, we are leaking}
  \CommentTok{// resources like crazy if this ever happens.}
  \CommentTok{// This is no better than process.on('uncaughtException')!}
  \KeywordTok{console}\NormalTok{.}\FunctionTok{log}\NormalTok{(}\CharTok{'error, but oh well'}\NormalTok{, }\KeywordTok{er}\NormalTok{.}\FunctionTok{message}\NormalTok{);}
\NormalTok{\});}
\KeywordTok{d}\NormalTok{.}\FunctionTok{run}\NormalTok{(}\KeywordTok{function}\NormalTok{() \{}
  \NormalTok{require(}\CharTok{'http'}\NormalTok{).}\FunctionTok{createServer}\NormalTok{(}\KeywordTok{function}\NormalTok{(req, res) \{}
    \NormalTok{handleRequest(req, res);}
  \NormalTok{\}).}\FunctionTok{listen}\NormalTok{(PORT);}
\NormalTok{\});}
\end{Highlighting}
\end{Shaded}

By using the context of a domain, and the resilience of separating our
program into multiple worker processes, we can react more appropriately,
and handle errors with much greater safety.

\begin{Shaded}
\begin{Highlighting}[]
\CommentTok{// Much better!}

\KeywordTok{var} \NormalTok{cluster = require(}\CharTok{'cluster'}\NormalTok{);}
\KeywordTok{var} \NormalTok{PORT = +}\KeywordTok{process.env}\NormalTok{.}\FunctionTok{PORT} \NormalTok{\textbar{}\textbar{} }\DecValTok{1337}\NormalTok{;}

\KeywordTok{if} \NormalTok{(}\KeywordTok{cluster}\NormalTok{.}\FunctionTok{isMaster}\NormalTok{) \{}
  \CommentTok{// In real life, you'd probably use more than just 2 workers,}
  \CommentTok{// and perhaps not put the master and worker in the same file.}
  \CommentTok{//}
  \CommentTok{// You can also of course get a bit fancier about logging, and}
  \CommentTok{// implement whatever custom logic you need to prevent DoS}
  \CommentTok{// attacks and other bad behavior.}
  \CommentTok{//}
  \CommentTok{// See the options in the cluster documentation.}
  \CommentTok{//}
  \CommentTok{// The important thing is that the master does very little,}
  \CommentTok{// increasing our resilience to unexpected errors.}

  \KeywordTok{cluster}\NormalTok{.}\FunctionTok{fork}\NormalTok{();}
  \KeywordTok{cluster}\NormalTok{.}\FunctionTok{fork}\NormalTok{();}

  \KeywordTok{cluster}\NormalTok{.}\FunctionTok{on}\NormalTok{(}\CharTok{'disconnect'}\NormalTok{, }\KeywordTok{function}\NormalTok{(worker) \{}
    \KeywordTok{console}\NormalTok{.}\FunctionTok{error}\NormalTok{(}\CharTok{'disconnect!'}\NormalTok{);}
    \KeywordTok{cluster}\NormalTok{.}\FunctionTok{fork}\NormalTok{();}
  \NormalTok{\});}

\NormalTok{\} }\KeywordTok{else} \NormalTok{\{}
  \CommentTok{// the worker}
  \CommentTok{//}
  \CommentTok{// This is where we put our bugs!}

  \KeywordTok{var} \NormalTok{domain = require(}\CharTok{'domain'}\NormalTok{);}

  \CommentTok{// See the cluster documentation for more details about using}
  \CommentTok{// worker processes to serve requests.  How it works, caveats, etc.}

  \KeywordTok{var} \NormalTok{server = require(}\CharTok{'http'}\NormalTok{).}\FunctionTok{createServer}\NormalTok{(}\KeywordTok{function}\NormalTok{(req, res) \{}
    \KeywordTok{var} \NormalTok{d = }\KeywordTok{domain}\NormalTok{.}\FunctionTok{create}\NormalTok{();}
    \KeywordTok{d}\NormalTok{.}\FunctionTok{on}\NormalTok{(}\CharTok{'error'}\NormalTok{, }\KeywordTok{function}\NormalTok{(er) \{}
      \KeywordTok{console}\NormalTok{.}\FunctionTok{error}\NormalTok{(}\CharTok{'error'}\NormalTok{, }\KeywordTok{er}\NormalTok{.}\FunctionTok{stack}\NormalTok{);}

      \CommentTok{// Note: we're in dangerous territory!}
      \CommentTok{// By definition, something unexpected occurred,}
      \CommentTok{// which we probably didn't want.}
      \CommentTok{// Anything can happen now!  Be very careful!}

      \KeywordTok{try} \NormalTok{\{}
        \CommentTok{// make sure we close down within 30 seconds}
        \KeywordTok{var} \NormalTok{killtimer = setTimeout(}\KeywordTok{function}\NormalTok{() \{}
          \KeywordTok{process}\NormalTok{.}\FunctionTok{exit}\NormalTok{(}\DecValTok{1}\NormalTok{);}
        \NormalTok{\}, }\DecValTok{30000}\NormalTok{);}
        \CommentTok{// But don't keep the process open just for that!}
        \KeywordTok{killtimer}\NormalTok{.}\FunctionTok{unref}\NormalTok{();}

        \CommentTok{// stop taking new requests.}
        \KeywordTok{server}\NormalTok{.}\FunctionTok{close}\NormalTok{();}

        \CommentTok{// Let the master know we're dead.  This will trigger a}
        \CommentTok{// 'disconnect' in the cluster master, and then it will fork}
        \CommentTok{// a new worker.}
        \KeywordTok{cluster.worker}\NormalTok{.}\FunctionTok{disconnect}\NormalTok{();}

        \CommentTok{// try to send an error to the request that triggered the problem}
        \KeywordTok{res}\NormalTok{.}\FunctionTok{statusCode} \NormalTok{= }\DecValTok{500}\NormalTok{;}
        \KeywordTok{res}\NormalTok{.}\FunctionTok{setHeader}\NormalTok{(}\CharTok{'content-type'}\NormalTok{, }\CharTok{'text/plain'}\NormalTok{);}
        \KeywordTok{res}\NormalTok{.}\FunctionTok{end}\NormalTok{(}\CharTok{'Oops, there was a problem!\textbackslash{}n'}\NormalTok{);}
      \NormalTok{\} }\KeywordTok{catch} \NormalTok{(er2) \{}
        \CommentTok{// oh well, not much we can do at this point.}
        \KeywordTok{console}\NormalTok{.}\FunctionTok{error}\NormalTok{(}\CharTok{'Error sending 500!'}\NormalTok{, }\KeywordTok{er2}\NormalTok{.}\FunctionTok{stack}\NormalTok{);}
      \NormalTok{\}}
    \NormalTok{\});}

    \CommentTok{// Because req and res were created before this domain existed,}
    \CommentTok{// we need to explicitly add them.}
    \CommentTok{// See the explanation of implicit vs explicit binding below.}
    \KeywordTok{d}\NormalTok{.}\FunctionTok{add}\NormalTok{(req);}
    \KeywordTok{d}\NormalTok{.}\FunctionTok{add}\NormalTok{(res);}

    \CommentTok{// Now run the handler function in the domain.}
    \KeywordTok{d}\NormalTok{.}\FunctionTok{run}\NormalTok{(}\KeywordTok{function}\NormalTok{() \{}
      \NormalTok{handleRequest(req, res);}
    \NormalTok{\});}
  \NormalTok{\});}
  \KeywordTok{server}\NormalTok{.}\FunctionTok{listen}\NormalTok{(PORT);}
\NormalTok{\}}

\CommentTok{// This part isn't important.  Just an example routing thing.}
\CommentTok{// You'd put your fancy application logic here.}
\KeywordTok{function} \NormalTok{handleRequest(req, res) \{}
  \KeywordTok{switch}\NormalTok{(}\KeywordTok{req}\NormalTok{.}\FunctionTok{url}\NormalTok{) \{}
    \KeywordTok{case} \CharTok{'/error'}\NormalTok{:}
      \CommentTok{// We do some async stuff, and then...}
      \NormalTok{setTimeout(}\KeywordTok{function}\NormalTok{() \{}
        \CommentTok{// Whoops!}
        \KeywordTok{flerb}\NormalTok{.}\FunctionTok{bark}\NormalTok{();}
      \NormalTok{\});}
      \KeywordTok{break}\NormalTok{;}
    \DataTypeTok{default}\NormalTok{:}
      \KeywordTok{res}\NormalTok{.}\FunctionTok{end}\NormalTok{(}\CharTok{'ok'}\NormalTok{);}
  \NormalTok{\}}
\NormalTok{\}}
\end{Highlighting}
\end{Shaded}

\subsection{Additions to Error objects}

Any time an Error object is routed through a domain, a few extra fields
are added to it.

\begin{itemize}
\item
  \texttt{error.domain} The domain that first handled the error.
\item
  \texttt{error.domainEmitter} The event emitter that emitted an `error'
  event with the error object.
\item
  \texttt{error.domainBound} The callback function which was bound to
  the domain, and passed an error as its first argument.
\item
  \texttt{error.domainThrown} A boolean indicating whether the error was
  thrown, emitted, or passed to a bound callback function.
\end{itemize}

\subsection{Implicit Binding}

If domains are in use, then all \textbf{new} EventEmitter objects
(including Stream objects, requests, responses, etc.) will be implicitly
bound to the active domain at the time of their creation.

Additionally, callbacks passed to lowlevel event loop requests (such as
to fs.open, or other callback-taking methods) will automatically be
bound to the active domain. If they throw, then the domain will catch
the error.

In order to prevent excessive memory usage, Domain objects themselves
are not implicitly added as children of the active domain. If they were,
then it would be too easy to prevent request and response objects from
being properly garbage collected.

If you \emph{want} to nest Domain objects as children of a parent
Domain, then you must explicitly add them.

Implicit binding routes thrown errors and \texttt{'error'} events to the
Domain's \texttt{error} event, but does not register the EventEmitter on
the Domain, so \texttt{domain.dispose()} will not shut down the
EventEmitter. Implicit binding only takes care of thrown errors and
\texttt{'error'} events.

\subsection{Explicit Binding}

Sometimes, the domain in use is not the one that ought to be used for a
specific event emitter. Or, the event emitter could have been created in
the context of one domain, but ought to instead be bound to some other
domain.

For example, there could be one domain in use for an HTTP server, but
perhaps we would like to have a separate domain to use for each request.

That is possible via explicit binding.

For example:

\begin{verbatim}
// create a top-level domain for the server
var serverDomain = domain.create();

serverDomain.run(function() {
  // server is created in the scope of serverDomain
  http.createServer(function(req, res) {
    // req and res are also created in the scope of serverDomain
    // however, we'd prefer to have a separate domain for each request.
    // create it first thing, and add req and res to it.
    var reqd = domain.create();
    reqd.add(req);
    reqd.add(res);
    reqd.on('error', function(er) {
      console.error('Error', er, req.url);
      try {
        res.writeHead(500);
        res.end('Error occurred, sorry.');
      } catch (er) {
        console.error('Error sending 500', er, req.url);
      }
    });
  }).listen(1337);
});
\end{verbatim}

\subsection{domain.create()}

\begin{itemize}
\item
  return: \{Domain\}
\end{itemize}

Returns a new Domain object.

\subsection{Class: Domain}

The Domain class encapsulates the functionality of routing errors and
uncaught exceptions to the active Domain object.

Domain is a child class of
\href{events.html\#events\_class\_events\_eventemitter}{EventEmitter}.
To handle the errors that it catches, listen to its \texttt{error}
event.

\subsubsection{domain.run(fn)}

\begin{itemize}
\item
  \texttt{fn} \{Function\}
\end{itemize}

Run the supplied function in the context of the domain, implicitly
binding all event emitters, timers, and lowlevel requests that are
created in that context.

This is the most basic way to use a domain.

Example:

\begin{verbatim}
var d = domain.create();
d.on('error', function(er) {
  console.error('Caught error!', er);
});
d.run(function() {
  process.nextTick(function() {
    setTimeout(function() { // simulating some various async stuff
      fs.open('non-existent file', 'r', function(er, fd) {
        if (er) throw er;
        // proceed...
      });
    }, 100);
  });
});
\end{verbatim}

In this example, the \texttt{d.on('error')} handler will be triggered,
rather than crashing the program.

\subsubsection{domain.members}

\begin{itemize}
\item
  \{Array\}
\end{itemize}

An array of timers and event emitters that have been explicitly added to
the domain.

\subsubsection{domain.add(emitter)}

\begin{itemize}
\item
  \texttt{emitter} \{EventEmitter \textbar{} Timer\} emitter or timer to
  be added to the domain
\end{itemize}

Explicitly adds an emitter to the domain. If any event handlers called
by the emitter throw an error, or if the emitter emits an \texttt{error}
event, it will be routed to the domain's \texttt{error} event, just like
with implicit binding.

This also works with timers that are returned from \texttt{setInterval}
and \texttt{setTimeout}. If their callback function throws, it will be
caught by the domain `error' handler.

If the Timer or EventEmitter was already bound to a domain, it is
removed from that one, and bound to this one instead.

\subsubsection{domain.remove(emitter)}

\begin{itemize}
\item
  \texttt{emitter} \{EventEmitter \textbar{} Timer\} emitter or timer to
  be removed from the domain
\end{itemize}

The opposite of \texttt{domain.add(emitter)}. Removes domain handling
from the specified emitter.

\subsubsection{domain.bind(callback)}

\begin{itemize}
\item
  \texttt{callback} \{Function\} The callback function
\item
  return: \{Function\} The bound function
\end{itemize}

The returned function will be a wrapper around the supplied callback
function. When the returned function is called, any errors that are
thrown will be routed to the domain's \texttt{error} event.

\paragraph{Example}

\begin{Shaded}
\begin{Highlighting}[]
\KeywordTok{var} \NormalTok{d = }\KeywordTok{domain}\NormalTok{.}\FunctionTok{create}\NormalTok{();}

\KeywordTok{function} \NormalTok{readSomeFile(filename, cb) \{}
  \KeywordTok{fs}\NormalTok{.}\FunctionTok{readFile}\NormalTok{(filename, }\CharTok{'utf8'}\NormalTok{, }\KeywordTok{d}\NormalTok{.}\FunctionTok{bind}\NormalTok{(}\KeywordTok{function}\NormalTok{(er, data) \{}
    \CommentTok{// if this throws, it will also be passed to the domain}
    \KeywordTok{return} \NormalTok{cb(er, data ? }\KeywordTok{JSON}\NormalTok{.}\FunctionTok{parse}\NormalTok{(data) : null);}
  \NormalTok{\}));}
\NormalTok{\}}

\KeywordTok{d}\NormalTok{.}\FunctionTok{on}\NormalTok{(}\CharTok{'error'}\NormalTok{, }\KeywordTok{function}\NormalTok{(er) \{}
  \CommentTok{// an error occurred somewhere.}
  \CommentTok{// if we throw it now, it will crash the program}
  \CommentTok{// with the normal line number and stack message.}
\NormalTok{\});}
\end{Highlighting}
\end{Shaded}

\subsubsection{domain.intercept(callback)}

\begin{itemize}
\item
  \texttt{callback} \{Function\} The callback function
\item
  return: \{Function\} The intercepted function
\end{itemize}

This method is almost identical to \texttt{domain.bind(callback)}.
However, in addition to catching thrown errors, it will also intercept
\texttt{Error} objects sent as the first argument to the function.

In this way, the common \texttt{if (er) return callback(er);} pattern
can be replaced with a single error handler in a single place.

\paragraph{Example}

\begin{Shaded}
\begin{Highlighting}[]
\KeywordTok{var} \NormalTok{d = }\KeywordTok{domain}\NormalTok{.}\FunctionTok{create}\NormalTok{();}

\KeywordTok{function} \NormalTok{readSomeFile(filename, cb) \{}
  \KeywordTok{fs}\NormalTok{.}\FunctionTok{readFile}\NormalTok{(filename, }\CharTok{'utf8'}\NormalTok{, }\KeywordTok{d}\NormalTok{.}\FunctionTok{intercept}\NormalTok{(}\KeywordTok{function}\NormalTok{(data) \{}
    \CommentTok{// note, the first argument is never passed to the}
    \CommentTok{// callback since it is assumed to be the 'Error' argument}
    \CommentTok{// and thus intercepted by the domain.}

    \CommentTok{// if this throws, it will also be passed to the domain}
    \CommentTok{// so the error-handling logic can be moved to the 'error'}
    \CommentTok{// event on the domain instead of being repeated throughout}
    \CommentTok{// the program.}
    \KeywordTok{return} \NormalTok{cb(null, }\KeywordTok{JSON}\NormalTok{.}\FunctionTok{parse}\NormalTok{(data));}
  \NormalTok{\}));}
\NormalTok{\}}

\KeywordTok{d}\NormalTok{.}\FunctionTok{on}\NormalTok{(}\CharTok{'error'}\NormalTok{, }\KeywordTok{function}\NormalTok{(er) \{}
  \CommentTok{// an error occurred somewhere.}
  \CommentTok{// if we throw it now, it will crash the program}
  \CommentTok{// with the normal line number and stack message.}
\NormalTok{\});}
\end{Highlighting}
\end{Shaded}

\subsubsection{domain.dispose()}

The dispose method destroys a domain, and makes a best effort attempt to
clean up any and all IO that is associated with the domain. Streams are
aborted, ended, closed, and/or destroyed. Timers are cleared. Explicitly
bound callbacks are no longer called. Any error events that are raised
as a result of this are ignored.

The intention of calling \texttt{dispose} is generally to prevent
cascading errors when a critical part of the Domain context is found to
be in an error state.

Once the domain is disposed the \texttt{dispose} event will emit.

Note that IO might still be performed. However, to the highest degree
possible, once a domain is disposed, further errors from the emitters in
that set will be ignored. So, even if some remaining actions are still
in flight, Node.js will not communicate further about them.
