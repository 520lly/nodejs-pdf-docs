\section{Assert}\label{assert}

\begin{Shaded}
\begin{Highlighting}[]
\NormalTok{Stability: }\DecValTok{5} \NormalTok{- Locked}
\end{Highlighting}
\end{Shaded}

This module is used for writing unit tests for your applications, you
can access it with
\texttt{require(\textquotesingle{}assert\textquotesingle{})}.

\subsection{assert.fail(actual, expected, message,
operator)}\label{assert.failactual-expected-message-operator}

Throws an exception that displays the values for \texttt{actual} and
\texttt{expected} separated by the provided operator.

\subsection{assert(value{[}, message{]}), assert.ok(value{[},
message{]})}\label{assertvalue-message-assert.okvalue-message}

Tests if value is truthy, it is equivalent to
\texttt{assert.equal(true,\ !!value,\ message);}

\subsection{assert.equal(actual, expected{[},
message{]})}\label{assert.equalactual-expected-message}

Tests shallow, coercive equality with the equal comparison operator (
\texttt{==} ).

\subsection{assert.notEqual(actual, expected{[},
message{]})}\label{assert.notequalactual-expected-message}

Tests shallow, coercive non-equality with the not equal comparison
operator ( \texttt{!=} ).

\subsection{assert.deepEqual(actual, expected{[},
message{]})}\label{assert.deepequalactual-expected-message}

Tests for deep equality.

\subsection{assert.notDeepEqual(actual, expected{[},
message{]})}\label{assert.notdeepequalactual-expected-message}

Tests for any deep inequality.

\subsection{assert.strictEqual(actual, expected{[},
message{]})}\label{assert.strictequalactual-expected-message}

Tests strict equality, as determined by the strict equality operator (
\texttt{===} )

\subsection{assert.notStrictEqual(actual, expected{[},
message{]})}\label{assert.notstrictequalactual-expected-message}

Tests strict non-equality, as determined by the strict not equal
operator ( \texttt{!==} )

\subsection{assert.throws(block{[}, error{]}{[},
message{]})}\label{assert.throwsblock-error-message}

Expects \texttt{block} to throw an error. \texttt{error} can be
constructor, \texttt{RegExp} or validation function.

Validate instanceof using constructor:

\begin{Shaded}
\begin{Highlighting}[]
\OtherTok{assert}\NormalTok{.}\FunctionTok{throws}\NormalTok{(}
  \KeywordTok{function}\NormalTok{() \{}
    \KeywordTok{throw} \KeywordTok{new} \FunctionTok{Error}\NormalTok{(}\StringTok{"Wrong value"}\NormalTok{);}
  \NormalTok{\},}
  \NormalTok{Error}
\NormalTok{);}
\end{Highlighting}
\end{Shaded}

Validate error message using RegExp:

\begin{Shaded}
\begin{Highlighting}[]
\OtherTok{assert}\NormalTok{.}\FunctionTok{throws}\NormalTok{(}
  \KeywordTok{function}\NormalTok{() \{}
    \KeywordTok{throw} \KeywordTok{new} \FunctionTok{Error}\NormalTok{(}\StringTok{"Wrong value"}\NormalTok{);}
  \NormalTok{\},}
  \OtherTok{/value/}
\NormalTok{);}
\end{Highlighting}
\end{Shaded}

Custom error validation:

\begin{Shaded}
\begin{Highlighting}[]
\OtherTok{assert}\NormalTok{.}\FunctionTok{throws}\NormalTok{(}
  \KeywordTok{function}\NormalTok{() \{}
    \KeywordTok{throw} \KeywordTok{new} \FunctionTok{Error}\NormalTok{(}\StringTok{"Wrong value"}\NormalTok{);}
  \NormalTok{\},}
  \KeywordTok{function}\NormalTok{(err) \{}
    \KeywordTok{if} \NormalTok{( (err }\KeywordTok{instanceof} \NormalTok{Error) && }\OtherTok{/value/}\NormalTok{.}\FunctionTok{test}\NormalTok{(err) ) \{}
      \KeywordTok{return} \KeywordTok{true}\NormalTok{;}
    \NormalTok{\}}
  \NormalTok{\},}
  \StringTok{"unexpected error"}
\NormalTok{);}
\end{Highlighting}
\end{Shaded}

\subsection{assert.doesNotThrow(block{[},
message{]})}\label{assert.doesnotthrowblock-message}

Expects \texttt{block} not to throw an error, see \texttt{assert.throws}
for details.

\subsection{assert.ifError(value)}\label{assert.iferrorvalue}

Tests if value is not a false value, throws if it is a true value.
Useful when testing the first argument, \texttt{error} in callbacks.
