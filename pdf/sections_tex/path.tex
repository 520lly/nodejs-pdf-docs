\section{Path}\label{path}

\begin{Shaded}
\begin{Highlighting}[]
\NormalTok{Stability: }\DecValTok{3} \NormalTok{- Stable}
\end{Highlighting}
\end{Shaded}

This module contains utilities for handling and transforming file paths.
Almost all these methods perform only string transformations. The file
system is not consulted to check whether paths are valid.

Use \texttt{require(\textquotesingle{}path\textquotesingle{})} to use
this module. The following methods are provided:

\subsection{path.normalize(p)}\label{path.normalizep}

Normalize a string path, taking care of
\texttt{\textquotesingle{}..\textquotesingle{}} and
\texttt{\textquotesingle{}.\textquotesingle{}} parts.

When multiple slashes are found, they're replaced by a single one; when
the path contains a trailing slash, it is preserved. On Windows
backslashes are used.

Example:

\begin{Shaded}
\begin{Highlighting}[]
\OtherTok{path}\NormalTok{.}\FunctionTok{normalize}\NormalTok{(}\StringTok{'/foo/bar//baz/asdf/quux/..'}\NormalTok{)}
\CommentTok{// returns}
\StringTok{'/foo/bar/baz/asdf'}
\end{Highlighting}
\end{Shaded}

\subsection{path.join({[}path1{]}{[}, path2{]}{[},
\ldots{}{]})}\label{path.joinpath1-path2}

Join all arguments together and normalize the resulting path.

Arguments must be strings. In v0.8, non-string arguments were silently
ignored. In v0.10 and up, an exception is thrown.

Example:

\begin{Shaded}
\begin{Highlighting}[]
\OtherTok{path}\NormalTok{.}\FunctionTok{join}\NormalTok{(}\StringTok{'/foo'}\NormalTok{, }\StringTok{'bar'}\NormalTok{, }\StringTok{'baz/asdf'}\NormalTok{, }\StringTok{'quux'}\NormalTok{, }\StringTok{'..'}\NormalTok{)}
\CommentTok{// returns}
\StringTok{'/foo/bar/baz/asdf'}

\OtherTok{path}\NormalTok{.}\FunctionTok{join}\NormalTok{(}\StringTok{'foo'}\NormalTok{, \{\}, }\StringTok{'bar'}\NormalTok{)}
\CommentTok{// throws exception}
\NormalTok{TypeError: Arguments to }\OtherTok{path}\NormalTok{.}\FunctionTok{join} \FunctionTok{must} \FunctionTok{be} \FunctionTok{strings}
\end{Highlighting}
\end{Shaded}

\subsection{path.resolve({[}from \ldots{}{]},
to)}\label{path.resolvefrom-to}

Resolves \texttt{to} to an absolute path.

If \texttt{to} isn't already absolute \texttt{from} arguments are
prepended in right to left order, until an absolute path is found. If
after using all \texttt{from} paths still no absolute path is found, the
current working directory is used as well. The resulting path is
normalized, and trailing slashes are removed unless the path gets
resolved to the root directory. Non-string \texttt{from} arguments are
ignored.

Another way to think of it is as a sequence of \texttt{cd} commands in a
shell.

\begin{Shaded}
\begin{Highlighting}[]
\OtherTok{path}\NormalTok{.}\FunctionTok{resolve}\NormalTok{(}\StringTok{'foo/bar'}\NormalTok{, }\StringTok{'/tmp/file/'}\NormalTok{, }\StringTok{'..'}\NormalTok{, }\StringTok{'a/../subfile'}\NormalTok{)}
\end{Highlighting}
\end{Shaded}

Is similar to:

\begin{Shaded}
\begin{Highlighting}[]
\NormalTok{cd foo/bar}
\NormalTok{cd /tmp/file/}
\OtherTok{cd} \NormalTok{..}
\NormalTok{cd a/../subfile}
\NormalTok{pwd}
\end{Highlighting}
\end{Shaded}

The difference is that the different paths don't need to exist and may
also be files.

Examples:

\begin{Shaded}
\begin{Highlighting}[]
\OtherTok{path}\NormalTok{.}\FunctionTok{resolve}\NormalTok{(}\StringTok{'/foo/bar'}\NormalTok{, }\StringTok{'./baz'}\NormalTok{)}
\CommentTok{// returns}
\StringTok{'/foo/bar/baz'}

\OtherTok{path}\NormalTok{.}\FunctionTok{resolve}\NormalTok{(}\StringTok{'/foo/bar'}\NormalTok{, }\StringTok{'/tmp/file/'}\NormalTok{)}
\CommentTok{// returns}
\StringTok{'/tmp/file'}

\OtherTok{path}\NormalTok{.}\FunctionTok{resolve}\NormalTok{(}\StringTok{'wwwroot'}\NormalTok{, }\StringTok{'static_files/png/'}\NormalTok{, }\StringTok{'../gif/image.gif'}\NormalTok{)}
\CommentTok{// if currently in /home/myself/node, it returns}
\StringTok{'/home/myself/node/wwwroot/static_files/gif/image.gif'}
\end{Highlighting}
\end{Shaded}

\subsection{path.isAbsolute(path)}\label{path.isabsolutepath}

Determines whether \texttt{path} is an absolute path. An absolute path
will always resolve to the same location, regardless of the working
directory.

Posix examples:

\begin{Shaded}
\begin{Highlighting}[]
\OtherTok{path}\NormalTok{.}\FunctionTok{isAbsolute}\NormalTok{(}\StringTok{'/foo/bar'}\NormalTok{) }\CommentTok{// true}
\OtherTok{path}\NormalTok{.}\FunctionTok{isAbsolute}\NormalTok{(}\StringTok{'/baz/..'}\NormalTok{)  }\CommentTok{// true}
\OtherTok{path}\NormalTok{.}\FunctionTok{isAbsolute}\NormalTok{(}\StringTok{'qux/'}\NormalTok{)     }\CommentTok{// false}
\OtherTok{path}\NormalTok{.}\FunctionTok{isAbsolute}\NormalTok{(}\StringTok{'.'}\NormalTok{)        }\CommentTok{// false}
\end{Highlighting}
\end{Shaded}

Windows examples:

\begin{Shaded}
\begin{Highlighting}[]
\OtherTok{path}\NormalTok{.}\FunctionTok{isAbsolute}\NormalTok{(}\StringTok{'//server'}\NormalTok{)  }\CommentTok{// true}
\OtherTok{path}\NormalTok{.}\FunctionTok{isAbsolute}\NormalTok{(}\StringTok{'C:/foo/..'}\NormalTok{) }\CommentTok{// true}
\OtherTok{path}\NormalTok{.}\FunctionTok{isAbsolute}\NormalTok{(}\StringTok{'bar}\CharTok{\textbackslash{}\textbackslash{}}\StringTok{baz'}\NormalTok{)   }\CommentTok{// false}
\OtherTok{path}\NormalTok{.}\FunctionTok{isAbsolute}\NormalTok{(}\StringTok{'.'}\NormalTok{)         }\CommentTok{// false}
\end{Highlighting}
\end{Shaded}

\subsection{path.relative(from, to)}\label{path.relativefrom-to}

Solve the relative path from \texttt{from} to \texttt{to}.

At times we have two absolute paths, and we need to derive the relative
path from one to the other. This is actually the reverse transform of
\texttt{path.resolve}, which means we see that:

\begin{Shaded}
\begin{Highlighting}[]
\OtherTok{path}\NormalTok{.}\FunctionTok{resolve}\NormalTok{(from, }\OtherTok{path}\NormalTok{.}\FunctionTok{relative}\NormalTok{(from, to)) == }\OtherTok{path}\NormalTok{.}\FunctionTok{resolve}\NormalTok{(to)}
\end{Highlighting}
\end{Shaded}

Examples:

\begin{Shaded}
\begin{Highlighting}[]
\OtherTok{path}\NormalTok{.}\FunctionTok{relative}\NormalTok{(}\StringTok{'C:}\CharTok{\textbackslash{}\textbackslash{}}\StringTok{orandea}\CharTok{\textbackslash{}\textbackslash{}}\StringTok{test}\CharTok{\textbackslash{}\textbackslash{}}\StringTok{aaa'}\NormalTok{, }\StringTok{'C:}\CharTok{\textbackslash{}\textbackslash{}}\StringTok{orandea}\CharTok{\textbackslash{}\textbackslash{}}\StringTok{impl}\CharTok{\textbackslash{}\textbackslash{}}\StringTok{bbb'}\NormalTok{)}
\CommentTok{// returns}
\StringTok{'..}\CharTok{\textbackslash{}\textbackslash{}}\StringTok{..}\CharTok{\textbackslash{}\textbackslash{}}\StringTok{impl}\CharTok{\textbackslash{}\textbackslash{}}\StringTok{bbb'}

\OtherTok{path}\NormalTok{.}\FunctionTok{relative}\NormalTok{(}\StringTok{'/data/orandea/test/aaa'}\NormalTok{, }\StringTok{'/data/orandea/impl/bbb'}\NormalTok{)}
\CommentTok{// returns}
\StringTok{'../../impl/bbb'}
\end{Highlighting}
\end{Shaded}

\subsection{path.dirname(p)}\label{path.dirnamep}

Return the directory name of a path. Similar to the Unix
\texttt{dirname} command.

Example:

\begin{Shaded}
\begin{Highlighting}[]
\OtherTok{path}\NormalTok{.}\FunctionTok{dirname}\NormalTok{(}\StringTok{'/foo/bar/baz/asdf/quux'}\NormalTok{)}
\CommentTok{// returns}
\StringTok{'/foo/bar/baz/asdf'}
\end{Highlighting}
\end{Shaded}

\subsection{path.basename(p{[}, ext{]})}\label{path.basenamep-ext}

Return the last portion of a path. Similar to the Unix \texttt{basename}
command.

Example:

\begin{Shaded}
\begin{Highlighting}[]
\OtherTok{path}\NormalTok{.}\FunctionTok{basename}\NormalTok{(}\StringTok{'/foo/bar/baz/asdf/quux.html'}\NormalTok{)}
\CommentTok{// returns}
\StringTok{'quux.html'}

\OtherTok{path}\NormalTok{.}\FunctionTok{basename}\NormalTok{(}\StringTok{'/foo/bar/baz/asdf/quux.html'}\NormalTok{, }\StringTok{'.html'}\NormalTok{)}
\CommentTok{// returns}
\StringTok{'quux'}
\end{Highlighting}
\end{Shaded}

\subsection{path.extname(p)}\label{path.extnamep}

Return the extension of the path, from the last `.' to end of string in
the last portion of the path. If there is no `.' in the last portion of
the path or the first character of it is `.', then it returns an empty
string. Examples:

\begin{Shaded}
\begin{Highlighting}[]
\OtherTok{path}\NormalTok{.}\FunctionTok{extname}\NormalTok{(}\StringTok{'index.html'}\NormalTok{)}
\CommentTok{// returns}
\StringTok{'.html'}

\OtherTok{path}\NormalTok{.}\FunctionTok{extname}\NormalTok{(}\StringTok{'index.coffee.md'}\NormalTok{)}
\CommentTok{// returns}
\StringTok{'.md'}

\OtherTok{path}\NormalTok{.}\FunctionTok{extname}\NormalTok{(}\StringTok{'index.'}\NormalTok{)}
\CommentTok{// returns}
\StringTok{'.'}

\OtherTok{path}\NormalTok{.}\FunctionTok{extname}\NormalTok{(}\StringTok{'index'}\NormalTok{)}
\CommentTok{// returns}
\StringTok{''}
\end{Highlighting}
\end{Shaded}

\subsection{path.sep}\label{path.sep}

The platform-specific file separator.
\texttt{\textquotesingle{}\textbackslash{}\textbackslash{}\textquotesingle{}}
or \texttt{\textquotesingle{}/\textquotesingle{}}.

An example on *nix:

\begin{Shaded}
\begin{Highlighting}[]
\StringTok{'foo/bar/baz'}\NormalTok{.}\FunctionTok{split}\NormalTok{(}\OtherTok{path}\NormalTok{.}\FunctionTok{sep}\NormalTok{)}
\CommentTok{// returns}
\NormalTok{[}\StringTok{'foo'}\NormalTok{, }\StringTok{'bar'}\NormalTok{, }\StringTok{'baz'}\NormalTok{]}
\end{Highlighting}
\end{Shaded}

An example on Windows:

\begin{Shaded}
\begin{Highlighting}[]
\StringTok{'foo}\CharTok{\textbackslash{}\textbackslash{}}\StringTok{bar}\CharTok{\textbackslash{}\textbackslash{}}\StringTok{baz'}\NormalTok{.}\FunctionTok{split}\NormalTok{(}\OtherTok{path}\NormalTok{.}\FunctionTok{sep}\NormalTok{)}
\CommentTok{// returns}
\NormalTok{[}\StringTok{'foo'}\NormalTok{, }\StringTok{'bar'}\NormalTok{, }\StringTok{'baz'}\NormalTok{]}
\end{Highlighting}
\end{Shaded}

\subsection{path.delimiter}\label{path.delimiter}

The platform-specific path delimiter, \texttt{;} or
\texttt{\textquotesingle{}:\textquotesingle{}}.

An example on *nix:

\begin{Shaded}
\begin{Highlighting}[]
\OtherTok{console}\NormalTok{.}\FunctionTok{log}\NormalTok{(}\OtherTok{process}\NormalTok{.}\OtherTok{env}\NormalTok{.}\FunctionTok{PATH}\NormalTok{)}
\CommentTok{// '/usr/bin:/bin:/usr/sbin:/sbin:/usr/local/bin'}

\OtherTok{process}\NormalTok{.}\OtherTok{env}\NormalTok{.}\OtherTok{PATH}\NormalTok{.}\FunctionTok{split}\NormalTok{(}\OtherTok{path}\NormalTok{.}\FunctionTok{delimiter}\NormalTok{)}
\CommentTok{// returns}
\NormalTok{[}\StringTok{'/usr/bin'}\NormalTok{, }\StringTok{'/bin'}\NormalTok{, }\StringTok{'/usr/sbin'}\NormalTok{, }\StringTok{'/sbin'}\NormalTok{, }\StringTok{'/usr/local/bin'}\NormalTok{]}
\end{Highlighting}
\end{Shaded}

An example on Windows:

\begin{Shaded}
\begin{Highlighting}[]
\OtherTok{console}\NormalTok{.}\FunctionTok{log}\NormalTok{(}\OtherTok{process}\NormalTok{.}\OtherTok{env}\NormalTok{.}\FunctionTok{PATH}\NormalTok{)}
\CommentTok{// 'C:\textbackslash{}Windows\textbackslash{}system32;C:\textbackslash{}Windows;C:\textbackslash{}Program Files\textbackslash{}nodejs\textbackslash{}'}

\OtherTok{process}\NormalTok{.}\OtherTok{env}\NormalTok{.}\OtherTok{PATH}\NormalTok{.}\FunctionTok{split}\NormalTok{(}\OtherTok{path}\NormalTok{.}\FunctionTok{delimiter}\NormalTok{)}
\CommentTok{// returns}
\NormalTok{[}\StringTok{'C:\textbackslash{}Windows\textbackslash{}system32'}\NormalTok{, }\StringTok{'C:\textbackslash{}Windows'}\NormalTok{, }\StringTok{'C:\textbackslash{}Program Files}\CharTok{\textbackslash{}n}\StringTok{odejs}\CharTok{\textbackslash{}'}\StringTok{]}
\end{Highlighting}
\end{Shaded}

\subsection{path.parse(pathString)}\label{path.parsepathstring}

Returns an object from a path string.

An example on *nix:

\begin{Shaded}
\begin{Highlighting}[]
\OtherTok{path}\NormalTok{.}\FunctionTok{parse}\NormalTok{(}\StringTok{'/home/user/dir/file.txt'}\NormalTok{)}
\CommentTok{// returns}
\NormalTok{\{}
    \DataTypeTok{root }\NormalTok{: }\StringTok{"/"}\NormalTok{,}
    \DataTypeTok{dir }\NormalTok{: }\StringTok{"/home/user/dir"}\NormalTok{,}
    \DataTypeTok{base }\NormalTok{: }\StringTok{"file.txt"}\NormalTok{,}
    \DataTypeTok{ext }\NormalTok{: }\StringTok{".txt"}\NormalTok{,}
    \DataTypeTok{name }\NormalTok{: }\StringTok{"file"}
\NormalTok{\}}
\end{Highlighting}
\end{Shaded}

An example on Windows:

\begin{Shaded}
\begin{Highlighting}[]
\OtherTok{path}\NormalTok{.}\FunctionTok{parse}\NormalTok{(}\StringTok{'C:}\CharTok{\textbackslash{}\textbackslash{}}\StringTok{path}\CharTok{\textbackslash{}\textbackslash{}}\StringTok{dir}\CharTok{\textbackslash{}\textbackslash{}}\StringTok{index.html'}\NormalTok{)}
\CommentTok{// returns}
\NormalTok{\{}
    \DataTypeTok{root }\NormalTok{: }\StringTok{"C:}\CharTok{\textbackslash{}"}\StringTok{,}
    \DataTypeTok{dir }\NormalTok{: }\StringTok{"C:\textbackslash{}path\textbackslash{}dir"}\NormalTok{,}
    \DataTypeTok{base }\NormalTok{: }\StringTok{"index.html"}\NormalTok{,}
    \DataTypeTok{ext }\NormalTok{: }\StringTok{".html"}\NormalTok{,}
    \DataTypeTok{name }\NormalTok{: }\StringTok{"index"}
\NormalTok{\}}
\end{Highlighting}
\end{Shaded}

\subsection{path.format(pathObject)}\label{path.formatpathobject}

Returns a path string from an object, the opposite of
\texttt{path.parse} above.

\begin{Shaded}
\begin{Highlighting}[]
\OtherTok{path}\NormalTok{.}\FunctionTok{format}\NormalTok{(\{}
    \DataTypeTok{root }\NormalTok{: }\StringTok{"/"}\NormalTok{,}
    \DataTypeTok{dir }\NormalTok{: }\StringTok{"/home/user/dir"}\NormalTok{,}
    \DataTypeTok{base }\NormalTok{: }\StringTok{"file.txt"}\NormalTok{,}
    \DataTypeTok{ext }\NormalTok{: }\StringTok{".txt"}\NormalTok{,}
    \DataTypeTok{name }\NormalTok{: }\StringTok{"file"}
\NormalTok{\})}
\CommentTok{// returns}
\StringTok{'/home/user/dir/file.txt'}
\end{Highlighting}
\end{Shaded}

\subsection{path.posix}\label{path.posix}

Provide access to aforementioned \texttt{path} methods but always
interact in a posix compatible way.

\subsection{path.win32}\label{path.win32}

Provide access to aforementioned \texttt{path} methods but always
interact in a win32 compatible way.
