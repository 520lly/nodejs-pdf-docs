\section{Executing JavaScript}\label{executing-javascript}

\begin{Shaded}
\begin{Highlighting}[]
\NormalTok{Stability: }\DecValTok{3} \NormalTok{- Stable}
\end{Highlighting}
\end{Shaded}

You can access this module with:

\begin{Shaded}
\begin{Highlighting}[]
\KeywordTok{var} \NormalTok{vm = }\FunctionTok{require}\NormalTok{(}\StringTok{'vm'}\NormalTok{);}
\end{Highlighting}
\end{Shaded}

JavaScript code can be compiled and run immediately or compiled, saved,
and run later.

\subsection{vm.runInThisContext(code{[},
options{]})}\label{vm.runinthiscontextcode-options}

\texttt{vm.runInThisContext()} compiles \texttt{code}, runs it and
returns the result. Running code does not have access to local scope,
but does have access to the current \texttt{global} object.

Example of using \texttt{vm.runInThisContext} and \texttt{eval} to run
the same code:

\begin{Shaded}
\begin{Highlighting}[]
\KeywordTok{var} \NormalTok{localVar = }\StringTok{'initial value'}\NormalTok{;}

\KeywordTok{var} \NormalTok{vmResult = }\OtherTok{vm}\NormalTok{.}\FunctionTok{runInThisContext}\NormalTok{(}\StringTok{'localVar = "vm";'}\NormalTok{);}
\OtherTok{console}\NormalTok{.}\FunctionTok{log}\NormalTok{(}\StringTok{'vmResult: '}\NormalTok{, vmResult);}
\OtherTok{console}\NormalTok{.}\FunctionTok{log}\NormalTok{(}\StringTok{'localVar: '}\NormalTok{, localVar);}

\KeywordTok{var} \NormalTok{evalResult = }\FunctionTok{eval}\NormalTok{(}\StringTok{'localVar = "eval";'}\NormalTok{);}
\OtherTok{console}\NormalTok{.}\FunctionTok{log}\NormalTok{(}\StringTok{'evalResult: '}\NormalTok{, evalResult);}
\OtherTok{console}\NormalTok{.}\FunctionTok{log}\NormalTok{(}\StringTok{'localVar: '}\NormalTok{, localVar);}

\CommentTok{// vmResult: 'vm', localVar: 'initial value'}
\CommentTok{// evalResult: 'eval', localVar: 'eval'}
\end{Highlighting}
\end{Shaded}

\texttt{vm.runInThisContext} does not have access to the local scope, so
\texttt{localVar} is unchanged. \texttt{eval} does have access to the
local scope, so \texttt{localVar} is changed.

In this way \texttt{vm.runInThisContext} is much like an
\href{http://es5.github.io/\#x10.4.2}{indirect \texttt{eval} call}, e.g.
\texttt{(0,eval)(\textquotesingle{}code\textquotesingle{})}. However, it
also has the following additional options:

\begin{itemize}
\itemsep1pt\parskip0pt\parsep0pt
\item
  \texttt{filename}: allows you to control the filename that shows up in
  any stack traces produced.
\item
  \texttt{displayErrors}: whether or not to print any errors to stderr,
  with the line of code that caused them highlighted, before throwing an
  exception. Will capture both syntax errors from compiling
  \texttt{code} and runtime errors thrown by executing the compiled
  code. Defaults to \texttt{true}.
\item
  \texttt{timeout}: a number of milliseconds to execute \texttt{code}
  before terminating execution. If execution is terminated, an
  \texttt{Error} will be thrown.
\end{itemize}

\subsection{vm.createContext({[}sandbox{]})}\label{vm.createcontextsandbox}

If given a \texttt{sandbox} object, will ``contextify'' that sandbox so
that it can be used in calls to \texttt{vm.runInContext} or
\texttt{script.runInContext}. Inside scripts run as such,
\texttt{sandbox} will be the global object, retaining all its existing
properties but also having the built-in objects and functions any
standard \href{http://es5.github.io/\#x15.1}{global object} has. Outside
of scripts run by the vm module, \texttt{sandbox} will be unchanged.

If not given a sandbox object, returns a new, empty contextified sandbox
object you can use.

This function is useful for creating a sandbox that can be used to run
multiple scripts, e.g.~if you were emulating a web browser it could be
used to create a single sandbox representing a window's global object,
then run all \texttt{\textless{}script\textgreater{}} tags together
inside that sandbox.

\subsection{vm.isContext(sandbox)}\label{vm.iscontextsandbox}

Returns whether or not a sandbox object has been contextified by calling
\texttt{vm.createContext} on it.

\subsection{vm.runInContext(code, contextifiedSandbox{[},
options{]})}\label{vm.runincontextcode-contextifiedsandbox-options}

\texttt{vm.runInContext} compiles \texttt{code}, then runs it in
\texttt{contextifiedSandbox} and returns the result. Running code does
not have access to local scope. The \texttt{contextifiedSandbox} object
must have been previously contextified via \texttt{vm.createContext}; it
will be used as the global object for \texttt{code}.

\texttt{vm.runInContext} takes the same options as
\texttt{vm.runInThisContext}.

Example: compile and execute different scripts in a single existing
context.

\begin{Shaded}
\begin{Highlighting}[]
\KeywordTok{var} \NormalTok{util = }\FunctionTok{require}\NormalTok{(}\StringTok{'util'}\NormalTok{);}
\KeywordTok{var} \NormalTok{vm = }\FunctionTok{require}\NormalTok{(}\StringTok{'vm'}\NormalTok{);}

\KeywordTok{var} \NormalTok{sandbox = \{ }\DataTypeTok{globalVar}\NormalTok{: }\DecValTok{1} \NormalTok{\};}
\OtherTok{vm}\NormalTok{.}\FunctionTok{createContext}\NormalTok{(sandbox);}

\KeywordTok{for} \NormalTok{(}\KeywordTok{var} \NormalTok{i = }\DecValTok{0}\NormalTok{; i < }\DecValTok{10}\NormalTok{; ++i) \{}
    \OtherTok{vm}\NormalTok{.}\FunctionTok{runInContext}\NormalTok{(}\StringTok{'globalVar *= 2;'}\NormalTok{, sandbox);}
\NormalTok{\}}
\OtherTok{console}\NormalTok{.}\FunctionTok{log}\NormalTok{(}\OtherTok{util}\NormalTok{.}\FunctionTok{inspect}\NormalTok{(sandbox));}

\CommentTok{// \{ globalVar: 1024 \}}
\end{Highlighting}
\end{Shaded}

Note that running untrusted code is a tricky business requiring great
care. \texttt{vm.runInContext} is quite useful, but safely running
untrusted code requires a separate process.

\subsection{vm.runInNewContext(code{[}, sandbox{]}{[},
options{]})}\label{vm.runinnewcontextcode-sandbox-options}

\texttt{vm.runInNewContext} compiles \texttt{code}, contextifies
\texttt{sandbox} if passed or creates a new contextified sandbox if it's
omitted, and then runs the code with the sandbox as the global object
and returns the result.

\texttt{vm.runInNewContext} takes the same options as
\texttt{vm.runInThisContext}.

Example: compile and execute code that increments a global variable and
sets a new one. These globals are contained in the sandbox.

\begin{Shaded}
\begin{Highlighting}[]
\KeywordTok{var} \NormalTok{util = }\FunctionTok{require}\NormalTok{(}\StringTok{'util'}\NormalTok{);}
\KeywordTok{var} \NormalTok{vm = }\FunctionTok{require}\NormalTok{(}\StringTok{'vm'}\NormalTok{),}

\KeywordTok{var} \NormalTok{sandbox = \{}
  \DataTypeTok{animal}\NormalTok{: }\StringTok{'cat'}\NormalTok{,}
  \DataTypeTok{count}\NormalTok{: }\DecValTok{2}
\NormalTok{\};}

\OtherTok{vm}\NormalTok{.}\FunctionTok{runInNewContext}\NormalTok{(}\StringTok{'count += 1; name = "kitty"'}\NormalTok{, sandbox);}
\OtherTok{console}\NormalTok{.}\FunctionTok{log}\NormalTok{(}\OtherTok{util}\NormalTok{.}\FunctionTok{inspect}\NormalTok{(sandbox));}

\CommentTok{// \{ animal: 'cat', count: 3, name: 'kitty' \}}
\end{Highlighting}
\end{Shaded}

Note that running untrusted code is a tricky business requiring great
care. \texttt{vm.runInNewContext} is quite useful, but safely running
untrusted code requires a separate process.

\subsection{vm.runInDebugContext(code)}\label{vm.runindebugcontextcode}

\texttt{vm.runInDebugContext} compiles and executes \texttt{code} inside
the V8 debug context. The primary use case is to get access to the V8
debug object:

\begin{Shaded}
\begin{Highlighting}[]
\KeywordTok{var} \NormalTok{Debug = }\OtherTok{vm}\NormalTok{.}\FunctionTok{runInDebugContext}\NormalTok{(}\StringTok{'Debug'}\NormalTok{);}
\OtherTok{Debug}\NormalTok{.}\FunctionTok{scripts}\NormalTok{().}\FunctionTok{forEach}\NormalTok{(}\KeywordTok{function}\NormalTok{(script) \{ }\OtherTok{console}\NormalTok{.}\FunctionTok{log}\NormalTok{(}\OtherTok{script}\NormalTok{.}\FunctionTok{name}\NormalTok{); \});}
\end{Highlighting}
\end{Shaded}

Note that the debug context and object are intrinsically tied to V8's
debugger implementation and may change (or even get removed) without
prior warning.

The debug object can also be exposed with the
\texttt{-\/-expose\_debug\_as=} switch.

\subsection{Class: Script}\label{class-script}

A class for holding precompiled scripts, and running them in specific
sandboxes.

\subsubsection{new vm.Script(code,
options)}\label{new-vm.scriptcode-options}

Creating a new \texttt{Script} compiles \texttt{code} but does not run
it. Instead, the created \texttt{vm.Script} object represents this
compiled code. This script can be run later many times using methods
below. The returned script is not bound to any global object. It is
bound before each run, just for that run.

The options when creating a script are:

\begin{itemize}
\itemsep1pt\parskip0pt\parsep0pt
\item
  \texttt{filename}: allows you to control the filename that shows up in
  any stack traces produced from this script.
\item
  \texttt{displayErrors}: whether or not to print any errors to stderr,
  with the line of code that caused them highlighted, before throwing an
  exception. Applies only to syntax errors compiling the code; errors
  while running the code are controlled by the options to the script's
  methods.
\end{itemize}

\subsubsection{script.runInThisContext({[}options{]})}\label{script.runinthiscontextoptions}

Similar to \texttt{vm.runInThisContext} but a method of a precompiled
\texttt{Script} object. \texttt{script.runInThisContext} runs
\texttt{script}'s compiled code and returns the result. Running code
does not have access to local scope, but does have access to the current
\texttt{global} object.

Example of using \texttt{script.runInThisContext} to compile code once
and run it multiple times:

\begin{Shaded}
\begin{Highlighting}[]
\KeywordTok{var} \NormalTok{vm = }\FunctionTok{require}\NormalTok{(}\StringTok{'vm'}\NormalTok{);}

\OtherTok{global}\NormalTok{.}\FunctionTok{globalVar} \NormalTok{= }\DecValTok{0}\NormalTok{;}

\KeywordTok{var} \NormalTok{script = }\KeywordTok{new} \OtherTok{vm}\NormalTok{.}\FunctionTok{Script}\NormalTok{(}\StringTok{'globalVar += 1'}\NormalTok{, \{ }\DataTypeTok{filename}\NormalTok{: }\StringTok{'myfile.vm'} \NormalTok{\});}

\KeywordTok{for} \NormalTok{(}\KeywordTok{var} \NormalTok{i = }\DecValTok{0}\NormalTok{; i < }\DecValTok{1000}\NormalTok{; ++i) \{}
  \OtherTok{script}\NormalTok{.}\FunctionTok{runInThisContext}\NormalTok{();}
\NormalTok{\}}

\OtherTok{console}\NormalTok{.}\FunctionTok{log}\NormalTok{(globalVar);}

\CommentTok{// 1000}
\end{Highlighting}
\end{Shaded}

The options for running a script are:

\begin{itemize}
\itemsep1pt\parskip0pt\parsep0pt
\item
  \texttt{displayErrors}: whether or not to print any runtime errors to
  stderr, with the line of code that caused them highlighted, before
  throwing an exception. Applies only to runtime errors executing the
  code; it is impossible to create a \texttt{Script} instance with
  syntax errors, as the constructor will throw.
\item
  \texttt{timeout}: a number of milliseconds to execute the script
  before terminating execution. If execution is terminated, an
  \texttt{Error} will be thrown.
\end{itemize}

\subsubsection{script.runInContext(contextifiedSandbox{[},
options{]})}\label{script.runincontextcontextifiedsandbox-options}

Similar to \texttt{vm.runInContext} but a method of a precompiled
\texttt{Script} object. \texttt{script.runInContext} runs
\texttt{script}'s compiled code in \texttt{contextifiedSandbox} and
returns the result. Running code does not have access to local scope.

\texttt{script.runInContext} takes the same options as
\texttt{script.runInThisContext}.

Example: compile code that increments a global variable and sets one,
then execute the code multiple times. These globals are contained in the
sandbox.

\begin{Shaded}
\begin{Highlighting}[]
\KeywordTok{var} \NormalTok{util = }\FunctionTok{require}\NormalTok{(}\StringTok{'util'}\NormalTok{);}
\KeywordTok{var} \NormalTok{vm = }\FunctionTok{require}\NormalTok{(}\StringTok{'vm'}\NormalTok{);}

\KeywordTok{var} \NormalTok{sandbox = \{}
  \DataTypeTok{animal}\NormalTok{: }\StringTok{'cat'}\NormalTok{,}
  \DataTypeTok{count}\NormalTok{: }\DecValTok{2}
\NormalTok{\};}

\KeywordTok{var} \NormalTok{script = }\KeywordTok{new} \OtherTok{vm}\NormalTok{.}\FunctionTok{Script}\NormalTok{(}\StringTok{'count += 1; name = "kitty"'}\NormalTok{);}

\KeywordTok{for} \NormalTok{(}\KeywordTok{var} \NormalTok{i = }\DecValTok{0}\NormalTok{; i < }\DecValTok{10}\NormalTok{; ++i) \{}
  \OtherTok{script}\NormalTok{.}\FunctionTok{runInContext}\NormalTok{(sandbox);}
\NormalTok{\}}

\OtherTok{console}\NormalTok{.}\FunctionTok{log}\NormalTok{(}\OtherTok{util}\NormalTok{.}\FunctionTok{inspect}\NormalTok{(sandbox));}

\CommentTok{// \{ animal: 'cat', count: 12, name: 'kitty' \}}
\end{Highlighting}
\end{Shaded}

Note that running untrusted code is a tricky business requiring great
care. \texttt{script.runInContext} is quite useful, but safely running
untrusted code requires a separate process.

\subsubsection{script.runInNewContext({[}sandbox{]}{[},
options{]})}\label{script.runinnewcontextsandbox-options}

Similar to \texttt{vm.runInNewContext} but a method of a precompiled
\texttt{Script} object. \texttt{script.runInNewContext} contextifies
\texttt{sandbox} if passed or creates a new contextified sandbox if it's
omitted, and then runs \texttt{script}'s compiled code with the sandbox
as the global object and returns the result. Running code does not have
access to local scope.

\texttt{script.runInNewContext} takes the same options as
\texttt{script.runInThisContext}.

Example: compile code that sets a global variable, then execute the code
multiple times in different contexts. These globals are set on and
contained in the sandboxes.

\begin{Shaded}
\begin{Highlighting}[]
\KeywordTok{var} \NormalTok{util = }\FunctionTok{require}\NormalTok{(}\StringTok{'util'}\NormalTok{);}
\KeywordTok{var} \NormalTok{vm = }\FunctionTok{require}\NormalTok{(}\StringTok{'vm'}\NormalTok{);}

\KeywordTok{var} \NormalTok{sandboxes = [\{\}, \{\}, \{\}];}

\KeywordTok{var} \NormalTok{script = }\KeywordTok{new} \OtherTok{vm}\NormalTok{.}\FunctionTok{Script}\NormalTok{(}\StringTok{'globalVar = "set"'}\NormalTok{);}

\OtherTok{sandboxes}\NormalTok{.}\FunctionTok{forEach}\NormalTok{(}\KeywordTok{function} \NormalTok{(sandbox) \{}
  \OtherTok{script}\NormalTok{.}\FunctionTok{runInNewContext}\NormalTok{(sandbox);}
\NormalTok{\});}

\OtherTok{console}\NormalTok{.}\FunctionTok{log}\NormalTok{(}\OtherTok{util}\NormalTok{.}\FunctionTok{inspect}\NormalTok{(sandboxes));}

\CommentTok{// [\{ globalVar: 'set' \}, \{ globalVar: 'set' \}, \{ globalVar: 'set' \}]}
\end{Highlighting}
\end{Shaded}

Note that running untrusted code is a tricky business requiring great
care. \texttt{script.runInNewContext} is quite useful, but safely
running untrusted code requires a separate process.
