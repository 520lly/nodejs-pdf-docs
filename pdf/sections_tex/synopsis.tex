\section{Synopsis}\label{synopsis}

An example of a \href{http.html}{web server} written with Node which
responds with `Hello World':

\begin{Shaded}
\begin{Highlighting}[]
\KeywordTok{var} \NormalTok{http = }\FunctionTok{require}\NormalTok{(}\StringTok{'http'}\NormalTok{);}

\OtherTok{http}\NormalTok{.}\FunctionTok{createServer}\NormalTok{(}\KeywordTok{function} \NormalTok{(request, response) \{}
  \OtherTok{response}\NormalTok{.}\FunctionTok{writeHead}\NormalTok{(}\DecValTok{200}\NormalTok{, \{}\StringTok{'Content-Type'}\NormalTok{: }\StringTok{'text/plain'}\NormalTok{\});}
  \OtherTok{response}\NormalTok{.}\FunctionTok{end}\NormalTok{(}\StringTok{'Hello World}\CharTok{\textbackslash{}n}\StringTok{'}\NormalTok{);}
\NormalTok{\}).}\FunctionTok{listen}\NormalTok{(}\DecValTok{8124}\NormalTok{);}

\OtherTok{console}\NormalTok{.}\FunctionTok{log}\NormalTok{(}\StringTok{'Server running at http://127.0.0.1:8124/'}\NormalTok{);}
\end{Highlighting}
\end{Shaded}

To run the server, put the code into a file called \texttt{example.js}
and execute it with the node program

\begin{Shaded}
\begin{Highlighting}[]
\NormalTok{> node }\OtherTok{example}\NormalTok{.}\FunctionTok{js}
\NormalTok{Server running at http:}\CommentTok{//127.0.0.1:8124/}
\end{Highlighting}
\end{Shaded}

All of the examples in the documentation can be run similarly.
