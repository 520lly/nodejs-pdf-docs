\section{Tracing}\label{tracing}

\begin{Shaded}
\begin{Highlighting}[]
\NormalTok{Stability: }\DecValTok{1} \NormalTok{- Experimental}
\end{Highlighting}
\end{Shaded}

The tracing module is designed for instrumenting your Node application.
It is not meant for general purpose use.

\textbf{\emph{Be very careful with callbacks used in conjunction with
this module}}

Many of these callbacks interact directly with asynchronous subsystems
in a synchronous fashion. That is to say, you may be in a callback where
a call to \texttt{console.log()} could result in an infinite recursive
loop. Also of note, many of these callbacks are in hot execution code
paths. That is to say your callbacks are executed quite often in the
normal operation of Node, so be wary of doing CPU bound or synchronous
workloads in these functions. Consider a ring buffer and a timer to
defer processing.

\texttt{require('tracing')} to use this module.

\subsection{v8}\label{v8}

The \texttt{v8} property is an
\href{events.html\#events_class_events_eventemitter}{EventEmitter}, it
exposes events and interfaces specific to the version of \texttt{v8}
built with node. These interfaces are subject to change by upstream and
are therefore not covered under the stability index.

\subsubsection{Event: `gc'}\label{event-gc}

\texttt{function (before, after) \{ \}}

Emitted each time a GC run is completed.

\texttt{before} and \texttt{after} are objects with the following
properties:

\begin{verbatim}
{
  type: 'mark-sweep-compact',
  flags: 0,
  timestamp: 905535650119053,
  total_heap_size: 6295040,
  total_heap_size_executable: 4194304,
  total_physical_size: 6295040,
  used_heap_size: 2855416,
  heap_size_limit: 1535115264
}
\end{verbatim}

\subsubsection{getHeapStatistics()}\label{getheapstatistics}

Returns an object with the following properties

\begin{verbatim}
{
  total_heap_size: 7326976,
  total_heap_size_executable: 4194304,
  total_physical_size: 7326976,
  used_heap_size: 3476208,
  heap_size_limit: 1535115264
}
\end{verbatim}

\section{Async Listeners}\label{async-listeners}

The \texttt{AsyncListener} API is the JavaScript interface for the
\texttt{AsyncWrap} class which allows developers to be notified about
key events in the lifetime of an asynchronous event. Node performs a lot
of asynchronous events internally, and significant use of this API may
have a \textbf{significant performance impact} on your application.

\subsection{tracing.createAsyncListener(callbacksObj{[},
userData{]})}\label{tracing.createasynclistenercallbacksobj-userdata}

\begin{itemize}
\itemsep1pt\parskip0pt\parsep0pt
\item
  \texttt{callbacksObj} \{Object\} Contains optional callbacks that will
  fire at specific times in the life cycle of the asynchronous event.
\item
  \texttt{userData} \{Value\} a value that will be passed to all
  callbacks.
\end{itemize}

Returns a constructed \texttt{AsyncListener} object.

To begin capturing asynchronous events pass either the
\texttt{callbacksObj} or pass an existing \texttt{AsyncListener}
instance to
\hyperref[tracingux5ftracingux5faddasynclistenerux5fasynclistener]{\texttt{tracing.addAsyncListener()}}.
The same \texttt{AsyncListener} instance can only be added once to the
active queue, and subsequent attempts to add the instance will be
ignored.

To stop capturing pass the \texttt{AsyncListener} instance to
\hyperref[tracingux5ftracingux5fremoveasynclistenerux5fasynclistener]{\texttt{tracing.removeAsyncListener()}}.
This does \emph{not} mean the \texttt{AsyncListener} previously added
will stop triggering callbacks. Once attached to an asynchronous event
it will persist with the lifetime of the asynchronous call stack.

Explanation of function parameters:

\texttt{callbacksObj}: An \texttt{Object} which may contain several
optional fields:

\begin{itemize}
\item
  \texttt{create(userData)}: A \texttt{Function} called when an
  asynchronous event is instantiated. If a \texttt{Value} is returned
  then it will be attached to the event and overwrite any value that had
  been passed to \texttt{tracing.createAsyncListener()}'s
  \texttt{userData} argument. If an initial \texttt{userData} was passed
  when created, then \texttt{create()} will receive that as a function
  argument.
\item
  \texttt{before(context, userData)}: A \texttt{Function} that is called
  immediately before the asynchronous callback is about to run. It will
  be passed both the \texttt{context} (i.e. \texttt{this}) of the
  calling function and the \texttt{userData} either returned from
  \texttt{create()} or passed during construction (if either occurred).
\item
  \texttt{after(context, userData)}: A \texttt{Function} called
  immediately after the asynchronous event's callback has run. Note this
  will not be called if the callback throws and the error is not
  handled.
\item
  \texttt{error(userData, error)}: A \texttt{Function} called if the
  event's callback threw. If this registered callback returns
  \texttt{true} then Node will assume the error has been properly
  handled and resume execution normally. When multiple \texttt{error()}
  callbacks have been registered only \textbf{one} of those callbacks
  needs to return \texttt{true} for \texttt{AsyncListener} to accept
  that the error has been handled, but all \texttt{error()} callbacks
  will always be run.
\end{itemize}

\texttt{userData}: A \texttt{Value} (i.e.~anything) that will be, by
default, attached to all new event instances. This will be overwritten
if a \texttt{Value} is returned by \texttt{create()}.

Here is an example of overwriting the \texttt{userData}:

\begin{Shaded}
\begin{Highlighting}[]
\OtherTok{tracing}\NormalTok{.}\FunctionTok{createAsyncListener}\NormalTok{(\{}
  \DataTypeTok{create}\NormalTok{: }\KeywordTok{function} \FunctionTok{listener}\NormalTok{(value) \{}
    \CommentTok{// value === true}
    \KeywordTok{return} \KeywordTok{false}\NormalTok{;}
\NormalTok{\}, \{}
  \DataTypeTok{before}\NormalTok{: }\KeywordTok{function} \FunctionTok{before}\NormalTok{(context, value) \{}
    \CommentTok{// value === false}
  \NormalTok{\}}
\NormalTok{\}, }\KeywordTok{true}\NormalTok{);}
\end{Highlighting}
\end{Shaded}

\textbf{Note:} The
\href{events.html\#events_class_events_eventemitter}{EventEmitter},
while used to emit status of an asynchronous event, is not itself
asynchronous. So \texttt{create()} will not fire when an event is added,
and \texttt{before()}/\texttt{after()} will not fire when emitted
callbacks are called.

\subsection{tracing.addAsyncListener(callbacksObj{[},
userData{]})}\label{tracing.addasynclistenercallbacksobj-userdata}

\subsection{tracing.addAsyncListener(asyncListener)}\label{tracing.addasynclistenerasynclistener}

Returns a constructed \texttt{AsyncListener} object and immediately adds
it to the listening queue to begin capturing asynchronous events.

Function parameters can either be the same as
\hyperref[tracingux5ftracingux5fcreateasynclistenerux5fasynclistenerux5fcallbacksobjux5fstoragevalue]{\texttt{tracing.createAsyncListener()}},
or a constructed \texttt{AsyncListener} object.

Example usage for capturing errors:

\begin{Shaded}
\begin{Highlighting}[]
\KeywordTok{var} \NormalTok{fs = }\FunctionTok{require}\NormalTok{(}\StringTok{'fs'}\NormalTok{);}

\KeywordTok{var} \NormalTok{cntr = }\DecValTok{0}\NormalTok{;}
\KeywordTok{var} \NormalTok{key = }\OtherTok{tracing}\NormalTok{.}\FunctionTok{addAsyncListener}\NormalTok{(\{}
  \DataTypeTok{create}\NormalTok{: }\KeywordTok{function} \FunctionTok{onCreate}\NormalTok{() \{}
    \KeywordTok{return} \NormalTok{\{ }\DataTypeTok{uid}\NormalTok{: cntr++ \};}
  \NormalTok{\},}
  \DataTypeTok{before}\NormalTok{: }\KeywordTok{function} \FunctionTok{onBefore}\NormalTok{(context, storage) \{}
    \CommentTok{// Write directly to stdout or we'll enter a recursive loop}
    \OtherTok{fs}\NormalTok{.}\FunctionTok{writeSync}\NormalTok{(}\DecValTok{1}\NormalTok{, }\StringTok{'uid: '} \NormalTok{+ }\OtherTok{storage}\NormalTok{.}\FunctionTok{uid} \NormalTok{+ }\StringTok{' is about to run}\CharTok{\textbackslash{}n}\StringTok{'}\NormalTok{);}
  \NormalTok{\},}
  \DataTypeTok{after}\NormalTok{: }\KeywordTok{function} \FunctionTok{onAfter}\NormalTok{(context, storage) \{}
    \OtherTok{fs}\NormalTok{.}\FunctionTok{writeSync}\NormalTok{(}\DecValTok{1}\NormalTok{, }\StringTok{'uid: '} \NormalTok{+ }\OtherTok{storage}\NormalTok{.}\FunctionTok{uid} \NormalTok{+ }\StringTok{' ran}\CharTok{\textbackslash{}n}\StringTok{'}\NormalTok{);}
  \NormalTok{\},}
  \DataTypeTok{error}\NormalTok{: }\KeywordTok{function} \FunctionTok{onError}\NormalTok{(storage, err) \{}
    \CommentTok{// Handle known errors}
    \KeywordTok{if} \NormalTok{(}\OtherTok{err}\NormalTok{.}\FunctionTok{message} \NormalTok{=== }\StringTok{'everything is fine'}\NormalTok{) \{}
      \CommentTok{// Writing to stderr this time.}
      \OtherTok{fs}\NormalTok{.}\FunctionTok{writeSync}\NormalTok{(}\DecValTok{2}\NormalTok{, }\StringTok{'handled error just threw:}\CharTok{\textbackslash{}n}\StringTok{'}\NormalTok{);}
      \OtherTok{fs}\NormalTok{.}\FunctionTok{writeSync}\NormalTok{(}\DecValTok{2}\NormalTok{, }\OtherTok{err}\NormalTok{.}\FunctionTok{stack} \NormalTok{+ }\StringTok{'}\CharTok{\textbackslash{}n}\StringTok{'}\NormalTok{);}
      \KeywordTok{return} \KeywordTok{true}\NormalTok{;}
    \NormalTok{\}}
  \NormalTok{\}}
\NormalTok{\});}

\OtherTok{process}\NormalTok{.}\FunctionTok{nextTick}\NormalTok{(}\KeywordTok{function}\NormalTok{() \{}
  \KeywordTok{throw} \KeywordTok{new} \FunctionTok{Error}\NormalTok{(}\StringTok{'everything is fine'}\NormalTok{);}
\NormalTok{\});}

\CommentTok{// Output:}
\CommentTok{// uid: 0 is about to run}
\CommentTok{// handled error just threw:}
\CommentTok{// Error: really, it's ok}
\CommentTok{//     at /tmp/test2.js:27:9}
\CommentTok{//     at process._tickCallback (node.js:583:11)}
\CommentTok{//     at Function.Module.runMain (module.js:492:11)}
\CommentTok{//     at startup (node.js:123:16)}
\CommentTok{//     at node.js:1012:3}
\end{Highlighting}
\end{Shaded}

\subsection{tracing.removeAsyncListener(asyncListener)}\label{tracing.removeasynclistenerasynclistener}

Removes the \texttt{AsyncListener} from the listening queue.

Removing the \texttt{AsyncListener} from the active queue does
\emph{not} mean the \texttt{asyncListener} callbacks will cease to fire
on the events they've been registered. Subsequently, any asynchronous
events fired during the execution of a callback will also have the same
\texttt{asyncListener} callbacks attached for future execution. For
example:

\begin{Shaded}
\begin{Highlighting}[]
\KeywordTok{var} \NormalTok{fs = }\FunctionTok{require}\NormalTok{(}\StringTok{'fs'}\NormalTok{);}

\KeywordTok{var} \NormalTok{key = }\OtherTok{tracing}\NormalTok{.}\FunctionTok{createAsyncListener}\NormalTok{(\{}
  \DataTypeTok{create}\NormalTok{: }\KeywordTok{function} \FunctionTok{asyncListener}\NormalTok{() \{}
    \CommentTok{// Write directly to stdout or we'll enter a recursive loop}
    \OtherTok{fs}\NormalTok{.}\FunctionTok{writeSync}\NormalTok{(}\DecValTok{1}\NormalTok{, }\StringTok{'You summoned me?}\CharTok{\textbackslash{}n}\StringTok{'}\NormalTok{);}
  \NormalTok{\}}
\NormalTok{\});}

\CommentTok{// We want to begin capturing async events some time in the future.}
\FunctionTok{setTimeout}\NormalTok{(}\KeywordTok{function}\NormalTok{() \{}
  \OtherTok{tracing}\NormalTok{.}\FunctionTok{addAsyncListener}\NormalTok{(key);}

  \CommentTok{// Perform a few additional async events.}
  \FunctionTok{setTimeout}\NormalTok{(}\KeywordTok{function}\NormalTok{() \{}
    \FunctionTok{setImmediate}\NormalTok{(}\KeywordTok{function}\NormalTok{() \{}
      \OtherTok{process}\NormalTok{.}\FunctionTok{nextTick}\NormalTok{(}\KeywordTok{function}\NormalTok{() \{ \});}
    \NormalTok{\});}
  \NormalTok{\});}

  \CommentTok{// Removing the listener doesn't mean to stop capturing events that}
  \CommentTok{// have already been added.}
  \OtherTok{tracing}\NormalTok{.}\FunctionTok{removeAsyncListener}\NormalTok{(key);}
\NormalTok{\}, }\DecValTok{100}\NormalTok{);}

\CommentTok{// Output:}
\CommentTok{// You summoned me?}
\CommentTok{// You summoned me?}
\CommentTok{// You summoned me?}
\CommentTok{// You summoned me?}
\end{Highlighting}
\end{Shaded}

The fact that we logged 4 asynchronous events is an implementation
detail of Node's \href{timers.html}{Timers}.

To stop capturing from a specific asynchronous event stack
\texttt{tracing.removeAsyncListener()} must be called from within the
call stack itself. For example:

\begin{Shaded}
\begin{Highlighting}[]
\KeywordTok{var} \NormalTok{fs = }\FunctionTok{require}\NormalTok{(}\StringTok{'fs'}\NormalTok{);}

\KeywordTok{var} \NormalTok{key = }\OtherTok{tracing}\NormalTok{.}\FunctionTok{createAsyncListener}\NormalTok{(\{}
  \DataTypeTok{create}\NormalTok{: }\KeywordTok{function} \FunctionTok{asyncListener}\NormalTok{() \{}
    \CommentTok{// Write directly to stdout or we'll enter a recursive loop}
    \OtherTok{fs}\NormalTok{.}\FunctionTok{writeSync}\NormalTok{(}\DecValTok{1}\NormalTok{, }\StringTok{'You summoned me?}\CharTok{\textbackslash{}n}\StringTok{'}\NormalTok{);}
  \NormalTok{\}}
\NormalTok{\});}

\CommentTok{// We want to begin capturing async events some time in the future.}
\FunctionTok{setTimeout}\NormalTok{(}\KeywordTok{function}\NormalTok{() \{}
  \OtherTok{tracing}\NormalTok{.}\FunctionTok{addAsyncListener}\NormalTok{(key);}

  \CommentTok{// Perform a few additional async events.}
  \FunctionTok{setImmediate}\NormalTok{(}\KeywordTok{function}\NormalTok{() \{}
    \CommentTok{// Stop capturing from this call stack.}
    \OtherTok{tracing}\NormalTok{.}\FunctionTok{removeAsyncListener}\NormalTok{(key);}

    \OtherTok{process}\NormalTok{.}\FunctionTok{nextTick}\NormalTok{(}\KeywordTok{function}\NormalTok{() \{ \});}
  \NormalTok{\});}
\NormalTok{\}, }\DecValTok{100}\NormalTok{);}

\CommentTok{// Output:}
\CommentTok{// You summoned me?}
\end{Highlighting}
\end{Shaded}

The user must be explicit and always pass the \texttt{AsyncListener}
they wish to remove. It is not possible to simply remove all listeners
at once.
