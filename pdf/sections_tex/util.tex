\section{util}

\begin{Shaded}
\begin{Highlighting}[]
\DataTypeTok{Stability}\NormalTok{: }\DecValTok{5} \NormalTok{- Locked}
\end{Highlighting}
\end{Shaded}

These functions are in the module \texttt{'util'}. Use
\texttt{require('util')} to access them.

\subsection{util.format(format, {[}\ldots{}{]})}

Returns a formatted string using the first argument as a
\texttt{printf}-like format.

The first argument is a string that contains zero or more
\emph{placeholders}. Each placeholder is replaced with the converted
value from its corresponding argument. Supported placeholders are:

\begin{itemize}
\item
  \texttt{\%s} - String.
\item
  \texttt{\%d} - Number (both integer and float).
\item
  \texttt{\%j} - JSON.
\item
  \texttt{\%\%} - single percent sign (\texttt{'\%'}). This does not
  consume an argument.
\end{itemize}

If the placeholder does not have a corresponding argument, the
placeholder is not replaced.

\begin{Shaded}
\begin{Highlighting}[]
\KeywordTok{util}\NormalTok{.}\FunctionTok{format}\NormalTok{(}\CharTok{'%s:%s'}\NormalTok{, }\CharTok{'foo'}\NormalTok{); }\CommentTok{// 'foo:%s'}
\end{Highlighting}
\end{Shaded}

If there are more arguments than placeholders, the extra arguments are
converted to strings with \texttt{util.inspect()} and these strings are
concatenated, delimited by a space.

\begin{Shaded}
\begin{Highlighting}[]
\KeywordTok{util}\NormalTok{.}\FunctionTok{format}\NormalTok{(}\CharTok{'%s:%s'}\NormalTok{, }\CharTok{'foo'}\NormalTok{, }\CharTok{'bar'}\NormalTok{, }\CharTok{'baz'}\NormalTok{); }\CommentTok{// 'foo:bar baz'}
\end{Highlighting}
\end{Shaded}

If the first argument is not a format string then \texttt{util.format()}
returns a string that is the concatenation of all its arguments
separated by spaces. Each argument is converted to a string with
\texttt{util.inspect()}.

\begin{Shaded}
\begin{Highlighting}[]
\KeywordTok{util}\NormalTok{.}\FunctionTok{format}\NormalTok{(}\DecValTok{1}\NormalTok{, }\DecValTok{2}\NormalTok{, }\DecValTok{3}\NormalTok{); }\CommentTok{// '1 2 3'}
\end{Highlighting}
\end{Shaded}

\subsection{util.debug(string)}

A synchronous output function. Will block the process and output
\texttt{string} immediately to \texttt{stderr}.

\begin{Shaded}
\begin{Highlighting}[]
\NormalTok{require(}\CharTok{'util'}\NormalTok{).}\FunctionTok{debug}\NormalTok{(}\CharTok{'message on stderr'}\NormalTok{);}
\end{Highlighting}
\end{Shaded}

\subsection{util.error({[}\ldots{}{]})}

Same as \texttt{util.debug()} except this will output all arguments
immediately to \texttt{stderr}.

\subsection{util.puts({[}\ldots{}{]})}

A synchronous output function. Will block the process and output all
arguments to \texttt{stdout} with newlines after each argument.

\subsection{util.print({[}\ldots{}{]})}

A synchronous output function. Will block the process, cast each
argument to a string then output to \texttt{stdout}. Does not place
newlines after each argument.

\subsection{util.log(string)}

Output with timestamp on \texttt{stdout}.

\begin{Shaded}
\begin{Highlighting}[]
\NormalTok{require(}\CharTok{'util'}\NormalTok{).}\FunctionTok{log}\NormalTok{(}\CharTok{'Timestamped message.'}\NormalTok{);}
\end{Highlighting}
\end{Shaded}

\subsection{util.inspect(object, {[}options{]})}

Return a string representation of \texttt{object}, which is useful for
debugging.

An optional \emph{options} object may be passed that alters certain
aspects of the formatted string:

\begin{itemize}
\item
  \texttt{showHidden} - if \texttt{true} then the object's
  non-enumerable properties will be shown too. Defaults to
  \texttt{false}.
\item
  \texttt{depth} - tells \texttt{inspect} how many times to recurse
  while formatting the object. This is useful for inspecting large
  complicated objects. Defaults to \texttt{2}. To make it recurse
  indefinitely pass \texttt{null}.
\item
  \texttt{colors} - if \texttt{true}, then the output will be styled
  with ANSI color codes. Defaults to \texttt{false}. Colors are
  customizable, see below.
\item
  \texttt{customInspect} - if \texttt{false}, then custom
  \texttt{inspect()} functions defined on the objects being inspected
  won't be called. Defaults to \texttt{true}.
\end{itemize}

Example of inspecting all properties of the \texttt{util} object:

\begin{Shaded}
\begin{Highlighting}[]
\KeywordTok{var} \NormalTok{util = require(}\CharTok{'util'}\NormalTok{);}

\KeywordTok{console}\NormalTok{.}\FunctionTok{log}\NormalTok{(}\KeywordTok{util}\NormalTok{.}\FunctionTok{inspect}\NormalTok{(util, \{ }\DataTypeTok{showHidden}\NormalTok{: }\KeywordTok{true}\NormalTok{, }\DataTypeTok{depth}\NormalTok{: null \}));}
\end{Highlighting}
\end{Shaded}

\subsubsection{Customizing \texttt{util.inspect} colors}

Color output (if enabled) of \texttt{util.inspect} is customizable
globally via \texttt{util.inspect.styles} and
\texttt{util.inspect.colors} objects.

\texttt{util.inspect.styles} is a map assigning each style a color from
\texttt{util.inspect.colors}. Highlighted styles and their default
values are: * \texttt{number} (yellow) * \texttt{boolean} (yellow) *
\texttt{string} (green) * \texttt{date} (magenta) * \texttt{regexp}
(red) * \texttt{null} (bold) * \texttt{undefined} (grey) *
\texttt{special} - only function at this time (cyan) * \texttt{name}
(intentionally no styling)

Predefined color codes are: \texttt{white}, \texttt{grey},
\texttt{black}, \texttt{blue}, \texttt{cyan}, \texttt{green},
\texttt{magenta}, \texttt{red} and \texttt{yellow}. There are also
\texttt{bold}, \texttt{italic}, \texttt{underline} and \texttt{inverse}
codes.

\subsubsection{Custom \texttt{inspect()} function on Objects}

Objects also may define their own \texttt{inspect(depth)} function which
\texttt{util.inspect()} will invoke and use the result of when
inspecting the object:

\begin{Shaded}
\begin{Highlighting}[]
\KeywordTok{var} \NormalTok{util = require(}\CharTok{'util'}\NormalTok{);}

\KeywordTok{var} \NormalTok{obj = \{ }\DataTypeTok{name}\NormalTok{: }\CharTok{'nate'} \NormalTok{\};}
\KeywordTok{obj}\NormalTok{.}\FunctionTok{inspect} \NormalTok{= }\KeywordTok{function}\NormalTok{(depth) \{}
  \KeywordTok{return} \CharTok{'\{'} \NormalTok{+ }\KeywordTok{this}\NormalTok{.}\FunctionTok{name} \NormalTok{+ }\CharTok{'\}'}\NormalTok{;}
\NormalTok{\};}

\KeywordTok{util}\NormalTok{.}\FunctionTok{inspect}\NormalTok{(obj);}
  \CommentTok{// "\{nate\}"}
\end{Highlighting}
\end{Shaded}

You may also return another Object entirely, and the returned String
will be formatted according to the returned Object. This is similar to
how \texttt{JSON.stringify()} works:

\begin{Shaded}
\begin{Highlighting}[]
\KeywordTok{var} \NormalTok{obj = \{ }\DataTypeTok{foo}\NormalTok{: }\CharTok{'this will not show up in the inspect() output'} \NormalTok{\};}
\KeywordTok{obj}\NormalTok{.}\FunctionTok{inspect} \NormalTok{= }\KeywordTok{function}\NormalTok{(depth) \{}
  \KeywordTok{return} \NormalTok{\{ }\DataTypeTok{bar}\NormalTok{: }\CharTok{'baz'} \NormalTok{\};}
\NormalTok{\};}

\KeywordTok{util}\NormalTok{.}\FunctionTok{inspect}\NormalTok{(obj);}
  \CommentTok{// "\{ bar: 'baz' \}"}
\end{Highlighting}
\end{Shaded}

\subsection{util.isArray(object)}

Returns \texttt{true} if the given ``object'' is an \texttt{Array}.
\texttt{false} otherwise.

\begin{Shaded}
\begin{Highlighting}[]
\KeywordTok{var} \NormalTok{util = require(}\CharTok{'util'}\NormalTok{);}

\KeywordTok{util}\NormalTok{.}\FunctionTok{isArray}\NormalTok{([])}
  \CommentTok{// true}
\KeywordTok{util}\NormalTok{.}\FunctionTok{isArray}\NormalTok{(}\KeywordTok{new} \KeywordTok{Array}\NormalTok{)}
  \CommentTok{// true}
\KeywordTok{util}\NormalTok{.}\FunctionTok{isArray}\NormalTok{(\{\})}
  \CommentTok{// false}
\end{Highlighting}
\end{Shaded}

\subsection{util.isRegExp(object)}

Returns \texttt{true} if the given ``object'' is a \texttt{RegExp}.
\texttt{false} otherwise.

\begin{Shaded}
\begin{Highlighting}[]
\KeywordTok{var} \NormalTok{util = require(}\CharTok{'util'}\NormalTok{);}

\KeywordTok{util}\NormalTok{.}\FunctionTok{isRegExp}\NormalTok{(}\OtherTok{/}\FloatTok{some regexp}\OtherTok{/}\NormalTok{)}
  \CommentTok{// true}
\KeywordTok{util}\NormalTok{.}\FunctionTok{isRegExp}\NormalTok{(}\KeywordTok{new} \KeywordTok{RegExp}\NormalTok{(}\CharTok{'another regexp'}\NormalTok{))}
  \CommentTok{// true}
\KeywordTok{util}\NormalTok{.}\FunctionTok{isRegExp}\NormalTok{(\{\})}
  \CommentTok{// false}
\end{Highlighting}
\end{Shaded}

\subsection{util.isDate(object)}

Returns \texttt{true} if the given ``object'' is a \texttt{Date}.
\texttt{false} otherwise.

\begin{Shaded}
\begin{Highlighting}[]
\KeywordTok{var} \NormalTok{util = require(}\CharTok{'util'}\NormalTok{);}

\KeywordTok{util}\NormalTok{.}\FunctionTok{isDate}\NormalTok{(}\KeywordTok{new} \KeywordTok{Date}\NormalTok{())}
  \CommentTok{// true}
\KeywordTok{util}\NormalTok{.}\FunctionTok{isDate}\NormalTok{(}\KeywordTok{Date}\NormalTok{())}
  \CommentTok{// false (without 'new' returns a String)}
\KeywordTok{util}\NormalTok{.}\FunctionTok{isDate}\NormalTok{(\{\})}
  \CommentTok{// false}
\end{Highlighting}
\end{Shaded}

\subsection{util.isError(object)}

Returns \texttt{true} if the given ``object'' is an \texttt{Error}.
\texttt{false} otherwise.

\begin{Shaded}
\begin{Highlighting}[]
\KeywordTok{var} \NormalTok{util = require(}\CharTok{'util'}\NormalTok{);}

\KeywordTok{util}\NormalTok{.}\FunctionTok{isError}\NormalTok{(}\KeywordTok{new} \NormalTok{Error())}
  \CommentTok{// true}
\KeywordTok{util}\NormalTok{.}\FunctionTok{isError}\NormalTok{(}\KeywordTok{new} \NormalTok{TypeError())}
  \CommentTok{// true}
\KeywordTok{util}\NormalTok{.}\FunctionTok{isError}\NormalTok{(\{ }\DataTypeTok{name}\NormalTok{: }\CharTok{'Error'}\NormalTok{, }\DataTypeTok{message}\NormalTok{: }\CharTok{'an error occurred'} \NormalTok{\})}
  \CommentTok{// false}
\end{Highlighting}
\end{Shaded}

\subsection{util.pump(readableStream, writableStream, {[}callback{]})}

\begin{Shaded}
\begin{Highlighting}[]
\DataTypeTok{Stability}\NormalTok{: }\DecValTok{0} \NormalTok{- }\DataTypeTok{Deprecated}\NormalTok{: Use }\KeywordTok{readableStream}\NormalTok{.}\FunctionTok{pipe}\NormalTok{(writableStream)}
\end{Highlighting}
\end{Shaded}

Read the data from \texttt{readableStream} and send it to the
\texttt{writableStream}. When \texttt{writableStream.write(data)}
returns \texttt{false} \texttt{readableStream} will be paused until the
\texttt{drain} event occurs on the \texttt{writableStream}.
\texttt{callback} gets an error as its only argument and is called when
\texttt{writableStream} is closed or when an error occurs.

\subsection{util.inherits(constructor, superConstructor)}

Inherit the prototype methods from one
\href{https://developer.mozilla.org/en/JavaScript/Reference/Global\_Objects/Object/constructor}{constructor}
into another. The prototype of \texttt{constructor} will be set to a new
object created from \texttt{superConstructor}.

As an additional convenience, \texttt{superConstructor} will be
accessible through the \texttt{constructor.super\_} property.

\begin{Shaded}
\begin{Highlighting}[]
\KeywordTok{var} \NormalTok{util = require(}\StringTok{"util"}\NormalTok{);}
\KeywordTok{var} \NormalTok{events = require(}\StringTok{"events"}\NormalTok{);}

\KeywordTok{function} \NormalTok{MyStream() \{}
    \KeywordTok{events.EventEmitter}\NormalTok{.}\FunctionTok{call}\NormalTok{(}\KeywordTok{this}\NormalTok{);}
\NormalTok{\}}

\KeywordTok{util}\NormalTok{.}\FunctionTok{inherits}\NormalTok{(MyStream, }\KeywordTok{events}\NormalTok{.}\FunctionTok{EventEmitter}\NormalTok{);}

\KeywordTok{MyStream.prototype}\NormalTok{.}\FunctionTok{write} \NormalTok{= }\KeywordTok{function}\NormalTok{(data) \{}
    \KeywordTok{this}\NormalTok{.}\FunctionTok{emit}\NormalTok{(}\StringTok{"data"}\NormalTok{, data);}
\NormalTok{\}}

\KeywordTok{var} \NormalTok{stream = }\KeywordTok{new} \NormalTok{MyStream();}

\KeywordTok{console}\NormalTok{.}\FunctionTok{log}\NormalTok{(stream instanceof }\KeywordTok{events}\NormalTok{.}\FunctionTok{EventEmitter}\NormalTok{); }\CommentTok{// true}
\KeywordTok{console}\NormalTok{.}\FunctionTok{log}\NormalTok{(}\KeywordTok{MyStream}\NormalTok{.}\FunctionTok{super}\NormalTok{_ === }\KeywordTok{events}\NormalTok{.}\FunctionTok{EventEmitter}\NormalTok{); }\CommentTok{// true}

\KeywordTok{stream}\NormalTok{.}\FunctionTok{on}\NormalTok{(}\StringTok{"data"}\NormalTok{, }\KeywordTok{function}\NormalTok{(data) \{}
    \KeywordTok{console}\NormalTok{.}\FunctionTok{log}\NormalTok{(}\CharTok{'Received data: "'} \NormalTok{+ data + }\CharTok{'"'}\NormalTok{);}
\NormalTok{\})}
\KeywordTok{stream}\NormalTok{.}\FunctionTok{write}\NormalTok{(}\StringTok{"It works!"}\NormalTok{); }\CommentTok{// Received data: "It works!"}
\end{Highlighting}
\end{Shaded}

