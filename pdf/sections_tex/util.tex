\section{util}\label{util}

\begin{Shaded}
\begin{Highlighting}[]
\NormalTok{Stability: }\DecValTok{4} \NormalTok{- API Frozen}
\end{Highlighting}
\end{Shaded}

These functions are in the module
\texttt{\textquotesingle{}util\textquotesingle{}}. Use
\texttt{require(\textquotesingle{}util\textquotesingle{})} to access
them.

The \texttt{util} module is primarily designed to support the needs of
Node's internal APIs. Many of these utilities are useful for your own
programs. If you find that these functions are lacking for your
purposes, however, you are encouraged to write your own utilities. We
are not interested in any future additions to the \texttt{util} module
that are unnecessary for Node's internal functionality.

\subsection{util.debuglog(section)}\label{util.debuglogsection}

\begin{itemize}
\itemsep1pt\parskip0pt\parsep0pt
\item
  \texttt{section} \{String\} The section of the program to be debugged
\item
  Returns: \{Function\} The logging function
\end{itemize}

This is used to create a function which conditionally writes to stderr
based on the existence of a \texttt{NODE\_DEBUG} environment variable.
If the \texttt{section} name appears in that environment variable, then
the returned function will be similar to \texttt{console.error()}. If
not, then the returned function is a no-op.

For example:

\begin{Shaded}
\begin{Highlighting}[]
\KeywordTok{var} \NormalTok{debuglog = }\OtherTok{util}\NormalTok{.}\FunctionTok{debuglog}\NormalTok{(}\StringTok{'foo'}\NormalTok{);}

\KeywordTok{var} \NormalTok{bar = }\DecValTok{123}\NormalTok{;}
\FunctionTok{debuglog}\NormalTok{(}\StringTok{'hello from foo [%d]'}\NormalTok{, bar);}
\end{Highlighting}
\end{Shaded}

If this program is run with \texttt{NODE\_DEBUG=foo} in the environment,
then it will output something like:

\begin{Shaded}
\begin{Highlighting}[]
\NormalTok{FOO }\DecValTok{3245}\NormalTok{: hello from foo [}\DecValTok{123}\NormalTok{]}
\end{Highlighting}
\end{Shaded}

where \texttt{3245} is the process id. If it is not run with that
environment variable set, then it will not print anything.

You may separate multiple \texttt{NODE\_DEBUG} environment variables
with a comma. For example, \texttt{NODE\_DEBUG=fs,net,tls}.

\subsection{util.format(format{[},
\ldots{}{]})}\label{util.formatformat}

Returns a formatted string using the first argument as a
\texttt{printf}-like format.

The first argument is a string that contains zero or more
\emph{placeholders}. Each placeholder is replaced with the converted
value from its corresponding argument. Supported placeholders are:

\begin{itemize}
\itemsep1pt\parskip0pt\parsep0pt
\item
  \texttt{\%s} - String.
\item
  \texttt{\%d} - Number (both integer and float).
\item
  \texttt{\%j} - JSON. Replaced with the string
  \texttt{\textquotesingle{}{[}Circular{]}\textquotesingle{}} if the
  argument contains circular references.
\item
  \texttt{\%\%} - single percent sign
  (\texttt{\textquotesingle{}\%\textquotesingle{}}). This does not
  consume an argument.
\end{itemize}

If the placeholder does not have a corresponding argument, the
placeholder is not replaced.

\begin{Shaded}
\begin{Highlighting}[]
\OtherTok{util}\NormalTok{.}\FunctionTok{format}\NormalTok{(}\StringTok{'%s:%s'}\NormalTok{, }\StringTok{'foo'}\NormalTok{); }\CommentTok{// 'foo:%s'}
\end{Highlighting}
\end{Shaded}

If there are more arguments than placeholders, the extra arguments are
converted to strings with \texttt{util.inspect()} and these strings are
concatenated, delimited by a space.

\begin{Shaded}
\begin{Highlighting}[]
\OtherTok{util}\NormalTok{.}\FunctionTok{format}\NormalTok{(}\StringTok{'%s:%s'}\NormalTok{, }\StringTok{'foo'}\NormalTok{, }\StringTok{'bar'}\NormalTok{, }\StringTok{'baz'}\NormalTok{); }\CommentTok{// 'foo:bar baz'}
\end{Highlighting}
\end{Shaded}

If the first argument is not a format string then \texttt{util.format()}
returns a string that is the concatenation of all its arguments
separated by spaces. Each argument is converted to a string with
\texttt{util.inspect()}.

\begin{Shaded}
\begin{Highlighting}[]
\OtherTok{util}\NormalTok{.}\FunctionTok{format}\NormalTok{(}\DecValTok{1}\NormalTok{, }\DecValTok{2}\NormalTok{, }\DecValTok{3}\NormalTok{); }\CommentTok{// '1 2 3'}
\end{Highlighting}
\end{Shaded}

\subsection{util.log(string)}\label{util.logstring}

Output with timestamp on \texttt{stdout}.

\begin{Shaded}
\begin{Highlighting}[]
\FunctionTok{require}\NormalTok{(}\StringTok{'util'}\NormalTok{).}\FunctionTok{log}\NormalTok{(}\StringTok{'Timestamped message.'}\NormalTok{);}
\end{Highlighting}
\end{Shaded}

\subsection{util.inspect(object{[},
options{]})}\label{util.inspectobject-options}

Return a string representation of \texttt{object}, which is useful for
debugging.

An optional \emph{options} object may be passed that alters certain
aspects of the formatted string:

\begin{itemize}
\item
  \texttt{showHidden} - if \texttt{true} then the object's
  non-enumerable properties will be shown too. Defaults to
  \texttt{false}.
\item
  \texttt{depth} - tells \texttt{inspect} how many times to recurse
  while formatting the object. This is useful for inspecting large
  complicated objects. Defaults to \texttt{2}. To make it recurse
  indefinitely pass \texttt{null}.
\item
  \texttt{colors} - if \texttt{true}, then the output will be styled
  with ANSI color codes. Defaults to \texttt{false}. Colors are
  customizable, see below.
\item
  \texttt{customInspect} - if \texttt{false}, then custom
  \texttt{inspect(depth,\ opts)} functions defined on the objects being
  inspected won't be called. Defaults to \texttt{true}.
\end{itemize}

Example of inspecting all properties of the \texttt{util} object:

\begin{Shaded}
\begin{Highlighting}[]
\KeywordTok{var} \NormalTok{util = }\FunctionTok{require}\NormalTok{(}\StringTok{'util'}\NormalTok{);}

\OtherTok{console}\NormalTok{.}\FunctionTok{log}\NormalTok{(}\OtherTok{util}\NormalTok{.}\FunctionTok{inspect}\NormalTok{(util, \{ }\DataTypeTok{showHidden}\NormalTok{: }\KeywordTok{true}\NormalTok{, }\DataTypeTok{depth}\NormalTok{: }\KeywordTok{null} \NormalTok{\}));}
\end{Highlighting}
\end{Shaded}

Values may supply their own custom \texttt{inspect(depth,\ opts)}
functions, when called they receive the current depth in the recursive
inspection, as well as the options object passed to
\texttt{util.inspect()}.

\subsubsection{\texorpdfstring{Customizing \texttt{util.inspect}
colors}{Customizing util.inspect colors}}\label{customizing-util.inspect-colors}

Color output (if enabled) of \texttt{util.inspect} is customizable
globally via \texttt{util.inspect.styles} and
\texttt{util.inspect.colors} objects.

\texttt{util.inspect.styles} is a map assigning each style a color from
\texttt{util.inspect.colors}. Highlighted styles and their default
values are: * \texttt{number} (yellow) * \texttt{boolean} (yellow) *
\texttt{string} (green) * \texttt{date} (magenta) * \texttt{regexp}
(red) * \texttt{null} (bold) * \texttt{undefined} (grey) *
\texttt{special} - only function at this time (cyan) * \texttt{name}
(intentionally no styling)

Predefined color codes are: \texttt{white}, \texttt{grey},
\texttt{black}, \texttt{blue}, \texttt{cyan}, \texttt{green},
\texttt{magenta}, \texttt{red} and \texttt{yellow}. There are also
\texttt{bold}, \texttt{italic}, \texttt{underline} and \texttt{inverse}
codes.

\subsubsection{\texorpdfstring{Custom \texttt{inspect()} function on
Objects}{Custom inspect() function on Objects}}\label{custom-inspect-function-on-objects}

Objects also may define their own \texttt{inspect(depth)} function which
\texttt{util.inspect()} will invoke and use the result of when
inspecting the object:

\begin{Shaded}
\begin{Highlighting}[]
\KeywordTok{var} \NormalTok{util = }\FunctionTok{require}\NormalTok{(}\StringTok{'util'}\NormalTok{);}

\KeywordTok{var} \NormalTok{obj = \{ }\DataTypeTok{name}\NormalTok{: }\StringTok{'nate'} \NormalTok{\};}
\OtherTok{obj}\NormalTok{.}\FunctionTok{inspect} \NormalTok{= }\KeywordTok{function}\NormalTok{(depth) \{}
  \KeywordTok{return} \StringTok{'\{'} \NormalTok{+ }\KeywordTok{this}\NormalTok{.}\FunctionTok{name} \NormalTok{+ }\StringTok{'\}'}\NormalTok{;}
\NormalTok{\};}

\OtherTok{util}\NormalTok{.}\FunctionTok{inspect}\NormalTok{(obj);}
  \CommentTok{// "\{nate\}"}
\end{Highlighting}
\end{Shaded}

You may also return another Object entirely, and the returned String
will be formatted according to the returned Object. This is similar to
how \texttt{JSON.stringify()} works:

\begin{Shaded}
\begin{Highlighting}[]
\KeywordTok{var} \NormalTok{obj = \{ }\DataTypeTok{foo}\NormalTok{: }\StringTok{'this will not show up in the inspect() output'} \NormalTok{\};}
\OtherTok{obj}\NormalTok{.}\FunctionTok{inspect} \NormalTok{= }\KeywordTok{function}\NormalTok{(depth) \{}
  \KeywordTok{return} \NormalTok{\{ }\DataTypeTok{bar}\NormalTok{: }\StringTok{'baz'} \NormalTok{\};}
\NormalTok{\};}

\OtherTok{util}\NormalTok{.}\FunctionTok{inspect}\NormalTok{(obj);}
  \CommentTok{// "\{ bar: 'baz' \}"}
\end{Highlighting}
\end{Shaded}

\subsection{util.isArray(object)}\label{util.isarrayobject}

Internal alias for Array.isArray.

Returns \texttt{true} if the given ``object'' is an \texttt{Array}.
\texttt{false} otherwise.

\begin{Shaded}
\begin{Highlighting}[]
\KeywordTok{var} \NormalTok{util = }\FunctionTok{require}\NormalTok{(}\StringTok{'util'}\NormalTok{);}

\OtherTok{util}\NormalTok{.}\FunctionTok{isArray}\NormalTok{([])}
  \CommentTok{// true}
\OtherTok{util}\NormalTok{.}\FunctionTok{isArray}\NormalTok{(}\KeywordTok{new} \NormalTok{Array)}
  \CommentTok{// true}
\OtherTok{util}\NormalTok{.}\FunctionTok{isArray}\NormalTok{(\{\})}
  \CommentTok{// false}
\end{Highlighting}
\end{Shaded}

\subsection{util.isRegExp(object)}\label{util.isregexpobject}

Returns \texttt{true} if the given ``object'' is a \texttt{RegExp}.
\texttt{false} otherwise.

\begin{Shaded}
\begin{Highlighting}[]
\KeywordTok{var} \NormalTok{util = }\FunctionTok{require}\NormalTok{(}\StringTok{'util'}\NormalTok{);}

\OtherTok{util}\NormalTok{.}\FunctionTok{isRegExp}\NormalTok{(}\OtherTok{/some regexp/}\NormalTok{)}
  \CommentTok{// true}
\OtherTok{util}\NormalTok{.}\FunctionTok{isRegExp}\NormalTok{(}\KeywordTok{new} \FunctionTok{RegExp}\NormalTok{(}\StringTok{'another regexp'}\NormalTok{))}
  \CommentTok{// true}
\OtherTok{util}\NormalTok{.}\FunctionTok{isRegExp}\NormalTok{(\{\})}
  \CommentTok{// false}
\end{Highlighting}
\end{Shaded}

\subsection{util.isDate(object)}\label{util.isdateobject}

Returns \texttt{true} if the given ``object'' is a \texttt{Date}.
\texttt{false} otherwise.

\begin{Shaded}
\begin{Highlighting}[]
\KeywordTok{var} \NormalTok{util = }\FunctionTok{require}\NormalTok{(}\StringTok{'util'}\NormalTok{);}

\OtherTok{util}\NormalTok{.}\FunctionTok{isDate}\NormalTok{(}\KeywordTok{new} \FunctionTok{Date}\NormalTok{())}
  \CommentTok{// true}
\OtherTok{util}\NormalTok{.}\FunctionTok{isDate}\NormalTok{(}\FunctionTok{Date}\NormalTok{())}
  \CommentTok{// false (without 'new' returns a String)}
\OtherTok{util}\NormalTok{.}\FunctionTok{isDate}\NormalTok{(\{\})}
  \CommentTok{// false}
\end{Highlighting}
\end{Shaded}

\subsection{util.isError(object)}\label{util.iserrorobject}

Returns \texttt{true} if the given ``object'' is an \texttt{Error}.
\texttt{false} otherwise.

\begin{Shaded}
\begin{Highlighting}[]
\KeywordTok{var} \NormalTok{util = }\FunctionTok{require}\NormalTok{(}\StringTok{'util'}\NormalTok{);}

\OtherTok{util}\NormalTok{.}\FunctionTok{isError}\NormalTok{(}\KeywordTok{new} \FunctionTok{Error}\NormalTok{())}
  \CommentTok{// true}
\OtherTok{util}\NormalTok{.}\FunctionTok{isError}\NormalTok{(}\KeywordTok{new} \FunctionTok{TypeError}\NormalTok{())}
  \CommentTok{// true}
\OtherTok{util}\NormalTok{.}\FunctionTok{isError}\NormalTok{(\{ }\DataTypeTok{name}\NormalTok{: }\StringTok{'Error'}\NormalTok{, }\DataTypeTok{message}\NormalTok{: }\StringTok{'an error occurred'} \NormalTok{\})}
  \CommentTok{// false}
\end{Highlighting}
\end{Shaded}

\subsection{util.inherits(constructor,
superConstructor)}\label{util.inheritsconstructor-superconstructor}

Inherit the prototype methods from one
\href{https://developer.mozilla.org/en/JavaScript/Reference/Global_Objects/Object/constructor}{constructor}
into another. The prototype of \texttt{constructor} will be set to a new
object created from \texttt{superConstructor}.

As an additional convenience, \texttt{superConstructor} will be
accessible through the \texttt{constructor.super\_} property.

\begin{Shaded}
\begin{Highlighting}[]
\KeywordTok{var} \NormalTok{util = }\FunctionTok{require}\NormalTok{(}\StringTok{"util"}\NormalTok{);}
\KeywordTok{var} \NormalTok{events = }\FunctionTok{require}\NormalTok{(}\StringTok{"events"}\NormalTok{);}

\KeywordTok{function} \FunctionTok{MyStream}\NormalTok{() \{}
    \OtherTok{events}\NormalTok{.}\OtherTok{EventEmitter}\NormalTok{.}\FunctionTok{call}\NormalTok{(}\KeywordTok{this}\NormalTok{);}
\NormalTok{\}}

\OtherTok{util}\NormalTok{.}\FunctionTok{inherits}\NormalTok{(MyStream, }\OtherTok{events}\NormalTok{.}\FunctionTok{EventEmitter}\NormalTok{);}

\OtherTok{MyStream}\NormalTok{.}\OtherTok{prototype}\NormalTok{.}\FunctionTok{write} \NormalTok{= }\KeywordTok{function}\NormalTok{(data) \{}
    \KeywordTok{this}\NormalTok{.}\FunctionTok{emit}\NormalTok{(}\StringTok{"data"}\NormalTok{, data);}
\NormalTok{\}}

\KeywordTok{var} \NormalTok{stream = }\KeywordTok{new} \FunctionTok{MyStream}\NormalTok{();}

\OtherTok{console}\NormalTok{.}\FunctionTok{log}\NormalTok{(stream }\KeywordTok{instanceof} \OtherTok{events}\NormalTok{.}\FunctionTok{EventEmitter}\NormalTok{); }\CommentTok{// true}
\OtherTok{console}\NormalTok{.}\FunctionTok{log}\NormalTok{(}\OtherTok{MyStream}\NormalTok{.}\FunctionTok{super_} \NormalTok{=== }\OtherTok{events}\NormalTok{.}\FunctionTok{EventEmitter}\NormalTok{); }\CommentTok{// true}

\OtherTok{stream}\NormalTok{.}\FunctionTok{on}\NormalTok{(}\StringTok{"data"}\NormalTok{, }\KeywordTok{function}\NormalTok{(data) \{}
    \OtherTok{console}\NormalTok{.}\FunctionTok{log}\NormalTok{(}\StringTok{'Received data: "'} \NormalTok{+ data + }\StringTok{'"'}\NormalTok{);}
\NormalTok{\})}
\OtherTok{stream}\NormalTok{.}\FunctionTok{write}\NormalTok{(}\StringTok{"It works!"}\NormalTok{); }\CommentTok{// Received data: "It works!"}
\end{Highlighting}
\end{Shaded}

\subsection{util.deprecate(function,
string)}\label{util.deprecatefunction-string}

Marks that a method should not be used any more.

\begin{Shaded}
\begin{Highlighting}[]
\OtherTok{exports}\NormalTok{.}\FunctionTok{puts} \NormalTok{= }\OtherTok{exports}\NormalTok{.}\FunctionTok{deprecate}\NormalTok{(}\KeywordTok{function}\NormalTok{() \{}
  \KeywordTok{for} \NormalTok{(}\KeywordTok{var} \NormalTok{i = }\DecValTok{0}\NormalTok{, len = }\OtherTok{arguments}\NormalTok{.}\FunctionTok{length}\NormalTok{; i < len; ++i) \{}
    \OtherTok{process}\NormalTok{.}\OtherTok{stdout}\NormalTok{.}\FunctionTok{write}\NormalTok{(arguments[i] + }\StringTok{'}\CharTok{\textbackslash{}n}\StringTok{'}\NormalTok{);}
  \NormalTok{\}}
\NormalTok{\}, }\StringTok{'util.puts: Use console.log instead'}\NormalTok{)}
\end{Highlighting}
\end{Shaded}

It returns a modified function which warns once by default. If
\texttt{-\/-no-deprecation} is set then this function is a NO-OP. If
\texttt{-\/-throw-deprecation} is set then the application will throw an
exception if the deprecated API is used.

\subsection{util.debug(string)}\label{util.debugstring}

\begin{Shaded}
\begin{Highlighting}[]
\NormalTok{Stability: }\DecValTok{0} \NormalTok{- Deprecated: use }\OtherTok{console}\NormalTok{.}\FunctionTok{error}\NormalTok{() }\OtherTok{instead}\NormalTok{.}
\end{Highlighting}
\end{Shaded}

Deprecated predecessor of \texttt{console.error}.

\subsection{util.error({[}\ldots{}{]})}\label{util.error}

\begin{Shaded}
\begin{Highlighting}[]
\NormalTok{Stability: }\DecValTok{0} \NormalTok{- Deprecated: Use }\OtherTok{console}\NormalTok{.}\FunctionTok{error}\NormalTok{() }\OtherTok{instead}\NormalTok{.}
\end{Highlighting}
\end{Shaded}

Deprecated predecessor of \texttt{console.error}.

\subsection{util.puts({[}\ldots{}{]})}\label{util.puts}

\begin{Shaded}
\begin{Highlighting}[]
\NormalTok{Stability: }\DecValTok{0} \NormalTok{- Deprecated: Use }\OtherTok{console}\NormalTok{.}\FunctionTok{log}\NormalTok{() }\OtherTok{instead}\NormalTok{.}
\end{Highlighting}
\end{Shaded}

Deprecated predecessor of \texttt{console.log}.

\subsection{util.print({[}\ldots{}{]})}\label{util.print}

\begin{Shaded}
\begin{Highlighting}[]
\NormalTok{Stability: }\DecValTok{0} \NormalTok{- Deprecated: Use `}\OtherTok{console}\NormalTok{.}\FunctionTok{log}\NormalTok{` }\OtherTok{instead}\NormalTok{.}
\end{Highlighting}
\end{Shaded}

Deprecated predecessor of \texttt{console.log}.

\subsection{util.pump(readableStream, writableStream{[},
callback{]})}\label{util.pumpreadablestream-writablestream-callback}

\begin{Shaded}
\begin{Highlighting}[]
\NormalTok{Stability: }\DecValTok{0} \NormalTok{- Deprecated: Use }\OtherTok{readableStream}\NormalTok{.}\FunctionTok{pipe}\NormalTok{(writableStream)}
\end{Highlighting}
\end{Shaded}

Deprecated predecessor of \texttt{stream.pipe()}.
