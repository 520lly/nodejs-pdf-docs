\section{Child Process}\label{child-process}

\begin{Shaded}
\begin{Highlighting}[]
\NormalTok{Stability: }\DecValTok{3} \NormalTok{- Stable}
\end{Highlighting}
\end{Shaded}

Node provides a tri-directional \texttt{popen(3)} facility through the
\texttt{child\_process} module.

It is possible to stream data through a child's \texttt{stdin},
\texttt{stdout}, and \texttt{stderr} in a fully non-blocking way. (Note
that some programs use line-buffered I/O internally. That doesn't affect
node.js but it means data you send to the child process is not
immediately consumed.)

To create a child process use \texttt{require('child\_process').spawn()}
or \texttt{require('child\_process').fork()}. The semantics of each are
slightly different, and explained below.

\subsection{Class: ChildProcess}\label{class-childprocess}

\texttt{ChildProcess} is an
\href{events.html\#events_class_events_eventemitter}{EventEmitter}.

Child processes always have three streams associated with them.
\texttt{child.stdin}, \texttt{child.stdout}, and \texttt{child.stderr}.
These may be shared with the stdio streams of the parent process, or
they may be separate stream objects which can be piped to and from.

The ChildProcess class is not intended to be used directly. Use the
\texttt{spawn()} or \texttt{fork()} methods to create a Child Process
instance.

\subsubsection{Event: `error'}\label{event-error}

\begin{itemize}
\itemsep1pt\parskip0pt\parsep0pt
\item
  \texttt{err} \{Error Object\} the error.
\end{itemize}

Emitted when:

\begin{enumerate}
\def\labelenumi{\arabic{enumi}.}
\itemsep1pt\parskip0pt\parsep0pt
\item
  The process could not be spawned, or
\item
  The process could not be killed, or
\item
  Sending a message to the child process failed for whatever reason.
\end{enumerate}

Note that the \texttt{exit}-event may or may not fire after an error has
occured. If you are listening on both events to fire a function,
remember to guard against calling your function twice.

See also
\hyperref[childux5fprocessux5fchildux5fkillux5fsignal]{\texttt{ChildProcess\#kill()}}
and
\hyperref[childux5fprocessux5fchildux5fsendux5fmessageux5fsendhandle]{\texttt{ChildProcess\#send()}}.

\subsubsection{Event: `exit'}\label{event-exit}

\begin{itemize}
\itemsep1pt\parskip0pt\parsep0pt
\item
  \texttt{code} \{Number\} the exit code, if it exited normally.
\item
  \texttt{signal} \{String\} the signal passed to kill the child
  process, if it was killed by the parent.
\end{itemize}

This event is emitted after the child process ends. If the process
terminated normally, \texttt{code} is the final exit code of the
process, otherwise \texttt{null}. If the process terminated due to
receipt of a signal, \texttt{signal} is the string name of the signal,
otherwise \texttt{null}.

Note that the child process stdio streams might still be open.

Also, note that node establishes signal handlers for \texttt{'SIGINT'}
and \texttt{'SIGTERM}', so it will not terminate due to receipt of those
signals, it will exit.

See \texttt{waitpid(2)}.

\subsubsection{Event: `close'}\label{event-close}

\begin{itemize}
\itemsep1pt\parskip0pt\parsep0pt
\item
  \texttt{code} \{Number\} the exit code, if it exited normally.
\item
  \texttt{signal} \{String\} the signal passed to kill the child
  process, if it was killed by the parent.
\end{itemize}

This event is emitted when the stdio streams of a child process have all
terminated. This is distinct from `exit', since multiple processes might
share the same stdio streams.

\subsubsection{Event: `disconnect'}\label{event-disconnect}

This event is emitted after calling the \texttt{.disconnect()} method in
the parent or in the child. After disconnecting it is no longer possible
to send messages, and the \texttt{.connected} property is false.

\subsubsection{Event: `message'}\label{event-message}

\begin{itemize}
\itemsep1pt\parskip0pt\parsep0pt
\item
  \texttt{message} \{Object\} a parsed JSON object or primitive value
\item
  \texttt{sendHandle} \{Handle object\} a Socket or Server object
\end{itemize}

Messages send by \texttt{.send(message, {[}sendHandle{]})} are obtained
using the \texttt{message} event.

\subsubsection{child.stdin}\label{child.stdin}

\begin{itemize}
\itemsep1pt\parskip0pt\parsep0pt
\item
  \{Stream object\}
\end{itemize}

A \texttt{Writable Stream} that represents the child process's
\texttt{stdin}. Closing this stream via \texttt{end()} often causes the
child process to terminate.

If the child stdio streams are shared with the parent, then this will
not be set.

\subsubsection{child.stdout}\label{child.stdout}

\begin{itemize}
\itemsep1pt\parskip0pt\parsep0pt
\item
  \{Stream object\}
\end{itemize}

A \texttt{Readable Stream} that represents the child process's
\texttt{stdout}.

If the child stdio streams are shared with the parent, then this will
not be set.

\subsubsection{child.stderr}\label{child.stderr}

\begin{itemize}
\itemsep1pt\parskip0pt\parsep0pt
\item
  \{Stream object\}
\end{itemize}

A \texttt{Readable Stream} that represents the child process's
\texttt{stderr}.

If the child stdio streams are shared with the parent, then this will
not be set.

\subsubsection{child.pid}\label{child.pid}

\begin{itemize}
\itemsep1pt\parskip0pt\parsep0pt
\item
  \{Integer\}
\end{itemize}

The PID of the child process.

Example:

\begin{Shaded}
\begin{Highlighting}[]
\KeywordTok{var} \NormalTok{spawn = }\FunctionTok{require}\NormalTok{(}\StringTok{'child_process'}\NormalTok{).}\FunctionTok{spawn}\NormalTok{,}
    \NormalTok{grep  = }\FunctionTok{spawn}\NormalTok{(}\StringTok{'grep'}\NormalTok{, [}\StringTok{'ssh'}\NormalTok{]);}

\OtherTok{console}\NormalTok{.}\FunctionTok{log}\NormalTok{(}\StringTok{'Spawned child pid: '} \NormalTok{+ }\OtherTok{grep}\NormalTok{.}\FunctionTok{pid}\NormalTok{);}
\OtherTok{grep}\NormalTok{.}\OtherTok{stdin}\NormalTok{.}\FunctionTok{end}\NormalTok{();}
\end{Highlighting}
\end{Shaded}

\subsubsection{child.connected}\label{child.connected}

\begin{itemize}
\itemsep1pt\parskip0pt\parsep0pt
\item
  \{Boolean\} Set to false after `.disconnect' is called
\end{itemize}

If \texttt{.connected} is false, it is no longer possible to send
messages.

\subsubsection{child.kill({[}signal{]})}\label{child.killsignal}

\begin{itemize}
\itemsep1pt\parskip0pt\parsep0pt
\item
  \texttt{signal} \{String\}
\end{itemize}

Send a signal to the child process. If no argument is given, the process
will be sent \texttt{'SIGTERM'}. See \texttt{signal(7)} for a list of
available signals.

\begin{Shaded}
\begin{Highlighting}[]
\KeywordTok{var} \NormalTok{spawn = }\FunctionTok{require}\NormalTok{(}\StringTok{'child_process'}\NormalTok{).}\FunctionTok{spawn}\NormalTok{,}
    \NormalTok{grep  = }\FunctionTok{spawn}\NormalTok{(}\StringTok{'grep'}\NormalTok{, [}\StringTok{'ssh'}\NormalTok{]);}

\OtherTok{grep}\NormalTok{.}\FunctionTok{on}\NormalTok{(}\StringTok{'close'}\NormalTok{, }\KeywordTok{function} \NormalTok{(code, signal) \{}
  \OtherTok{console}\NormalTok{.}\FunctionTok{log}\NormalTok{(}\StringTok{'child process terminated due to receipt of signal '}\NormalTok{+signal);}
\NormalTok{\});}

\CommentTok{// send SIGHUP to process}
\OtherTok{grep}\NormalTok{.}\FunctionTok{kill}\NormalTok{(}\StringTok{'SIGHUP'}\NormalTok{);}
\end{Highlighting}
\end{Shaded}

May emit an \texttt{'error'} event when the signal cannot be delivered.
Sending a signal to a child process that has already exited is not an
error but may have unforeseen consequences: if the PID (the process ID)
has been reassigned to another process, the signal will be delivered to
that process instead. What happens next is anyone's guess.

Note that while the function is called \texttt{kill}, the signal
delivered to the child process may not actually kill it. \texttt{kill}
really just sends a signal to a process.

See \texttt{kill(2)}

\subsubsection{child.send(message,
{[}sendHandle{]})}\label{child.sendmessage-sendhandle}

\begin{itemize}
\itemsep1pt\parskip0pt\parsep0pt
\item
  \texttt{message} \{Object\}
\item
  \texttt{sendHandle} \{Handle object\}
\end{itemize}

When using \texttt{child\_process.fork()} you can write to the child
using \texttt{child.send(message, {[}sendHandle{]})} and messages are
received by a \texttt{'message'} event on the child.

For example:

\begin{Shaded}
\begin{Highlighting}[]
\KeywordTok{var} \NormalTok{cp = }\FunctionTok{require}\NormalTok{(}\StringTok{'child_process'}\NormalTok{);}

\KeywordTok{var} \NormalTok{n = }\OtherTok{cp}\NormalTok{.}\FunctionTok{fork}\NormalTok{(__dirname + }\StringTok{'/sub.js'}\NormalTok{);}

\OtherTok{n}\NormalTok{.}\FunctionTok{on}\NormalTok{(}\StringTok{'message'}\NormalTok{, }\KeywordTok{function}\NormalTok{(m) \{}
  \OtherTok{console}\NormalTok{.}\FunctionTok{log}\NormalTok{(}\StringTok{'PARENT got message:'}\NormalTok{, m);}
\NormalTok{\});}

\OtherTok{n}\NormalTok{.}\FunctionTok{send}\NormalTok{(\{ }\DataTypeTok{hello}\NormalTok{: }\StringTok{'world'} \NormalTok{\});}
\end{Highlighting}
\end{Shaded}

And then the child script, \texttt{'sub.js'} might look like this:

\begin{Shaded}
\begin{Highlighting}[]
\OtherTok{process}\NormalTok{.}\FunctionTok{on}\NormalTok{(}\StringTok{'message'}\NormalTok{, }\KeywordTok{function}\NormalTok{(m) \{}
  \OtherTok{console}\NormalTok{.}\FunctionTok{log}\NormalTok{(}\StringTok{'CHILD got message:'}\NormalTok{, m);}
\NormalTok{\});}

\OtherTok{process}\NormalTok{.}\FunctionTok{send}\NormalTok{(\{ }\DataTypeTok{foo}\NormalTok{: }\StringTok{'bar'} \NormalTok{\});}
\end{Highlighting}
\end{Shaded}

In the child the \texttt{process} object will have a \texttt{send()}
method, and \texttt{process} will emit objects each time it receives a
message on its channel.

Please note that the \texttt{send()} method on both the parent and child
are synchronous - sending large chunks of data is not advised (pipes can
be used instead, see
\hyperref[childux5fprocessux5fchildux5fprocessux5fspawnux5fcommandux5fargsux5foptions]{\texttt{child\_process.spawn}}).

There is a special case when sending a \texttt{\{cmd: 'NODE\_foo'\}}
message. All messages containing a \texttt{NODE\_} prefix in its
\texttt{cmd} property will not be emitted in the \texttt{message} event,
since they are internal messages used by node core. Messages containing
the prefix are emitted in the \texttt{internalMessage} event, you should
by all means avoid using this feature, it is subject to change without
notice.

The \texttt{sendHandle} option to \texttt{child.send()} is for sending a
TCP server or socket object to another process. The child will receive
the object as its second argument to the \texttt{message} event.

Emits an \texttt{'error'} event if the message cannot be sent, for
example because the child process has already exited.

\paragraph{Example: sending server
object}\label{example-sending-server-object}

Here is an example of sending a server:

\begin{Shaded}
\begin{Highlighting}[]
\KeywordTok{var} \NormalTok{child = }\FunctionTok{require}\NormalTok{(}\StringTok{'child_process'}\NormalTok{).}\FunctionTok{fork}\NormalTok{(}\StringTok{'child.js'}\NormalTok{);}

\CommentTok{// Open up the server object and send the handle.}
\KeywordTok{var} \NormalTok{server = }\FunctionTok{require}\NormalTok{(}\StringTok{'net'}\NormalTok{).}\FunctionTok{createServer}\NormalTok{();}
\OtherTok{server}\NormalTok{.}\FunctionTok{on}\NormalTok{(}\StringTok{'connection'}\NormalTok{, }\KeywordTok{function} \NormalTok{(socket) \{}
  \OtherTok{socket}\NormalTok{.}\FunctionTok{end}\NormalTok{(}\StringTok{'handled by parent'}\NormalTok{);}
\NormalTok{\});}
\OtherTok{server}\NormalTok{.}\FunctionTok{listen}\NormalTok{(}\DecValTok{1337}\NormalTok{, }\KeywordTok{function}\NormalTok{() \{}
  \OtherTok{child}\NormalTok{.}\FunctionTok{send}\NormalTok{(}\StringTok{'server'}\NormalTok{, server);}
\NormalTok{\});}
\end{Highlighting}
\end{Shaded}

And the child would the receive the server object as:

\begin{Shaded}
\begin{Highlighting}[]
\OtherTok{process}\NormalTok{.}\FunctionTok{on}\NormalTok{(}\StringTok{'message'}\NormalTok{, }\KeywordTok{function}\NormalTok{(m, server) \{}
  \KeywordTok{if} \NormalTok{(m === }\StringTok{'server'}\NormalTok{) \{}
    \OtherTok{server}\NormalTok{.}\FunctionTok{on}\NormalTok{(}\StringTok{'connection'}\NormalTok{, }\KeywordTok{function} \NormalTok{(socket) \{}
      \OtherTok{socket}\NormalTok{.}\FunctionTok{end}\NormalTok{(}\StringTok{'handled by child'}\NormalTok{);}
    \NormalTok{\});}
  \NormalTok{\}}
\NormalTok{\});}
\end{Highlighting}
\end{Shaded}

Note that the server is now shared between the parent and child, this
means that some connections will be handled by the parent and some by
the child.

For \texttt{dgram} servers the workflow is exactly the same. Here you
listen on a \texttt{message} event instead of \texttt{connection} and
use \texttt{server.bind} instead of \texttt{server.listen}. (Currently
only supported on UNIX platforms.)

\paragraph{Example: sending socket
object}\label{example-sending-socket-object}

Here is an example of sending a socket. It will spawn two children and
handle connections with the remote address \texttt{74.125.127.100} as
VIP by sending the socket to a ``special'' child process. Other sockets
will go to a ``normal'' process.

\begin{Shaded}
\begin{Highlighting}[]
\KeywordTok{var} \NormalTok{normal = }\FunctionTok{require}\NormalTok{(}\StringTok{'child_process'}\NormalTok{).}\FunctionTok{fork}\NormalTok{(}\StringTok{'child.js'}\NormalTok{, [}\StringTok{'normal'}\NormalTok{]);}
\KeywordTok{var} \NormalTok{special = }\FunctionTok{require}\NormalTok{(}\StringTok{'child_process'}\NormalTok{).}\FunctionTok{fork}\NormalTok{(}\StringTok{'child.js'}\NormalTok{, [}\StringTok{'special'}\NormalTok{]);}

\CommentTok{// Open up the server and send sockets to child}
\KeywordTok{var} \NormalTok{server = }\FunctionTok{require}\NormalTok{(}\StringTok{'net'}\NormalTok{).}\FunctionTok{createServer}\NormalTok{();}
\OtherTok{server}\NormalTok{.}\FunctionTok{on}\NormalTok{(}\StringTok{'connection'}\NormalTok{, }\KeywordTok{function} \NormalTok{(socket) \{}

  \CommentTok{// if this is a VIP}
  \KeywordTok{if} \NormalTok{(}\OtherTok{socket}\NormalTok{.}\FunctionTok{remoteAddress} \NormalTok{=== }\StringTok{'74.125.127.100'}\NormalTok{) \{}
    \OtherTok{special}\NormalTok{.}\FunctionTok{send}\NormalTok{(}\StringTok{'socket'}\NormalTok{, socket);}
    \KeywordTok{return}\NormalTok{;}
  \NormalTok{\}}
  \CommentTok{// just the usual dudes}
  \OtherTok{normal}\NormalTok{.}\FunctionTok{send}\NormalTok{(}\StringTok{'socket'}\NormalTok{, socket);}
\NormalTok{\});}
\OtherTok{server}\NormalTok{.}\FunctionTok{listen}\NormalTok{(}\DecValTok{1337}\NormalTok{);}
\end{Highlighting}
\end{Shaded}

The \texttt{child.js} could look like this:

\begin{Shaded}
\begin{Highlighting}[]
\OtherTok{process}\NormalTok{.}\FunctionTok{on}\NormalTok{(}\StringTok{'message'}\NormalTok{, }\KeywordTok{function}\NormalTok{(m, socket) \{}
  \KeywordTok{if} \NormalTok{(m === }\StringTok{'socket'}\NormalTok{) \{}
    \OtherTok{socket}\NormalTok{.}\FunctionTok{end}\NormalTok{(}\StringTok{'You were handled as a '} \NormalTok{+ }\OtherTok{process}\NormalTok{.}\FunctionTok{argv}\NormalTok{[}\DecValTok{2}\NormalTok{] + }\StringTok{' person'}\NormalTok{);}
  \NormalTok{\}}
\NormalTok{\});}
\end{Highlighting}
\end{Shaded}

Note that once a single socket has been sent to a child the parent can
no longer keep track of when the socket is destroyed. To indicate this
condition the \texttt{.connections} property becomes \texttt{null}. It
is also recommended not to use \texttt{.maxConnections} in this
condition.

\subsubsection{child.disconnect()}\label{child.disconnect}

Close the IPC channel between parent and child, allowing the child to
exit gracefully once there are no other connections keeping it alive.
After calling this method the \texttt{.connected} flag will be set to
\texttt{false} in both the parent and child, and it is no longer
possible to send messages.

The `disconnect' event will be emitted when there are no messages in the
process of being received, most likely immediately.

Note that you can also call \texttt{process.disconnect()} in the child
process.

\subsection{child\_process.spawn(command, {[}args{]},
{[}options{]})}\label{childux5fprocess.spawncommand-args-options}

\begin{itemize}
\itemsep1pt\parskip0pt\parsep0pt
\item
  \texttt{command} \{String\} The command to run
\item
  \texttt{args} \{Array\} List of string arguments
\item
  \texttt{options} \{Object\}
\item
  \texttt{cwd} \{String\} Current working directory of the child process
\item
  \texttt{stdio} \{Array\textbar{}String\} Child's stdio configuration.
  (See below)
\item
  \texttt{customFds} \{Array\} \textbf{Deprecated} File descriptors for
  the child to use for stdio. (See below)
\item
  \texttt{env} \{Object\} Environment key-value pairs
\item
  \texttt{detached} \{Boolean\} The child will be a process group
  leader. (See below)
\item
  \texttt{uid} \{Number\} Sets the user identity of the process. (See
  setuid(2).)
\item
  \texttt{gid} \{Number\} Sets the group identity of the process. (See
  setgid(2).)
\item
  return: \{ChildProcess object\}
\end{itemize}

Launches a new process with the given \texttt{command}, with command
line arguments in \texttt{args}. If omitted, \texttt{args} defaults to
an empty Array.

The third argument is used to specify additional options, which defaults
to:

\begin{Shaded}
\begin{Highlighting}[]
\NormalTok{\{ }\DataTypeTok{cwd}\NormalTok{: }\KeywordTok{undefined}\NormalTok{,}
  \DataTypeTok{env}\NormalTok{: }\OtherTok{process}\NormalTok{.}\FunctionTok{env}
\NormalTok{\}}
\end{Highlighting}
\end{Shaded}

\texttt{cwd} allows you to specify the working directory from which the
process is spawned. Use \texttt{env} to specify environment variables
that will be visible to the new process.

Example of running \texttt{ls -lh /usr}, capturing \texttt{stdout},
\texttt{stderr}, and the exit code:

\begin{Shaded}
\begin{Highlighting}[]
\KeywordTok{var} \NormalTok{spawn = }\FunctionTok{require}\NormalTok{(}\StringTok{'child_process'}\NormalTok{).}\FunctionTok{spawn}\NormalTok{,}
    \NormalTok{ls    = }\FunctionTok{spawn}\NormalTok{(}\StringTok{'ls'}\NormalTok{, [}\StringTok{'-lh'}\NormalTok{, }\StringTok{'/usr'}\NormalTok{]);}

\OtherTok{ls}\NormalTok{.}\OtherTok{stdout}\NormalTok{.}\FunctionTok{on}\NormalTok{(}\StringTok{'data'}\NormalTok{, }\KeywordTok{function} \NormalTok{(data) \{}
  \OtherTok{console}\NormalTok{.}\FunctionTok{log}\NormalTok{(}\StringTok{'stdout: '} \NormalTok{+ data);}
\NormalTok{\});}

\OtherTok{ls}\NormalTok{.}\OtherTok{stderr}\NormalTok{.}\FunctionTok{on}\NormalTok{(}\StringTok{'data'}\NormalTok{, }\KeywordTok{function} \NormalTok{(data) \{}
  \OtherTok{console}\NormalTok{.}\FunctionTok{log}\NormalTok{(}\StringTok{'stderr: '} \NormalTok{+ data);}
\NormalTok{\});}

\OtherTok{ls}\NormalTok{.}\FunctionTok{on}\NormalTok{(}\StringTok{'close'}\NormalTok{, }\KeywordTok{function} \NormalTok{(code) \{}
  \OtherTok{console}\NormalTok{.}\FunctionTok{log}\NormalTok{(}\StringTok{'child process exited with code '} \NormalTok{+ code);}
\NormalTok{\});}
\end{Highlighting}
\end{Shaded}

Example: A very elaborate way to run `ps ax \textbar{} grep ssh'

\begin{Shaded}
\begin{Highlighting}[]
\KeywordTok{var} \NormalTok{spawn = }\FunctionTok{require}\NormalTok{(}\StringTok{'child_process'}\NormalTok{).}\FunctionTok{spawn}\NormalTok{,}
    \NormalTok{ps    = }\FunctionTok{spawn}\NormalTok{(}\StringTok{'ps'}\NormalTok{, [}\StringTok{'ax'}\NormalTok{]),}
    \NormalTok{grep  = }\FunctionTok{spawn}\NormalTok{(}\StringTok{'grep'}\NormalTok{, [}\StringTok{'ssh'}\NormalTok{]);}

\OtherTok{ps}\NormalTok{.}\OtherTok{stdout}\NormalTok{.}\FunctionTok{on}\NormalTok{(}\StringTok{'data'}\NormalTok{, }\KeywordTok{function} \NormalTok{(data) \{}
  \OtherTok{grep}\NormalTok{.}\OtherTok{stdin}\NormalTok{.}\FunctionTok{write}\NormalTok{(data);}
\NormalTok{\});}

\OtherTok{ps}\NormalTok{.}\OtherTok{stderr}\NormalTok{.}\FunctionTok{on}\NormalTok{(}\StringTok{'data'}\NormalTok{, }\KeywordTok{function} \NormalTok{(data) \{}
  \OtherTok{console}\NormalTok{.}\FunctionTok{log}\NormalTok{(}\StringTok{'ps stderr: '} \NormalTok{+ data);}
\NormalTok{\});}

\OtherTok{ps}\NormalTok{.}\FunctionTok{on}\NormalTok{(}\StringTok{'close'}\NormalTok{, }\KeywordTok{function} \NormalTok{(code) \{}
  \KeywordTok{if} \NormalTok{(code !== }\DecValTok{0}\NormalTok{) \{}
    \OtherTok{console}\NormalTok{.}\FunctionTok{log}\NormalTok{(}\StringTok{'ps process exited with code '} \NormalTok{+ code);}
  \NormalTok{\}}
  \OtherTok{grep}\NormalTok{.}\OtherTok{stdin}\NormalTok{.}\FunctionTok{end}\NormalTok{();}
\NormalTok{\});}

\OtherTok{grep}\NormalTok{.}\OtherTok{stdout}\NormalTok{.}\FunctionTok{on}\NormalTok{(}\StringTok{'data'}\NormalTok{, }\KeywordTok{function} \NormalTok{(data) \{}
  \OtherTok{console}\NormalTok{.}\FunctionTok{log}\NormalTok{(}\StringTok{''} \NormalTok{+ data);}
\NormalTok{\});}

\OtherTok{grep}\NormalTok{.}\OtherTok{stderr}\NormalTok{.}\FunctionTok{on}\NormalTok{(}\StringTok{'data'}\NormalTok{, }\KeywordTok{function} \NormalTok{(data) \{}
  \OtherTok{console}\NormalTok{.}\FunctionTok{log}\NormalTok{(}\StringTok{'grep stderr: '} \NormalTok{+ data);}
\NormalTok{\});}

\OtherTok{grep}\NormalTok{.}\FunctionTok{on}\NormalTok{(}\StringTok{'close'}\NormalTok{, }\KeywordTok{function} \NormalTok{(code) \{}
  \KeywordTok{if} \NormalTok{(code !== }\DecValTok{0}\NormalTok{) \{}
    \OtherTok{console}\NormalTok{.}\FunctionTok{log}\NormalTok{(}\StringTok{'grep process exited with code '} \NormalTok{+ code);}
  \NormalTok{\}}
\NormalTok{\});}
\end{Highlighting}
\end{Shaded}

Example of checking for failed exec:

\begin{Shaded}
\begin{Highlighting}[]
\KeywordTok{var} \NormalTok{spawn = }\FunctionTok{require}\NormalTok{(}\StringTok{'child_process'}\NormalTok{).}\FunctionTok{spawn}\NormalTok{,}
    \NormalTok{child = }\FunctionTok{spawn}\NormalTok{(}\StringTok{'bad_command'}\NormalTok{);}

\OtherTok{child}\NormalTok{.}\FunctionTok{on}\NormalTok{(}\StringTok{'error'}\NormalTok{, }\KeywordTok{function} \NormalTok{(err) \{}
  \OtherTok{console}\NormalTok{.}\FunctionTok{log}\NormalTok{(}\StringTok{'Failed to start child process.'}\NormalTok{);}
\NormalTok{\});}
\end{Highlighting}
\end{Shaded}

Note that if spawn receives an empty options object, it will result in
spawning the process with an empty environment rather than using
\texttt{process.env}. This due to backwards compatibility issues with a
deprecated API.

The `stdio' option to \texttt{child\_process.spawn()} is an array where
each index corresponds to a fd in the child. The value is one of the
following:

\begin{enumerate}
\def\labelenumi{\arabic{enumi}.}
\itemsep1pt\parskip0pt\parsep0pt
\item
  \texttt{'pipe'} - Create a pipe between the child process and the
  parent process. The parent end of the pipe is exposed to the parent as
  a property on the \texttt{child\_process} object as
  \texttt{ChildProcess.stdio{[}fd{]}}. Pipes created for fds 0 - 2 are
  also available as ChildProcess.stdin, ChildProcess.stdout and
  ChildProcess.stderr, respectively.
\item
  \texttt{'ipc'} - Create an IPC channel for passing messages/file
  descriptors between parent and child. A ChildProcess may have at most
  \emph{one} IPC stdio file descriptor. Setting this option enables the
  ChildProcess.send() method. If the child writes JSON messages to this
  file descriptor, then this will trigger ChildProcess.on(`message'). If
  the child is a Node.js program, then the presence of an IPC channel
  will enable process.send() and process.on(`message').
\item
  \texttt{'ignore'} - Do not set this file descriptor in the child. Note
  that Node will always open fd 0 - 2 for the processes it spawns. When
  any of these is ignored node will open \texttt{/dev/null} and attach
  it to the child's fd.
\item
  \texttt{Stream} object - Share a readable or writable stream that
  refers to a tty, file, socket, or a pipe with the child process. The
  stream's underlying file descriptor is duplicated in the child process
  to the fd that corresponds to the index in the \texttt{stdio} array.
  Note that the stream must have an underlying descriptor (file streams
  do not until the \texttt{'open'} event has occurred).
\item
  Positive integer - The integer value is interpreted as a file
  descriptor that is is currently open in the parent process. It is
  shared with the child process, similar to how \texttt{Stream} objects
  can be shared.
\item
  \texttt{null}, \texttt{undefined} - Use default value. For stdio fds
  0, 1 and 2 (in other words, stdin, stdout, and stderr) a pipe is
  created. For fd 3 and up, the default is \texttt{'ignore'}.
\end{enumerate}

As a shorthand, the \texttt{stdio} argument may also be one of the
following strings, rather than an array:

\begin{itemize}
\itemsep1pt\parskip0pt\parsep0pt
\item
  \texttt{ignore} - \texttt{{[}'ignore', 'ignore', 'ignore'{]}}
\item
  \texttt{pipe} - \texttt{{[}'pipe', 'pipe', 'pipe'{]}}
\item
  \texttt{inherit} -
  \texttt{{[}process.stdin, process.stdout, process.stderr{]}} or
  \texttt{{[}0,1,2{]}}
\end{itemize}

Example:

\begin{Shaded}
\begin{Highlighting}[]
\KeywordTok{var} \NormalTok{spawn = }\FunctionTok{require}\NormalTok{(}\StringTok{'child_process'}\NormalTok{).}\FunctionTok{spawn}\NormalTok{;}

\CommentTok{// Child will use parent's stdios}
\FunctionTok{spawn}\NormalTok{(}\StringTok{'prg'}\NormalTok{, [], \{ }\DataTypeTok{stdio}\NormalTok{: }\StringTok{'inherit'} \NormalTok{\});}

\CommentTok{// Spawn child sharing only stderr}
\FunctionTok{spawn}\NormalTok{(}\StringTok{'prg'}\NormalTok{, [], \{ }\DataTypeTok{stdio}\NormalTok{: [}\StringTok{'pipe'}\NormalTok{, }\StringTok{'pipe'}\NormalTok{, }\OtherTok{process}\NormalTok{.}\FunctionTok{stderr}\NormalTok{] \});}

\CommentTok{// Open an extra fd=4, to interact with programs present a}
\CommentTok{// startd-style interface.}
\FunctionTok{spawn}\NormalTok{(}\StringTok{'prg'}\NormalTok{, [], \{ }\DataTypeTok{stdio}\NormalTok{: [}\StringTok{'pipe'}\NormalTok{, }\KeywordTok{null}\NormalTok{, }\KeywordTok{null}\NormalTok{, }\KeywordTok{null}\NormalTok{, }\StringTok{'pipe'}\NormalTok{] \});}
\end{Highlighting}
\end{Shaded}

If the \texttt{detached} option is set, the child process will be made
the leader of a new process group. This makes it possible for the child
to continue running after the parent exits.

By default, the parent will wait for the detached child to exit. To
prevent the parent from waiting for a given \texttt{child}, use the
\texttt{child.unref()} method, and the parent's event loop will not
include the child in its reference count.

Example of detaching a long-running process and redirecting its output
to a file:

\begin{Shaded}
\begin{Highlighting}[]
 \KeywordTok{var} \NormalTok{fs = }\FunctionTok{require}\NormalTok{(}\StringTok{'fs'}\NormalTok{),}
     \NormalTok{spawn = }\FunctionTok{require}\NormalTok{(}\StringTok{'child_process'}\NormalTok{).}\FunctionTok{spawn}\NormalTok{,}
     \NormalTok{out = }\OtherTok{fs}\NormalTok{.}\FunctionTok{openSync}\NormalTok{(}\StringTok{'./out.log'}\NormalTok{, }\StringTok{'a'}\NormalTok{),}
     \NormalTok{err = }\OtherTok{fs}\NormalTok{.}\FunctionTok{openSync}\NormalTok{(}\StringTok{'./out.log'}\NormalTok{, }\StringTok{'a'}\NormalTok{);}

 \KeywordTok{var} \NormalTok{child = }\FunctionTok{spawn}\NormalTok{(}\StringTok{'prg'}\NormalTok{, [], \{}
   \DataTypeTok{detached}\NormalTok{: }\KeywordTok{true}\NormalTok{,}
   \DataTypeTok{stdio}\NormalTok{: [ }\StringTok{'ignore'}\NormalTok{, out, err ]}
 \NormalTok{\});}

 \OtherTok{child}\NormalTok{.}\FunctionTok{unref}\NormalTok{();}
\end{Highlighting}
\end{Shaded}

When using the \texttt{detached} option to start a long-running process,
the process will not stay running in the background unless it is
provided with a \texttt{stdio} configuration that is not connected to
the parent. If the parent's \texttt{stdio} is inherited, the child will
remain attached to the controlling terminal.

There is a deprecated option called \texttt{customFds} which allows one
to specify specific file descriptors for the stdio of the child process.
This API was not portable to all platforms and therefore removed. With
\texttt{customFds} it was possible to hook up the new process'
\texttt{{[}stdin, stdout, stderr{]}} to existing streams; \texttt{-1}
meant that a new stream should be created. Use at your own risk.

See also: \texttt{child\_process.exec()} and
\texttt{child\_process.fork()}

\subsection{child\_process.exec(command, {[}options{]},
callback)}\label{childux5fprocess.execcommand-options-callback}

\begin{itemize}
\itemsep1pt\parskip0pt\parsep0pt
\item
  \texttt{command} \{String\} The command to run, with space-separated
  arguments
\item
  \texttt{options} \{Object\}
\item
  \texttt{cwd} \{String\} Current working directory of the child process
\item
  \texttt{env} \{Object\} Environment key-value pairs
\item
  \texttt{encoding} \{String\} (Default: `utf8')
\item
  \texttt{shell} \{String\} Shell to execute the command with (Default:
  `/bin/sh' on UNIX, `cmd.exe' on Windows, The shell should understand
  the \texttt{-c} switch on UNIX or \texttt{/s /c} on Windows. On
  Windows, command line parsing should be compatible with
  \texttt{cmd.exe}.)
\item
  \texttt{timeout} \{Number\} (Default: 0)
\item
  \texttt{maxBuffer} \{Number\} (Default: \texttt{200*1024})
\item
  \texttt{killSignal} \{String\} (Default: `SIGTERM')
\item
  \texttt{callback} \{Function\} called with the output when process
  terminates
\item
  \texttt{error} \{Error\}
\item
  \texttt{stdout} \{Buffer\}
\item
  \texttt{stderr} \{Buffer\}
\item
  Return: ChildProcess object
\end{itemize}

Runs a command in a shell and buffers the output.

\begin{Shaded}
\begin{Highlighting}[]
\KeywordTok{var} \NormalTok{exec = }\FunctionTok{require}\NormalTok{(}\StringTok{'child_process'}\NormalTok{).}\FunctionTok{exec}\NormalTok{,}
    \NormalTok{child;}

\NormalTok{child = }\FunctionTok{exec}\NormalTok{(}\StringTok{'cat *.js bad_file | wc -l'}\NormalTok{,}
  \KeywordTok{function} \NormalTok{(error, stdout, stderr) \{}
    \OtherTok{console}\NormalTok{.}\FunctionTok{log}\NormalTok{(}\StringTok{'stdout: '} \NormalTok{+ stdout);}
    \OtherTok{console}\NormalTok{.}\FunctionTok{log}\NormalTok{(}\StringTok{'stderr: '} \NormalTok{+ stderr);}
    \KeywordTok{if} \NormalTok{(error !== }\KeywordTok{null}\NormalTok{) \{}
      \OtherTok{console}\NormalTok{.}\FunctionTok{log}\NormalTok{(}\StringTok{'exec error: '} \NormalTok{+ error);}
    \NormalTok{\}}
\NormalTok{\});}
\end{Highlighting}
\end{Shaded}

The callback gets the arguments \texttt{(error, stdout, stderr)}. On
success, \texttt{error} will be \texttt{null}. On error, \texttt{error}
will be an instance of \texttt{Error} and \texttt{error.code} will be
the exit code of the child process, and \texttt{error.signal} will be
set to the signal that terminated the process.

There is a second optional argument to specify several options. The
default options are

\begin{Shaded}
\begin{Highlighting}[]
\NormalTok{\{ }\DataTypeTok{encoding}\NormalTok{: }\StringTok{'utf8'}\NormalTok{,}
  \DataTypeTok{timeout}\NormalTok{: }\DecValTok{0}\NormalTok{,}
  \DataTypeTok{maxBuffer}\NormalTok{: }\DecValTok{200}\NormalTok{*}\DecValTok{1024}\NormalTok{,}
  \DataTypeTok{killSignal}\NormalTok{: }\StringTok{'SIGTERM'}\NormalTok{,}
  \DataTypeTok{cwd}\NormalTok{: }\KeywordTok{null}\NormalTok{,}
  \DataTypeTok{env}\NormalTok{: }\KeywordTok{null} \NormalTok{\}}
\end{Highlighting}
\end{Shaded}

If \texttt{timeout} is greater than 0, then it will kill the child
process if it runs longer than \texttt{timeout} milliseconds. The child
process is killed with \texttt{killSignal} (default:
\texttt{'SIGTERM'}). \texttt{maxBuffer} specifies the largest amount of
data allowed on stdout or stderr - if this value is exceeded then the
child process is killed.

\subsection{child\_process.execFile(file, {[}args{]}, {[}options{]},
{[}callback{]})}\label{childux5fprocess.execfilefile-args-options-callback}

\begin{itemize}
\itemsep1pt\parskip0pt\parsep0pt
\item
  \texttt{file} \{String\} The filename of the program to run
\item
  \texttt{args} \{Array\} List of string arguments
\item
  \texttt{options} \{Object\}
\item
  \texttt{cwd} \{String\} Current working directory of the child process
\item
  \texttt{env} \{Object\} Environment key-value pairs
\item
  \texttt{encoding} \{String\} (Default: `utf8')
\item
  \texttt{timeout} \{Number\} (Default: 0)
\item
  \texttt{maxBuffer} \{Number\} (Default: 200*1024)
\item
  \texttt{killSignal} \{String\} (Default: `SIGTERM')
\item
  \texttt{callback} \{Function\} called with the output when process
  terminates
\item
  \texttt{error} \{Error\}
\item
  \texttt{stdout} \{Buffer\}
\item
  \texttt{stderr} \{Buffer\}
\item
  Return: ChildProcess object
\end{itemize}

This is similar to \texttt{child\_process.exec()} except it does not
execute a subshell but rather the specified file directly. This makes it
slightly leaner than \texttt{child\_process.exec}. It has the same
options.

\subsection{child\_process.fork(modulePath, {[}args{]},
{[}options{]})}\label{childux5fprocess.forkmodulepath-args-options}

\begin{itemize}
\itemsep1pt\parskip0pt\parsep0pt
\item
  \texttt{modulePath} \{String\} The module to run in the child
\item
  \texttt{args} \{Array\} List of string arguments
\item
  \texttt{options} \{Object\}
\item
  \texttt{cwd} \{String\} Current working directory of the child process
\item
  \texttt{env} \{Object\} Environment key-value pairs
\item
  \texttt{encoding} \{String\} (Default: `utf8')
\item
  \texttt{execPath} \{String\} Executable used to create the child
  process
\item
  \texttt{execArgv} \{Array\} List of string arguments passed to the
  executable (Default: \texttt{process.execArgv})
\item
  \texttt{silent} \{Boolean\} If true, stdin, stdout, and stderr of the
  child will be piped to the parent, otherwise they will be inherited
  from the parent, see the ``pipe'' and ``inherit'' options for
  \texttt{spawn()}'s \texttt{stdio} for more details (default is false)
\item
  Return: ChildProcess object
\end{itemize}

This is a special case of the \texttt{spawn()} functionality for
spawning Node processes. In addition to having all the methods in a
normal ChildProcess instance, the returned object has a communication
channel built-in. See \texttt{child.send(message, {[}sendHandle{]})} for
details.

These child Nodes are still whole new instances of V8. Assume at least
30ms startup and 10mb memory for each new Node. That is, you cannot
create many thousands of them.

The \texttt{execPath} property in the \texttt{options} object allows for
a process to be created for the child rather than the current
\texttt{node} executable. This should be done with care and by default
will talk over the fd represented an environmental variable
\texttt{NODE\_CHANNEL\_FD} on the child process. The input and output on
this fd is expected to be line delimited JSON objects.

\subsection{child\_process.spawnSync(command, {[}args{]},
{[}options{]})}\label{childux5fprocess.spawnsynccommand-args-options}

\begin{itemize}
\itemsep1pt\parskip0pt\parsep0pt
\item
  \texttt{command} \{String\} The command to run
\item
  \texttt{args} \{Array\} List of string arguments
\item
  \texttt{options} \{Object\}
\item
  \texttt{cwd} \{String\} Current working directory of the child process
\item
  \texttt{input} \{String\textbar{}Buffer\} The value which will be
  passed as stdin to the spawned process

  \begin{itemize}
  \itemsep1pt\parskip0pt\parsep0pt
  \item
    supplying this value will override \texttt{stdio{[}0{]}}
  \end{itemize}
\item
  \texttt{stdio} \{Array\} Child's stdio configuration.
\item
  \texttt{env} \{Object\} Environment key-value pairs
\item
  \texttt{uid} \{Number\} Sets the user identity of the process. (See
  setuid(2).)
\item
  \texttt{gid} \{Number\} Sets the group identity of the process. (See
  setgid(2).)
\item
  \texttt{timeout} \{Number\} In milliseconds the maximum amount of time
  the process is allowed to run. (Default: undefined)
\item
  \texttt{killSignal} \{String\} The signal value to be used when the
  spawned process will be killed. (Default: `SIGTERM')
\item
  \texttt{maxBuffer} \{Number\}
\item
  \texttt{encoding} \{String\} The encoding used for all stdio inputs
  and outputs. (Default: `buffer')
\item
  return: \{Object\}
\item
  \texttt{pid} \{Number\} Pid of the child process
\item
  \texttt{output} \{Array\} Array of results from stdio output
\item
  \texttt{stdout} \{Buffer\textbar{}String\} The contents of
  \texttt{output{[}1{]}}
\item
  \texttt{stderr} \{Buffer\textbar{}String\} The contents of
  \texttt{output{[}2{]}}
\item
  \texttt{status} \{Number\} The exit code of the child process
\item
  \texttt{signal} \{String\} The signal used to kill the child process
\item
  \texttt{error} \{Error\} The error object if the child process failed
  or timedout
\end{itemize}

\texttt{spawnSync} will not return until the child process has fully
closed. When a timeout has been encountered and \texttt{killSignal} is
sent, the method won't return until the process has completely exited.
That is to say, if the process handles the \texttt{SIGTERM} signal and
doesn't exit, your process will wait until the child process has exited.

\subsection{child\_process.execFileSync(command, {[}args{]},
{[}options{]})}\label{childux5fprocess.execfilesynccommand-args-options}

\begin{itemize}
\itemsep1pt\parskip0pt\parsep0pt
\item
  \texttt{command} \{String\} The command to run
\item
  \texttt{args} \{Array\} List of string arguments
\item
  \texttt{options} \{Object\}
\item
  \texttt{cwd} \{String\} Current working directory of the child process
\item
  \texttt{input} \{String\textbar{}Buffer\} The value which will be
  passed as stdin to the spawned process

  \begin{itemize}
  \itemsep1pt\parskip0pt\parsep0pt
  \item
    supplying this value will override \texttt{stdio{[}0{]}}
  \end{itemize}
\item
  \texttt{stdio} \{Array\} Child's stdio configuration. (Default:
  `pipe')

  \begin{itemize}
  \itemsep1pt\parskip0pt\parsep0pt
  \item
    \texttt{stderr} by default will be output to the parent process'
    stderr unless \texttt{stdio} is specified
  \end{itemize}
\item
  \texttt{env} \{Object\} Environment key-value pairs
\item
  \texttt{uid} \{Number\} Sets the user identity of the process. (See
  setuid(2).)
\item
  \texttt{gid} \{Number\} Sets the group identity of the process. (See
  setgid(2).)
\item
  \texttt{timeout} \{Number\} In milliseconds the maximum amount of time
  the process is allowed to run. (Default: undefined)
\item
  \texttt{killSignal} \{String\} The signal value to be used when the
  spawned process will be killed. (Default: `SIGTERM')
\item
  \texttt{maxBuffer} \{Number\}
\item
  \texttt{encoding} \{String\} The encoding used for all stdio inputs
  and outputs. (Default: `buffer')
\item
  return: \{Buffer\textbar{}String\} The stdout from the command
\end{itemize}

\texttt{execFileSync} will not return until the child process has fully
closed. When a timeout has been encountered and \texttt{killSignal} is
sent, the method won't return until the process has completely exited.
That is to say, if the process handles the \texttt{SIGTERM} signal and
doesn't exit, your process will wait until the child process has exited.

If the process times out, or has a non-zero exit code, this method
\textbf{\emph{will}} throw. The \texttt{Error} object will contain the
entire result from
\hyperref[childux5fprocessux5fchildux5fprocessux5fspawnsyncux5fcommandux5fargsux5foptions]{\texttt{child\_process.spawnSync}}

\subsection{child\_process.execSync(command,
{[}options{]})}\label{childux5fprocess.execsynccommand-options}

\begin{itemize}
\itemsep1pt\parskip0pt\parsep0pt
\item
  \texttt{command} \{String\} The command to run
\item
  \texttt{options} \{Object\}
\item
  \texttt{cwd} \{String\} Current working directory of the child process
\item
  \texttt{input} \{String\textbar{}Buffer\} The value which will be
  passed as stdin to the spawned process

  \begin{itemize}
  \itemsep1pt\parskip0pt\parsep0pt
  \item
    supplying this value will override \texttt{stdio{[}0{]}}
  \end{itemize}
\item
  \texttt{stdio} \{Array\} Child's stdio configuration. (Default:
  `pipe')

  \begin{itemize}
  \itemsep1pt\parskip0pt\parsep0pt
  \item
    \texttt{stderr} by default will be output to the parent process'
    stderr unless \texttt{stdio} is specified
  \end{itemize}
\item
  \texttt{env} \{Object\} Environment key-value pairs
\item
  \texttt{uid} \{Number\} Sets the user identity of the process. (See
  setuid(2).)
\item
  \texttt{gid} \{Number\} Sets the group identity of the process. (See
  setgid(2).)
\item
  \texttt{timeout} \{Number\} In milliseconds the maximum amount of time
  the process is allowed to run. (Default: undefined)
\item
  \texttt{killSignal} \{String\} The signal value to be used when the
  spawned process will be killed. (Default: `SIGTERM')
\item
  \texttt{maxBuffer} \{Number\}
\item
  \texttt{encoding} \{String\} The encoding used for all stdio inputs
  and outputs. (Default: `buffer')
\item
  return: \{Buffer\textbar{}String\} The stdout from the command
\end{itemize}

\texttt{execSync} will not return until the child process has fully
closed. When a timeout has been encountered and \texttt{killSignal} is
sent, the method won't return until the process has completely exited.
That is to say, if the process handles the \texttt{SIGTERM} signal and
doesn't exit, your process will wait until the child process has exited.

If the process times out, or has a non-zero exit code, this method
\textbf{\emph{will}} throw. The \texttt{Error} object will contain the
entire result from
\hyperref[childux5fprocessux5fchildux5fprocessux5fspawnsyncux5fcommandux5fargsux5foptions]{\texttt{child\_process.spawnSync}}
