\section{punycode}\label{punycode}

\begin{Shaded}
\begin{Highlighting}[]
\NormalTok{Stability: }\DecValTok{2} \NormalTok{- Unstable}
\end{Highlighting}
\end{Shaded}

\href{http://mths.be/punycode}{Punycode.js} is bundled with Node.js
v0.6.2+. Use \texttt{require('punycode')} to access it. (To use it with
other Node.js versions, use npm to install the \texttt{punycode} module
first.)

\subsection{punycode.decode(string)}\label{punycode.decodestring}

Converts a Punycode string of ASCII-only symbols to a string of Unicode
symbols.

\begin{Shaded}
\begin{Highlighting}[]
\CommentTok{// decode domain name parts}
\OtherTok{punycode}\NormalTok{.}\FunctionTok{decode}\NormalTok{(}\StringTok{'maana-pta'}\NormalTok{); }\CommentTok{// 'mañana'}
\OtherTok{punycode}\NormalTok{.}\FunctionTok{decode}\NormalTok{(}\StringTok{'--dqo34k'}\NormalTok{); }\CommentTok{// '☃-⌘'}
\end{Highlighting}
\end{Shaded}

\subsection{punycode.encode(string)}\label{punycode.encodestring}

Converts a string of Unicode symbols to a Punycode string of ASCII-only
symbols.

\begin{Shaded}
\begin{Highlighting}[]
\CommentTok{// encode domain name parts}
\OtherTok{punycode}\NormalTok{.}\FunctionTok{encode}\NormalTok{(}\StringTok{'mañana'}\NormalTok{); }\CommentTok{// 'maana-pta'}
\OtherTok{punycode}\NormalTok{.}\FunctionTok{encode}\NormalTok{(}\StringTok{'☃-⌘'}\NormalTok{); }\CommentTok{// '--dqo34k'}
\end{Highlighting}
\end{Shaded}

\subsection{punycode.toUnicode(domain)}\label{punycode.tounicodedomain}

Converts a Punycode string representing a domain name to Unicode. Only
the Punycoded parts of the domain name will be converted, i.e.~it
doesn't matter if you call it on a string that has already been
converted to Unicode.

\begin{Shaded}
\begin{Highlighting}[]
\CommentTok{// decode domain names}
\OtherTok{punycode}\NormalTok{.}\FunctionTok{toUnicode}\NormalTok{(}\StringTok{'xn--maana-pta.com'}\NormalTok{); }\CommentTok{// 'mañana.com'}
\OtherTok{punycode}\NormalTok{.}\FunctionTok{toUnicode}\NormalTok{(}\StringTok{'xn----dqo34k.com'}\NormalTok{); }\CommentTok{// '☃-⌘.com'}
\end{Highlighting}
\end{Shaded}

\subsection{punycode.toASCII(domain)}\label{punycode.toasciidomain}

Converts a Unicode string representing a domain name to Punycode. Only
the non-ASCII parts of the domain name will be converted, i.e.~it
doesn't matter if you call it with a domain that's already in ASCII.

\begin{Shaded}
\begin{Highlighting}[]
\CommentTok{// encode domain names}
\OtherTok{punycode}\NormalTok{.}\FunctionTok{toASCII}\NormalTok{(}\StringTok{'mañana.com'}\NormalTok{); }\CommentTok{// 'xn--maana-pta.com'}
\OtherTok{punycode}\NormalTok{.}\FunctionTok{toASCII}\NormalTok{(}\StringTok{'☃-⌘.com'}\NormalTok{); }\CommentTok{// 'xn----dqo34k.com'}
\end{Highlighting}
\end{Shaded}

\subsection{punycode.ucs2}\label{punycode.ucs2}

\subsubsection{punycode.ucs2.decode(string)}\label{punycode.ucs2.decodestring}

Creates an array containing the numeric code point values of each
Unicode symbol in the string. While
\href{http://mathiasbynens.be/notes/javascript-encoding}{JavaScript uses
UCS-2 internally}, this function will convert a pair of surrogate halves
(each of which UCS-2 exposes as separate characters) into a single code
point, matching UTF-16.

\begin{Shaded}
\begin{Highlighting}[]
\OtherTok{punycode}\NormalTok{.}\OtherTok{ucs2}\NormalTok{.}\FunctionTok{decode}\NormalTok{(}\StringTok{'abc'}\NormalTok{); }\CommentTok{// [0x61, 0x62, 0x63]}
\CommentTok{// surrogate pair for U+1D306 tetragram for centre:}
\OtherTok{punycode}\NormalTok{.}\OtherTok{ucs2}\NormalTok{.}\FunctionTok{decode}\NormalTok{(}\StringTok{'\textbackslash{}uD834\textbackslash{}uDF06'}\NormalTok{); }\CommentTok{// [0x1D306]}
\end{Highlighting}
\end{Shaded}

\subsubsection{punycode.ucs2.encode(codePoints)}\label{punycode.ucs2.encodecodepoints}

Creates a string based on an array of numeric code point values.

\begin{Shaded}
\begin{Highlighting}[]
\OtherTok{punycode}\NormalTok{.}\OtherTok{ucs2}\NormalTok{.}\FunctionTok{encode}\NormalTok{([}\BaseNTok{0x61}\NormalTok{, }\BaseNTok{0x62}\NormalTok{, }\BaseNTok{0x63}\NormalTok{]); }\CommentTok{// 'abc'}
\OtherTok{punycode}\NormalTok{.}\OtherTok{ucs2}\NormalTok{.}\FunctionTok{encode}\NormalTok{([}\BaseNTok{0x1D306}\NormalTok{]); }\CommentTok{// '\textbackslash{}uD834\textbackslash{}uDF06'}
\end{Highlighting}
\end{Shaded}

\subsection{punycode.version}\label{punycode.version}

A string representing the current Punycode.js version number.
