\section{net}

\begin{Shaded}
\begin{Highlighting}[]
\DataTypeTok{Stability}\NormalTok{: }\DecValTok{3} \NormalTok{- Stable}
\end{Highlighting}
\end{Shaded}

The \texttt{net} module provides you with an asynchronous network
wrapper. It contains methods for creating both servers and clients
(called streams). You can include this module with
\texttt{require('net');}

\subsection{net.createServer({[}options{]}, {[}connectionListener{]})}

Creates a new TCP server. The \texttt{connectionListener} argument is
automatically set as a listener for the
\hyperref[net_event_connection]{`connection'} event.

\texttt{options} is an object with the following defaults:

\begin{Shaded}
\begin{Highlighting}[]
\NormalTok{\{ }\DataTypeTok{allowHalfOpen}\NormalTok{: }\KeywordTok{false}
\NormalTok{\}}
\end{Highlighting}
\end{Shaded}

If \texttt{allowHalfOpen} is \texttt{true}, then the socket won't
automatically send a FIN packet when the other end of the socket sends a
FIN packet. The socket becomes non-readable, but still writable. You
should call the \texttt{end()} method explicitly. See
\hyperref[net_event_end]{`end'} event for more information.

Here is an example of a echo server which listens for connections on
port 8124:

\begin{Shaded}
\begin{Highlighting}[]
\KeywordTok{var} \NormalTok{net = require(}\CharTok{'net'}\NormalTok{);}
\KeywordTok{var} \NormalTok{server = }\KeywordTok{net}\NormalTok{.}\FunctionTok{createServer}\NormalTok{(}\KeywordTok{function}\NormalTok{(c) \{ }\CommentTok{//'connection' listener}
  \KeywordTok{console}\NormalTok{.}\FunctionTok{log}\NormalTok{(}\CharTok{'server connected'}\NormalTok{);}
  \KeywordTok{c}\NormalTok{.}\FunctionTok{on}\NormalTok{(}\CharTok{'end'}\NormalTok{, }\KeywordTok{function}\NormalTok{() \{}
    \KeywordTok{console}\NormalTok{.}\FunctionTok{log}\NormalTok{(}\CharTok{'server disconnected'}\NormalTok{);}
  \NormalTok{\});}
  \KeywordTok{c}\NormalTok{.}\FunctionTok{write}\NormalTok{(}\CharTok{'hello\textbackslash{}r\textbackslash{}n'}\NormalTok{);}
  \KeywordTok{c}\NormalTok{.}\FunctionTok{pipe}\NormalTok{(c);}
\NormalTok{\});}
\KeywordTok{server}\NormalTok{.}\FunctionTok{listen}\NormalTok{(}\DecValTok{8124}\NormalTok{, }\KeywordTok{function}\NormalTok{() \{ }\CommentTok{//'listening' listener}
  \KeywordTok{console}\NormalTok{.}\FunctionTok{log}\NormalTok{(}\CharTok{'server bound'}\NormalTok{);}
\NormalTok{\});}
\end{Highlighting}
\end{Shaded}

Test this by using \texttt{telnet}:

\begin{Shaded}
\begin{Highlighting}[]
\NormalTok{telnet localhost }\DecValTok{8124}
\end{Highlighting}
\end{Shaded}

To listen on the socket \texttt{/tmp/echo.sock} the third line from the
last would just be changed to

\begin{Shaded}
\begin{Highlighting}[]
\KeywordTok{server}\NormalTok{.}\FunctionTok{listen}\NormalTok{(}\CharTok{'/tmp/echo.sock'}\NormalTok{, }\KeywordTok{function}\NormalTok{() \{ }\CommentTok{//'listening' listener}
\end{Highlighting}
\end{Shaded}

Use \texttt{nc} to connect to a UNIX domain socket server:

\begin{Shaded}
\begin{Highlighting}[]
\NormalTok{nc -U /tmp/}\KeywordTok{echo}\NormalTok{.}\FunctionTok{sock}
\end{Highlighting}
\end{Shaded}

\subsection{net.connect(options, {[}connectionListener{]})}

\subsection{net.createConnection(options, {[}connectionListener{]})}

Constructs a new socket object and opens the socket to the given
location. When the socket is established, the
\hyperref[net_event_connect]{`connect'} event will be emitted.

For TCP sockets, \texttt{options} argument should be an object which
specifies:

\begin{itemize}
\item
  \texttt{port}: Port the client should connect to (Required).
\item
  \texttt{host}: Host the client should connect to. Defaults to
  \texttt{'localhost'}.
\item
  \texttt{localAddress}: Local interface to bind to for network
  connections.
\end{itemize}

For UNIX domain sockets, \texttt{options} argument should be an object
which specifies:

\begin{itemize}
\item
  \texttt{path}: Path the client should connect to (Required).
\end{itemize}

Common options are:

\begin{itemize}
\item
  \texttt{allowHalfOpen}: if \texttt{true}, the socket won't
  automatically send a FIN packet when the other end of the socket sends
  a FIN packet. Defaults to \texttt{false}. See
  \hyperref[net_event_end]{`end'} event for more information.
\end{itemize}

The \texttt{connectListener} parameter will be added as an listener for
the \hyperref[net_event_connect]{`connect'} event.

Here is an example of a client of echo server as described previously:

\begin{Shaded}
\begin{Highlighting}[]
\KeywordTok{var} \NormalTok{net = require(}\CharTok{'net'}\NormalTok{);}
\KeywordTok{var} \NormalTok{client = }\KeywordTok{net}\NormalTok{.}\FunctionTok{connect}\NormalTok{(\{}\DataTypeTok{port}\NormalTok{: }\DecValTok{8124}\NormalTok{\},}
    \KeywordTok{function}\NormalTok{() \{ }\CommentTok{//'connect' listener}
  \KeywordTok{console}\NormalTok{.}\FunctionTok{log}\NormalTok{(}\CharTok{'client connected'}\NormalTok{);}
  \KeywordTok{client}\NormalTok{.}\FunctionTok{write}\NormalTok{(}\CharTok{'world!\textbackslash{}r\textbackslash{}n'}\NormalTok{);}
\NormalTok{\});}
\KeywordTok{client}\NormalTok{.}\FunctionTok{on}\NormalTok{(}\CharTok{'data'}\NormalTok{, }\KeywordTok{function}\NormalTok{(data) \{}
  \KeywordTok{console}\NormalTok{.}\FunctionTok{log}\NormalTok{(}\KeywordTok{data}\NormalTok{.}\FunctionTok{toString}\NormalTok{());}
  \KeywordTok{client}\NormalTok{.}\FunctionTok{end}\NormalTok{();}
\NormalTok{\});}
\KeywordTok{client}\NormalTok{.}\FunctionTok{on}\NormalTok{(}\CharTok{'end'}\NormalTok{, }\KeywordTok{function}\NormalTok{() \{}
  \KeywordTok{console}\NormalTok{.}\FunctionTok{log}\NormalTok{(}\CharTok{'client disconnected'}\NormalTok{);}
\NormalTok{\});}
\end{Highlighting}
\end{Shaded}

To connect on the socket \texttt{/tmp/echo.sock} the second line would
just be changed to

\begin{Shaded}
\begin{Highlighting}[]
\KeywordTok{var} \NormalTok{client = }\KeywordTok{net}\NormalTok{.}\FunctionTok{connect}\NormalTok{(\{}\DataTypeTok{path}\NormalTok{: }\CharTok{'/tmp/echo.sock'}\NormalTok{\},}
\end{Highlighting}
\end{Shaded}

\subsection{net.connect(port, {[}host{]}, {[}connectListener{]})}

\subsection{net.createConnection(port, {[}host{]},
{[}connectListener{]})}

Creates a TCP connection to \texttt{port} on \texttt{host}. If
\texttt{host} is omitted, \texttt{'localhost'} will be assumed. The
\texttt{connectListener} parameter will be added as an listener for the
\hyperref[net_event_connect]{`connect'} event.

\subsection{net.connect(path, {[}connectListener{]})}

\subsection{net.createConnection(path, {[}connectListener{]})}

Creates unix socket connection to \texttt{path}. The
\texttt{connectListener} parameter will be added as an listener for the
\hyperref[net_event_connect]{`connect'} event.

\subsection{Class: net.Server}

This class is used to create a TCP or UNIX server. A server is a
\texttt{net.Socket} that can listen for new incoming connections.

\subsubsection{server.listen(port, {[}host{]}, {[}backlog{]},
{[}listeningListener{]})}

Begin accepting connections on the specified \texttt{port} and
\texttt{host}. If the \texttt{host} is omitted, the server will accept
connections directed to any IPv4 address (\texttt{INADDR\_ANY}). A port
value of zero will assign a random port.

Backlog is the maximum length of the queue of pending connections. The
actual length will be determined by your OS through sysctl settings such
as \texttt{tcp\_max\_syn\_backlog} and \texttt{somaxconn} on linux. The
default value of this parameter is 511 (not 512).

This function is asynchronous. When the server has been bound,
\hyperref[net_event_listening]{`listening'} event will be emitted. The
last parameter \texttt{listeningListener} will be added as an listener
for the \hyperref[net_event_listening]{`listening'} event.

One issue some users run into is getting \texttt{EADDRINUSE} errors.
This means that another server is already running on the requested port.
One way of handling this would be to wait a second and then try again.
This can be done with

\begin{Shaded}
\begin{Highlighting}[]
\KeywordTok{server}\NormalTok{.}\FunctionTok{on}\NormalTok{(}\CharTok{'error'}\NormalTok{, }\KeywordTok{function} \NormalTok{(e) \{}
  \KeywordTok{if} \NormalTok{(}\KeywordTok{e}\NormalTok{.}\FunctionTok{code} \NormalTok{== }\CharTok{'EADDRINUSE'}\NormalTok{) \{}
    \KeywordTok{console}\NormalTok{.}\FunctionTok{log}\NormalTok{(}\CharTok{'Address in use, retrying...'}\NormalTok{);}
    \NormalTok{setTimeout(}\KeywordTok{function} \NormalTok{() \{}
      \KeywordTok{server}\NormalTok{.}\FunctionTok{close}\NormalTok{();}
      \KeywordTok{server}\NormalTok{.}\FunctionTok{listen}\NormalTok{(PORT, HOST);}
    \NormalTok{\}, }\DecValTok{1000}\NormalTok{);}
  \NormalTok{\}}
\NormalTok{\});}
\end{Highlighting}
\end{Shaded}

(Note: All sockets in Node set \texttt{SO\_REUSEADDR} already)

\subsubsection{server.listen(path, {[}listeningListener{]})}

Start a UNIX socket server listening for connections on the given
\texttt{path}.

This function is asynchronous. When the server has been bound,
\hyperref[net_event_listening]{`listening'} event will be emitted. The
last parameter \texttt{listeningListener} will be added as an listener
for the \hyperref[net_event_listening]{`listening'} event.

\subsubsection{server.listen(handle, {[}listeningListener{]})}

\begin{itemize}
\item
  \texttt{handle} \{Object\}
\item
  \texttt{listeningListener} \{Function\}
\end{itemize}

The \texttt{handle} object can be set to either a server or socket
(anything with an underlying \texttt{\_handle} member), or a
\texttt{\{fd: \textless{}n\textgreater{}\}} object.

This will cause the server to accept connections on the specified
handle, but it is presumed that the file descriptor or handle has
already been bound to a port or domain socket.

Listening on a file descriptor is not supported on Windows.

This function is asynchronous. When the server has been bound,
\hyperref[event_listening_]{`listening'} event will be emitted. the
last parameter \texttt{listeningListener} will be added as an listener
for the \hyperref[event_listening_]{`listening'} event.

\subsubsection{server.close({[}cb{]})}

Stops the server from accepting new connections and keeps existing
connections. This function is asynchronous, the server is finally closed
when all connections are ended and the server emits a \texttt{'close'}
event. Optionally, you can pass a callback to listen for the
\texttt{'close'} event.

\subsubsection{server.address()}

Returns the bound address, the address family name and port of the
server as reported by the operating system. Useful to find which port
was assigned when giving getting an OS-assigned address. Returns an
object with three properties, e.g.
\texttt{\{ port: 12346, family: 'IPv4', address: '127.0.0.1' \}}

Example:

\begin{Shaded}
\begin{Highlighting}[]
\KeywordTok{var} \NormalTok{server = }\KeywordTok{net}\NormalTok{.}\FunctionTok{createServer}\NormalTok{(}\KeywordTok{function} \NormalTok{(socket) \{}
  \KeywordTok{socket}\NormalTok{.}\FunctionTok{end}\NormalTok{(}\StringTok{"goodbye}\CharTok{\textbackslash{}n}\StringTok{"}\NormalTok{);}
\NormalTok{\});}

\CommentTok{// grab a random port.}
\KeywordTok{server}\NormalTok{.}\FunctionTok{listen}\NormalTok{(}\KeywordTok{function}\NormalTok{() \{}
  \NormalTok{address = }\KeywordTok{server}\NormalTok{.}\FunctionTok{address}\NormalTok{();}
  \KeywordTok{console}\NormalTok{.}\FunctionTok{log}\NormalTok{(}\StringTok{"opened server on %j"}\NormalTok{, address);}
\NormalTok{\});}
\end{Highlighting}
\end{Shaded}

Don't call \texttt{server.address()} until the \texttt{'listening'}
event has been emitted.

\subsubsection{server.maxConnections}

Set this property to reject connections when the server's connection
count gets high.

It is not recommended to use this option once a socket has been sent to
a child with \texttt{child\_process.fork()}.

\subsubsection{server.connections}

The number of concurrent connections on the server.

This becomes \texttt{null} when sending a socket to a child with
\texttt{child\_process.fork()}.

\texttt{net.Server} is an
\href{events.html\#events\_class\_events\_eventemitter}{EventEmitter}
with the following events:

\subsubsection{Event: `listening'}

Emitted when the server has been bound after calling
\texttt{server.listen}.

\subsubsection{Event: `connection'}

\begin{itemize}
\item
  \{Socket object\} The connection object
\end{itemize}

Emitted when a new connection is made. \texttt{socket} is an instance of
\texttt{net.Socket}.

\subsubsection{Event: `close'}

Emitted when the server closes. Note that if connections exist, this
event is not emitted until all connections are ended.

\subsubsection{Event: `error'}

\begin{itemize}
\item
  \{Error Object\}
\end{itemize}

Emitted when an error occurs. The \texttt{'close'} event will be called
directly following this event. See example in discussion of
\texttt{server.listen}.

\subsection{Class: net.Socket}

This object is an abstraction of a TCP or UNIX socket.
\texttt{net.Socket} instances implement a duplex Stream interface. They
can be created by the user and used as a client (with
\texttt{connect()}) or they can be created by Node and passed to the
user through the \texttt{'connection'} event of a server.

\subsubsection{new net.Socket({[}options{]})}

Construct a new socket object.

\texttt{options} is an object with the following defaults:

\begin{Shaded}
\begin{Highlighting}[]
\NormalTok{\{ }\DataTypeTok{fd}\NormalTok{: null}
  \DataTypeTok{type}\NormalTok{: null}
  \DataTypeTok{allowHalfOpen}\NormalTok{: }\KeywordTok{false}
\NormalTok{\}}
\end{Highlighting}
\end{Shaded}

\texttt{fd} allows you to specify the existing file descriptor of
socket. \texttt{type} specified underlying protocol. It can be
\texttt{'tcp4'}, \texttt{'tcp6'}, or \texttt{'unix'}. About
\texttt{allowHalfOpen}, refer to \texttt{createServer()} and
\texttt{'end'} event.

\subsubsection{socket.connect(port, {[}host{]}, {[}connectListener{]})}

\subsubsection{socket.connect(path, {[}connectListener{]})}

Opens the connection for a given socket. If \texttt{port} and
\texttt{host} are given, then the socket will be opened as a TCP socket,
if \texttt{host} is omitted, \texttt{localhost} will be assumed. If a
\texttt{path} is given, the socket will be opened as a unix socket to
that path.

Normally this method is not needed, as \texttt{net.createConnection}
opens the socket. Use this only if you are implementing a custom Socket
or if a Socket is closed and you want to reuse it to connect to another
server.

This function is asynchronous. When the
\hyperref[net_event_connect]{`connect'} event is emitted the socket is
established. If there is a problem connecting, the \texttt{'connect'}
event will not be emitted, the \texttt{'error'} event will be emitted
with the exception.

The \texttt{connectListener} parameter will be added as an listener for
the \hyperref[net_event_connect]{`connect'} event.

\subsubsection{socket.bufferSize}

\texttt{net.Socket} has the property that \texttt{socket.write()} always
works. This is to help users get up and running quickly. The computer
cannot always keep up with the amount of data that is written to a
socket - the network connection simply might be too slow. Node will
internally queue up the data written to a socket and send it out over
the wire when it is possible. (Internally it is polling on the socket's
file descriptor for being writable).

The consequence of this internal buffering is that memory may grow. This
property shows the number of characters currently buffered to be
written. (Number of characters is approximately equal to the number of
bytes to be written, but the buffer may contain strings, and the strings
are lazily encoded, so the exact number of bytes is not known.)

Users who experience large or growing \texttt{bufferSize} should attempt
to ``throttle'' the data flows in their program with \texttt{pause()}
and \texttt{resume()}.

\subsubsection{socket.setEncoding({[}encoding{]})}

Set the encoding for the socket as a Readable Stream. See
\href{stream.html\#stream\_stream\_setencoding\_encoding}{stream.setEncoding()}
for more information.

\subsubsection{socket.write(data, {[}encoding{]}, {[}callback{]})}

Sends data on the socket. The second parameter specifies the encoding in
the case of a string--it defaults to UTF8 encoding.

Returns \texttt{true} if the entire data was flushed successfully to the
kernel buffer. Returns \texttt{false} if all or part of the data was
queued in user memory. \texttt{'drain'} will be emitted when the buffer
is again free.

The optional \texttt{callback} parameter will be executed when the data
is finally written out - this may not be immediately.

\subsubsection{socket.end({[}data{]}, {[}encoding{]})}

Half-closes the socket. i.e., it sends a FIN packet. It is possible the
server will still send some data.

If \texttt{data} is specified, it is equivalent to calling
\texttt{socket.write(data, encoding)} followed by \texttt{socket.end()}.

\subsubsection{socket.destroy()}

Ensures that no more I/O activity happens on this socket. Only necessary
in case of errors (parse error or so).

\subsubsection{socket.pause()}

Pauses the reading of data. That is, \texttt{'data'} events will not be
emitted. Useful to throttle back an upload.

\subsubsection{socket.resume()}

Resumes reading after a call to \texttt{pause()}.

\subsubsection{socket.setTimeout(timeout, {[}callback{]})}

Sets the socket to timeout after \texttt{timeout} milliseconds of
inactivity on the socket. By default \texttt{net.Socket} do not have a
timeout.

When an idle timeout is triggered the socket will receive a
\texttt{'timeout'} event but the connection will not be severed. The
user must manually \texttt{end()} or \texttt{destroy()} the socket.

If \texttt{timeout} is 0, then the existing idle timeout is disabled.

The optional \texttt{callback} parameter will be added as a one time
listener for the \texttt{'timeout'} event.

\subsubsection{socket.setNoDelay({[}noDelay{]})}

Disables the Nagle algorithm. By default TCP connections use the Nagle
algorithm, they buffer data before sending it off. Setting \texttt{true}
for \texttt{noDelay} will immediately fire off data each time
\texttt{socket.write()} is called. \texttt{noDelay} defaults to
\texttt{true}.

\subsubsection{socket.setKeepAlive({[}enable{]}, {[}initialDelay{]})}

Enable/disable keep-alive functionality, and optionally set the initial
delay before the first keepalive probe is sent on an idle socket.
\texttt{enable} defaults to \texttt{false}.

Set \texttt{initialDelay} (in milliseconds) to set the delay between the
last data packet received and the first keepalive probe. Setting 0 for
initialDelay will leave the value unchanged from the default (or
previous) setting. Defaults to \texttt{0}.

\subsubsection{socket.address()}

Returns the bound address, the address family name and port of the
socket as reported by the operating system. Returns an object with three
properties, e.g.
\texttt{\{ port: 12346, family: 'IPv4', address: '127.0.0.1' \}}

\subsubsection{socket.remoteAddress}

The string representation of the remote IP address. For example,
\texttt{'74.125.127.100'} or \texttt{'2001:4860:a005::68'}.

\subsubsection{socket.remotePort}

The numeric representation of the remote port. For example, \texttt{80}
or \texttt{21}.

\subsubsection{socket.bytesRead}

The amount of received bytes.

\subsubsection{socket.bytesWritten}

The amount of bytes sent.

\texttt{net.Socket} instances are
\href{events.html\#events\_class\_events\_eventemitter}{EventEmitter}
with the following events:

\subsubsection{Event: `connect'}

Emitted when a socket connection is successfully established. See
\texttt{connect()}.

\subsubsection{Event: `data'}

\begin{itemize}
\item
  \{Buffer object\}
\end{itemize}

Emitted when data is received. The argument \texttt{data} will be a
\texttt{Buffer} or \texttt{String}. Encoding of data is set by
\texttt{socket.setEncoding()}. (See the
\href{stream.html\#stream\_readable\_stream}{Readable Stream} section
for more information.)

Note that the \textbf{data will be lost} if there is no listener when a
\texttt{Socket} emits a \texttt{'data'} event.

\subsubsection{Event: `end'}

Emitted when the other end of the socket sends a FIN packet.

By default (\texttt{allowHalfOpen == false}) the socket will destroy its
file descriptor once it has written out its pending write queue.
However, by setting \texttt{allowHalfOpen == true} the socket will not
automatically \texttt{end()} its side allowing the user to write
arbitrary amounts of data, with the caveat that the user is required to
\texttt{end()} their side now.

\subsubsection{Event: `timeout'}

Emitted if the socket times out from inactivity. This is only to notify
that the socket has been idle. The user must manually close the
connection.

See also: \texttt{socket.setTimeout()}

\subsubsection{Event: `drain'}

Emitted when the write buffer becomes empty. Can be used to throttle
uploads.

See also: the return values of \texttt{socket.write()}

\subsubsection{Event: `error'}

\begin{itemize}
\item
  \{Error object\}
\end{itemize}

Emitted when an error occurs. The \texttt{'close'} event will be called
directly following this event.

\subsubsection{Event: `close'}

\begin{itemize}
\item
  \texttt{had\_error} \{Boolean\} true if the socket had a transmission
  error
\end{itemize}

Emitted once the socket is fully closed. The argument
\texttt{had\_error} is a boolean which says if the socket was closed due
to a transmission error.

\subsection{net.isIP(input)}

Tests if input is an IP address. Returns 0 for invalid strings, returns
4 for IP version 4 addresses, and returns 6 for IP version 6 addresses.

\subsection{net.isIPv4(input)}

Returns true if input is a version 4 IP address, otherwise returns
false.

\subsection{net.isIPv6(input)}

Returns true if input is a version 6 IP address, otherwise returns
false.
