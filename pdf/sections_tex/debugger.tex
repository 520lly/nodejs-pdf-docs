\section{Debugger}

\begin{Shaded}
\begin{Highlighting}[]
\DataTypeTok{Stability}\NormalTok{: }\DecValTok{3} \NormalTok{- Stable}
\end{Highlighting}
\end{Shaded}

V8 comes with an extensive debugger which is accessible out-of-process
via a simple
\href{http://code.google.com/p/v8/wiki/DebuggerProtocol}{TCP protocol}.
Node has a built-in client for this debugger. To use this, start Node
with the \texttt{debug} argument; a prompt will appear:

\begin{Shaded}
\begin{Highlighting}[]
\NormalTok{% node debug }\KeywordTok{myscript}\NormalTok{.}\FunctionTok{js}
\NormalTok{< debugger listening on port }\DecValTok{5858}
\KeywordTok{connecting..}\NormalTok{. ok}
\KeywordTok{break} \KeywordTok{in} \NormalTok{/home/indutny/Code/git/indutny/}\DataTypeTok{myscript.js}\NormalTok{:}\DecValTok{1}
  \DecValTok{1} \NormalTok{x = }\DecValTok{5}\NormalTok{;}
  \DecValTok{2} \NormalTok{setTimeout(}\KeywordTok{function} \NormalTok{() \{}
  \DecValTok{3}   \NormalTok{debugger;}
\NormalTok{debug>}
\end{Highlighting}
\end{Shaded}

Node's debugger client doesn't support the full range of commands, but
simple step and inspection is possible. By putting the statement
\texttt{debugger;} into the source code of your script, you will enable
a breakpoint.

For example, suppose \texttt{myscript.js} looked like this:

\begin{Shaded}
\begin{Highlighting}[]
\CommentTok{// myscript.js}
\NormalTok{x = }\DecValTok{5}\NormalTok{;}
\NormalTok{setTimeout(}\KeywordTok{function} \NormalTok{() \{}
  \NormalTok{debugger;}
  \KeywordTok{console}\NormalTok{.}\FunctionTok{log}\NormalTok{(}\StringTok{"world"}\NormalTok{);}
\NormalTok{\}, }\DecValTok{1000}\NormalTok{);}
\KeywordTok{console}\NormalTok{.}\FunctionTok{log}\NormalTok{(}\StringTok{"hello"}\NormalTok{);}
\end{Highlighting}
\end{Shaded}

Then once the debugger is run, it will break on line 4.

\begin{Shaded}
\begin{Highlighting}[]
\NormalTok{% node debug }\KeywordTok{myscript}\NormalTok{.}\FunctionTok{js}
\NormalTok{< debugger listening on port }\DecValTok{5858}
\KeywordTok{connecting..}\NormalTok{. ok}
\KeywordTok{break} \KeywordTok{in} \NormalTok{/home/indutny/Code/git/indutny/}\DataTypeTok{myscript.js}\NormalTok{:}\DecValTok{1}
  \DecValTok{1} \NormalTok{x = }\DecValTok{5}\NormalTok{;}
  \DecValTok{2} \NormalTok{setTimeout(}\KeywordTok{function} \NormalTok{() \{}
  \DecValTok{3}   \NormalTok{debugger;}
\NormalTok{debug> cont}
\NormalTok{< hello}
\KeywordTok{break} \KeywordTok{in} \NormalTok{/home/indutny/Code/git/indutny/}\DataTypeTok{myscript.js}\NormalTok{:}\DecValTok{3}
  \DecValTok{1} \NormalTok{x = }\DecValTok{5}\NormalTok{;}
  \DecValTok{2} \NormalTok{setTimeout(}\KeywordTok{function} \NormalTok{() \{}
  \DecValTok{3}   \NormalTok{debugger;}
  \DecValTok{4}   \KeywordTok{console}\NormalTok{.}\FunctionTok{log}\NormalTok{(}\StringTok{"world"}\NormalTok{);}
  \DecValTok{5} \NormalTok{\}, }\DecValTok{1000}\NormalTok{);}
\NormalTok{debug> next}
\KeywordTok{break} \KeywordTok{in} \NormalTok{/home/indutny/Code/git/indutny/}\DataTypeTok{myscript.js}\NormalTok{:}\DecValTok{4}
  \DecValTok{2} \NormalTok{setTimeout(}\KeywordTok{function} \NormalTok{() \{}
  \DecValTok{3}   \NormalTok{debugger;}
  \DecValTok{4}   \KeywordTok{console}\NormalTok{.}\FunctionTok{log}\NormalTok{(}\StringTok{"world"}\NormalTok{);}
  \DecValTok{5} \NormalTok{\}, }\DecValTok{1000}\NormalTok{);}
  \DecValTok{6} \KeywordTok{console}\NormalTok{.}\FunctionTok{log}\NormalTok{(}\StringTok{"hello"}\NormalTok{);}
\NormalTok{debug> repl}
\NormalTok{Press Ctrl + C to leave debug repl}
\NormalTok{> x}
\DecValTok{5}
\NormalTok{> }\DecValTok{2+2}
\DecValTok{4}
\NormalTok{debug> next}
\NormalTok{< world}
\KeywordTok{break} \KeywordTok{in} \NormalTok{/home/indutny/Code/git/indutny/}\DataTypeTok{myscript.js}\NormalTok{:}\DecValTok{5}
  \DecValTok{3}   \NormalTok{debugger;}
  \DecValTok{4}   \KeywordTok{console}\NormalTok{.}\FunctionTok{log}\NormalTok{(}\StringTok{"world"}\NormalTok{);}
  \DecValTok{5} \NormalTok{\}, }\DecValTok{1000}\NormalTok{);}
  \DecValTok{6} \KeywordTok{console}\NormalTok{.}\FunctionTok{log}\NormalTok{(}\StringTok{"hello"}\NormalTok{);}
  \DecValTok{7}
\NormalTok{debug> quit}
\NormalTok{%}
\end{Highlighting}
\end{Shaded}

The \texttt{repl} command allows you to evaluate code remotely. The
\texttt{next} command steps over to the next line. There are a few other
commands available and more to come. Type \texttt{help} to see others.

\subsection{Watchers}

You can watch expression and variable values while debugging your code.
On every breakpoint each expression from the watchers list will be
evaluated in the current context and displayed just before the
breakpoint's source code listing.

To start watching an expression, type \texttt{watch("my\_expression")}.
\texttt{watchers} prints the active watchers. To remove a watcher, type
\texttt{unwatch("my\_expression")}.

\subsection{Commands reference}

\subsubsection{Stepping}

\begin{itemize}
\item
  \texttt{cont}, \texttt{c} - Continue execution
\item
  \texttt{next}, \texttt{n} - Step next
\item
  \texttt{step}, \texttt{s} - Step in
\item
  \texttt{out}, \texttt{o} - Step out
\item
  \texttt{pause} - Pause running code (like pause button in Developer
  TOols)
\end{itemize}

\subsubsection{Breakpoints}

\begin{itemize}
\item
  \texttt{setBreakpoint()}, \texttt{sb()} - Set breakpoint on current
  line
\item
  \texttt{setBreakpoint(line)}, \texttt{sb(line)} - Set breakpoint on
  specific line
\item
  \texttt{setBreakpoint('fn()')}, \texttt{sb(...)} - Set breakpoint on a
  first statement in functions body
\item
  \texttt{setBreakpoint('script.js', 1)}, \texttt{sb(...)} - Set
  breakpoint on first line of script.js
\item
  \texttt{clearBreakpoint}, \texttt{cb(...)} - Clear breakpoint
\end{itemize}

\subsubsection{Info}

\begin{itemize}
\item
  \texttt{backtrace}, \texttt{bt} - Print backtrace of current execution
  frame
\item
  \texttt{list(5)} - List scripts source code with 5 line context (5
  lines before and after)
\item
  \texttt{watch(expr)} - Add expression to watch list
\item
  \texttt{unwatch(expr)} - Remove expression from watch list
\item
  \texttt{watchers} - List all watchers and their values (automatically
  listed on each breakpoint)
\item
  \texttt{repl} - Open debugger's repl for evaluation in debugging
  script's context
\end{itemize}

\subsubsection{Execution control}

\begin{itemize}
\item
  \texttt{run} - Run script (automatically runs on debugger's start)
\item
  \texttt{restart} - Restart script
\item
  \texttt{kill} - Kill script
\end{itemize}

\subsubsection{Various}

\begin{itemize}
\item
  \texttt{scripts} - List all loaded scripts
\item
  \texttt{version} - Display v8's version
\end{itemize}

\subsection{Advanced Usage}

The V8 debugger can be enabled and accessed either by starting Node with
the \texttt{-{}-debug} command-line flag or by signaling an existing
Node process with \texttt{SIGUSR1}.

Once a process has been set in debug mode with this it can be connected
to with the node debugger. Either connect to the \texttt{pid} or the URI
to the debugger. The syntax is:

\begin{itemize}
\item
  \texttt{node debug -p \textless{}pid\textgreater{}} - Connects to the
  process via the \texttt{pid}
\item
  `node debug - Connects to the process via the URI such as
  localhost:5858
\end{itemize}
