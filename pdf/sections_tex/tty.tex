\section{TTY}\label{tty}

\begin{Shaded}
\begin{Highlighting}[]
\NormalTok{Stability: }\DecValTok{2} \NormalTok{- Unstable}
\end{Highlighting}
\end{Shaded}

The \texttt{tty} module houses the \texttt{tty.ReadStream} and
\texttt{tty.WriteStream} classes. In most cases, you will not need to
use this module directly.

When node detects that it is being run inside a TTY context, then
\texttt{process.stdin} will be a \texttt{tty.ReadStream} instance and
\texttt{process.stdout} will be a \texttt{tty.WriteStream} instance. The
preferred way to check if node is being run in a TTY context is to check
\texttt{process.stdout.isTTY}:

\begin{Shaded}
\begin{Highlighting}[]
\NormalTok{$ node -p -e }\StringTok{"Boolean(process.stdout.isTTY)"}
\KeywordTok{true}
\NormalTok{$ node -p -e }\StringTok{"Boolean(process.stdout.isTTY)"} \NormalTok{| cat}
\KeywordTok{false}
\end{Highlighting}
\end{Shaded}

\subsection{tty.isatty(fd)}\label{tty.isattyfd}

Returns \texttt{true} or \texttt{false} depending on if the \texttt{fd}
is associated with a terminal.

\subsection{tty.setRawMode(mode)}\label{tty.setrawmodemode}

Deprecated. Use \texttt{tty.ReadStream\#setRawMode()} (i.e.
\texttt{process.stdin.setRawMode()}) instead.

\subsection{Class: ReadStream}\label{class-readstream}

A \texttt{net.Socket} subclass that represents the readable portion of a
tty. In normal circumstances, \texttt{process.stdin} will be the only
\texttt{tty.ReadStream} instance in any node program (only when
\texttt{isatty(0)} is true).

\subsubsection{rs.isRaw}\label{rs.israw}

A \texttt{Boolean} that is initialized to \texttt{false}. It represents
the current ``raw'' state of the \texttt{tty.ReadStream} instance.

\subsubsection{rs.setRawMode(mode)}\label{rs.setrawmodemode}

\texttt{mode} should be \texttt{true} or \texttt{false}. This sets the
properties of the \texttt{tty.ReadStream} to act either as a raw device
or default. \texttt{isRaw} will be set to the resulting mode.

\subsection{Class: WriteStream}\label{class-writestream}

A \texttt{net.Socket} subclass that represents the writable portion of a
tty. In normal circumstances, \texttt{process.stdout} will be the only
\texttt{tty.WriteStream} instance ever created (and only when
\texttt{isatty(1)} is true).

\subsubsection{ws.columns}\label{ws.columns}

A \texttt{Number} that gives the number of columns the TTY currently
has. This property gets updated on ``resize'' events.

\subsubsection{ws.rows}\label{ws.rows}

A \texttt{Number} that gives the number of rows the TTY currently has.
This property gets updated on ``resize'' events.

\subsubsection{\texorpdfstring{Event:
`resize'}{Event: resize}}\label{event-resize}

\texttt{function\ ()\ \{\}}

Emitted by \texttt{refreshSize()} when either of the \texttt{columns} or
\texttt{rows} properties has changed.

\begin{Shaded}
\begin{Highlighting}[]
\OtherTok{process}\NormalTok{.}\OtherTok{stdout}\NormalTok{.}\FunctionTok{on}\NormalTok{(}\StringTok{'resize'}\NormalTok{, }\KeywordTok{function}\NormalTok{() \{}
  \OtherTok{console}\NormalTok{.}\FunctionTok{log}\NormalTok{(}\StringTok{'screen size has changed!'}\NormalTok{);}
  \OtherTok{console}\NormalTok{.}\FunctionTok{log}\NormalTok{(}\OtherTok{process}\NormalTok{.}\OtherTok{stdout}\NormalTok{.}\FunctionTok{columns} \NormalTok{+ }\StringTok{'x'} \NormalTok{+ }\OtherTok{process}\NormalTok{.}\OtherTok{stdout}\NormalTok{.}\FunctionTok{rows}\NormalTok{);}
\NormalTok{\});}
\end{Highlighting}
\end{Shaded}

