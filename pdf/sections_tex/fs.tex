\section{File System}

\begin{Shaded}
\begin{Highlighting}[]
\DataTypeTok{Stability}\NormalTok{: }\DecValTok{3} \NormalTok{- Stable}
\end{Highlighting}
\end{Shaded}

File I/O is provided by simple wrappers around standard POSIX functions.
To use this module do \texttt{require('fs')}. All the methods have
asynchronous and synchronous forms.

The asynchronous form always take a completion callback as its last
argument. The arguments passed to the completion callback depend on the
method, but the first argument is always reserved for an exception. If
the operation was completed successfully, then the first argument will
be \texttt{null} or \texttt{undefined}.

When using the synchronous form any exceptions are immediately thrown.
You can use try/catch to handle exceptions or allow them to bubble up.

Here is an example of the asynchronous version:

\begin{Shaded}
\begin{Highlighting}[]
\KeywordTok{var} \NormalTok{fs = require(}\CharTok{'fs'}\NormalTok{);}

\KeywordTok{fs}\NormalTok{.}\FunctionTok{unlink}\NormalTok{(}\CharTok{'/tmp/hello'}\NormalTok{, }\KeywordTok{function} \NormalTok{(err) \{}
  \KeywordTok{if} \NormalTok{(err) }\KeywordTok{throw} \NormalTok{err;}
  \KeywordTok{console}\NormalTok{.}\FunctionTok{log}\NormalTok{(}\CharTok{'successfully deleted /tmp/hello'}\NormalTok{);}
\NormalTok{\});}
\end{Highlighting}
\end{Shaded}

Here is the synchronous version:

\begin{Shaded}
\begin{Highlighting}[]
\KeywordTok{var} \NormalTok{fs = require(}\CharTok{'fs'}\NormalTok{);}

\KeywordTok{fs}\NormalTok{.}\FunctionTok{unlinkSync}\NormalTok{(}\CharTok{'/tmp/hello'}\NormalTok{)}
\KeywordTok{console}\NormalTok{.}\FunctionTok{log}\NormalTok{(}\CharTok{'successfully deleted /tmp/hello'}\NormalTok{);}
\end{Highlighting}
\end{Shaded}

With the asynchronous methods there is no guaranteed ordering. So the
following is prone to error:

\begin{Shaded}
\begin{Highlighting}[]
\KeywordTok{fs}\NormalTok{.}\FunctionTok{rename}\NormalTok{(}\CharTok{'/tmp/hello'}\NormalTok{, }\CharTok{'/tmp/world'}\NormalTok{, }\KeywordTok{function} \NormalTok{(err) \{}
  \KeywordTok{if} \NormalTok{(err) }\KeywordTok{throw} \NormalTok{err;}
  \KeywordTok{console}\NormalTok{.}\FunctionTok{log}\NormalTok{(}\CharTok{'renamed complete'}\NormalTok{);}
\NormalTok{\});}
\KeywordTok{fs}\NormalTok{.}\FunctionTok{stat}\NormalTok{(}\CharTok{'/tmp/world'}\NormalTok{, }\KeywordTok{function} \NormalTok{(err, stats) \{}
  \KeywordTok{if} \NormalTok{(err) }\KeywordTok{throw} \NormalTok{err;}
  \KeywordTok{console}\NormalTok{.}\FunctionTok{log}\NormalTok{(}\CharTok{'stats: '} \NormalTok{+ }\KeywordTok{JSON}\NormalTok{.}\FunctionTok{stringify}\NormalTok{(stats));}
\NormalTok{\});}
\end{Highlighting}
\end{Shaded}

It could be that \texttt{fs.stat} is executed before \texttt{fs.rename}.
The correct way to do this is to chain the callbacks.

\begin{Shaded}
\begin{Highlighting}[]
\KeywordTok{fs}\NormalTok{.}\FunctionTok{rename}\NormalTok{(}\CharTok{'/tmp/hello'}\NormalTok{, }\CharTok{'/tmp/world'}\NormalTok{, }\KeywordTok{function} \NormalTok{(err) \{}
  \KeywordTok{if} \NormalTok{(err) }\KeywordTok{throw} \NormalTok{err;}
  \KeywordTok{fs}\NormalTok{.}\FunctionTok{stat}\NormalTok{(}\CharTok{'/tmp/world'}\NormalTok{, }\KeywordTok{function} \NormalTok{(err, stats) \{}
    \KeywordTok{if} \NormalTok{(err) }\KeywordTok{throw} \NormalTok{err;}
    \KeywordTok{console}\NormalTok{.}\FunctionTok{log}\NormalTok{(}\CharTok{'stats: '} \NormalTok{+ }\KeywordTok{JSON}\NormalTok{.}\FunctionTok{stringify}\NormalTok{(stats));}
  \NormalTok{\});}
\NormalTok{\});}
\end{Highlighting}
\end{Shaded}

In busy processes, the programmer is \emph{strongly encouraged} to use
the asynchronous versions of these calls. The synchronous versions will
block the entire process until they complete--halting all connections.

Relative path to filename can be used, remember however that this path
will be relative to \texttt{process.cwd()}.

Most fs functions let you omit the callback argument. If you do, a
default callback is used that rethrows errors. To get a trace to the
original call site, set the NODE\_DEBUG environment variable:

\begin{Shaded}
\begin{Highlighting}[]
\NormalTok{$ cat }\KeywordTok{script}\NormalTok{.}\FunctionTok{js}
\FunctionTok{function} \NormalTok{bad() \{}
  \NormalTok{require(}\CharTok{'fs'}\NormalTok{).}\FunctionTok{readFile}\NormalTok{(}\CharTok{'/'}\NormalTok{);}
\NormalTok{\}}
\NormalTok{bad();}

\NormalTok{$ env NODE_DEBUG=fs node }\KeywordTok{script}\NormalTok{.}\FunctionTok{js}
\FunctionTok{fs}\NormalTok{.}\FunctionTok{js}\NormalTok{:}\DecValTok{66}
        \KeywordTok{throw} \NormalTok{err;}
              \NormalTok{^}
\DataTypeTok{Error}\NormalTok{: EISDIR, read}
    \NormalTok{at rethrow (}\DataTypeTok{fs.js}\NormalTok{:}\DecValTok{61}\NormalTok{:}\DecValTok{21}\NormalTok{)}
    \NormalTok{at maybeCallback (}\DataTypeTok{fs.js}\NormalTok{:}\DecValTok{79}\NormalTok{:}\DecValTok{42}\NormalTok{)}
    \NormalTok{at }\KeywordTok{Object}\NormalTok{.}\FunctionTok{fs}\NormalTok{.}\FunctionTok{readFile} \NormalTok{(}\DataTypeTok{fs.js}\NormalTok{:}\DecValTok{153}\NormalTok{:}\DecValTok{18}\NormalTok{)}
    \NormalTok{at bad (}\OtherTok{/}\FloatTok{path}\OtherTok{/}\NormalTok{to/}\DataTypeTok{script.js}\NormalTok{:}\DecValTok{2}\NormalTok{:}\DecValTok{17}\NormalTok{)}
    \NormalTok{at }\KeywordTok{Object}\NormalTok{.<anonymous> (}\OtherTok{/}\FloatTok{path}\OtherTok{/}\NormalTok{to/}\DataTypeTok{script.js}\NormalTok{:}\DecValTok{5}\NormalTok{:}\DecValTok{1}\NormalTok{)}
    \NormalTok{<}\KeywordTok{etc}\NormalTok{.>}
\end{Highlighting}
\end{Shaded}

\subsection{fs.rename(oldPath, newPath, callback)}

Asynchronous rename(2). No arguments other than a possible exception are
given to the completion callback.

\subsection{fs.renameSync(oldPath, newPath)}

Synchronous rename(2).

\subsection{fs.ftruncate(fd, len, callback)}

Asynchronous ftruncate(2). No arguments other than a possible exception
are given to the completion callback.

\subsection{fs.ftruncateSync(fd, len)}

Synchronous ftruncate(2).

\subsection{fs.truncate(path, len, callback)}

Asynchronous truncate(2). No arguments other than a possible exception
are given to the completion callback.

\subsection{fs.truncateSync(path, len)}

Synchronous truncate(2).

\subsection{fs.chown(path, uid, gid, callback)}

Asynchronous chown(2). No arguments other than a possible exception are
given to the completion callback.

\subsection{fs.chownSync(path, uid, gid)}

Synchronous chown(2).

\subsection{fs.fchown(fd, uid, gid, callback)}

Asynchronous fchown(2). No arguments other than a possible exception are
given to the completion callback.

\subsection{fs.fchownSync(fd, uid, gid)}

Synchronous fchown(2).

\subsection{fs.lchown(path, uid, gid, callback)}

Asynchronous lchown(2). No arguments other than a possible exception are
given to the completion callback.

\subsection{fs.lchownSync(path, uid, gid)}

Synchronous lchown(2).

\subsection{fs.chmod(path, mode, callback)}

Asynchronous chmod(2). No arguments other than a possible exception are
given to the completion callback.

\subsection{fs.chmodSync(path, mode)}

Synchronous chmod(2).

\subsection{fs.fchmod(fd, mode, callback)}

Asynchronous fchmod(2). No arguments other than a possible exception are
given to the completion callback.

\subsection{fs.fchmodSync(fd, mode)}

Synchronous fchmod(2).

\subsection{fs.lchmod(path, mode, callback)}

Asynchronous lchmod(2). No arguments other than a possible exception are
given to the completion callback.

Only available on Mac OS X.

\subsection{fs.lchmodSync(path, mode)}

Synchronous lchmod(2).

\subsection{fs.stat(path, callback)}

Asynchronous stat(2). The callback gets two arguments
\texttt{(err, stats)} where \texttt{stats} is a
\hyperref[fs\_class\_fs\_stats]{fs.Stats} object. See the
\hyperref[fs\_class\_fs\_stats]{fs.Stats} section below for more
information.

\subsection{fs.lstat(path, callback)}

Asynchronous lstat(2). The callback gets two arguments
\texttt{(err, stats)} where \texttt{stats} is a \texttt{fs.Stats}
object. \texttt{lstat()} is identical to \texttt{stat()}, except that if
\texttt{path} is a symbolic link, then the link itself is stat-ed, not
the file that it refers to.

\subsection{fs.fstat(fd, callback)}

Asynchronous fstat(2). The callback gets two arguments
\texttt{(err, stats)} where \texttt{stats} is a \texttt{fs.Stats}
object. \texttt{fstat()} is identical to \texttt{stat()}, except that
the file to be stat-ed is specified by the file descriptor \texttt{fd}.

\subsection{fs.statSync(path)}

Synchronous stat(2). Returns an instance of \texttt{fs.Stats}.

\subsection{fs.lstatSync(path)}

Synchronous lstat(2). Returns an instance of \texttt{fs.Stats}.

\subsection{fs.fstatSync(fd)}

Synchronous fstat(2). Returns an instance of \texttt{fs.Stats}.

\subsection{fs.link(srcpath, dstpath, callback)}

Asynchronous link(2). No arguments other than a possible exception are
given to the completion callback.

\subsection{fs.linkSync(srcpath, dstpath)}

Synchronous link(2).

\subsection{fs.symlink(srcpath, dstpath, {[}type{]}, callback)}

Asynchronous symlink(2). No arguments other than a possible exception
are given to the completion callback. \texttt{type} argument can be
either \texttt{'dir'}, \texttt{'file'}, or \texttt{'junction'} (default
is \texttt{'file'}). It is only used on Windows (ignored on other
platforms). Note that Windows junction points require the destination
path to be absolute. When using \texttt{'junction'}, the
\texttt{destination} argument will automatically be normalized to
absolute path.

\subsection{fs.symlinkSync(srcpath, dstpath, {[}type{]})}

Synchronous symlink(2).

\subsection{fs.readlink(path, callback)}

Asynchronous readlink(2). The callback gets two arguments
\texttt{(err, linkString)}.

\subsection{fs.readlinkSync(path)}

Synchronous readlink(2). Returns the symbolic link's string value.

\subsection{fs.realpath(path, {[}cache{]}, callback)}

Asynchronous realpath(2). The \texttt{callback} gets two arguments
\texttt{(err, resolvedPath)}. May use \texttt{process.cwd} to resolve
relative paths. \texttt{cache} is an object literal of mapped paths that
can be used to force a specific path resolution or avoid additional
\texttt{fs.stat} calls for known real paths.

Example:

\begin{Shaded}
\begin{Highlighting}[]
\KeywordTok{var} \NormalTok{cache = \{}\CharTok{'/etc'}\NormalTok{:}\CharTok{'/private/etc'}\NormalTok{\};}
\KeywordTok{fs}\NormalTok{.}\FunctionTok{realpath}\NormalTok{(}\CharTok{'/etc/passwd'}\NormalTok{, cache, }\KeywordTok{function} \NormalTok{(err, resolvedPath) \{}
  \KeywordTok{if} \NormalTok{(err) }\KeywordTok{throw} \NormalTok{err;}
  \KeywordTok{console}\NormalTok{.}\FunctionTok{log}\NormalTok{(resolvedPath);}
\NormalTok{\});}
\end{Highlighting}
\end{Shaded}

\subsection{fs.realpathSync(path, {[}cache{]})}

Synchronous realpath(2). Returns the resolved path.

\subsection{fs.unlink(path, callback)}

Asynchronous unlink(2). No arguments other than a possible exception are
given to the completion callback.

\subsection{fs.unlinkSync(path)}

Synchronous unlink(2).

\subsection{fs.rmdir(path, callback)}

Asynchronous rmdir(2). No arguments other than a possible exception are
given to the completion callback.

\subsection{fs.rmdirSync(path)}

Synchronous rmdir(2).

\subsection{fs.mkdir(path, {[}mode{]}, callback)}

Asynchronous mkdir(2). No arguments other than a possible exception are
given to the completion callback. \texttt{mode} defaults to
\texttt{0777}.

\subsection{fs.mkdirSync(path, {[}mode{]})}

Synchronous mkdir(2).

\subsection{fs.readdir(path, callback)}

Asynchronous readdir(3). Reads the contents of a directory. The callback
gets two arguments \texttt{(err, files)} where \texttt{files} is an
array of the names of the files in the directory excluding \texttt{'.'}
and \texttt{'..'}.

\subsection{fs.readdirSync(path)}

Synchronous readdir(3). Returns an array of filenames excluding
\texttt{'.'} and \texttt{'..'}.

\subsection{fs.close(fd, callback)}

Asynchronous close(2). No arguments other than a possible exception are
given to the completion callback.

\subsection{fs.closeSync(fd)}

Synchronous close(2).

\subsection{fs.open(path, flags, {[}mode{]}, callback)}

Asynchronous file open. See open(2). \texttt{flags} can be:

\begin{itemize}
\item
  \texttt{'r'} - Open file for reading. An exception occurs if the file
  does not exist.
\item
  \texttt{'r+'} - Open file for reading and writing. An exception occurs
  if the file does not exist.
\item
  \texttt{'rs'} - Open file for reading in synchronous mode. Instructs
  the operating system to bypass the local file system cache.
\end{itemize}

This is primarily useful for opening files on NFS mounts as it allows
you to skip the potentially stale local cache. It has a very real impact
on I/O performance so don't use this mode unless you need it.

Note that this doesn't turn \texttt{fs.open()} into a synchronous
blocking call. If that's what you want then you should be using
\texttt{fs.openSync()}

\begin{itemize}
\item
  \texttt{'rs+'} - Open file for reading and writing, telling the OS to
  open it synchronously. See notes for \texttt{'rs'} about using this
  with caution.
\item
  \texttt{'w'} - Open file for writing. The file is created (if it does
  not exist) or truncated (if it exists).
\item
  \texttt{'wx'} - Like \texttt{'w'} but opens the file in exclusive
  mode.
\item
  \texttt{'w+'} - Open file for reading and writing. The file is created
  (if it does not exist) or truncated (if it exists).
\item
  \texttt{'wx+'} - Like \texttt{'w+'} but opens the file in exclusive
  mode.
\item
  \texttt{'a'} - Open file for appending. The file is created if it does
  not exist.
\item
  \texttt{'ax'} - Like \texttt{'a'} but opens the file in exclusive
  mode.
\item
  \texttt{'a+'} - Open file for reading and appending. The file is
  created if it does not exist.
\item
  \texttt{'ax+'} - Like \texttt{'a+'} but opens the file in exclusive
  mode.
\end{itemize}

\texttt{mode} defaults to \texttt{0666}. The callback gets two arguments
\texttt{(err, fd)}.

Exclusive mode (\texttt{O\_EXCL}) ensures that \texttt{path} is newly
created. \texttt{fs.open()} fails if a file by that name already exists.
On POSIX systems, symlinks are not followed. Exclusive mode may or may
not work with network file systems.

On Linux, positional writes don't work when the file is opened in append
mode. The kernel ignores the position argument and always appends the
data to the end of the file.

\subsection{fs.openSync(path, flags, {[}mode{]})}

Synchronous open(2).

\subsection{fs.utimes(path, atime, mtime, callback)}

\subsection{fs.utimesSync(path, atime, mtime)}

Change file timestamps of the file referenced by the supplied path.

\subsection{fs.futimes(fd, atime, mtime, callback)}

\subsection{fs.futimesSync(fd, atime, mtime)}

Change the file timestamps of a file referenced by the supplied file
descriptor.

\subsection{fs.fsync(fd, callback)}

Asynchronous fsync(2). No arguments other than a possible exception are
given to the completion callback.

\subsection{fs.fsyncSync(fd)}

Synchronous fsync(2).

\subsection{fs.write(fd, buffer, offset, length, position, callback)}

Write \texttt{buffer} to the file specified by \texttt{fd}.

\texttt{offset} and \texttt{length} determine the part of the buffer to
be written.

\texttt{position} refers to the offset from the beginning of the file
where this data should be written. If \texttt{position} is
\texttt{null}, the data will be written at the current position. See
pwrite(2).

The callback will be given three arguments
\texttt{(err, written, buffer)} where \texttt{written} specifies how
many \emph{bytes} were written from \texttt{buffer}.

Note that it is unsafe to use \texttt{fs.write} multiple times on the
same file without waiting for the callback. For this scenario,
\texttt{fs.createWriteStream} is strongly recommended.

On Linux, positional writes don't work when the file is opened in append
mode. The kernel ignores the position argument and always appends the
data to the end of the file.

\subsection{fs.writeSync(fd, buffer, offset, length, position)}

Synchronous version of \texttt{fs.write()}. Returns the number of bytes
written.

\subsection{fs.read(fd, buffer, offset, length, position, callback)}

Read data from the file specified by \texttt{fd}.

\texttt{buffer} is the buffer that the data will be written to.

\texttt{offset} is offset within the buffer where reading will start.

\texttt{length} is an integer specifying the number of bytes to read.

\texttt{position} is an integer specifying where to begin reading from
in the file. If \texttt{position} is \texttt{null}, data will be read
from the current file position.

The callback is given the three arguments,
\texttt{(err, bytesRead, buffer)}.

\subsection{fs.readSync(fd, buffer, offset, length, position)}

Synchronous version of \texttt{fs.read}. Returns the number of
\texttt{bytesRead}.

\subsection{fs.readFile(filename, {[}options{]}, callback)}

\begin{itemize}
\item
  \texttt{filename} \{String\}
\item
  \texttt{options} \{Object\}
\item
  \texttt{encoding} \{String \textbar{} Null\} default = \texttt{null}
\item
  \texttt{flag} \{String\} default = \texttt{'r'}
\item
  \texttt{callback} \{Function\}
\end{itemize}

Asynchronously reads the entire contents of a file. Example:

\begin{Shaded}
\begin{Highlighting}[]
\KeywordTok{fs}\NormalTok{.}\FunctionTok{readFile}\NormalTok{(}\CharTok{'/etc/passwd'}\NormalTok{, }\KeywordTok{function} \NormalTok{(err, data) \{}
  \KeywordTok{if} \NormalTok{(err) }\KeywordTok{throw} \NormalTok{err;}
  \KeywordTok{console}\NormalTok{.}\FunctionTok{log}\NormalTok{(data);}
\NormalTok{\});}
\end{Highlighting}
\end{Shaded}

The callback is passed two arguments \texttt{(err, data)}, where
\texttt{data} is the contents of the file.

If no encoding is specified, then the raw buffer is returned.

\subsection{fs.readFileSync(filename, {[}options{]})}

Synchronous version of \texttt{fs.readFile}. Returns the contents of the
\texttt{filename}.

If the \texttt{encoding} option is specified then this function returns
a string. Otherwise it returns a buffer.

\subsection{fs.writeFile(filename, data, {[}options{]}, callback)}

\begin{itemize}
\item
  \texttt{filename} \{String\}
\item
  \texttt{data} \{String \textbar{} Buffer\}
\item
  \texttt{options} \{Object\}
\item
  \texttt{encoding} \{String \textbar{} Null\} default = \texttt{'utf8'}
\item
  \texttt{mode} \{Number\} default = \texttt{438} (aka \texttt{0666} in
  Octal)
\item
  \texttt{flag} \{String\} default = \texttt{'w'}
\item
  \texttt{callback} \{Function\}
\end{itemize}

Asynchronously writes data to a file, replacing the file if it already
exists. \texttt{data} can be a string or a buffer.

The \texttt{encoding} option is ignored if \texttt{data} is a buffer. It
defaults to \texttt{'utf8'}.

Example:

\begin{Shaded}
\begin{Highlighting}[]
\KeywordTok{fs}\NormalTok{.}\FunctionTok{writeFile}\NormalTok{(}\CharTok{'message.txt'}\NormalTok{, }\CharTok{'Hello Node'}\NormalTok{, }\KeywordTok{function} \NormalTok{(err) \{}
  \KeywordTok{if} \NormalTok{(err) }\KeywordTok{throw} \NormalTok{err;}
  \KeywordTok{console}\NormalTok{.}\FunctionTok{log}\NormalTok{(}\CharTok{'It\textbackslash{}'s saved!'}\NormalTok{);}
\NormalTok{\});}
\end{Highlighting}
\end{Shaded}

\subsection{fs.writeFileSync(filename, data, {[}options{]})}

The synchronous version of \texttt{fs.writeFile}.

\subsection{fs.appendFile(filename, data, {[}options{]}, callback)}

\begin{itemize}
\item
  \texttt{filename} \{String\}
\item
  \texttt{data} \{String \textbar{} Buffer\}
\item
  \texttt{options} \{Object\}
\item
  \texttt{encoding} \{String \textbar{} Null\} default = \texttt{'utf8'}
\item
  \texttt{mode} \{Number\} default = \texttt{438} (aka \texttt{0666} in
  Octal)
\item
  \texttt{flag} \{String\} default = \texttt{'a'}
\item
  \texttt{callback} \{Function\}
\end{itemize}

Asynchronously append data to a file, creating the file if it not yet
exists. \texttt{data} can be a string or a buffer.

Example:

\begin{Shaded}
\begin{Highlighting}[]
\KeywordTok{fs}\NormalTok{.}\FunctionTok{appendFile}\NormalTok{(}\CharTok{'message.txt'}\NormalTok{, }\CharTok{'data to append'}\NormalTok{, }\KeywordTok{function} \NormalTok{(err) \{}
  \KeywordTok{if} \NormalTok{(err) }\KeywordTok{throw} \NormalTok{err;}
  \KeywordTok{console}\NormalTok{.}\FunctionTok{log}\NormalTok{(}\CharTok{'The "data to append" was appended to file!'}\NormalTok{);}
\NormalTok{\});}
\end{Highlighting}
\end{Shaded}

\subsection{fs.appendFileSync(filename, data, {[}options{]})}

The synchronous version of \texttt{fs.appendFile}.

\subsection{fs.watchFile(filename, {[}options{]}, listener)}

\begin{Shaded}
\begin{Highlighting}[]
\DataTypeTok{Stability}\NormalTok{: }\DecValTok{2} \NormalTok{- }\KeywordTok{Unstable}\NormalTok{.  Use }\KeywordTok{fs}\NormalTok{.}\FunctionTok{watch} \NormalTok{instead, }\KeywordTok{if} \KeywordTok{possible}\NormalTok{.}
\end{Highlighting}
\end{Shaded}

Watch for changes on \texttt{filename}. The callback \texttt{listener}
will be called each time the file is accessed.

The second argument is optional. The \texttt{options} if provided should
be an object containing two members a boolean, \texttt{persistent}, and
\texttt{interval}. \texttt{persistent} indicates whether the process
should continue to run as long as files are being watched.
\texttt{interval} indicates how often the target should be polled, in
milliseconds. The default is
\texttt{\{ persistent: true, interval: 5007 \}}.

The \texttt{listener} gets two arguments the current stat object and the
previous stat object:

\begin{Shaded}
\begin{Highlighting}[]
\KeywordTok{fs}\NormalTok{.}\FunctionTok{watchFile}\NormalTok{(}\CharTok{'message.text'}\NormalTok{, }\KeywordTok{function} \NormalTok{(curr, prev) \{}
  \KeywordTok{console}\NormalTok{.}\FunctionTok{log}\NormalTok{(}\CharTok{'the current mtime is: '} \NormalTok{+ }\KeywordTok{curr}\NormalTok{.}\FunctionTok{mtime}\NormalTok{);}
  \KeywordTok{console}\NormalTok{.}\FunctionTok{log}\NormalTok{(}\CharTok{'the previous mtime was: '} \NormalTok{+ }\KeywordTok{prev}\NormalTok{.}\FunctionTok{mtime}\NormalTok{);}
\NormalTok{\});}
\end{Highlighting}
\end{Shaded}

These stat objects are instances of \texttt{fs.Stat}.

If you want to be notified when the file was modified, not just accessed
you need to compare \texttt{curr.mtime} and \texttt{prev.mtime}.

\subsection{fs.unwatchFile(filename, {[}listener{]})}

\begin{Shaded}
\begin{Highlighting}[]
\DataTypeTok{Stability}\NormalTok{: }\DecValTok{2} \NormalTok{- }\KeywordTok{Unstable}\NormalTok{.  Use }\KeywordTok{fs}\NormalTok{.}\FunctionTok{watch} \NormalTok{instead, }\KeywordTok{if} \KeywordTok{available}\NormalTok{.}
\end{Highlighting}
\end{Shaded}

Stop watching for changes on \texttt{filename}. If \texttt{listener} is
specified, only that particular listener is removed. Otherwise,
\emph{all} listeners are removed and you have effectively stopped
watching \texttt{filename}.

Calling \texttt{fs.unwatchFile()} with a filename that is not being
watched is a no-op, not an error.

\subsection{fs.watch(filename, {[}options{]}, {[}listener{]})}

\begin{Shaded}
\begin{Highlighting}[]
\DataTypeTok{Stability}\NormalTok{: }\DecValTok{2} \NormalTok{- }\KeywordTok{Unstable}\NormalTok{.}
\end{Highlighting}
\end{Shaded}

Watch for changes on \texttt{filename}, where \texttt{filename} is
either a file or a directory. The returned object is a
\hyperref[fs\_class\_fs\_fswatcher]{fs.FSWatcher}.

The second argument is optional. The \texttt{options} if provided should
be an object containing a boolean member \texttt{persistent}, which
indicates whether the process should continue to run as long as files
are being watched. The default is \texttt{\{ persistent: true \}}.

The listener callback gets two arguments \texttt{(event, filename)}.
\texttt{event} is either `rename' or `change', and \texttt{filename} is
the name of the file which triggered the event.

\subsubsection{Caveats}

The \texttt{fs.watch} API is not 100\% consistent across platforms, and
is unavailable in some situations.

\paragraph{Availability}

This feature depends on the underlying operating system providing a way
to be notified of filesystem changes.

\begin{itemize}
\item
  On Linux systems, this uses \texttt{inotify}.
\item
  On BSD systems (including OS X), this uses \texttt{kqueue}.
\item
  On SunOS systems (including Solaris and SmartOS), this uses
  \texttt{event ports}.
\item
  On Windows systems, this feature depends on
  \texttt{ReadDirectoryChangesW}.
\end{itemize}

If the underlying functionality is not available for some reason, then
\texttt{fs.watch} will not be able to function. For example, watching
files or directories on network file systems (NFS, SMB, etc.) often
doesn't work reliably or at all.

You can still use \texttt{fs.watchFile}, which uses stat polling, but it
is slower and less reliable.

\paragraph{Filename Argument}

Providing \texttt{filename} argument in the callback is not supported on
every platform (currently it's only supported on Linux and Windows).
Even on supported platforms \texttt{filename} is not always guaranteed
to be provided. Therefore, don't assume that \texttt{filename} argument
is always provided in the callback, and have some fallback logic if it
is null.

\begin{Shaded}
\begin{Highlighting}[]
\KeywordTok{fs}\NormalTok{.}\FunctionTok{watch}\NormalTok{(}\CharTok{'somedir'}\NormalTok{, }\KeywordTok{function} \NormalTok{(event, filename) \{}
  \KeywordTok{console}\NormalTok{.}\FunctionTok{log}\NormalTok{(}\CharTok{'event is: '} \NormalTok{+ event);}
  \KeywordTok{if} \NormalTok{(filename) \{}
    \KeywordTok{console}\NormalTok{.}\FunctionTok{log}\NormalTok{(}\CharTok{'filename provided: '} \NormalTok{+ filename);}
  \NormalTok{\} }\KeywordTok{else} \NormalTok{\{}
    \KeywordTok{console}\NormalTok{.}\FunctionTok{log}\NormalTok{(}\CharTok{'filename not provided'}\NormalTok{);}
  \NormalTok{\}}
\NormalTok{\});}
\end{Highlighting}
\end{Shaded}

\subsection{fs.exists(path, callback)}

Test whether or not the given path exists by checking with the file
system. Then call the \texttt{callback} argument with either true or
false. Example:

\begin{Shaded}
\begin{Highlighting}[]
\KeywordTok{fs}\NormalTok{.}\FunctionTok{exists}\NormalTok{(}\CharTok{'/etc/passwd'}\NormalTok{, }\KeywordTok{function} \NormalTok{(exists) \{}
  \KeywordTok{util}\NormalTok{.}\FunctionTok{debug}\NormalTok{(exists ? }\StringTok{"it's there"} \NormalTok{: }\StringTok{"no passwd!"}\NormalTok{);}
\NormalTok{\});}
\end{Highlighting}
\end{Shaded}

\subsection{fs.existsSync(path)}

Synchronous version of \texttt{fs.exists}.

\subsection{Class: fs.Stats}

Objects returned from \texttt{fs.stat()}, \texttt{fs.lstat()} and
\texttt{fs.fstat()} and their synchronous counterparts are of this type.

\begin{itemize}
\item
  \texttt{stats.isFile()}
\item
  \texttt{stats.isDirectory()}
\item
  \texttt{stats.isBlockDevice()}
\item
  \texttt{stats.isCharacterDevice()}
\item
  \texttt{stats.isSymbolicLink()} (only valid with \texttt{fs.lstat()})
\item
  \texttt{stats.isFIFO()}
\item
  \texttt{stats.isSocket()}
\end{itemize}

For a regular file \texttt{util.inspect(stats)} would return a string
very similar to this:

\begin{Shaded}
\begin{Highlighting}[]
\NormalTok{\{ }\DataTypeTok{dev}\NormalTok{: }\DecValTok{2114}\NormalTok{,}
  \DataTypeTok{ino}\NormalTok{: }\DecValTok{48064969}\NormalTok{,}
  \DataTypeTok{mode}\NormalTok{: }\DecValTok{33188}\NormalTok{,}
  \DataTypeTok{nlink}\NormalTok{: }\DecValTok{1}\NormalTok{,}
  \DataTypeTok{uid}\NormalTok{: }\DecValTok{85}\NormalTok{,}
  \DataTypeTok{gid}\NormalTok{: }\DecValTok{100}\NormalTok{,}
  \DataTypeTok{rdev}\NormalTok{: }\DecValTok{0}\NormalTok{,}
  \DataTypeTok{size}\NormalTok{: }\DecValTok{527}\NormalTok{,}
  \DataTypeTok{blksize}\NormalTok{: }\DecValTok{4096}\NormalTok{,}
  \DataTypeTok{blocks}\NormalTok{: }\DecValTok{8}\NormalTok{,}
  \DataTypeTok{atime}\NormalTok{: Mon, }\DecValTok{10} \NormalTok{Oct }\DecValTok{2011} \DecValTok{23}\NormalTok{:}\DecValTok{24}\NormalTok{:}\DecValTok{11} \NormalTok{GMT,}
  \DataTypeTok{mtime}\NormalTok{: Mon, }\DecValTok{10} \NormalTok{Oct }\DecValTok{2011} \DecValTok{23}\NormalTok{:}\DecValTok{24}\NormalTok{:}\DecValTok{11} \NormalTok{GMT,}
  \DataTypeTok{ctime}\NormalTok{: Mon, }\DecValTok{10} \NormalTok{Oct }\DecValTok{2011} \DecValTok{23}\NormalTok{:}\DecValTok{24}\NormalTok{:}\DecValTok{11} \NormalTok{GMT \}}
\end{Highlighting}
\end{Shaded}

Please note that \texttt{atime}, \texttt{mtime} and \texttt{ctime} are
instances of
\href{https://developer.mozilla.org/en/JavaScript/Reference/Global\_Objects/Date}{Date}
object and to compare the values of these objects you should use
appropriate methods. For most general uses
\href{https://developer.mozilla.org/en/JavaScript/Reference/Global\_Objects/Date/getTime}{getTime()}
will return the number of milliseconds elapsed since \emph{1 January
1970 00:00:00 UTC} and this integer should be sufficient for any
comparison, however there additional methods which can be used for
displaying fuzzy information. More details can be found in the
\href{https://developer.mozilla.org/en/JavaScript/Reference/Global\_Objects/Date}{MDN
JavaScript Reference} page.

\subsection{fs.createReadStream(path, {[}options{]})}

Returns a new ReadStream object (See \texttt{Readable Stream}).

\texttt{options} is an object with the following defaults:

\begin{Shaded}
\begin{Highlighting}[]
\NormalTok{\{ }\DataTypeTok{flags}\NormalTok{: }\CharTok{'r'}\NormalTok{,}
  \DataTypeTok{encoding}\NormalTok{: null,}
  \DataTypeTok{fd}\NormalTok{: null,}
  \DataTypeTok{mode}\NormalTok{: }\DecValTok{0666}\NormalTok{,}
  \DataTypeTok{bufferSize}\NormalTok{: }\DecValTok{64} \NormalTok{* }\DecValTok{1024}\NormalTok{,}
  \DataTypeTok{autoClose}\NormalTok{: }\KeywordTok{true}
\NormalTok{\}}
\end{Highlighting}
\end{Shaded}

\texttt{options} can include \texttt{start} and \texttt{end} values to
read a range of bytes from the file instead of the entire file. Both
\texttt{start} and \texttt{end} are inclusive and start at 0. The
\texttt{encoding} can be \texttt{'utf8'}, \texttt{'ascii'}, or
\texttt{'base64'}.

If \texttt{autoClose} is false, then the file descriptor won't be
closed, even if there's an error. It is your responsibility to close it
and make sure there's no file descriptor leak. If \texttt{autoClose} is
set to true (default behavior), on \texttt{error} or \texttt{end} the
file descriptor will be closed automatically.

An example to read the last 10 bytes of a file which is 100 bytes long:

\begin{Shaded}
\begin{Highlighting}[]
\KeywordTok{fs}\NormalTok{.}\FunctionTok{createReadStream}\NormalTok{(}\CharTok{'sample.txt'}\NormalTok{, \{}\DataTypeTok{start}\NormalTok{: }\DecValTok{90}\NormalTok{, }\DataTypeTok{end}\NormalTok{: }\DecValTok{99}\NormalTok{\});}
\end{Highlighting}
\end{Shaded}

\subsection{Class: fs.ReadStream}

\texttt{ReadStream} is a
\href{stream.html\#stream\_readable\_stream}{Readable Stream}.

\subsubsection{Event: `open'}

\begin{itemize}
\item
  \texttt{fd} \{Integer\} file descriptor used by the ReadStream.
\end{itemize}

Emitted when the ReadStream's file is opened.

\subsection{fs.createWriteStream(path, {[}options{]})}

Returns a new WriteStream object (See \texttt{Writable Stream}).

\texttt{options} is an object with the following defaults:

\begin{Shaded}
\begin{Highlighting}[]
\NormalTok{\{ }\DataTypeTok{flags}\NormalTok{: }\CharTok{'w'}\NormalTok{,}
  \DataTypeTok{encoding}\NormalTok{: null,}
  \DataTypeTok{mode}\NormalTok{: }\DecValTok{0666} \NormalTok{\}}
\end{Highlighting}
\end{Shaded}

\texttt{options} may also include a \texttt{start} option to allow
writing data at some position past the beginning of the file. Modifying
a file rather than replacing it may require a \texttt{flags} mode of
\texttt{r+} rather than the default mode \texttt{w}.

\subsection{fs.WriteStream}

\texttt{WriteStream} is a
\href{stream.html\#stream\_writable\_stream}{Writable Stream}.

\subsubsection{Event: `open'}

\begin{itemize}
\item
  \texttt{fd} \{Integer\} file descriptor used by the WriteStream.
\end{itemize}

Emitted when the WriteStream's file is opened.

\subsubsection{file.bytesWritten}

The number of bytes written so far. Does not include data that is still
queued for writing.

\subsection{Class: fs.FSWatcher}

Objects returned from \texttt{fs.watch()} are of this type.

\subsubsection{watcher.close()}

Stop watching for changes on the given \texttt{fs.FSWatcher}.

\subsubsection{Event: `change'}

\begin{itemize}
\item
  \texttt{event} \{String\} The type of fs change
\item
  \texttt{filename} \{String\} The filename that changed (if
  relevant/available)
\end{itemize}

Emitted when something changes in a watched directory or file. See more
details in
\hyperref[fs\_fs\_watch\_filename\_options\_listener]{fs.watch}.

\subsubsection{Event: `error'}

\begin{itemize}
\item
  \texttt{error} \{Error object\}
\end{itemize}

Emitted when an error occurs.
