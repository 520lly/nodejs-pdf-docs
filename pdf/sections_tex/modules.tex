\section{Modules}\label{modules}

\begin{Shaded}
\begin{Highlighting}[]
\NormalTok{Stability: }\DecValTok{5} \NormalTok{- Locked}
\end{Highlighting}
\end{Shaded}

Node has a simple module loading system. In Node, files and modules are
in one-to-one correspondence. As an example, \texttt{foo.js} loads the
module \texttt{circle.js} in the same directory.

The contents of \texttt{foo.js}:

\begin{Shaded}
\begin{Highlighting}[]
\KeywordTok{var} \NormalTok{circle = }\FunctionTok{require}\NormalTok{(}\StringTok{'./circle.js'}\NormalTok{);}
\OtherTok{console}\NormalTok{.}\FunctionTok{log}\NormalTok{( }\StringTok{'The area of a circle of radius 4 is '}
           \NormalTok{+ }\OtherTok{circle}\NormalTok{.}\FunctionTok{area}\NormalTok{(}\DecValTok{4}\NormalTok{));}
\end{Highlighting}
\end{Shaded}

The contents of \texttt{circle.js}:

\begin{Shaded}
\begin{Highlighting}[]
\KeywordTok{var} \NormalTok{PI = }\OtherTok{Math}\NormalTok{.}\FunctionTok{PI}\NormalTok{;}

\OtherTok{exports}\NormalTok{.}\FunctionTok{area} \NormalTok{= }\KeywordTok{function} \NormalTok{(r) \{}
  \KeywordTok{return} \NormalTok{PI * r * r;}
\NormalTok{\};}

\OtherTok{exports}\NormalTok{.}\FunctionTok{circumference} \NormalTok{= }\KeywordTok{function} \NormalTok{(r) \{}
  \KeywordTok{return} \DecValTok{2} \NormalTok{* PI * r;}
\NormalTok{\};}
\end{Highlighting}
\end{Shaded}

The module \texttt{circle.js} has exported the functions \texttt{area()}
and \texttt{circumference()}. To add functions and objects to the root
of your module, you can add them to the special \texttt{exports} object.

Variables local to the module will be private, as though the module was
wrapped in a function. In this example the variable \texttt{PI} is
private to \texttt{circle.js}.

If you want the root of your module's export to be a function (such as a
constructor) or if you want to export a complete object in one
assignment instead of building it one property at a time, assign it to
\texttt{module.exports} instead of \texttt{exports}.

Below, \texttt{bar.js} makes use of the \texttt{square} module, which
exports a constructor:

\begin{Shaded}
\begin{Highlighting}[]
\KeywordTok{var} \NormalTok{square = }\FunctionTok{require}\NormalTok{(}\StringTok{'./square.js'}\NormalTok{);}
\KeywordTok{var} \NormalTok{mySquare = }\FunctionTok{square}\NormalTok{(}\DecValTok{2}\NormalTok{);}
\OtherTok{console}\NormalTok{.}\FunctionTok{log}\NormalTok{(}\StringTok{'The area of my square is '} \NormalTok{+ }\OtherTok{mySquare}\NormalTok{.}\FunctionTok{area}\NormalTok{());}
\end{Highlighting}
\end{Shaded}

The \texttt{square} module is defined in \texttt{square.js}:

\begin{Shaded}
\begin{Highlighting}[]
\CommentTok{// assigning to exports will not modify module, must use module.exports}
\OtherTok{module}\NormalTok{.}\FunctionTok{exports} \NormalTok{= }\KeywordTok{function}\NormalTok{(width) \{}
  \KeywordTok{return} \NormalTok{\{}
    \DataTypeTok{area}\NormalTok{: }\KeywordTok{function}\NormalTok{() \{}
      \KeywordTok{return} \NormalTok{width * width;}
    \NormalTok{\}}
  \NormalTok{\};}
\NormalTok{\}}
\end{Highlighting}
\end{Shaded}

The module system is implemented in the \texttt{require("module")}
module.

\subsection{Cycles}\label{cycles}

When there are circular \texttt{require()} calls, a module might not
have finished executing when it is returned.

Consider this situation:

\texttt{a.js}:

\begin{Shaded}
\begin{Highlighting}[]
\OtherTok{console}\NormalTok{.}\FunctionTok{log}\NormalTok{(}\StringTok{'a starting'}\NormalTok{);}
\OtherTok{exports}\NormalTok{.}\FunctionTok{done} \NormalTok{= }\KeywordTok{false}\NormalTok{;}
\KeywordTok{var} \NormalTok{b = }\FunctionTok{require}\NormalTok{(}\StringTok{'./b.js'}\NormalTok{);}
\OtherTok{console}\NormalTok{.}\FunctionTok{log}\NormalTok{(}\StringTok{'in a, b.done = %j'}\NormalTok{, }\OtherTok{b}\NormalTok{.}\FunctionTok{done}\NormalTok{);}
\OtherTok{exports}\NormalTok{.}\FunctionTok{done} \NormalTok{= }\KeywordTok{true}\NormalTok{;}
\OtherTok{console}\NormalTok{.}\FunctionTok{log}\NormalTok{(}\StringTok{'a done'}\NormalTok{);}
\end{Highlighting}
\end{Shaded}

\texttt{b.js}:

\begin{Shaded}
\begin{Highlighting}[]
\OtherTok{console}\NormalTok{.}\FunctionTok{log}\NormalTok{(}\StringTok{'b starting'}\NormalTok{);}
\OtherTok{exports}\NormalTok{.}\FunctionTok{done} \NormalTok{= }\KeywordTok{false}\NormalTok{;}
\KeywordTok{var} \NormalTok{a = }\FunctionTok{require}\NormalTok{(}\StringTok{'./a.js'}\NormalTok{);}
\OtherTok{console}\NormalTok{.}\FunctionTok{log}\NormalTok{(}\StringTok{'in b, a.done = %j'}\NormalTok{, }\OtherTok{a}\NormalTok{.}\FunctionTok{done}\NormalTok{);}
\OtherTok{exports}\NormalTok{.}\FunctionTok{done} \NormalTok{= }\KeywordTok{true}\NormalTok{;}
\OtherTok{console}\NormalTok{.}\FunctionTok{log}\NormalTok{(}\StringTok{'b done'}\NormalTok{);}
\end{Highlighting}
\end{Shaded}

\texttt{main.js}:

\begin{Shaded}
\begin{Highlighting}[]
\OtherTok{console}\NormalTok{.}\FunctionTok{log}\NormalTok{(}\StringTok{'main starting'}\NormalTok{);}
\KeywordTok{var} \NormalTok{a = }\FunctionTok{require}\NormalTok{(}\StringTok{'./a.js'}\NormalTok{);}
\KeywordTok{var} \NormalTok{b = }\FunctionTok{require}\NormalTok{(}\StringTok{'./b.js'}\NormalTok{);}
\OtherTok{console}\NormalTok{.}\FunctionTok{log}\NormalTok{(}\StringTok{'in main, a.done=%j, b.done=%j'}\NormalTok{, }\OtherTok{a}\NormalTok{.}\FunctionTok{done}\NormalTok{, }\OtherTok{b}\NormalTok{.}\FunctionTok{done}\NormalTok{);}
\end{Highlighting}
\end{Shaded}

When \texttt{main.js} loads \texttt{a.js}, then \texttt{a.js} in turn
loads \texttt{b.js}. At that point, \texttt{b.js} tries to load
\texttt{a.js}. In order to prevent an infinite loop, an
\textbf{unfinished copy} of the \texttt{a.js} exports object is returned
to the \texttt{b.js} module. \texttt{b.js} then finishes loading, and
its \texttt{exports} object is provided to the \texttt{a.js} module.

By the time \texttt{main.js} has loaded both modules, they're both
finished. The output of this program would thus be:

\begin{Shaded}
\begin{Highlighting}[]
\NormalTok{$ node }\OtherTok{main}\NormalTok{.}\FunctionTok{js}
\NormalTok{main starting}
\NormalTok{a starting}
\NormalTok{b starting}
\KeywordTok{in} \NormalTok{b, }\OtherTok{a}\NormalTok{.}\FunctionTok{done} \NormalTok{= }\KeywordTok{false}
\NormalTok{b done}
\KeywordTok{in} \NormalTok{a, }\OtherTok{b}\NormalTok{.}\FunctionTok{done} \NormalTok{= }\KeywordTok{true}
\NormalTok{a done}
\KeywordTok{in} \NormalTok{main, }\OtherTok{a}\NormalTok{.}\FunctionTok{done}\NormalTok{=}\KeywordTok{true}\NormalTok{, }\OtherTok{b}\NormalTok{.}\FunctionTok{done}\NormalTok{=}\KeywordTok{true}
\end{Highlighting}
\end{Shaded}

If you have cyclic module dependencies in your program, make sure to
plan accordingly.

\subsection{Core Modules}\label{core-modules}

Node has several modules compiled into the binary. These modules are
described in greater detail elsewhere in this documentation.

The core modules are defined in node's source in the \texttt{lib/}
folder.

Core modules are always preferentially loaded if their identifier is
passed to \texttt{require()}. For instance,
\texttt{require(\textquotesingle{}http\textquotesingle{})} will always
return the built in HTTP module, even if there is a file by that name.

\subsection{File Modules}\label{file-modules}

If the exact filename is not found, then node will attempt to load the
required filename with the added extension of \texttt{.js},
\texttt{.json}, and then \texttt{.node}.

\texttt{.js} files are interpreted as JavaScript text files, and
\texttt{.json} files are parsed as JSON text files. \texttt{.node} files
are interpreted as compiled addon modules loaded with \texttt{dlopen}.

A module prefixed with \texttt{\textquotesingle{}/\textquotesingle{}} is
an absolute path to the file. For example,
\texttt{require(\textquotesingle{}/home/marco/foo.js\textquotesingle{})}
will load the file at \texttt{/home/marco/foo.js}.

A module prefixed with \texttt{\textquotesingle{}./\textquotesingle{}}
is relative to the file calling \texttt{require()}. That is,
\texttt{circle.js} must be in the same directory as \texttt{foo.js} for
\texttt{require(\textquotesingle{}./circle\textquotesingle{})} to find
it.

Without a leading `/' or `./' to indicate a file, the module is either a
``core module'' or is loaded from a \texttt{node\_modules} folder.

If the given path does not exist, \texttt{require()} will throw an Error
with its \texttt{code} property set to
\texttt{\textquotesingle{}MODULE\_NOT\_FOUND\textquotesingle{}}.

\subsection{\texorpdfstring{Loading from \texttt{node\_modules}
Folders}{Loading from node\_modules Folders}}\label{loading-from-nodeux5fmodules-folders}

If the module identifier passed to \texttt{require()} is not a native
module, and does not begin with
\texttt{\textquotesingle{}/\textquotesingle{}},
\texttt{\textquotesingle{}../\textquotesingle{}}, or
\texttt{\textquotesingle{}./\textquotesingle{}}, then node starts at the
parent directory of the current module, and adds
\texttt{/node\_modules}, and attempts to load the module from that
location.

If it is not found there, then it moves to the parent directory, and so
on, until the root of the file system is reached.

For example, if the file at
\texttt{\textquotesingle{}/home/ry/projects/foo.js\textquotesingle{}}
called \texttt{require(\textquotesingle{}bar.js\textquotesingle{})},
then node would look in the following locations, in this order:

\begin{itemize}
\itemsep1pt\parskip0pt\parsep0pt
\item
  \texttt{/home/ry/projects/node\_modules/bar.js}
\item
  \texttt{/home/ry/node\_modules/bar.js}
\item
  \texttt{/home/node\_modules/bar.js}
\item
  \texttt{/node\_modules/bar.js}
\end{itemize}

This allows programs to localize their dependencies, so that they do not
clash.

You can require specific files or sub modules distributed with a module
by including a path suffix after the module name. For instance
\texttt{require(\textquotesingle{}example-module/path/to/file\textquotesingle{})}
would resolve \texttt{path/to/file} relative to where
\texttt{example-module} is located. The suffixed path follows the same
module resolution semantics.

\subsection{Folders as Modules}\label{folders-as-modules}

It is convenient to organize programs and libraries into self-contained
directories, and then provide a single entry point to that library.
There are three ways in which a folder may be passed to
\texttt{require()} as an argument.

The first is to create a \texttt{package.json} file in the root of the
folder, which specifies a \texttt{main} module. An example package.json
file might look like this:

\begin{Shaded}
\begin{Highlighting}[]
\NormalTok{\{ }\StringTok{"name"} \NormalTok{: }\StringTok{"some-library"}\NormalTok{,}
  \StringTok{"main"} \NormalTok{: }\StringTok{"./lib/some-library.js"} \NormalTok{\}}
\end{Highlighting}
\end{Shaded}

If this was in a folder at \texttt{./some-library}, then
\texttt{require(\textquotesingle{}./some-library\textquotesingle{})}
would attempt to load \texttt{./some-library/lib/some-library.js}.

This is the extent of Node's awareness of package.json files.

If there is no package.json file present in the directory, then node
will attempt to load an \texttt{index.js} or \texttt{index.node} file
out of that directory. For example, if there was no package.json file in
the above example, then
\texttt{require(\textquotesingle{}./some-library\textquotesingle{})}
would attempt to load:

\begin{itemize}
\itemsep1pt\parskip0pt\parsep0pt
\item
  \texttt{./some-library/index.js}
\item
  \texttt{./some-library/index.node}
\end{itemize}

\subsection{Caching}\label{caching}

Modules are cached after the first time they are loaded. This means
(among other things) that every call to
\texttt{require(\textquotesingle{}foo\textquotesingle{})} will get
exactly the same object returned, if it would resolve to the same file.

Multiple calls to
\texttt{require(\textquotesingle{}foo\textquotesingle{})} may not cause
the module code to be executed multiple times. This is an important
feature. With it, ``partially done'' objects can be returned, thus
allowing transitive dependencies to be loaded even when they would cause
cycles.

If you want to have a module execute code multiple times, then export a
function, and call that function.

\subsubsection{Module Caching Caveats}\label{module-caching-caveats}

Modules are cached based on their resolved filename. Since modules may
resolve to a different filename based on the location of the calling
module (loading from \texttt{node\_modules} folders), it is not a
\emph{guarantee} that
\texttt{require(\textquotesingle{}foo\textquotesingle{})} will always
return the exact same object, if it would resolve to different files.

\subsection{\texorpdfstring{The \texttt{module}
Object}{The module Object}}\label{the-module-object}

\begin{itemize}
\itemsep1pt\parskip0pt\parsep0pt
\item
  \{Object\}
\end{itemize}

In each module, the \texttt{module} free variable is a reference to the
object representing the current module. For convenience,
\texttt{module.exports} is also accessible via the \texttt{exports}
module-global. \texttt{module} isn't actually a global but rather local
to each module.

\subsubsection{module.exports}\label{module.exports}

\begin{itemize}
\itemsep1pt\parskip0pt\parsep0pt
\item
  \{Object\}
\end{itemize}

The \texttt{module.exports} object is created by the Module system.
Sometimes this is not acceptable; many want their module to be an
instance of some class. To do this, assign the desired export object to
\texttt{module.exports}. Note that assigning the desired object to
\texttt{exports} will simply rebind the local \texttt{exports} variable,
which is probably not what you want to do.

For example suppose we were making a module called \texttt{a.js}

\begin{Shaded}
\begin{Highlighting}[]
\KeywordTok{var} \NormalTok{EventEmitter = }\FunctionTok{require}\NormalTok{(}\StringTok{'events'}\NormalTok{).}\FunctionTok{EventEmitter}\NormalTok{;}

\OtherTok{module}\NormalTok{.}\FunctionTok{exports} \NormalTok{= }\KeywordTok{new} \FunctionTok{EventEmitter}\NormalTok{();}

\CommentTok{// Do some work, and after some time emit}
\CommentTok{// the 'ready' event from the module itself.}
\FunctionTok{setTimeout}\NormalTok{(}\KeywordTok{function}\NormalTok{() \{}
  \OtherTok{module}\NormalTok{.}\OtherTok{exports}\NormalTok{.}\FunctionTok{emit}\NormalTok{(}\StringTok{'ready'}\NormalTok{);}
\NormalTok{\}, }\DecValTok{1000}\NormalTok{);}
\end{Highlighting}
\end{Shaded}

Then in another file we could do

\begin{Shaded}
\begin{Highlighting}[]
\KeywordTok{var} \NormalTok{a = }\FunctionTok{require}\NormalTok{(}\StringTok{'./a'}\NormalTok{);}
\OtherTok{a}\NormalTok{.}\FunctionTok{on}\NormalTok{(}\StringTok{'ready'}\NormalTok{, }\KeywordTok{function}\NormalTok{() \{}
  \OtherTok{console}\NormalTok{.}\FunctionTok{log}\NormalTok{(}\StringTok{'module a is ready'}\NormalTok{);}
\NormalTok{\});}
\end{Highlighting}
\end{Shaded}

Note that assignment to \texttt{module.exports} must be done
immediately. It cannot be done in any callbacks. This does not work:

x.js:

\begin{Shaded}
\begin{Highlighting}[]
\FunctionTok{setTimeout}\NormalTok{(}\KeywordTok{function}\NormalTok{() \{}
  \OtherTok{module}\NormalTok{.}\FunctionTok{exports} \NormalTok{= \{ }\DataTypeTok{a}\NormalTok{: }\StringTok{"hello"} \NormalTok{\};}
\NormalTok{\}, }\DecValTok{0}\NormalTok{);}
\end{Highlighting}
\end{Shaded}

y.js:

\begin{Shaded}
\begin{Highlighting}[]
\KeywordTok{var} \NormalTok{x = }\FunctionTok{require}\NormalTok{(}\StringTok{'./x'}\NormalTok{);}
\OtherTok{console}\NormalTok{.}\FunctionTok{log}\NormalTok{(}\OtherTok{x}\NormalTok{.}\FunctionTok{a}\NormalTok{);}
\end{Highlighting}
\end{Shaded}

\paragraph{exports alias}\label{exports-alias}

The \texttt{exports} variable that is available within a module starts
as a reference to \texttt{module.exports}. As with any variable, if you
assign a new value to it, it is no longer bound to the previous value.

To illustrate the behaviour, imagine this hypothetical implementation of
\texttt{require()}:

\begin{Shaded}
\begin{Highlighting}[]
\KeywordTok{function} \FunctionTok{require}\NormalTok{(...) \{}
  \CommentTok{// ...}
  \KeywordTok{function} \NormalTok{(module, exports) \{}
    \CommentTok{// Your module code here}
    \NormalTok{exports = some_func;        }\CommentTok{// re-assigns exports, exports is no longer}
                                \CommentTok{// a shortcut, and nothing is exported.}
    \OtherTok{module}\NormalTok{.}\FunctionTok{exports} \NormalTok{= some_func; }\CommentTok{// makes your module export 0}
  \NormalTok{\} (module, }\OtherTok{module}\NormalTok{.}\FunctionTok{exports}\NormalTok{);}
  \KeywordTok{return} \NormalTok{module;}
\NormalTok{\}}
\end{Highlighting}
\end{Shaded}

As a guideline, if the relationship between \texttt{exports} and
\texttt{module.exports} seems like magic to you, ignore \texttt{exports}
and only use \texttt{module.exports}.

\subsubsection{module.require(id)}\label{module.requireid}

\begin{itemize}
\itemsep1pt\parskip0pt\parsep0pt
\item
  \texttt{id} \{String\}
\item
  Return: \{Object\} \texttt{module.exports} from the resolved module
\end{itemize}

The \texttt{module.require} method provides a way to load a module as if
\texttt{require()} was called from the original module.

Note that in order to do this, you must get a reference to the
\texttt{module} object. Since \texttt{require()} returns the
\texttt{module.exports}, and the \texttt{module} is typically
\emph{only} available within a specific module's code, it must be
explicitly exported in order to be used.

\subsubsection{module.id}\label{module.id}

\begin{itemize}
\itemsep1pt\parskip0pt\parsep0pt
\item
  \{String\}
\end{itemize}

The identifier for the module. Typically this is the fully resolved
filename.

\subsubsection{module.filename}\label{module.filename}

\begin{itemize}
\itemsep1pt\parskip0pt\parsep0pt
\item
  \{String\}
\end{itemize}

The fully resolved filename to the module.

\subsubsection{module.loaded}\label{module.loaded}

\begin{itemize}
\itemsep1pt\parskip0pt\parsep0pt
\item
  \{Boolean\}
\end{itemize}

Whether or not the module is done loading, or is in the process of
loading.

\subsubsection{module.parent}\label{module.parent}

\begin{itemize}
\itemsep1pt\parskip0pt\parsep0pt
\item
  \{Module Object\}
\end{itemize}

The module that required this one.

\subsubsection{module.children}\label{module.children}

\begin{itemize}
\itemsep1pt\parskip0pt\parsep0pt
\item
  \{Array\}
\end{itemize}

The module objects required by this one.

\subsection{All Together\ldots{}}\label{all-together}

To get the exact filename that will be loaded when \texttt{require()} is
called, use the \texttt{require.resolve()} function.

Putting together all of the above, here is the high-level algorithm in
pseudocode of what require.resolve does:

\begin{Shaded}
\begin{Highlighting}[]
\FunctionTok{require}\NormalTok{(X) from module at path Y}
\DecValTok{1}\NormalTok{. }\FunctionTok{If} \FunctionTok{X} \FunctionTok{is} \FunctionTok{a} \FunctionTok{core} \FunctionTok{module}\NormalTok{,}
   \OtherTok{a}\NormalTok{. }\FunctionTok{return} \FunctionTok{the} \FunctionTok{core} \FunctionTok{module}
   \OtherTok{b}\NormalTok{. }\FunctionTok{STOP}
\DecValTok{2}\NormalTok{. }\FunctionTok{If} \FunctionTok{X} \FunctionTok{begins} \FunctionTok{with} \StringTok{'./'} \NormalTok{or }\StringTok{'/'} \NormalTok{or }\StringTok{'../'}
   \OtherTok{a}\NormalTok{. }\FunctionTok{LOAD_AS_FILE}\NormalTok{(Y + X)}
   \OtherTok{b}\NormalTok{. }\FunctionTok{LOAD_AS_DIRECTORY}\NormalTok{(Y + X)}
\DecValTok{3}\NormalTok{. }\FunctionTok{LOAD_NODE_MODULES}\NormalTok{(X, }\FunctionTok{dirname}\NormalTok{(Y))}
\DecValTok{4}\NormalTok{. }\FunctionTok{THROW} \StringTok{"not found"}

\FunctionTok{LOAD_AS_FILE}\NormalTok{(X)}
\DecValTok{1}\NormalTok{. }\FunctionTok{If} \FunctionTok{X} \FunctionTok{is} \FunctionTok{a} \FunctionTok{file}\NormalTok{, load X as JavaScript }\OtherTok{text}\NormalTok{.  }\FunctionTok{STOP}
\DecValTok{2}\NormalTok{. }\FunctionTok{If} \OtherTok{X}\NormalTok{.}\FunctionTok{js} \FunctionTok{is} \FunctionTok{a} \FunctionTok{file}\NormalTok{, load }\OtherTok{X}\NormalTok{.}\FunctionTok{js} \FunctionTok{as} \FunctionTok{JavaScript} \OtherTok{text}\NormalTok{.  }\FunctionTok{STOP}
\DecValTok{3}\NormalTok{. }\FunctionTok{If} \OtherTok{X}\NormalTok{.}\FunctionTok{json} \FunctionTok{is} \FunctionTok{a} \FunctionTok{file}\NormalTok{, parse }\OtherTok{X}\NormalTok{.}\FunctionTok{json} \FunctionTok{to} \FunctionTok{a} \FunctionTok{JavaScript} \OtherTok{Object}\NormalTok{.  }\FunctionTok{STOP}
\DecValTok{4}\NormalTok{. }\FunctionTok{If} \OtherTok{X}\NormalTok{.}\FunctionTok{node} \FunctionTok{is} \FunctionTok{a} \FunctionTok{file}\NormalTok{, load }\OtherTok{X}\NormalTok{.}\FunctionTok{node} \FunctionTok{as} \FunctionTok{binary} \OtherTok{addon}\NormalTok{.  }\FunctionTok{STOP}

\FunctionTok{LOAD_AS_DIRECTORY}\NormalTok{(X)}
\DecValTok{1}\NormalTok{. }\FunctionTok{If} \FunctionTok{X}\NormalTok{/}\KeywordTok{package}\NormalTok{.}\FunctionTok{json} \FunctionTok{is} \FunctionTok{a} \FunctionTok{file}\NormalTok{,}
   \OtherTok{a}\NormalTok{. }\FunctionTok{Parse} \FunctionTok{X}\NormalTok{/}\KeywordTok{package}\NormalTok{.}\FunctionTok{json}\NormalTok{, and look }\KeywordTok{for} \StringTok{"main"} \OtherTok{field}\NormalTok{.}
   \OtherTok{b}\NormalTok{. }\FunctionTok{let} \FunctionTok{M} \NormalTok{= X + (json main field)}
   \OtherTok{c}\NormalTok{. }\FunctionTok{LOAD_AS_FILE}\NormalTok{(M)}
\DecValTok{2}\NormalTok{. }\FunctionTok{If} \FunctionTok{X}\NormalTok{/}\OtherTok{index}\NormalTok{.}\FunctionTok{js} \FunctionTok{is} \FunctionTok{a} \FunctionTok{file}\NormalTok{, load X/}\OtherTok{index}\NormalTok{.}\FunctionTok{js} \FunctionTok{as} \FunctionTok{JavaScript} \OtherTok{text}\NormalTok{.  }\FunctionTok{STOP}
\DecValTok{3}\NormalTok{. }\FunctionTok{If} \FunctionTok{X}\NormalTok{/}\OtherTok{index}\NormalTok{.}\FunctionTok{json} \FunctionTok{is} \FunctionTok{a} \FunctionTok{file}\NormalTok{, parse X/}\OtherTok{index}\NormalTok{.}\FunctionTok{json} \FunctionTok{to} \FunctionTok{a} \FunctionTok{JavaScript} \OtherTok{object}\NormalTok{. }\FunctionTok{STOP}
\DecValTok{4}\NormalTok{. }\FunctionTok{If} \FunctionTok{X}\NormalTok{/}\OtherTok{index}\NormalTok{.}\FunctionTok{node} \FunctionTok{is} \FunctionTok{a} \FunctionTok{file}\NormalTok{, load X/}\OtherTok{index}\NormalTok{.}\FunctionTok{node} \FunctionTok{as} \FunctionTok{binary} \OtherTok{addon}\NormalTok{.  }\FunctionTok{STOP}

\FunctionTok{LOAD_NODE_MODULES}\NormalTok{(X, START)}
\DecValTok{1}\NormalTok{. }\FunctionTok{let} \FunctionTok{DIRS}\NormalTok{=}\FunctionTok{NODE_MODULES_PATHS}\NormalTok{(START)}
\DecValTok{2}\NormalTok{. }\FunctionTok{for} \FunctionTok{each} \FunctionTok{DIR} \FunctionTok{in} \FunctionTok{DIRS}\NormalTok{:}
   \OtherTok{a}\NormalTok{. }\FunctionTok{LOAD_AS_FILE}\NormalTok{(DIR/X)}
   \OtherTok{b}\NormalTok{. }\FunctionTok{LOAD_AS_DIRECTORY}\NormalTok{(DIR/X)}

\FunctionTok{NODE_MODULES_PATHS}\NormalTok{(START)}
\DecValTok{1}\NormalTok{. }\FunctionTok{let} \FunctionTok{PARTS} \NormalTok{= path }\FunctionTok{split}\NormalTok{(START)}
\DecValTok{2}\NormalTok{. }\FunctionTok{let} \FunctionTok{I} \NormalTok{= count of PARTS - }\DecValTok{1}
\DecValTok{3}\NormalTok{. }\FunctionTok{let} \FunctionTok{DIRS} \NormalTok{= []}
\DecValTok{4}\NormalTok{. }\FunctionTok{while} \FunctionTok{I} \NormalTok{>= }\DecValTok{0}\NormalTok{,}
   \OtherTok{a}\NormalTok{. }\FunctionTok{if} \FunctionTok{PARTS}\NormalTok{[I] = }\StringTok{"node_modules"} \NormalTok{CONTINUE}
   \OtherTok{c}\NormalTok{. }\FunctionTok{DIR} \NormalTok{= path }\FunctionTok{join}\NormalTok{(PARTS[}\DecValTok{0} \NormalTok{.. }\FunctionTok{I}\NormalTok{] + }\StringTok{"node_modules"}\NormalTok{)}
   \OtherTok{b}\NormalTok{. }\FunctionTok{DIRS} \NormalTok{= DIRS + DIR}
   \OtherTok{c}\NormalTok{. }\FunctionTok{let} \FunctionTok{I} \NormalTok{= I - }\DecValTok{1}
\DecValTok{5}\NormalTok{. }\FunctionTok{return} \FunctionTok{DIRS}
\end{Highlighting}
\end{Shaded}

\subsection{Loading from the global
folders}\label{loading-from-the-global-folders}

If the \texttt{NODE\_PATH} environment variable is set to a
colon-delimited list of absolute paths, then node will search those
paths for modules if they are not found elsewhere. (Note: On Windows,
\texttt{NODE\_PATH} is delimited by semicolons instead of colons.)

Additionally, node will search in the following locations:

\begin{itemize}
\itemsep1pt\parskip0pt\parsep0pt
\item
  1: \texttt{\$HOME/.node\_modules}
\item
  2: \texttt{\$HOME/.node\_libraries}
\item
  3: \texttt{\$PREFIX/lib/node}
\end{itemize}

Where \texttt{\$HOME} is the user's home directory, and
\texttt{\$PREFIX} is node's configured \texttt{node\_prefix}.

These are mostly for historic reasons. You are highly encouraged to
place your dependencies locally in \texttt{node\_modules} folders. They
will be loaded faster, and more reliably.

\subsection{Accessing the main module}\label{accessing-the-main-module}

When a file is run directly from Node, \texttt{require.main} is set to
its \texttt{module}. That means that you can determine whether a file
has been run directly by testing

\begin{Shaded}
\begin{Highlighting}[]
\OtherTok{require}\NormalTok{.}\FunctionTok{main} \NormalTok{=== module}
\end{Highlighting}
\end{Shaded}

For a file \texttt{foo.js}, this will be \texttt{true} if run via
\texttt{node\ foo.js}, but \texttt{false} if run by
\texttt{require(\textquotesingle{}./foo\textquotesingle{})}.

Because \texttt{module} provides a \texttt{filename} property (normally
equivalent to \texttt{\_\_filename}), the entry point of the current
application can be obtained by checking \texttt{require.main.filename}.

\subsection{Addenda: Package Manager
Tips}\label{addenda-package-manager-tips}

The semantics of Node's \texttt{require()} function were designed to be
general enough to support a number of sane directory structures. Package
manager programs such as \texttt{dpkg}, \texttt{rpm}, and \texttt{npm}
will hopefully find it possible to build native packages from Node
modules without modification.

Below we give a suggested directory structure that could work:

Let's say that we wanted to have the folder at
\texttt{/usr/lib/node/\textless{}some-package\textgreater{}/\textless{}some-version\textgreater{}}
hold the contents of a specific version of a package.

Packages can depend on one another. In order to install package
\texttt{foo}, you may have to install a specific version of package
\texttt{bar}. The \texttt{bar} package may itself have dependencies, and
in some cases, these dependencies may even collide or form cycles.

Since Node looks up the \texttt{realpath} of any modules it loads (that
is, resolves symlinks), and then looks for their dependencies in the
\texttt{node\_modules} folders as described above, this situation is
very simple to resolve with the following architecture:

\begin{itemize}
\itemsep1pt\parskip0pt\parsep0pt
\item
  \texttt{/usr/lib/node/foo/1.2.3/} - Contents of the \texttt{foo}
  package, version 1.2.3.
\item
  \texttt{/usr/lib/node/bar/4.3.2/} - Contents of the \texttt{bar}
  package that \texttt{foo} depends on.
\item
  \texttt{/usr/lib/node/foo/1.2.3/node\_modules/bar} - Symbolic link to
  \texttt{/usr/lib/node/bar/4.3.2/}.
\item
  \texttt{/usr/lib/node/bar/4.3.2/node\_modules/*} - Symbolic links to
  the packages that \texttt{bar} depends on.
\end{itemize}

Thus, even if a cycle is encountered, or if there are dependency
conflicts, every module will be able to get a version of its dependency
that it can use.

When the code in the \texttt{foo} package does
\texttt{require(\textquotesingle{}bar\textquotesingle{})}, it will get
the version that is symlinked into
\texttt{/usr/lib/node/foo/1.2.3/node\_modules/bar}. Then, when the code
in the \texttt{bar} package calls
\texttt{require(\textquotesingle{}quux\textquotesingle{})}, it'll get
the version that is symlinked into
\texttt{/usr/lib/node/bar/4.3.2/node\_modules/quux}.

Furthermore, to make the module lookup process even more optimal, rather
than putting packages directly in \texttt{/usr/lib/node}, we could put
them in
\texttt{/usr/lib/node\_modules/\textless{}name\textgreater{}/\textless{}version\textgreater{}}.
Then node will not bother looking for missing dependencies in
\texttt{/usr/node\_modules} or \texttt{/node\_modules}.

In order to make modules available to the node REPL, it might be useful
to also add the \texttt{/usr/lib/node\_modules} folder to the
\texttt{\$NODE\_PATH} environment variable. Since the module lookups
using \texttt{node\_modules} folders are all relative, and based on the
real path of the files making the calls to \texttt{require()}, the
packages themselves can be anywhere.
